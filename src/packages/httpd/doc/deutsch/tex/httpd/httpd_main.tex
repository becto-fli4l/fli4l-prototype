% Last Update: $Id: httpd_main.tex 48040 2017-05-11 17:26:15Z lanspezi $

\section{HTTPD - Status-Webserver}

\marklabel{OPTHTTPD}{\subsection{OPT\_HTTPD - Mini-Webserver als Statusmonitor }}\index{OPT\_HTTPD}

  Wer aus irgendeinem Grund keine Möglichkeit hat, den IMONC zu benutzen,
  weil er z.B. einen Mac benutzt, kann den Webserver benutzen, um den
  Status des fli4l-Routers abzurufen oder zu ändern. Mit \var{OPT\_HTTPD='yes'}
  kann man den Statusmonitor verwenden.

  Um den Status abzurufen, muss man in seinen Browser eine der folgenden
  Adressen eingeben:

\begin{example}
\begin{verbatim}
    http://fli4l/
    http://fli4l.domain.lan/
    http://192.168.6.1/
\end{verbatim}
\end{example}

  Hat der fli4l-Router einen abweichenden Namen, muss dieser statt ``fli4l''
  verwendet werden. Dies gilt auch für den Domain-Namen und die obige
  IP-Adresse. Hat man den Webserver auf einen anderen Port gelegt (per
  \var{HTTPD\_\-PORT}), muss man diesen mit angeben:

\begin{example}
\begin{verbatim}
    http://fli4l:81/
\end{verbatim}
\end{example}

  Es wird seit der Version 2.1.12 eine Login-Seite angezeigt, die nicht
  passwortgeschützt ist. Die passwortgeschützten Seiten befinden sich im
  Unterverzeichnis admin, also beispielsweise:
\begin{example}
\begin{verbatim}
    http://fli4l.domain.lan/admin/
\end{verbatim}
\end{example}

   Der Webserver lässt sich über folgende Variablen anpassen:

\begin{description}

\config{HTTPD\_GUI\_LANG}{HTTPD\_GUI\_LANG}{HTTPDGUILANG}

    {Hiermit wird die Sprache eingestellt, in der das Webinterface
    dargestellt werden soll. Wird hier 'auto' eingetragen, wird die 
    Spracheinstellung der Variablen \var{LOCALE} (in der base.txt) 
    verwendet.}

\config{HTTPD\_LISTENIP}{HTTPD\_LISTENIP}{HTTPDLISTENIP}

Der Webserver bindet sich normalerweise an eine sogenannte
Wildcard-Adresse, so dass er auf einem beliebigen Interface
angesprochen werden kann. Soll er sich nur an eine IP-Adresse binden,
so kann es mit diesem Parameter eingestellt werden. Dazu die
IP-Adressse wie folgt eintragen: \var{IP\_NET\_x\_IPADDR}.  Normalerweise
bleibt dieser Parameter leer, damit die Standardeinstellung
(ansprechbar auf einer beliebigen IP) greift.

Dieser Parameter dient lediglich dazu, den httpd nur an eine IP zu
binden, so dass sich andere Instanzen an die anderen IPs des Routers
binden können. Er kann nicht ohne weiteres dazu genutzt werden, den
Zugang einzelner Subnetze zum Webinterface des Routers zu
sperren. Dazu benötigt man weiterhin die Hilfe des Paketfilters.

Es ist auch möglich, hier durch Leerzeichen getrennt mehrere IP anzugeben.

\config{HTTPD\_PORT}{HTTPD\_PORT}{HTTPDPORT}

    {Soll der Webserver auf einem anderen Port laufen als 80, so ist
    dieser Wert anzupassen. Das ist normalerweise nicht zu empfehlen, da
    dann z.B. mit http://fli4l:81/ auf den Webserver zugegriffen werden
    muß.}

\config{HTTPD\_PORTFW}{HTTPD\_PORTFW}{HTTPDPORTFW} { Setzt man diese
  Variable auf 'yes', kann man über das Webinterface Änderungen an der
  Portweiterleitung vornehmen. Es können Regeln gelöscht und
  hinzugefügt werden, Änderungen werden sofort wirksam. Änderungen an
  den Regeln gelten nur für die Laufzeit des Routers. Wird der Router
  neu gestartet, sind die Änderungen weg. 

  Diese Variable hat einen Defaultwert von 'yes'.}

\config{HTTPD\_ARPING}{HTTPD\_ARPING}{HTTPDARPING} { Der Webserver
  stellt den Online-Zustand der mit \var{HOST\_x} aufgelisteten Hosts
  dar. Dazu verwendet er den \emph{``Arp-Cache''}, ein Zwischenspeicher
  der die Adressen der lokalen Hosts zwischenspeichert. Hat ein
  Rechner lange nicht mehr mit dem Router kommuniziert, verschwindet
  seine Adresse aus dem \emph{``Arp-Cache''} und der Host scheint aus
  zu sein. Will man den \emph{``Arp-Cache''} aktuell halten (also das
  rausfallen eigentlich nicht benötigter Einträge verhindern), kann
  man \var{HTTPD\_ARPING} auf \emph{'yes'} setzen.}
  
\config{HTTPD\_ARPING\_IGNORE\_N}{HTTPD\_ARPING\_IGNORE\_N}{HTTPDARPINGIGNOREN}
  {Legt die Anzahl der zu Ignorierenden Einträge fest}
  
\config{HTTPD\_ARPING\_IGNORE\_x}{HTTPD\_ARPING\_IGNORE\_x}{HTTPDARPINGIGNOREx}
  { IP-Adresse oder Name des Hosts der nicht bei den ARPING-Test geprüft werden soll.
  Dies kann z.B. sinnvoll sein, bei Hosts die durch die regelmäßigen Netzwerpakete 
  Ihren Akku schneller verbrauchen (Handys im WLAN).}
  
\end{description}

\subsection{Nutzerverwaltung }
  Der Webserver bietet eine ausgefeilte Benutzerverwaltung:
\begin{description}
\config{HTTPD\_USER\_N}{HTTPD\_USER\_N}{HTTPDUSERN}
    {Hiermit wird die Anzahl der Benutzer eingestellt. Wird diese Variable
    auf 0 gesetzt, wird die Benutzerverwaltung komplett deaktiviert und
    jeder hat die Möglichkeit, auf den Webserver zuzugreifen.}

\configlabel{HTTPD\_USER\_x\_PASSWORD}{HTTPDUSERxPASSWORD}
\configlabel{HTTPD\_USER\_x\_RIGHTS}{HTTPDUSERxRIGHTS}
\config{HTTPD\_USER\_x\_USERNAME  HTTPD\_USER\_x\_PASSWORD  HTTPD\_USER\_x\_RIGHTS}
       {HTTPD\_USER\_x\_USERNAME}
       {HTTPDUSERxUSERNAME}
   
    {Hier werden Benutzername und Passwort der einzelnen Benutzer
    eingetragen. Desweiteren wird für jeden Benutzer angegeben, auf
    welche Funktionen des Webservers er zugreifen darf. Diese Funktion
    wird mit der Variable \var{HTTPD\_\-RIGHTS\_\-x} geregelt. Im einfachsten Fall
    steht dort nur 'all', was bedeutet, dass der entsprechende Benutzer
    alles darf. Ansonsten hat die Variable den folgenden Aufbau:

\begin{example}
\begin{verbatim}
       'bereich1:recht1,recht2,... bereich2:...'
\end{verbatim}
\end{example}

    Statt für einen Bereich die einzelnen Rechte anzugeben, darf auch
    hier das Wort ``all'' eingesetzt werden, was wiederum heißt, das
    dieser Benutzer in diesem Bereich alle Rechte hat. Dabei gibt es
    folgende Bereiche und Rechte:

      \begin{description}
      \item[Bereich ``status''] Alles, was im Menü Status zu sehen ist.
        \begin{description}
        \item[view] Der Benutzer darf alle Menüpunkte aufrufen.
        \item[dial] Der Benutzer darf wählen und auflegen.
        \item[boot] Der Benutzer darf den Router herunterfahren \&
                    neu starten.
        \item[link] Der Benutzer darf Kanalbündlung an- und abschalten.
        \item[circuit] Der Benutzer darf den Circuit wechseln.
        \item[dialmode] Der Benutzer daf den Dialmode (Auto, Manual,
                        Off) ändern.
        \item[conntrack] Der Benutzer darf die aktuell über den Router laufenden
	                  Verbindungen ansehen.
        \item[dyndns] Der Benutzer darf Log-Meldungen des \jump{sec:dyndns}{\var{DYNDNS}} Paketes sehen.
       \end{description}

      \item[Bereich ``logs''] Alles, was mit Logdateien zu tun hat
            (Verbindungen, Anrufe, Syslog)
        \begin{description}
        \item[view] Der Benutzer darf die Logdateien betrachten.
        \item[reset] Der Benutzer darf die Logdateien löschen.
        \end{description}

      \item[Bereich ``support''] Alles, was nützlich ist, wenn man beispielsweise
                    in der Newsgroup Hilfestellung sucht.
        \begin{description}
        \item[view] Der Benutzer darf die Links zur Doku, fli4l-Webseite, usw. abrufen
        \item[systeminfo] Der Benutzer darf detaillierte Informationen zur Konfiguration
	                  und zum aktuellen Status des Routers (z. B.: Firewall) abfragen.
        \end{description}

      \end{description}

    Hier noch einige Beispiele:
        
      \begin{description}
      \item[HTTPD\_USER\_1\_RIGHTS='all']
        Diese Angabe erlaubt einem Benutzer alles!

      \item[HTTPD\_USER\_2\_RIGHTS='status:view logs:view support:all']
        Dieser Benutzer darf sich zwar alles ansehen, aber nichts
        ändern.

      \item[HTTPD\_USER\_3\_RIGHTS='status:view,dial,link']
        Dieser Benutzer darf sich den Status der Internetverbindung
        ansehen, wählen und Kanalbündelung ein- und ausschalten.

      \item[HTTPD\_USER\_4\_RIGHTS='status:all']
        Dieser Benutzer darf alles mit der Internetverbindung machen
        und neu starten (natürlich auch herunterfahren). Er darf aber
        nicht die Logdateien sehen oder löschen, die Timetable darf
        er auch nicht sehen...
      \end{description}
    }

\end{description}

\subsection{OPT\_OAC - Online Access Control}

\begin{description}

\config{OPT\_OAC}{OPT\_OAC}{OPTOAC} (optionale Variable)

    Aktiviert das Modul 'Online Access Control'.
    Hierüber kann der Internet-Zugang jedes im Paket \jump{sec:dnsdhcp}{dns\_dhcp}
    konfigurierten Rechners selektiv deaktiviert werden.

    Es gibt auch ein Kommandozeilen-Tool, welches die Steuerung über
    andere Pakete wie z.B. EasyCron möglich macht:

    /usr/local/bin/oac.sh

    Die Optionen werden beim Aufruf angezeigt.

\config{OAC\_WANDEVICE}{OAC\_WANDEVICE}{OACWANDEVICE} (optionale Variable)

    Schränkt die Online Zugangssperre auf Verbindungen über dieses
    Netzwerkdevice ein. z.B. 'pppoe'

\config{OAC\_INPUT}{OAC\_INPUT}{OACINPUT} (optionale Variable)

    Bietet Schutz vor der Umgehung via Proxy.
    
    OAC\_INPUT='default' sperrt die konfigurierten Ports von:
    Privoxy, Squid, Tor, SS5, Transproxy.
    
    OAC\_INPUT='tcp:8080 tcp:3128' sperrt TCP Port 8080 und 3128.
    Dies ist eine Space-separierte Liste an zu sperrenden Ports mit
    dazugehörendem Protokoll (udp, tcp). Fehlt das Protokoll, werden
    udp und tcp Ports gesperrt.
    
    Weglassen dieser Variable oder Inhalt 'no' deaktiviert die Funktion.

\config{OAC\_ALL\_INVISIBLE}{OAC\_ALL\_INVISIBLE}{OACALLINVISIBLE} (optionale Variable)

    Schaltet die Gesamtübersicht aus, wenn mindestens eine sichtbare Gruppe existiert.
    Existiert nicht mindestens eine sichtbare Gruppe, so hat diese Variable keine Wirkung.

\config{OAC\_LIMITS}{OAC\_LIMITS}{OACLIMITS} (optionale Variable)

    Gibt eine durch Leerzeichen getrennte Liste der Zeitlimits an, die zur Auswahl stehen.
    Die Limits werden in Minuten angegeben. Damit kann eine zeitlich limitierte Sperre oder
    Freigabe erreicht werden.

    Default: '30 60 90 120 180 360 540'
    
\config{OAC\_MODE}{OAC\_MODE}{OACMODE} (optionale Variable)

    Mögliche Werte: \var{'DROP'} oder \var{'REJECT'} (default)

\config{OAC\_GROUP\_N}{OAC\_GROUP\_N}{OACGROUPN} (optionale Variable)

    Anzahl der Clientgruppen. Dient der Übersichtlichkeit, erlaubt aber
    auch über das Web-Interface eine gesamte Gruppe gesammelt freizugeben
    oder zu sperren.

\config{OAC\_GROUP\_x\_NAME}{OAC\_GROUP\_x\_NAME}{OACGROUPxNAME} (optionale Variable)

    Name der Gruppe - Dieser Name wird im Web-Interface angezeigt und ist auch
    über die das Kommandozeilenscript 'oac.sh' nutzbar.

\config{OAC\_GROUP\_x\_BOOTBLOCK}{OAC\_GROUP\_x\_BOOTBLOCK}{OACGROUPxBOOTBLOCK} (optionale Variable)

    Wenn hier 'yes' steht, werden alle Clients der Gruppe beim Bootvorgang bereits
    gesperrt. Hilfreich, wenn Rechner idr. gesperrt sein sollen und nur ausnahmsweise nicht.

\config{OAC\_GROUP\_x\_INVISIBLE}{OAC\_GROUP\_x\_INVISIBLE}{OACGROUPxINVISIBLE} (optionale Variable)

    Markiert die Gruppe als unsichtbar. Sinnvoll, wenn einige Rechner vorgesperrt werden sollen
    aber diese nicht als eigene Gruppe im Web-If sichtbar sein sollen.
    Das Kommandozeilentool oac.sh kann diese trotzdem ansprechen, wenn man das braucht, z.B. von
    easycron aus.

\config{OAC\_GROUP\_x\_CLIENT\_N}{OAC\_GROUP\_x\_CLIENT\_N}{OACGROUPxCLIENTN} (optionale Variable)

    Anzahl der Clients in der Gruppe.

\config{OAC\_GROUP\_x\_CLIENT\_x}{OAC\_GROUP\_x\_CLIENT\_x}{OACGROUPxCLIENTx} (optionale Variable)

    Name des Clients wie in {\var{HOST\_x\_NAME}} des Pakets \jump{sec:dnsdhcp}{dns\_dhcp} definiert.

\config{OAC\_BLOCK\_UNKNOWN\_IF\_x}{OAC\_BLOCK\_UNKNOWN\_IF\_x}{OACBLOCKUNKNOWNIF} (optionale Variable)

    Liste der in base.txt definierten Interfaces, auf welchen nur in dns\_dhcp.txt definierte Hosts
    ins Internet dürfen. Nicht-definierte Hosts sind hier damit generell gesperrt.

\end{description}

