% Synchronized to r48039
\section{HTTPD - Webserver For Status-Display}

\marklabel{OPTHTTPD}{\subsection{OPT\_HTTPD - Mini-Webserver As Status-Display}}\index{OPT\_HTTPD}

  The webserver can be used to display or change the status of fli4l 
  routers (IMONC can be used too). The status monitor can be activated by 
  setting \verb*?OPT_HTTPD='yes'?.

  If you are using the default configuration set your browser to one 
  of the following addresses:

\begin{example}
\begin{verbatim}
    http://fli4l/
    http://fli4l.domain.lan/ 
    http://192.168.6.1/
\end{verbatim}
\end{example}

  If you configured fli4l to use another name and/or domain these have 
  to be used. If the webserver is set to listen on another port than the 
  default one specify it like this:

\begin{example}
\begin{verbatim}
    http://fli4l:81/
\end{verbatim}
\end{example}

  As of version 2.1.12  a login page will be displayed which is not 
  protected by a password. Protected pages are in the subdirectory 
  admin, for example:
\begin{example}
\begin{verbatim}
    http://fli4l.domain.lan/admin/
\end{verbatim}
\end{example}

   The web server can be configured by setting the following variables:

\begin{description}

\config{HTTPD\_GUI\_LANG}{HTTPD\_GUI\_LANG}{HTTPDGUILANG}

    {This specifies the language in which the web interface is shown. 
    If set to 'auto' the language setting is taken from the variable
    \var{LOCALE} in base.txt.}

\config{HTTPD\_LISTENIP}{HTTPD\_LISTENIP}{HTTPDLISTENIP}

The web server usually binds to a so-called wildcard address 
in order to be accessed on any router interface. Set the web server 
with this parameter to only bind to one IP address. The corresponding 
IP address is given here: \var{IP\_NET\_x\_IPADDR}. Normally
this parameter is left blank, so the default (Accessible on any 
interface IP) is used.

This parameter is used to bind the httpd to only one IP so that other 
instances can bind to other IPs on the router. It can not be used 
to limit access to the web interface of the router. This would need 
additional configuring of the packet filter, too.

It is also possible to specify multiple IP addresses here (separated 
by spaces).

\config{HTTPD\_PORT}{HTTPD\_PORT}{HTTPDPORT}

    {Set this value if the web server should run on another port than 
    80. This is usually not recommended, since then it must be 
    accessed through the browser by adding the port number. Example 
    http://fli4l:81/.}

\config{HTTPD\_PORTFW}{HTTPD\_PORTFW}{HTTPDPORTFW} {If setting this
  variable to 'yes' you can change port forwarding via the web 
  interface. Rules can be erased and/or added. Changes take effect 
  immediately and only apply during router runtime. If the router is 
  restarted, the changes are gone. The default setting is 'yes'.}

\config{HTTPD\_ARPING}{HTTPD\_ARPING}{HTTPDARPING} {The web server displays
  the online/offline state of the hosts listed in \var{HOST\_x}. To achieve 
  this it uses the \emph{``Arp-Cache''}, a cache that buffers the addresses 
  of the local hosts. If a host does not communicate with the router for 
  longer its address will disappear from the \emph{``Arp-Cache''} and it 
  appears to be offline. If you want to keep the \emph{``Arp-Cache''} up-to-date 
  (keep the online hosts that are not communicating with the router in it) 
  set \var{HTTPD\_ARPING='yes'?}.}

\config{HTTPD\_ARPING\_IGNORE\_N}{HTTPD\_ARPING\_IGNORE\_N}{HTTPDARPINGIGNOREN}
  {Sets the number of entries to be ignored when arping}
  
\config{HTTPD\_ARPING\_IGNORE\_x}{HTTPD\_ARPING\_IGNORE\_x}{HTTPDARPINGIGNOREx}
  {IP-adress or name of the hosts not to be included in ARPING-tests.
  This may be useful for battery based hosts consuming too much power by 
  regular network queries (i.e. laptops or cell phones in WLAN).}
  
\end{description}

\subsection{User Management}
  The webserver provides a sophisticated user management:
\begin{description}
\config{HTTPD\_USER\_N}{HTTPD\_USER\_N}{HTTPDUSERN}
    {Specify the number of users. If set to 0 user management 
    will be switched off completely so everybody can access the 
    web server.}

\configlabel{HTTPD\_USER\_x\_PASSWORD}{HTTPDUSERxPASSWORD}
\configlabel{HTTPD\_USER\_x\_RIGHTS}{HTTPDUSERxRIGHTS}
\config{HTTPD\_USER\_x\_USERNAME  HTTPD\_USER\_x\_PASSWORD  HTTPD\_USER\_x\_RIGHTS}
       {HTTPD\_USER\_x\_USERNAME}
       {HTTPDUSERxUSERNAME}
   
    {enter username and password for each user here. On top specify for each 
    user which functions of the the web server should be accessible to him. 
    Functions are controlled via the variable \var{HTTPD\_\-RIGHTS\_\-x}. 
    In the simplest case it is 'all', which means that the corresponding user
    is allowed to access everything. The variable has the following 
    structure:

\begin{example}
\begin{verbatim}
       'Range1: right1,right2,... Range2:...'
\end{verbatim}
\end{example}

    Instead of adding all rights for a certain range the word ``all'' can be used.
    This means that the user has all rights in this range. The following ranges and 
    rights exist:

      \begin{description}
      \item[Range ``status''] Everything in menu 'Status'.
        \begin{description}
        \item[view] User can access all menu items.
        \item[dial] User can dial and hang up connections.
        \item[boot] User can reboot and shut down the router.
        \item[link] User can switch channel bundeling.
        \item[circuit] User can switch circuits.
        \item[dialmode] User can switch dialmodes (Auto, Manual, Off).
        \item[conntrack] User can view currently active connections.
        \item[dyndns] User can view logfiles of package \jump{sec:dyndns}{\var{DYNDNS}}.
       \end{description}

      \item[Range ``logs''] Everything concerning log files (connections, calls, syslog)
        \begin{description}
        \item[view] User can view logfiles.
        \item[reset] User can delete logfiles.
        \end{description}

      \item[Range ``support''] Everything of use for getting help (Newsgroups a.s.o.).
        \begin{description}
        \item[view] User can access links to documentation, fli4l-website, a.s.o.
        \item[systeminfo] User can query detailed informations about configuration 
	                  and actual status of the router (i.e.: firewall).
        \end{description}

      \end{description}

    Some examples:
        
      \begin{description}
      \item[HTTPD\_USER\_1\_RIGHTS='all']
        Allow a user to access everything in the webserver!

      \item[HTTPD\_USER\_2\_RIGHTS='status:view logs:view support:all']
        User can view everything but can't change settings.

      \item[HTTPD\_USER\_3\_RIGHTS='status:view,dial,link']
        User can view the status of internet connections, dial and switch channel bundeling.

      \item[HTTPD\_USER\_4\_RIGHTS='status:all']
        User can do everything concerning internet connections
        and reboot/shutdown. He is not allowed to view or delete 
	logfiles (nor timetables).
      \end{description}
    }

\end{description}

\subsection{OPT\_OAC - Online Access Control}

\begin{description}

\config{OPT\_OAC}{OPT\_OAC}{OPTOAC} (optional)
 
    Activates 'Online Access Control'. By using this internet access of each host 
    configured in package \jump{sec:dnsdhcp}{dns\_dhcp} can be controlled selectively.

    A console tool is available too, providing an interface to other
    packages like EasyCron:

    /usr/local/bin/oac.sh

    Options will be shown when executed on a console.

\config{OAC\_WANDEVICE}{OAC\_WANDEVICE}{OACWANDEVICE} (optional)

    Restricts the online access control to connections on this
    network device (i.e. 'Pppoe').

\config{OAC\_INPUT}{OAC\_INPUT}{OACINPUT} (optional)

    Provides protection against circumvention via proxy.
    
    OAC\_INPUT='default' blocks default ports for Privoxy, Squid, Tor, SS5, Transproxy.
    
    OAC\_INPUT='tcp:8080 tcp:3128' blocks TCP Port 8080 and 3128.
    This is a space separated list of ports to be blocked and their respective 
    protocol (udp, tcp). Omitting protocols blocks both udp and tcp.
    
    Omitting this variable or setting it to \verb*?'no'? deactivates the function.

\config{OAC\_ALL\_INVISIBLE}{OAC\_ALL\_INVISIBLE}{OACALLINVISIBLE} (optional)

    Turns off overview if at least one group exists. If no groups exist the 
    variable is without effect.
    
\config{OAC\_LIMITS}{OAC\_LIMITS}{OACLIMITS} (optional)

    List of available time limits separated by spaces. Limits are set in minutes. 
    This allows time period based blocking or access definition.

    Default: '30 60 90 120 180 360 540'
    
\config{OAC\_MODE}{OAC\_MODE}{OACMODE} (optional)

    Possible values: \var{'DROP'} or \var{'REJECT'} (default)

\config{OAC\_GROUP\_N}{OAC\_GROUP\_N}{OACGROUPN} (optional)

    Number of client groups. Used for clarity but also allows to block or allow 
    access for a whole group at once over the web interface.

\config{OAC\_GROUP\_x\_NAME}{OAC\_GROUP\_x\_NAME}{OACGROUPxNAME} (optional)

    Name of the group - this name will be displayed in the web interface and may be 
    used in console script 'oac.sh'.

\config{OAC\_GROUP\_x\_BOOTBLOCK}{OAC\_GROUP\_x\_BOOTBLOCK}{OACGROUPxBOOTBLOCK} (optional)

    If set to \verb*?'yes'? all clients of the group are blocked at boot. Useful if 
    PCs should be blocked in general.

\config{OAC\_GROUP\_x\_INVISIBLE}{OAC\_GROUP\_x\_INVISIBLE}{OACGROUPxINVISIBLE} (optional)

    Marks the group as invisible. Useful to block a PC in general which should 
    not be visible in the web interface. The console script oac.sh is not affected 
    by this (for use in easycron).

\config{OAC\_GROUP\_x\_CLIENT\_N}{OAC\_GROUP\_x\_CLIENT\_N}{OACGROUPxCLIENTN} (optional)

    Number of clients in the group.

\config{OAC\_GROUP\_x\_CLIENT\_x}{OAC\_GROUP\_x\_CLIENT\_x}{OACGROUPxCLIENTx} (optional)

    Name of the client as defined in {\var{HOST\_x\_NAME}} in package \jump{sec:dnsdhcp}{dns\_dhcp}.

\config{OAC\_BLOCK\_UNKNOWN\_IF\_x}{OAC\_BLOCK\_UNKNOWN\_IF\_x}{OACBLOCKUNKNOWNIF} (optional)

    List of interfaces defined in base.txt allowing internet access only to 
    hosts defined in dns\_dhcp.txt. Hosts not defined are blocked in general.

\end{description}

