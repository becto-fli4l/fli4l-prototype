% Last Update: $Id: ppp_main.tex 46045 2016-08-12 20:49:28Z kristov $
\marklabel{sec:optppp}{
\section {PPP - Unterstützung für das Point-to-Point-Protokoll}
}

Mit Hilfe dieses Pakets kann der Router Verbindungen zu anderen Rechnern
aufbauen, welche die PPP-Protokollfamilie als zugrunde liegendes
Layer-2-Protokoll nutzen und PPP für die Aushandlung von IP-Adressen,
Kompression, Verschlüsselung etc. verwenden. Die meisten Internet-Anbindungen in
Deutschland nutzen PPP auf die eine oder andere Art. Wenn Sie einen
Internet-Zugang via DSL, ISDN oder UMTS nutzen, benötigen Sie aller Voraussicht
nach dieses Paket.

Dieses Paket bringt zum einen generelle Unterstützung für PPP mit. Zum anderen
werden mit dem Paket gleich drei spezielle PPP-Module (``bundle'', ``serial''
und ``serial-server'') aktiviert, mit denen PPP-Verbindungen gepündelt (PPP-Typ
``bundle'') oder ohne Modem über die serielle Schnittstelle mit Hilfe eines so
genannten Null-Modem-Kabels\footnote{Die Anschluss-Belegung ist im Anhang der
Dokumentation des base-Pakets beschrieben.} (PPP-Typen ``serial'' und
``serial-server'') genutzt werden können. Im Folgenden werden sowohl die
allgemeinen Einstellungen für alle PPP-Circuit-Typen erläutert als auch die
speziellen Einstellungen für die serielle Anbindung.

\marklabel{sect:ppp-circuits}{\subsection{Circuits vom Typ ``ppp''}}

Ein Circuit vom Typ ``ppp'' baut eine Verbindung zwischen zwei Rechnern mit
Hilfe des Point-to-Point-Protokolls (PPP)\footnote{siehe RFC 1661}
auf. PPP ist ein recht komplexes Protokoll, mehr eine Protokoll-Familie, und
erlaubt eine Aushandlung von u.\,a. der zu verwendenden IP-Adressen, Kompression
und/oder Verschlüsselung. Auch können mehrere PPP-Verbindungen zum selben
Zielrechner (``Peer'' genannt) gebündelt werden (MP).\footnote{
``Multilink PPP'', siehe RFC 1990 und den Abschnitt
\jump{sect:multilink-ppp}{Multilink PPP}}.

Das ppp-Paket bringt die nötige allgemeine Infrastruktur für PPP-Verbindungen
mit. Diese wird über die folgende Variable eingebunden und aktiviert:

\begin{description}
\config{OPT\_PPP}{OPT\_PPP}{OPTPPP}

Diese Variable aktiviert den PPP-Dämon, der für alle PPP-Verbindungen notwendig
ist.

Standard-Einstellung: \verb+OPT_PPP='no'+

Beispiel: \verb+OPT_PPP='yes'+

\end{description}

Zusätzlich werden spezielle \emph{PPP-Module} benötigt, die spezifisch für den
jeweiligen Typ von Verbindung sind. Die folgende Tabelle listet die zugehörigen
fli4l-Pakete samt Typ und Beschreibung auf.

\begin{longtable}{|l|l|l|p{0.5\textwidth}|}
    \hline
    \multicolumn{1}{|l}{\textbf{PPP-Typ}} &
    \multicolumn{1}{|l}{\textbf{Paket}} &
    \multicolumn{1}{|l}{\textbf{OPT}} &
    \multicolumn{1}{|l|}{\textbf{Beschreibung}} \\
    \hline
    \endhead
    \hline
    \endfoot
    \endlastfoot
bundle & ppp (dieses Paket) & \var{OPT\_MULTILINK\_PPP} &
Mit Hilfe von bundle-Circuits ist es möglich, PPP-Kanäle zu bündeln. Mehr dazu
steht im Abschnitt \jump{sect:multilink-ppp}{Multilink PPP}.
    \\
    \hline
serial & ppp (dieses Paket) & \var{OPT\_PPP\_SERIAL} &
Mit Hilfe von serial-Circuits ist eine Verbindung zu einem anderen Rechner
über ein serielles Null-Modem-Kabel möglich.
    \\
    \hline
ethernet & pppoe & \var{OPT\_PPP\_ETHERNET} &
Mit Hilfe von ethernet-Circuits ist eine DSL-Einwahl ins Internet über
PPPoE\footnote{``Point-to-Point Protocol over Ethernet'', siehe RFC 2516}
möglich. Dabei ist der Router über eine normale Ethernet-Netzanbindung mit
einem DSL-Modem verbunden, das dann die physikalische Einwahl durchführt.
    \\
    \hline
dslmodem & dslmodem & \var{OPT\_PPP\_DSLMODEM} &
Mit Hilfe von dslmodem-Circuits ist eine DSL-Einwahl ins Internet über ein
im fli4l integriertes DSL-Modem möglich. Dabei ist der Router direkt mit dem
DSL-Splitter verbunden. Momentan wird nur der Fritz!DSL-Adapter der Firma AVM
unterstützt.
    \\
    \hline
pptp & vpn & \var{OPT\_PPP\_PPTP} &
Mit Hilfe von pptp-Circuits ist eine DSL-Einwahl ins Internet über
PPTP\footnote{``Point-to-Point Tunneling Protocol'', siehe RFC 2637}
möglich. Dabei ist der Router über eine normale Ethernet-Netzanbindung mit einem
DSL-Modem verbunden, das dann die physikalische Einwahl durchführt. Abgesehen
von der Internet-Einwahl können pptp-Circuits auch für VPN-Tunnel eingesetzt
werden, wobei auf Grund von Sicherheitsproblemen bei der Authentifizierung von
ihrem Gebrauch abgeraten werden muss.
    \\
    \hline
isdn & isdn & \var{OPT\_ISDN} &
Mit Hilfe von isdn-Circuits ist eine ISDN-Einwahl ins Internet über das
HDLC-Protokoll\footnote{``High-level data link control'', siehe RFC 1618}
möglich. Dabei ist der Router mit Hilfe eines ISDN-Adapters mit einem NTBA
verbunden. Die physikalische Einwahl wird von dem ISDN-Adapter durchgeführt.
    \\
    \hline
umts & umts & \var{OPT\_UMTS} &
Mit Hilfe von umts-Circuits ist eine Einwahl ins Internet via UMTS oder LTE,
also über eine geeignete Funkverbindung, möglich. Ein spezielles Funkmodem,
das in der Regel als USB-Stick vorliegt, stellt die physikalische Anbindung
her.
    \\
    \hline
modem & modem & \var{OPT\_MODEM} &
Mit Hilfe von modem-Circuits ist eine Einwahl ins Internet über eine serielle
Leitung möglich. Dabei stellt ein analoges Modem die physikalische Anbindung
her.
    \\
    \hline
    \caption{Verfügbare PPP-Typen}\marklabel{circuit:ppp-types}{}
\end{longtable}

\subsubsection{Allgemeine PPP-Konfigurationsvariablen}

Neben den allgemeinen Circuit-Variablen sind Angaben über die folgenden
Variablen nötig:

\begin{description}

\config{CIRC\_x\_PPP\_TYPE}{CIRC\_x\_PPP\_TYPE}{CIRCxPPPTYPE}

Jeder ppp-Circuit hat einen Typ. Dieser Typ bestimmt, wie das
Wählen und Auflegen eines ppp-Circuits konkret funktioniert.

Es sind immer nur die Typen der aktivierten Pakete verfügbar. Wenn Sie also
z.\,B. einen PPP-over-Ethernet-Circuit (PPP-Typ ``ethernet'') verwenden möchten,
dann müssen Sie das Paket pppoe herunterladen und PPP/ethernet-Circuits mit
\verb+OPT_PPP_ETHERNET='yes'+ aktivieren. Andernfalls bekommen Sie bei der
Verwendung des Typs ``ethernet'' beim Bauen der Installationsarchive eine
Fehlermeldung.

Beispiel: \verb+CIRC_1_PPP_TYPE='ethernet'+

\config{CIRC\_x\_PPP\_MTU}{CIRC\_x\_PPP\_MTU}{CIRCxPPPMTU}

Mit dieser optionalen Variable kann die Größe der so genannten \emph{MTU}
(Maximum Transmission Unit) in Bytes eingestellt werden. Normalerweise beträgt
die MTU 1492 Bytes, dies ist gleichzeitig die größtmögliche Angabe. Diese
Einstellung sollte nur in Sonderfällen geändert werden!

Standard-Einstellung: je nach PPP-Typ

Beispiel: \verb+CIRC_1_PPP_MTU='1488'+

\config{CIRC\_x\_PPP\_MRU}{CIRC\_x\_PPP\_MRU}{CIRCxPPPMRU}

Mit dieser optionalen Variable kann die Größe der so genannten \emph{MRU}
(Maximum Receive Unit) in Bytes eingestellt werden. Normalerweise beträgt
die MRU 1492 Bytes, dies ist gleichzeitig die größtmögliche Angabe. Diese
Einstellung sollte nur in Sonderfällen geändert werden!

Standard-Einstellung: je nach PPP-Typ

Beispiel: \verb+CIRC_1_PPP_MRU='1488'+

\config{CIRC\_x\_PPP\_NF\_MSS}{CIRC\_x\_PPP\_NF\_MSS}{CIRCxPPPNFMSS}

Bei manchen Providern treten Effekte folgender Art auf:
\begin{itemize}
\item der Webbrowser bekommt eine Verbindung, macht aber danach nichts mehr
\item eine kleine E-Mail kann verschickt werden, eine große E-Mail nicht
\item ssh funktioniert, scp hängt nach dem initialen Verbindungsaufbau
\end{itemize}

Um diese Probleme zu umgehen, manipuliert fli4l standardmäßig die MTU. In
einigen Fällen reicht das allerdings nicht, daher gestattet fli4l explizit das
Setzen der MSS (Message Segment Size) für TCP-Verbindungen auf einen vom
Provider vorgegebenen Wert. Falls der Provider nichts vorgibt, ist 1412 ein
guter Startwert zum Probieren; falls er die MTU vorgibt, sind hier 40 Bytes
(für IPv4-Verbindungen)\footnote{siehe RFC 879} bzw. 60 Bytes (falls auch IPv6
genutzt werden soll)\footnote{siehe RFC 2460} weniger einzutragen.

Fehlt diese Variable oder wird sie leer gelassen, wird die MSS automatisch
vom Linux-Kern aus der PMTU errechnet.\footnote{Die PMTU oder Pfad-MTU
(``Path MTU'') ist die MTU für den gesamten Netzwerk\emph{pfad} zwischen Sender
und Empfänger und berechnet sich aus dem Minimum aller MTUs auf dem Weg zwischen
Sender und Empfänger.}

Beispiel: \verb+CIRC_1_PPP_NF_MSS='1412'+

\config{CIRC\_x\_PPP\_LOCALIP}{CIRC\_x\_PPP\_LOCALIP}{CIRCxPPPLOCALIP}

Diese Variable legt die zu verwendende IPv4-Adresse auf Client-Seite für die
PPP-Verbindung fest. Wird diese Variable weggelassen, wird eine temporäre
IPv4-Adresse aus dem Adress-Bereich 169.254.0.0/16 genommen und gleichzeitig
dem Peer signalisiert, dass man ``offen für Vorschläge'' ist, d.h. der Peer kann
eine alternative IPv4-Adresse für unsere Seite der Verbindung vorschlagen. Dies
ist z.B. bei Internet-Anbindungen üblich.

Beispiel: \verb+CIRC_1_PPP_LOCALIP='192.168.11.22'+

\config{CIRC\_x\_PPP\_REMOTEIP}{CIRC\_x\_PPP\_REMOTEIP}{CIRCxPPPREMOTEIP}

Diese Variable legt die zu verwendende IPv4-Adresse auf Peer-Seite für die
PPP-Verbindung fest. Wird diese Variable weggelassen, wird eine temporäre
IPv4-Adresse aus dem Adress-Bereich 169.254.0.0/16 genommen und gleichzeitig
dem Peer signalisiert, dass man ``offen für Vorschläge'' ist, d.h. der Peer kann
eine alternative IPv4-Adresse für seine Seite der Verbindung vorschlagen. Dies
ist z.B. bei Internet-Anbindungen üblich.

Beispiel: \verb+CIRC_1_PPP_REMOTEIP='192.168.11.23'+

\config{CIRC\_x\_PPP\_LOCALIP6}{CIRC\_x\_PPP\_LOCALIP6}{CIRCxPPPLOCALIP6}

Diese Variable legt das Suffix der zu verwendenden IPv6-Adresse auf
Client-Seite für die PPP-Verbindung fest. Wird diese Variable weggelassen,
wird \verb+::2+ genommen. In jedem Fall wird dem Peer signalisiert, dass man
``offen für Vorschläge'' ist, d.h. der Peer kann eine alternative IPv6-Adresse
für unsere Seite der Verbindung vorschlagen. Dies ist z.B. bei
Internet-Anbindungen üblich.

Beispiel: \verb+CIRC_1_PPP_LOCALIP6='::2'+

\config{CIRC\_x\_PPP\_REMOTEIP6}{CIRC\_x\_PPP\_REMOTEIP6}{CIRCxPPPREMOTEIP6}

Diese Variable legt die zu verwendende IPv6-Adresse auf Peer-Seite für die
PPP-Verbindung fest. Wird diese Variable weggelassen, wird \verb+::1+ genommen.
In jedem Fall wird dem Peer signalisiert, dass man ``offen für Vorschläge''
ist, d.h. der Peer kann eine alternative IPv4-Adresse für seine Seite der
Verbindung vorschlagen. Dies ist z.B. bei Internet-Anbindungen üblich.

Beispiel: \verb+CIRC_1_PPP_REMOTEIP6='::1'+

\config{CIRC\_x\_PPP\_FILTER}{CIRC\_x\_PPP\_FILTER}{CIRCxPPPFILTER}

Der fli4l legt automatisch auf, wenn während der über den Hangup-Timeout
angegebenen Zeit keine Daten über die PPP-Schnittstelle gehen.
Leider wertet der Linux-Kern auch Datentransfers mit, die von außen
kommen, z.\,B. durch Verbindungsversuche eines P2P-Clients wie
eDonkey. Da man heutzutage eigentlich pemanent von anderen kontaktiert
wird, kann es passieren, dass fli4l die DSL-Verbindung nie beendet.

Hier hilft die Option \var{CIRC\_x\_PPP\_FILTER}. Setzt man die Variable auf
`yes', wird nur noch Verkehr gewertet, der von der eigenen Maschine generiert
wird, und von außen eingehende Daten werden komplett ignoriert. Da von draußen
hereinkommender Datenverkehr in der Regel dazu führt, dass der Router oder
dahinter liegende Rechner reagieren, indem sie z.\,B. Verbindungswünsche
ablehnen, werden zusätzlich noch einige hinausgehende Pakete ignoriert.
\footnote{Wie das genau funktioniert, kann man unter
\altlink{http://www.fli4l.de/hilfe/howtos/basteleien/hangup-problem-loesen/} und
\altlink{http://web.archive.org/web/20061107225118/http://www.linux-bayreuth.de/dcforum/DCForumID2/46.html}
nachlesen.}

Standard-Einstellung: \verb+CIRC_x_PPP_FILTER='no'+

Beispiel: \verb+CIRC_1_PPP_FILTER='yes'+

\config{CIRC\_x\_PPP\_FILTER\_EXPR}{CIRC\_x\_PPP\_FILTER\_EXPR}{CIRCxPPPFILTEREXPR}

Hier steht der zu nutzende Filter, wenn \var{CIRC\_x\_PPP\_FILTER} auf `yes'
gesetzt ist. Ist die Variable undefiniert oder leer, wird der Standard-Filter
genutzt. Dieser lautet:

\begin{example}
\begin{verbatim}
    outbound and not icmp[0] != 8 and not tcp[13] & 4 != 0
\end{verbatim}
\end{example}

Zu beachten ist, dass der Standard-Ausdruck oder der hier angegebene Ausdruck
(je nachdem) unter Umständen erweitert wird, wenn andere OPTs aktiviert werden.
So verhindert das IPv6-OPT, dass bestimmte IPv6-Multicast-Pakete, die vom
Linux-Kern über \emph{jede} IPv6-fähige Schnittstelle versandt werden, die
Verbindung aufrecht erhalten, und das Chrony-OPT erweitert den Ausdruck
dergestalt, dass Zeitabfragen für die Messung der Netzaktivität ignoriert
werden, damit die regelmäßige Zeitsynchronisation via NTP nicht zu einer
Dauereinwahl führt. Der hier angegebene Ausdruck kann sich somit von dem
letztlich verwendeten unterscheiden.

Standard-Einstellung: \verb+CIRC_x_PPP_FILTER_EXPR=''+

Beispiel: \verb+CIRC_1_PPP_FILTER_EXPR='outbound and tcp dst port 80'+

Dieser (nicht unbedingt praxistaugliche) Ausdruck berücksichtigt nur ausgehende
HTTP-Verbindungen (TCP-Verbindungen zu Port 80). Alle anderen Datenpakete
fließen in die Bewertung der Netzaktivität nicht mit ein.

\end{description}

\subsubsection{Authentifizierung}

PPP ermöglicht es, dass die Rechner auf beiden Seiten einer Verbindung sich
gegenseitig authentifizieren, d.h. mit Hilfe einer jeweils vom anderen Rechner
akzeptierten Kombination eines Benutzernamens und eines Passworts ausweisen. Die
im Folgenden erläuterten Einstellungen erlauben es, die PPP-Authentifizierung
wunschgemäß zu konfigurieren.

\begin{description}

\config{CIRC\_x\_PPP\_USERID}{CIRC\_x\_PPP\_USERID}{CIRCxPPPUSERID}
\config{CIRC\_x\_PPP\_PASSWORD}{CIRC\_x\_PPP\_PASSWORD}{CIRCxPPPPASSWORD}

Hier sind Benutzerkennung und Passwort für den jeweils benutzten PPP-Peer
(das ist i.\,d.\,R. der Internet-Provider) anzugeben. \var{CIRC\_x\_PPP\_USERID}
enthält die Benutzerkennung, \var{CIRC\_x\_PPP\_PASSWORD} das Passwort.

Für einen T-Online-Zugang ist Folgendes zu beachten:

Die Benutzerkennung \texttt{AAAAAAAAAAAATTTTTTTTTTTTMMMM@t-online.de} setzt
sich aus der zwölfstelligen Anschlusskennung \texttt{AAAAAA...}, der
T-Online-Nummer \texttt{TTTTTT...}, der vierstelligen Mitbenutzernummer
\texttt{MMMM} und dem Suffix \texttt{@t-online.de} zusammen. Hinter der
T-Online-Nummer muss ein \texttt{\#} angegeben werden, wenn deren Länge
kürzer als zwölf Zeichen ist.

Sollte dies in Einzelfällen nicht zum Erfolg führen (offenbar abhängig
von der Vermittlungsstelle), muss zusätzlich zwischen der Anschlusskennung und
der T-Online-Nummer ein weiteres \texttt{\#}-Zeichen eingefügt werden.

Ansonsten (T-Online-Nummer ist zwölfstellig) sind keine \texttt{\#}-Zeichen
anzugeben.

Beispiel:
\begin{example}
\begin{verbatim}
CIRC_1_PPP_USERID='111111111111222222#0001@t-online.de'
CIRC_1_PPP_PASSWORD='12345678'
\end{verbatim}
\end{example}

Informationen über Benutzerkennungen bei anderen Providern finden sich in der
FAQ (\altlink{http://extern.fli4l.de/fli4l_faqengine/faq.php?list=category&catnr=3&prog=1}).

\config{CIRC\_x\_PPP\_PEER\_AUTH}{CIRC\_x\_PPP\_PEER\_AUTH}{CIRCxPPPPEERAUTH}

Mit dieser optionalen Variable kann der PPP-Peer aufgefordert werden, sich zu
authentifizieren. Dazu muss er einen gültigen Benutzernamen und ein gültiges
Passwort übermitteln. Welche Benutzername/Passwort-Kombinationen für welche
Circuits gültig sind, wird über das \verb+OPT_PPP_PEERS+ eingestellt. Generell
ist dies eher für PPP-Circuits auf Server-Seite interessant.

Standard-Einstellung: \verb+CIRC_x_PPP_PEER_AUTH='no'+

Beispiel: \verb+CIRC_1_PPP_PEER_AUTH='yes'+

\end{description}

\subsubsection{TCP/IP-Kompression}

PPP bietet die Möglichkeit, für TCP/IPv4-Verbindungen die Van Jacobson-
Header-Kompri\-mierung nach RFC 1144 zu aktivieren und somit die effektive
Bandbreite zu erhöhen. Dies ist vor allem bei eher langsamer physikalischer
Anbindung (serielle Leitung, ISDN) nützlich.

Generell gilt, dass diese Form der Kompression dazu führt, dass im Falle
verloren gegangener Pakete es etwas dauert, bis der Fehler bemerkt und auf
beiden Seiten entsprechende Korrekturmaßnahmen eingeleitet worden sind.

\begin{description}

\config{CIRC\_x\_PPP\_VJ}{CIRC\_x\_PPP\_VJ}{CIRCxPPPVJ}

Diese Variable aktiviert die Van Jacobson-Komprimierung für TCP/IPv4-Header.

Standard-Einstellung: \verb+CIRC_x_PPP_VJ='no'+

Beispiel: \verb+CIRC_1_PPP_VJ='yes'+

\config{CIRC\_x\_PPP\_VJCCOMP}{CIRC\_x\_PPP\_VJCCOMP}{CIRCxPPPVJCCOMP}

Mit dieser Option kann man erreichen, dass trotz aktivierter
Van Jacobson-Komprimierung (\verb+CIRC_x_PPP_VJ='yes'+) die Verbindungsnummer
(``Connection ID'') Teil eines jeden Pakets ist. Dies erreicht man durch
\verb+CIRC_x_PPP_VJCCOMP='no'+, ein Weglassen der Verbindungsnummer erreicht man
durch \verb+CIRC_x_PPP_VJCCOMP='yes'+. Das unbedingte Senden der
Verbindungsnummer (\verb+CIRC_x_PPP_VJCCOMP='no'+) kann dazu führen, dass
im Falle verloren gegangener Pakete es nicht ganz so lange dauert, bis der
Fehler bemerkt und auf beiden Seiten entsprechende Korrekturmaßnahmen
eingeleitet worden sind.

Standard-Einstellung: \verb+CIRC_x_PPP_VJCCOMP='no'+

Beispiel: \verb+CIRC_1_PPP_VJCCOMP='yes'+

\end{description}

\subsubsection{PPP-Kompression}

PPP bietet die Möglichkeit, unabhängig für jede Verbindungsrichtung das
Komprimieren (Packen) von Paketen auszuhandeln. Dies wird über die folgenden
Variablen konfiguriert. Das Aktivieren der \verb+CIRC_x_PPP_COMP_y+-Variablen
hat zur Folge, dass dem Verbindungspartner (Peer) angeboten wird, ihm die Pakete
komprimiert zu schicken (was er verweigern kann, etwa wenn er die vorgeschlagene
Kompression nicht versteht). Im Gegenzug führt das Aktivieren der
\verb+CIRC_x_PPP_DECOMP_y+-Variablen dazu, dass dem Peer mitgeteilt wird, dass
man in der Lage ist, bestimmte komprimierte Pakete verarbeiten (d.\,h.
entpacken) zu können. Es handelt sich in beiden Fällen jeweils um
\emph{Vorschläge}, die entsprechend zurückgewiesen werden können~-- man kann
den Peer nicht dazu zwingen, die Pakete komprimiert zu verschicken oder
komprimierte Pakete zu verarbeiten.

Generell ist das Entpacken effizienter zu realisieren als das Packen. Ist die
Hardware des fli4l-Routers nicht sehr leistungsfähig, ist es somit u.\,U.\ zu
empfehlen, höchstens das Entpacken anzubieten (\verb+CIRC_x_PPP_DECOMP_y='yes'+)
und das Packen nicht (\verb+CIRC_x_PPP_COMP_y='no'+).

\begin{description}

\config{CIRC\_x\_PPP\_COMP\_BSDCOMP}{CIRC\_x\_PPP\_COMP\_BSDCOMP}{CIRCxPPPCOMPBSDCOMP}
\config{CIRC\_x\_PPP\_DECOMP\_BSDCOMP}{CIRC\_x\_PPP\_DECOMP\_BSDCOMP}{CIRCxPPPDECOMPBSDCOMP}

Diese Variablen aktivieren die Möglichkeit, Pakete vor dem Versenden an den Peer
mit Hilfe des BSD-Kompressionsalgorithmus zu komprimieren
(\verb+CIRC_x_PPP_COMP_BSDCOMP='yes'+) bzw. entsprechend komprimierte Pakete
vom Peer zu empfangen (\verb+CIRC_x_PPP_DECOMP_BSDCOMP='yes'+).

Standard-Einstellung:

\verb+CIRC_x_PPP_COMP_BSDCOMP='no'+\\
\verb+CIRC_x_PPP_DECOMP_BSDCOMP='no'+

Beispiel:

\verb+CIRC_1_PPP_COMP_BSDCOMP='yes'+\\
\verb+CIRC_1_PPP_DECOMP_BSDCOMP='yes'+

\config{CIRC\_x\_PPP\_COMP\_BSDCOMP\_LEVEL}{CIRC\_x\_PPP\_COMP\_BSDCOMP\_LEVEL}{CIRCxPPPCOMPBSDCOMPLEVEL}
\config{CIRC\_x\_PPP\_DECOMP\_BSDCOMP\_LEVEL}{CIRC\_x\_PPP\_DECOMP\_BSDCOMP\_LEVEL}{CIRCxPPPDECOMPBSDCOMPLEVEL}

Ist die BSD-Kompression mit Hilfe der Variablen
\verb+CIRC_x_PPP_COMP_BSDCOMP='yes'+ bzw. \verb+CIRC_x_PPP_DECOMP_BSDCOMP='yes'+
aktiviert worden, kann mit Hilfe der Variablen
\verb+CIRC_x_PPP_COMP_BSDCOMP_LEVEL+ bzw. \verb+CIRC_x_PPP_DECOMP_BSDCOMP_LEVEL+
die zu verwendende Kompressionsstufe konfiguriert werden. Diese Variablen
erwarten einen Wert von 9 (schwächste Kompression, am schnellsten) bis 15
(stärkste Kompression, am langsamsten).

Diese Stufe ist jeweils als Vorschlag einer oberen Grenze zu sehen; der Peer
kann eine schwächere Stufe aushandeln, falls er mit dem Vorschlag nicht
einverstanden ist.

Standard-Einstellung:

\verb+CIRC_x_PPP_COMP_BSDCOMP_LEVEL='12'+\\
\verb+CIRC_x_PPP_DECOMP_BSDCOMP_LEVEL='12'+

Beispiel (langsame Hardware, nur schwache Kompression möglich):

\verb+CIRC_1_PPP_COMP_BSDCOMP_LEVEL='9'+\\
\verb+CIRC_1_PPP_DECOMP_BSDCOMP_LEVEL='15'+

\config{CIRC\_x\_PPP\_COMP\_DEFLATE}{CIRC\_x\_PPP\_COMP\_DEFLATE}{CIRCxPPPCOMPDEFLATE}
\config{CIRC\_x\_PPP\_DECOMP\_DEFLATE}{CIRC\_x\_PPP\_DECOMP\_DEFLATE}{CIRCxPPPDECOMPDEFLATE}

Diese Variablen aktivieren die Möglichkeit, Pakete vor dem Versenden an den Peer
mit Hilfe des Deflate-Kompressionsalgorithmus zu komprimieren
(\verb+CIRC_x_PPP_COMP_DEFLATE='yes'+) bzw. entsprechend komprimierte Pakete
vom Peer zu empfangen
(\verb+CIRC_x_PPP_DECOMP_DEFLATE='yes'+).

Standard-Einstellung:

\verb+CIRC_x_PPP_COMP_DEFLATE='no'+\\
\verb+CIRC_x_PPP_DECOMP_DEFLATE='no'+

Beispiel:

\verb+CIRC_1_PPP_COMP_DEFLATE='yes'+\\
\verb+CIRC_1_PPP_DECOMP_DEFLATE='yes'+

\config{CIRC\_x\_PPP\_COMP\_DEFLATE\_LEVEL}{CIRC\_x\_PPP\_COMP\_DEFLATE\_LEVEL}{CIRCxPPPCOMPDEFLATELEVEL}
\config{CIRC\_x\_PPP\_DECOMP\_DEFLATE\_LEVEL}{CIRC\_x\_PPP\_DECOMP\_DEFLATE\_LEVEL}{CIRCxPPPDECOMPDEFLATELEVEL}

Ist die Deflate-Kompression mit Hilfe der Variablen
\verb+CIRC_x_PPP_COMP_DEFLATE='yes'+ bzw. \verb+CIRC_x_PPP_DECOMP_DEFLATE='yes'+
aktiviert worden, kann mit Hilfe der Variablen
\verb+CIRC_x_PPP_COMP_DEFLATE_LEVEL+ bzw. \verb+CIRC_x_PPP_DECOMP_DEFLATE_LEVEL+
die zu verwendende Kompressionsstufe konfiguriert werden. Diese Variablen
erwarten einen Wert von 9 (schwächste Kompression, am schnellsten) bis 15
(stärkste Kompression, am langsamsten).

Diese Stufe ist jeweils als Vorschlag einer oberen Grenze zu sehen; der Peer
kann eine schwächere Stufe aushandeln, falls er mit dem Vorschlag nicht
einverstanden ist.

Standard-Einstellung:

\verb+CIRC_x_PPP_COMP_DEFLATE_LEVEL='12'+\\
\verb+CIRC_x_PPP_DECOMP_DEFLATE_LEVEL='12'+

Beispiel (langsame Hardware, nur schwache Kompression möglich):

\verb+CIRC_1_PPP_COMP_DEFLATE_LEVEL='9'+\\
\verb+CIRC_1_PPP_DECOMP_DEFLATE_LEVEL='15'+

\config{CIRC\_x\_PPP\_COMP\_LZSCOMP}{CIRC\_x\_PPP\_COMP\_LZSCOMP}{CIRCxPPPCOMPLZSCOMP}
\config{CIRC\_x\_PPP\_DECOMP\_LZSCOMP}{CIRC\_x\_PPP\_DECOMP\_LZSCOMP}{CIRCxPPPDECOMPLZSCOMP}

Diese Variablen aktivieren die Möglichkeit, Pakete vor dem Versenden an den Peer
mit Hilfe des Stac/HiFn-LZS-Kompressionsalgorithmus zu komprimieren
(\verb+CIRC_x_PPP_COMP_LZSCOMP='yes'+) bzw. entsprechend komprimierte Pakete vom
Peer zu empfangen (\verb+CIRC_x_PPP_DECOMP_LZSCOMP='yes'+).

Der Nutzung dieses Algorithmus ist lizenzrechtlich problematisch. Die
Implementierung für das ISDN4Linux-Subsystem des Linux-Kerns enthält die
folgende Passage:

\begin{verbatim}
The code is Copyleft 1998-2001 by Andre Beck under terms of the
GNU GPL. While the code is provided free and under GPL, it may infringe
patents Stac has applied for in the US and maybe other countries. Verify
that this is not the case in your country before using it. This code is a
private development related in no way whatsoever to my employer, written
in Germany where software patents are still considered bogus. This
situation might change any day due to EU law. Linux distributors who
supply this stuff shrink wrapped should reconsider doing this (not only
for isdn_lzscomp, but as well for isdn_bsdcomp, which is likely to
infringe on the infamous Unisys LZW patent) with their law departement.
[...]
Please check whether using this code constitutes a crime in your country.
While I just implemented a module that can source and sink the data
format described by Stac/HiFn in RFC1974, they claim to have US Patents
on practically every aspect of this format. Thus I assume that selling
this code violates these US Patents. As this code is

a) Freeware under GPL and
b) Developed in Germany

I hope I wont get any terror from the lawyers of the patent holders. I'm
sure that the availability of this code will actually help HiFn sales
because more users (the Linux community) will ask their ISPs for LZS
support on their NASes and these NASes will contain HiFn chips for that
purpose. Then again, after they claimed even deflate would be covered by
their patents and tried to sue Netscape for using deflate in the PNG code
of Mozilla, they are probably not on the planet for good will games.
We'll see.
\end{verbatim}

Außerdem gibt es Probleme, wenn die Datenübertragung nicht sehr stabil ist
und gelegentlich Pakete verloren gehen: Die dann nötige Resynchronisation der
Zustände des Kompressors/Dekompressors reduziert die
Übertragungsgeschwindigkeit derart, dass der Geschwindigkeitsgewinn u.\,U.
gänzlich aufgefressen wird.

Man sollte den Algorithmus somit nur verwenden, wenn man muss, wenn der
PPP-Peer also keinen anderen, frei verfügbaren Algorithmus anbietet bzw.
versteht und man die Komprimierung gerne nutzen möchte (was sich vor allem bei
eher langsamen Verbindungen wie ISDN merklich auf die Übertragungsgeschwindigkeit
auswirkt). Ansonsten sollte man ausprobieren, ob mit eingeschalteter Kompression
die Datenübertragung in den eigenen typischen Anwendungsfällen wirklich
schneller ist.

Standard-Einstellung:

\verb+CIRC_x_PPP_COMP_LZSCOMP='no'+\\
\verb+CIRC_x_PPP_DECOMP_LZSCOMP='no'+

Beispiel:

\verb+CIRC_1_PPP_COMP_LZSCOMP='yes'+\\
\verb+CIRC_1_PPP_DECOMP_LZSCOMP='yes'+

\config{CIRC\_x\_PPP\_COMP\_LZSCOMP\_NHIST}{CIRC\_x\_PPP\_COMP\_LZSCOMP\_NHIST}{CIRCxPPPCOMPLZSCOMPNHIST}
\config{CIRC\_x\_PPP\_DECOMP\_LZSCOMP\_NHIST}{CIRC\_x\_PPP\_DECOMP\_LZSCOMP\_NHIST}{CIRCxPPPDECOMPLZSCOMPNHIST}

Ist die Stac-LZS-Kompression mit Hilfe der Variablen
\verb+CIRC_x_PPP_COMP_LZSCOMP='yes'+ bzw. \verb+CIRC_x_PPP_DECOMP_LZSCOMP='yes'+
aktiviert worden, kann mit Hilfe der Variablen
\verb+CIRC_x_PPP_COMP_LZSCOMP_NHIST+ bzw. \verb+CIRC_x_PPP_DECOMP_LZSCOMP_NHIST+
eingestellt werden, wie viele verschiedene Historien verwendet werden sollen.
Verschiedene Historien erlauben es, verschiedene TCP-Verbindungen unabhängig
voneinander zu komprimieren. Aufgrund von Einschränkungen im Linux-Kern werden
momentan jedoch nur zwei Einstellungen unterstützt: 0 (keine Historie) und 1
(eine Historie). Bei 0 wird jedes Paket individuell komprimiert. Das senkt den
Kompressionsgewinn erheblich, erhöht jedoch die Robustheit, da ein Verlust eines
Pakets sich nicht auf die anderen übertragenen Pakete auswirkt.

Dieser Wert ist jeweils als Vorschlag einer oberen Grenze zu sehen; der Peer
kann weniger Historien aushandeln, falls er mit dem Vorschlag nicht
einverstanden ist.

Standard-Einstellung:

\verb+CIRC_x_PPP_COMP_LZSCOMP_NHIST='1'+\\
\verb+CIRC_x_PPP_DECOMP_LZSCOMP_NHIST='1'+

Beispiel (hochgradig unzuverlässige Verbindung):

\verb+CIRC_1_PPP_COMP_LZSCOMP_NHIST='0'+\\
\verb+CIRC_1_PPP_DECOMP_LZSCOMP_NHIST='0'+

\config{PPP\_COMP\_LZSCOMP\_LEVEL}{PPP\_COMP\_LZSCOMP\_LEVEL}{PPPCOMPLZSCOMPLEVEL}

Diese globale, nicht Circuit-spezifische Variable gibt an, welche
Stufe beim Stac/HiFn-LZS-Algorithmus für die Komprimierung von Paketen verwendet
werden soll. Diese Variable erwartet einen Wert von 1 (schwächste Kompression,
am schnellsten) bis 9 (stärkste Kompression, am langsamsten).

Diese Einstellung ist global, weil sie nicht Teil der Aushandlung der
PPP-Verbindungs\-eigenschaften ist und beim Laden des entsprechenden Kern-Moduls
fest angegeben werden muss.

Stufe 9 ist nicht unbedingt für den produktiven Betrieb geeignet. Die
Dokumentation besagt:

\begin{verbatim}
Try 9 only for fun, it is rather useless (compresses only 1 or 2%
better than 8, but a P100 facing that amount of brute force becomes
sluggish, jumping mouse etc) - just not worth the burned CPU (even
if you have a GHz oven).
\end{verbatim}

Standard-Einstellung:

\verb+PPP_COMP_LZSCOMP_LEVEL='8'+

Beispiel (langsame Hardware, nur schwache Kompression möglich):

\verb+PPP_COMP_LZSCOMP_LEVEL='1'+

\config{CIRC\_x\_PPP\_COMP\_MPPE}{CIRC\_x\_PPP\_COMP\_MPPE}{CIRCxPPPCOMPMPPE}

Diese Variable aktiviert die Möglichkeit, Pakete vor dem Versenden an den Peer
mit Hilfe des MPPE-Algorithmus zu komprimieren und zu verschlüsseln. Im
Gegensatz zu den anderen Algorithmen gibt es keine separate Einstellung für
die Pakete vom Peer zum fli4l-Router, weil MPPE immer symmetrisch ausgehandelt
wird, d.\,h. dass MPPE entweder für beide Richtungen aktiv ist oder für keine.

MPPE wird vor allem in Zusammenhang mit PPTP-Tunneln genutzt. Für andere
Einsatzzwecke sollte MPPE aufgrund bekannter kryptographischer Schwachstellen
nicht verwendet werden.

Standard-Einstellung: \verb+CIRC_x_PPP_COMP_MPPE='no'+

Beispiel: \verb+CIRC_1_PPP_COMP_MPPE='yes'+

\config{CIRC\_x\_PPP\_COMP\_MPPE\_KEY\_LEN}{CIRC\_x\_PPP\_COMP\_MPPE\_KEY\_LEN}{CIRCxPPPCOMPMPPEKEYLEN}

Ist die MPPE-Verschlüsselung mit Hilfe der Variable
\verb+CIRC_x_PPP_COMP_MPPE='yes'+ aktiviert, ist es u.\,U. nötig, die
erlaubten Schlüssellängen einzustellen. Dies kann mit Hilfe der
Variable \verb+CIRC_x_PPP_COMP_MPPE_KEY_LEN+ konfiguriert werden. Es gibt drei
Möglichkeiten:

\begin{longtable}{|l|p{0.7\textwidth}|}
    \hline
    \multicolumn{1}{|l}{\textbf{Wert}} &
    \multicolumn{1}{|l|}{\textbf{Bedeutung}} \\
    \hline
    \endhead
    \hline
    \endfoot
    \endlastfoot
\texttt{40}   & Es soll ein 40-Bit-Schlüssel verwendet werden. Die
                Aushandlung eines 128-Bit-Schlüssels wird abgelehnt.
                \emph{(nicht empfohlen)}\\
\texttt{128}  & Es soll ein 128-Bit-Schlüssel verwendet werden. Die
                Aushandlung eines 40-Bit-Schlüssels wird abgelehnt.
                \emph{(empfohlen)}\\
\texttt{both} & Es können sowohl 40-Bit- als auch 128-Bit-Schlüssel genutzt
                werden.
                \emph{(nicht empfohlen)}\\
    \hline
    \caption{Mit \var{CIRC\_x\_PPP\_COMP\_MPPE\_KEY\_LEN} konfigurierbare MPPE-Schlüssellängen}
\end{longtable}

Generell gilt: Je länger der Schlüssel, desto sicherer die Verschlüsselung.
Wenn möglich, sollten somit immer 128-Bit-Schlüssel verwendet werden.

Standard-Einstellung:

\verb+CIRC_x_PPP_COMP_MPPE_KEY_LEN='128'+

Beispiel (für die Nutzung eines Peers, der nur mit 40-Bit-Schlüsseln umgehen
kann):

\verb+CIRC_1_PPP_COMP_MPPE_KEY_LEN='40'+

\end{description}

\marklabel{sect:multilink-ppp}{\subsection{Multilink PPP: PPP-Circuits vom Typ ``bundle''}}

PPP erlaubt es, mehrere Verbindungen zum selben Peer zu \emph{bündeln}. Das
bedeutet, dass mehrere unterschiedliche physische Verbindungen (z.\,B.
ISDN-Kanäle oder Ethernet-Netzwerk\-verbindungen) logisch zu einer
\emph{einzigen} Verbindung (``Bündel'' genannt) zusammengefasst werden können.
Somit handelt es sich bei Multilink PPP um eine Variante des
BONDings.\footnote{BOND steht für ``Bandwidth ON Demand'' und bezeichnet eine
Technik, bei der dynamisch~-- je nach Auslastung der (logischen) Verbindung~--
weitere physische Übertragungskanäle hinzugefügt oder bestehende entfernt
werden, ohne dass die logische Verbindung etwas davon mitbekommt.} Pakete, die
über ein Multilink-Bündel transportiert werden, werden zuerst in einzelne
Fragmente aufgeteilt. Diese Fragmente werden dann über die zum Bündel gehörenden
PPP-Verbindungen verschickt und auf der Gegenseite vom Peer wieder in der
richtigen Reihenfolge zusammengebaut.\footnote{Die Anzahl der Fragmente sowie
die Auswahl der Verbindungen erfolgt automatisch vom Linux-Kern und ist nicht
beeinflussbar.} Bei genügend viel Datenverkehr werden alle Verbindungen im
Bündel gleichmäßig ausgelastet, was zur Erhöhung der Übertragungsgeschwindigkeit
führt, weil mehr Daten pro Zeiteinheit übertragen werden können.

Um Multilink PPP zu konfigurieren, muss ein so genanntes \emph{PPP-Bündel}
konfiguriert werden. Das ist ein spezieller PPP-Circuit vom Typ ``bundle''.
Alle eigentlichen PPP-Kanäle, die gebündelt werden sollen, müssen dann dieses
Bündel mit Hilfe der Variable \verb+CIRC_x_BUNDLE+ referenzieren. Dies
funktioniert gleichermaßen auf Client- wie auf Server-Seite. Wird weiterhin die
Dial-on-Demand-Funktionalität des fli4l für diesen Bündel-Circuit aktiviert
(d.h. dass erst bei tatsächlich ausgehenden Netzwerkpaketen gewählt werden
soll), muss einer dieser Circuits im Bündel als so genannter ``Master''
ausgezeichnet werden, der auf diese Pakete ``horcht''; dies wird mit Hilfe der
Variablen \verb+CIRC_x_PPP_MASTER+ vorgenommen.

Generell können die PPP-Kanäle eines PPP-Bündels beliebig hinzu- und
herausgenommen werden: So lange auch nur ein Kanal im Bündel verbleibt, bleibt
das gesamte Bündel intakt. Im Falle von Dial-on-Demand-Bündeln (siehe oben) ist
die Situation etwas anders: Legt der Master-Circuit auf, so legen auch alle
anderen Kanäle desselben Bündels auf.

Bei Multilink PPP muss man zum einen beachten, dass mehrere physisch
\emph{unabhängige} Übertragungskanäle genutzt werden (müssen). Dies bedeutet
in der Regel, dass die Kosten sich entsprechend vervielfachen (wenn man z.\,B.
mehrere ISDN-Verbindungen oder mehrere DSL-Verbindungen nutzt). Zum anderen
müssen \emph{beide} Verbindungspartner Multilink PPP unterstützen. Wenn der
eigene Internet-Provider kein Multilink PPP anbietet, hat man also Pech...
Typischerweise bieten Internet-Provider bei ISDN-Verbindungen Multilink PPP an,
während dies bei DSL-Verbindungen die Seltenheit ist.

Multilink PPP ist aber natürlich nicht auf die Verbindung zum Internet-Provider
beschränkt, sondern kann auch für andere Zwecke verwendet werden, etwa um
lokale Netze übers Internet mit Hilfe mehrerer, unabhängiger Verbindungen
performant zu verbinden. Wenn jedoch keine spezielle PPP-Funktionalität wie
Kompression oder IP-Adress-Vergabe benötigt wird, ist im Falle von
Ethernet-Verbindungen das von Linux angebotene Ethernet-Bonding effizienter,
weil der Overhead des PPP-Protokolls entfällt.

Viele Einstellungen müssen für das gesamte PPP-Bündel identisch sein. Dies wird
so realisiert, dass einige Einstellungen nur beim PPP-Bündel-Circuit
konfiguriert werden (können) und die zugehörigen PPP-Kanäle diese übernehmen.
Deswegen ist es erforderlich, dass der Bündel-Circuit \emph{vor} den Circuits
für die einzelnen PPP-Kanäle konfiguriert wird, d.\,h. einen niedrigeren Index
besitzt. Die folgenden Einstellungen werden von den Kanälen übernommen bzw.
haben nur auf den Bündel-Circuit Auswirkungen und dürfen folglich \emph{nicht}
für PPP-Kanäle separat konfiguriert werden:

\begin{itemize}
\item alles, was mit Netzwerk-Eigenschaften zu tun hat (Protokolle, Routen
etc.):
\begin{itemize}
\item \verb+CIRC_x_NETS_IPV4_%+
\item \verb+CIRC_x_NETS_IPV6_%+
\item \verb+CIRC_x_PROTOCOLS+
\item \verb+CIRC_x_USEPEERDNS+
\item \verb+CIRC_x_PPP_NF_MSS+
\end{itemize}
\item Authentifizierung:
\begin{itemize}
\item \verb+CIRC_x_PPP_USERID+
\item \verb+CIRC_x_PPP_PASSWORD+
\item \verb+CIRC_x_PPP_PEER_AUTH+
\end{itemize}
\item Paketfilter und Hangup-Timeout:
\begin{itemize}
\item \verb+CIRC_x_HUP_TIMEOUT+
\item \verb+CIRC_x_PPP_FILTER+
\end{itemize}
\item alle PPP-Kompressionseinstellungen\footnote{Der verwendete PPP-Dämon
unterstützt Kompression nur auf Bündel- und nicht auf Verbindungsebene.}
\item \verb+CIRC_x_PPP_MRRU+
\end{itemize}

Typische individuelle Einstellungen für einen PPP-Kanal sind die
\jump{CIRCxPPPMTU}{MTU} und die \jump{CIRCxPPPMRU}{MRU}, die für jede physische
Verbindung separat konfiguriert werden (können) sowie Einstellungen, die den
jeweiligen PPP-Typ betreffen (also z.\,B. die Netzwerk-Schnittstelle bei
einer PPPoE-Verbindung) und somit ebenfalls für jede physische Verbindung
unterschiedlich sind bzw. sein können.

Die Unterstützung für PPP-Bündel wird über die folgende Variable eingebunden
und aktiviert:

\begin{description}
\config{OPT\_MULTILINK\_PPP}{OPT\_MULTILINK\_PPP}{OPTMULTILINKPPP}

Diese Variable aktiviert die PPP-Multilink-Unterstützung. Zusätzlich muss das
\verb+OPT_BUNDLE+ aus dem base-Paket aktiviert werden.

Standard-Einstellung: \verb+OPT_MULTILINK_PPP='no'+

Beispiel:
\begin{example}
\begin{verbatim}
OPT_BUNDLE='yes'
OPT_MULTILINK_PPP='yes'
\end{verbatim}
\end{example}

\end{description}

Bei einem Bündel-Circuit kommen zu den allgemeinen Circuit- und
PPP-Circuit-Variablen noch die folgenden hinzu, die spezifisch für
PPP/bundle-Circuits sind:

\begin{description}

\config{CIRC\_x\_PPP\_MASTER}{CIRC\_x\_PPP\_MASTER}{CIRCxPPPMASTER}

Falls das PPP-Bündel für Dial-on-Demand konfiguriert ist (d.h. falls
im Bündel \verb+CIRC_x_HUP_TIMEOUT+ größer null ist), dann wird mit Hilfe
dieser Variable festgelegt, welcher PPP-Kanal zuerst aufgebaut wird. Dies ist
nötig, weil nicht alle PPP-Kanäle gleichzeitig auf Netzwerkaktivität horchen
können. Der Wert kann ein Bezeichner der Form \emph{circ<Index>} oder der Name
(\verb+CIRC_+\emph{Index}\verb+_NAME='+\emph{Name}\verb+')+ des jeweiligen
Circuits sein. Der referenzierte Circuit muss ein PPP-Circuit sein, der Teil
dieses Bündels ist.

Bei Circuits, die sich sofort einwählen (\verb+CIRC_x_HUP_TIMEOUT='0'+), wird
diese Einstellung ignoriert.

Beispiel: \verb+CIRC_1_PPP_MASTER='ppp-serial1'+

\config{CIRC\_x\_PPP\_MRRU}{CIRC\_x\_PPP\_MRRU}{CIRCxPPPMRRU}

Die MRRU ist die \jump{CIRCxPPPMRU}{MRU} auf der Ebene des Multilink-Bündels.
Sie sagt aus, wie groß ein aus Fragmenten zusammengesetztes PPP-Paket höchstens
werden darf. Die MRRU entspricht somit auf der Seite des Peers der maximalen
Größe eines PPP-Pakets, bevor es fragmentiert und über die einzelnen
Verbindungen des Bündels verschickt wird.

Wird die MRRU nicht konfiguriert oder wird sie auf null gesetzt, wird die MRRU
aus den MRUs der zum Bündel gehörenden Circuits automatisch berechnet.

Die MRRU ist als Vorschlag einer oberen Grenze zu sehen; der Peer kann eine
kleinere MRRU aushandeln, falls er mit dem Vorschlag nicht einverstanden ist.

Standard-Einstellung: \verb+CIRC_x_PPP_MRRU='0'+

Beispiel: \verb+CIRC_1_PPP_MRRU='yes'+

\end{description}

Das nachfolgende Beispiel zeigt, wie man auf Client- und Server-Seite eine
PPP-Multilink-Verbindung definiert. Dabei wird das Paket pppoe vorausgesetzt,
welches die entsprechenden PPP-Module ``ethernet'' und ``ethernet-server''
mitbringt.

Beispiel (Client): Der Client ``wotan'' baut zwei zu einem Bündel
zusammengefasste PPPoE-Verbindungen zum Server ``odin'' auf und authentifiziert
sich mit Benutzernamen ``user'' und Passwort ``pass''.

\begin{example}
\begin{verbatim}
    HOSTNAME='wotan'
    OPT_BUNDLE='yes'
    OPT_PPP='yes'
    OPT_MULTILINK_PPP='yes'
    OPT_PPP_ETHERNET='yes'
    #
    CIRC_N='3'
    #
    CIRC_1_NAME='ppp2odin'
    CIRC_1_TYPE='ppp'
    CIRC_1_ENABLED='yes'
    CIRC_1_UP='yes'
    CIRC_1_NETS_IPV4_N='1'
    CIRC_1_NETS_IPV4_1='192.168.12.0/24'
    CIRC_1_PPP_TYPE='bundle'
    CIRC_1_PPP_USERID='user'
    CIRC_1_PPP_PASSWORD='pass'
    #
    CIRC_2_NAME='ppp2odin-eth1'
    CIRC_2_TYPE='ppp'
    CIRC_2_ENABLED='yes'
    CIRC_2_UP='yes'
    CIRC_2_BUNDLE='ppp2odin'
    CIRC_2_PPP_TYPE='ethernet'
    CIRC_2_PPP_ETHERNET_DEV='eth1'
    CIRC_2_PPP_ETHERNET_TYPE='kernel'
    #
    CIRC_3_NAME='ppp2odin-eth2'
    CIRC_3_TYPE='ppp'
    CIRC_3_ENABLED='yes'
    CIRC_3_UP='yes'
    CIRC_3_BUNDLE='ppp2odin'
    CIRC_3_PPP_TYPE='ethernet'
    CIRC_3_PPP_ETHERNET_DEV='eth2'
    CIRC_3_PPP_ETHERNET_TYPE='kernel'
\end{verbatim}
\end{example}

Beispiel (Server): Der Server ``odin'' wartet auf zwei Ethernet-Schnittstellen
auf PPPoE-Verbindungsanfragen von Client ``wotan'' und verlangt eine
Authentifizierung mit Benutzernamen ``user'' und Passwort ``pass''. Diese
PPPoE-Kanäle werden zu einem PPP-Bündel zusammengefasst. Weil keine Netze hin
zum Client geroutet werden (\verb+CIRC_1_NETS_IPV4_N+ ist nicht definiert), muss
via \verb+CIRC_1_PROTOCOLS='ipv4'+ explizit mitgeteilt werden, dass IPv4 über
diese PPP-Verbindungen genutzt werden soll.  Via \verb+CIRC_1_PPP_LOCALIP+ und
\verb+CIRC_1_PPP_REMOTEIP+ wird konfiguriert, welche IPv4-Adressen der Client
und der Server für diese PPP-Verbindung nutzen sollen.

\begin{example}
\begin{verbatim}
    HOSTNAME='odin'
    OPT_BUNDLE='yes'
    OPT_PPP='yes'
    OPT_MULTILINK_PPP='yes'
    OPT_PPP_ETHERNET_SERVER='yes'
    OPT_PPP_PEERS='yes'
    PPP_PEER_N='1'
    PPP_PEER_1_USERID='user'
    PPP_PEER_1_PASSWORD='pass'
    PPP_PEER_1_CIRCUITS='ppp2wotan'
    #
    CIRC_N='3'
    #
    CIRC_1_NAME='ppp2wotan'
    CIRC_1_TYPE='ppp'
    CIRC_1_ENABLED='yes'
    CIRC_1_UP='yes'
    CIRC_1_PROTOCOLS='ipv4'
    CIRC_1_PPP_TYPE='bundle'
    CIRC_1_PPP_PEER_AUTH='yes'
    CIRC_1_PPP_LOCALIP='192.168.222.1'
    CIRC_1_PPP_REMOTEIP='192.168.222.2'
    #
    CIRC_2_NAME='ppp2wotan-eth1'
    CIRC_2_TYPE='ppp'
    CIRC_2_ENABLED='yes'
    CIRC_2_UP='yes'
    CIRC_2_BUNDLE='ppp2wotan'
    CIRC_2_PPP_TYPE='ethernet-server'
    CIRC_2_PPP_ETHERNET_DEV='eth1'
    CIRC_2_PPP_ETHERNET_TYPE='kernel'
    #
    CIRC_3_NAME='ppp2wotan-eth2'
    CIRC_3_TYPE='ppp'
    CIRC_3_ENABLED='yes'
    CIRC_3_UP='yes'
    CIRC_3_BUNDLE='ppp2wotan'
    CIRC_3_PPP_TYPE='ethernet-server'
    CIRC_3_PPP_ETHERNET_DEV='eth2'
    CIRC_3_PPP_ETHERNET_TYPE='kernel'
\end{verbatim}
\end{example}

\marklabel{sect:ppp-serial}{\subsection{PPP-Circuits vom Typ ``serial''}}

Dieser Circuit-Typ erlaubt es, PPP-Verbindungen zu einem anderen Rechner über
eine serielle Leitung (ohne Modem) aufbauen zu können. Das Paket wird über die
Variable \var{OPT\_PPP\_SERIAL} aktiviert:

\begin{description}
\config{OPT\_PPP\_SERIAL}{OPT\_PPP\_SERIAL}{OPTPPPSERIAL}

Diese Variable muss auf 'yes' gesetzt werden, um PPP-Circuits vom Typ ``serial''
anlegen zu können.

Standard-Einstellung: \verb+OPT_PPP_SERIAL='no'+
\end{description}

Soll eine konkrete serielle Verbindung aufgebaut werden können, so muss ein
passender Circuit konfiguriert werden, siehe Abschnitt
\jump{sect:circuits}{Circuit-Konfiguration}. Der zu verwendende Typ in
\var{CIRC\_x\_TYPE} lautet ``ppp'', der zu verwendende PPP-Typ in
\var{CIRC\_x\_PPP\_TYPE} lautet ``serial''. Zu den allgemeinen Circuit-Variablen
kommen die folgenden hinzu, die spezifisch für PPP/serial-Circuits sind:

\begin{description}

\config{CIRC\_x\_PPP\_SERIAL\_DEV}{CIRC\_x\_PPP\_SERIAL\_DEV}{CIRCxPPPSERIALDEV}

Hier ist die serielle Schnittstelle anzugeben, die verwendet werden soll. Sie
können entweder ``com1'' bis ``com4'' oder alternativ ``ttyS0'' bis ``ttyS3''
für die seriellen Schnittstellen 1--4 eintragen.

Beispiel: \verb+CIRC_1_PPP_SERIAL_DEV='com2'+

\config{CIRC\_x\_PPP\_SERIAL\_SPEED}{CIRC\_x\_PPP\_SERIAL\_SPEED}{CIRCxPPPSERIALSPEED}

Hier muss die Übertragungsrate in Baud eingetragen werden. 38400 wird auch von
alten Schnittstellen unterstützt. Eventuell kann es Probleme geben, wenn man die
Rate auf 57600 oder gar 115200 Baud einstellt.

Beispiel: \verb+CIRC_1_PPP_SERIAL_SPEED='38400'+

\end{description}

\marklabel{sect:ppp-serial-server}{\subsection{PPP-Circuits vom Typ
``serial-server''}}

Dieser Circuit-Typ erlaubt es, PPP-Verbindungen von einem anderen Rechner über
eine serielle Leitung (ohne Modem) annehmen zu können. Es entspricht in seiner
Funktionsweise dem \verb+OPT_PPP+ in älteren fli4l-Versionen. Ein sinnvolles
Anwendungsgebiet ist beispielsweise das Einbinden eines Notebook ohne
Netzwerkkarte in das Netzwerk. Das Paket wird über die Variable
\var{OPT\_PPP\_SERIAL\_SERVER} aktiviert:

\begin{description}
\config{OPT\_PPP\_SERIAL\_SERVER}{OPT\_PPP\_SERIAL\_SERVER}{OPTPPPSERIALSERVER}

Diese Variable muss auf 'yes' gesetzt werden, um PPP-Circuits vom Typ
``serial-server'' anlegen zu können.

Standard-Einstellung: \verb+OPT_PPP_SERIAL_SERVER='no'+
\end{description}

Soll eine konkrete serielle Verbindung angenommen werden können, so muss ein
passender Circuit konfiguriert werden, siehe Abschnitt
\jump{sect:circuits}{Circuit-Konfiguration}. Der zu verwendende Typ in
\var{CIRC\_x\_TYPE} lautet ``ppp'', der zu verwendende PPP-Typ in
\var{CIRC\_x\_PPP\_TYPE} lautet ``serial-server''. Zu den allgemeinen
Circuit-Variablen kommen die folgenden hinzu, die spezifisch für
PPP/serial-server-Circuits sind:

\begin{description}

\config{CIRC\_x\_PPP\_SERIAL\_SERVER\_DEV}{CIRC\_x\_PPP\_SERIAL\_SERVER\_DEV}{CIRCxPPPSERIALSERVERDEV}

Hier ist die serielle Schnittstelle anzugeben, die verwendet werden soll. Sie
können entweder ``com1'' bis ``com4'' oder alternativ ``ttyS0'' bis ``ttyS3''
für die seriellen Schnittstellen 1--4 eintragen.

Beispiel: \verb+CIRC_1_PPP_SERIAL_SERVER_DEV='com2'+

\config{CIRC\_x\_PPP\_SERIAL\_SERVER\_SPEED}{CIRC\_x\_PPP\_SERIAL\_SERVER\_SPEED}{CIRCxPPPSERIALSERVERSPEED}

Hier muss die Übertragungsrate in Baud eingetragen werden. 38400 wird auch von
alten Schnittstellen unterstützt. Eventuell kann es Probleme geben, wenn man die
Rate auf 57600 oder gar 115200 Baud einstellt.

Beispiel: \verb+CIRC_1_PPP_SERIAL_SERVER_SPEED='38400'+

\end{description}

Beispiel (Client): Der Client ``wotan'' baut eine PPP-Verbindung zum Server
``odin'' auf und authentifiziert sich mit Benutzernamen ``user'' und Passwort
``pass''.

\begin{example}
\begin{verbatim}
    HOSTNAME='wotan'
    OPT_PPP='yes'
    OPT_PPP_SERIAL='yes'
    #
    CIRC_N='1'
    CIRC_1_NAME='ppp2odin'
    CIRC_1_TYPE='ppp'
    CIRC_1_ENABLED='yes'
    CIRC_1_NETS_IPV4_N='1'
    CIRC_1_NETS_IPV4_1='192.168.12.0/24'
    CIRC_1_UP='yes'
    CIRC_1_PPP_TYPE='serial'
    CIRC_1_PPP_USERID='user'
    CIRC_1_PPP_PASSWORD='pass'
    CIRC_1_PPP_SERIAL_DEV='com1'
    CIRC_1_PPP_SERIAL_SPEED='38400'
\end{verbatim}
\end{example}

Beispiel (Server): Der Server ``odin'' wartet auf eine PPP-Verbindungsanfrage
von Client ``wotan'' und verlangt eine Authentifizierung mit Benutzernamen
``user'' und Passwort ``pass''. Weil keine Netze hin zum Client geroutet werden
(\verb+CIRC_1_NETS_IPV4_N+ ist nicht definiert), muss via
\verb+CIRC_1_PROTOCOLS='ipv4'+ explizit mitgeteilt werden, dass IPv4 über diese
PPP-Verbindung genutzt werden soll.  Via \verb+CIRC_1_PPP_LOCALIP+ und
\verb+CIRC_1_PPP_REMOTEIP+ wird konfiguriert, welche IPv4-Adressen der Client
und der Server für diese PPP-Verbindung nutzen sollen.

\begin{example}
\begin{verbatim}
    HOSTNAME='odin'
    OPT_PPP='yes'
    OPT_PPP_SERIAL_SERVER='yes'
    OPT_PPP_PEERS='yes'
    PPP_PEER_N='1'
    PPP_PEER_1_USERID='user'
    PPP_PEER_1_PASSWORD='pass'
    PPP_PEER_1_CIRCUITS='ppp2wotan'
    #
    CIRC_N='1'
    CIRC_1_NAME='ppp2wotan'
    CIRC_1_TYPE='ppp'
    CIRC_1_ENABLED='yes'
    CIRC_1_PROTOCOLS='ipv4'
    CIRC_1_UP='yes'
    CIRC_1_PPP_PEER_AUTH='yes'
    CIRC_1_PPP_LOCALIP='192.168.222.1'
    CIRC_1_PPP_REMOTEIP='192.168.222.2'
    CIRC_1_PPP_SERIAL_DEV='com1'
    CIRC_1_PPP_SERIAL_SPEED='38400'
\end{verbatim}
\end{example}
