% Synchronized to r46350
\marklabel{sec:tools}
{
\section{TOOLS - Additional Tools For Debugging}
}

    Package TOOLS provides some Unix programs mostly used for administration
    and debugging. Other ones like wget are used i.e. for catching the first
    (ad-)page of some providers. By setting the value 'yes' the mentioned
    program is copied to the fli4l router. Default setting is 'no'. The programs
    are only described in short, how to use them is described in their repective
    man pages which can be found in your favourite linux distribution or online at:
    \altlink{http://www.linuxmanpages.com}

\subsection{Networking-Tools}

\begin{description}

\config{OPT\_BMON}{OPT\_BMON}{OPTBMON} Bandwidth Monitor

    bmon is a monitoring and debugging tool to capture networking related
    statistics and prepare them visually in a human friendly way. It features
    various output methods including an interactive curses user interface and a
    programmable text output for scripting.  

\config{OPT\_CURL}{OPT\_CURL}{OPTCURL} Data Transfer Tool

    The program curl allows the transfer of data to or from a server running
    a number of supported protocols. These include, amongst others, FTP, HTTP(S),
    SCP, SFTP, and TFTP.

    The program also provides support for user authentication, data transfer
    via proxy, FTP upload, HTTP POST requests, SSL connections, cookies,
    resumption of interrupted file transfers and more.

    \wichtig{In order to establish TLS connections, the Mozilla X.509 root
    certificates should be installed by setting \var{CERT\_X509\_MOZILLA='yes'}
    in the CERT package.}

\config{OPT\_DIG}{OPT\_DIG}{OPTDIG} DNS Multitool

    The command dig allows to execute various DNS queries.

\config{OPT\_FTP}{OPT\_FTP}{OPTFTP} FTP-Client

    The ftp program can connect fli4l to a FTP server to move files between the
    two of them.

\config{FTP\_PF\_ENABLE\_ACTIVE}{FTP\_PF\_ENABLE\_ACTIVE}{FTPPFENABLEACTIVE}

    The setting \verb+FTP_PF_ENABLE_ACTIVE='yes'+ adds a rule to the packet filter
    that enables initiated active FTP.
    For \verb+FTP_PF_ENABLE_ACTIVE='no'+ such a rule has to be added to the
    \var{PF\_OUTPUT\_\%}-array manually (if needed), for an example look
    \jump{sec:masqueradingmodule}{here}.

    Passive FTP is always possible, neither this variable nor an explicit packet
    filter rule is needed then.

\config{OPT\_IFTOP}{OPT\_IFTOP}{OPTIFTOP} Netzwerküberwachung

    The iftop program lists all active network connections and
    their throughput directly on the fli4l router.

    After login to the router iftop is simply started by executing it on the
    console.

\config{OPT\_IMONC}{OPT\_IMONC}{OPTIMONC} Text based control program for imond

    This program is a text based frontend for the router to control imond.

\config{OPT\_IPERF}{OPT\_IPERF}{OPTIPERF} Performance checks for the network

    The command iperf can do performance measuring for networks. The program
    has to be started on the two machines involved. On the iperf server execute

\begin{example}
\begin{alltt}
fli4l-server \version~\# iperf -s
------------------------------------------------------------
Server listening on TCP port 5001
TCP window size: 85.3 KByte (default)
------------------------------------------------------------
\end{alltt}
\end{example}

    to start. The server will wait for a connection from an iperf client then. Start
    the client on the second machine with

\begin{example}
\begin{alltt}
fli4l-client \version~\# iperf -c 1.2.3.4
------------------------------------------------------------
Client connecting to 1.2.3.4, TCP port 5001
TCP window size: 16.0 KByte (default)
------------------------------------------------------------
[  3] local 1.2.3.5 port 50311 connected with 1.2.3.4 port 5001
[ ID] Interval       Transfer     Bandwidth
[  3]  0.0-10.0 sec    985 MBytes    826 Mbits/sec
\end{alltt}
\end{example}

    A performance check will start immedeately showing first results. Some options
    can be added to iperf, for details evaluate the informations on its homepage \\
    \altlink{http://iperf.sourceforge.net/}.

\config{OPT\_LNSTAT}{OPT\_LNSTAT}{OPTLNSTAT} Advanced network informations

    This OPT installs \texttt{lnstat}, a tool for processing
    informations from  \texttt{/proc/net/stat}. The name stands for ``Linux
    Network Statistics''.

    Examples from \texttt{lnstat}'s manual:
    \begin{example}
    \begin{alltt}
    \# \textbf{lnstat -d}
        \textsl{Get a list of supported statistics files.}
    \# \textbf{lnstat -k arp_cache:entries,rt_cache:in_hit,arp_cache:destroys}
        \textsl{Select the specified files and keys.}
    \# \textbf{lnstat -i 10}
        \textsl{Use an interval of 10 seconds.}
    \# \textbf{lnstat -f ip_conntrack}
        \textsl{Use only the specified file for statistics.}
    \# \textbf{lnstat -s 0}
        \textsl{Do not print a header at all.}
    \# \textbf{lnstat -s 20}
        \textsl{Print a header at start and every 20 lines.}
    \# \textbf{lnstat -c -1 -i 1 -f rt_cache -k entries,in_hit,in_slow_tot}
        \textsl{Display statistics for keys entries, in_hit and in_slow_tot of field
        rt_cache every second.}
    \end{alltt}
    \end{example}

\config{OPT\_NETCAT}{OPT\_NETCAT}{OPTNETCAT} Transfer data to TCP based servers

\config{OPT\_NETIO}{OPT\_NETIO}{OPTNETIO} Network performance test

    Just like ``iperf'' the program ``netio'' can do network performance tests.
    The program has to be started on both systems taking part in the test, one
    acting as the server, one as the client side.
    On the server side the program has to be started via
    \verb+netio -s -t+ (for data transfer via TCP) resp. \verb+netio -s -u+
    (for data transfer via UDP). Via \verb+netio <server> -t+ resp. \verb+netio <server> -u+
    the ``netio'' client process is started on the client side, contacting the server side
    and starting the performance measurement.

\config{OPT\_NGREP}{OPT\_NGREP}{OPTNGREP} A grep that is able to work directly with network devices.

\config{OPT\_NMAP}{OPT\_NMAP}{OPTNMAP} Port Scanner

    By the help of nmap operating systems can be tested for open ports.
    In addition the program provides further information, e.g. about
    MAC addresses or the operating system used.

\config{OPT\_NTTCP}{OPT\_NTTCP}{OPTNTTCP} Network checks

    The program NTTCP can check network speed. On one side a server is started
    and on the other side the client.

    Start the server by executing nttcp -i -v. The server will wait for
    client requests. To test i.e. speed execute nttcp -t $<$IP
    Adress of the server$>$ on the client.

    This is how a started nttcp server looks like:

\begin{example}
\begin{alltt}
fli4l-server \version~\# nttcp -i -v
nttcp-l: nttcp, version 1.47
nttcp-l: running in inetd mode on port 5037 - ignoring options beside -v and -p
\end{alltt}
\end{example}

    This is how a test with a nttcp client looks like:

\begin{example}
\begin{alltt}
fli4l-client \version~\# nttcp -t 192.168.77.77
l~~8388608~~~~4.77~~~~0.06~~~~~14.0713~~~1118.4811~~~~2048~~~~429.42~~~34133.3
1~~8388608~~~~4.81~~~~0.28~~~~~13.9417~~~~239.6745~~~~6971~~~1448.21~~~24896.4
\end{alltt}
\end{example}

    The nttcp help shows all further parameters:

\begin{verbatim}
Usage: nttcp [local options] host [remote options]
       local/remote options are:
        -t      transmit data (default for local side)
        -r      receive data
        -l#     length of bufs written to network (default 4k)
        -m      use IP/multicasting for transmit (enforces -t -u)
        -n#     number of source bufs written to network (default 2048)
        -u      use UDP instead of TCP
        -g#us   gap in micro seconds between UDP packets (default 0s)
        -d      set SO_DEBUG in sockopt
        -D      don't buffer TCP writes (sets TCP_NODELAY socket option)
        -w#     set the send buffer space to #kilobytes, which is
                dependent on the system - default is 16k
        -T      print title line (default no)
        -f      give own format of what and how to print
        -c      compares each received buffer with expected value
        -s      force stream pattern for UDP transmission
        -S      give another initialisation for pattern generator
        -p#     specify another service port
        -i      behave as if started via inetd
        -R#     calculate the getpid()/s rate from # getpid() calls
        -v      more verbose output
        -V      print version number and exit
        -?      print this help
        -N      remote number (internal use only)
        default format is: %9b%8.2rt%8.2ct%12.4rbr%12.4cbr%8c%10.2rcr%10.1ccr
\end{verbatim}

\config{OPT\_RTMON}{OPT\_RTMON}{OPTRTMON}
        Installs a tool that will track changes in routing tables.
        Primary used for debugging.

\config{OPT\_SOCAT}{OPT\_SOCAT}{OPTSOCAT}

    The program ``socat'' is more or less an enhanced version of the
    \jump{OPTNETCAT}{``netcat'' program} with more functionality. By using
    ``socat'' you may not only establish or accept various types of
    network connections, but also sent data to or read data from UNIX
    sockets, devices, FIFOs, and so on. In addition sources and destinations
    of \emph{different} types may be connected:
    An example would be a Network server listening on a TCP port and
    then writing received data to a local FIFO or reading data from the FIFO
    and then transmitting it over the network to a client. See
    \altlink{http://www.dest-unreach.org/socat/doc/socat.html}
    for more information and application examples.

\config{OPT\_TCPDUMP}{OPT\_TCPDUMP}{OPTTCPDUMP} debug

    The program tcpdump can watch, interpret and record network traffic.
    Find more on this by feeding Google with
    \grqq{}tcpdump man\grqq{}

    tcpdump $<$parameter$>$

\config{OPT\_TRACEPATH}{OPT\_TRACEPATH}{OPTTRACEPATH} Determining the PMTU

    The program \texttt{tracepath} is used to determine
    the so-called ``Path MTU''. This is defined as the maximum usable packet size
    for the path used from the fli4l router to the destination host. Larger packets
    must either be fragmented (IPv4) or discarded (IPv6).
    Typically, the Linux kernel cares for the determination of the correct
    Path MTU (`` Path MTU Discovery ''). Occasionally, however, it is
    useful to find out this Path MTU to diagnose problems in the network.
    
    An example for IPv4:
    \begin{example}
    \begin{alltt}
    sandbox 4.0.0-r46077M \# tracepath -4 fli4l.de
     1?: [LOCALHOST]                                         pmtu 1500
     1:  fritz.box                                             0.703ms 
     1:  fritz.box                                             0.588ms 
     2:  a89-182-53-190.net-htp.de                             0.702ms pmtu 1492
     2:  a81-14-248-243.net-htp.de                            33.692ms 
     3:  a81-14-249-82.net-htp.de                             32.089ms asymm  4 
     4:  xe-4-1-2.edge4.Berlin1.Level3.net                    35.936ms
     5:  SYSELEVEN-G.edge4.Berlin1.Level3.net                 74.944ms asymm  8
     6:  ecix.dus.octalus.in-berlin.de                        49.693ms asymm  7
     7:  virtualhost.in-berlin.de                             50.269ms reached
         Resume: pmtu 1492 hops 7 back 57
    \end{alltt}
    \end{example}
    In this example the Internet connection is established via a DSL connection between
    an AVM Fritz! Box and a BRAS belonging to the Internet provider htp. As for DSL in
    Germany PPP over Ethernet (PPPoE) is used, the Path MTU is only 1492 bytes in size.
    
    An example for IPv6:
    \begin{example}
    \begin{alltt}
    sandbox 4.0.0-r46077M \# tracepath -6 fli4l.de
     1?: [LOCALHOST]                        0.046ms pmtu 1280
     1:  gw-1362.ham-01.de.sixxs.net                          43.586ms
     1:  gw-1362.ham-01.de.sixxs.net                          42.832ms
     2:  2001:6f8:862:1::c2e9:c729                            43.565ms asymm  1
     3:  2001:6f8:862:1::c2e9:c72c                            44.313ms asymm  2
     4:  30gigabitethernet4-3.core1.fra1.he.net               64.501ms asymm  6
     5:  no reply
     6:  virtualhost.in-berlin.de                             65.949ms reached
         Resume: pmtu 1280 hops 6 back 56
    \end{alltt}
    \end{example}
    Here, the Internet connection is established via a 6in4 tunnel by SixXS.
    In the configuration of the tunnel a tunnel MTU of 1280 is fixed. Since this is
    the smallest possible MTU for IPv6 connections, it is not further reduced on
    the path to the destination host.

\config{OPT\_DHCPDUMP}{OPT\_DHCPDUMP}{OPTDHCPDUMP} DHCP packet dumper

    The program ``dhcpdump'' can be used to analyse DHCP packets more
    accurately. It is based on tcpdump and produces easily readable output.

    Usage:

\begin{verbatim}
    dhcpdump -i interface [-h regular-expression]
\end{verbatim}

    The program can be executed using the following command:

\begin{verbatim}
    dhcpdump -i eth0
\end{verbatim}

    If desired, the analysis can be cut down to a specific MAC address using
    regular expressions. The command would be:

\begin{verbatim}
    dhcpdump -i eth0 -h ^00:a1:c4
\end{verbatim}

    The output could look like this:

\begin{example}
\begin{verbatim}
         TIME: 15:45:02.084272
           IP: 0.0.0.0.68 (0:c0:4f:82:ac:7f) > 255.255.255.255.67 (ff:ff:ff:ff:ff:ff)
           OP: 1 (BOOTPREQUEST)
        HTYPE: 1 (Ethernet)
         HLEN: 6
         HOPS: 0
          XID: 28f61b03
         SECS: 0
        FLAGS: 0
       CIADDR: 0.0.0.0
       YIADDR: 0.0.0.0
       SIADDR: 0.0.0.0
       GIADDR: 0.0.0.0
       CHADDR: 00:c0:4f:82:ac:7f:00:00:00:00:00:00:00:00:00:00
        SNAME: .
        FNAME: .
       OPTION:  53 (  1) DHCP message type         3 (DHCPREQUEST)
       OPTION:  54 (  4) Server identifier         130.139.64.101
       OPTION:  50 (  4) Request IP address        130.139.64.143
       OPTION:  55 (  7) Parameter Request List      1 (Subnet mask)
                                                     3 (Routers)
                                                    58 (T1)
                                                    59 (T2)
\end{verbatim}
\end{example}
    
\config{OPT\_WGET}{OPT\_WGET}{OPTWGET} http/ftp Client

    The program wget can fetch data from web servers in batch mode.
    More useful is (and that is why wget is included here) that it can
    catch redirections to your provider's own webserver while
    establishig a connection to the internet in a simple way i.e.
    for Freenet. A (german) Mini-HowTo on this topic can be found here:\\
\altlink{http://www.fli4l.de/hilfe/howtos/einsteiger/wget-und-freenet/}

    \wichtig{In order to establish TLS connections, the Mozilla X.509 root
    certificates should be installed by setting \var{CERT\_X509\_MOZILLA='yes'}
    in the CERT package.}

\end{description}

\subsection{Hardware Identification}

Often you do not know exactly what hardware is in your own computer
or which driver you should use for i.e. your network card or the USB
chipset. The hardware itself can help here. It provides a list of
devices in the computer and the associated driver if possible. You
can choose whether the recognition is done at boot (which is recommended
before the first installation) or later when the computer is running,
comfortably triggered from the Web interface. The output might look
like this:

\begin{example}
\begin{alltt}
fli4l \version # cat /bootmsg.txt
#
# PCI Devices and drivers
#
Host bridge: Advanced Micro Devices [AMD] CS5536 [Geode companion] Host Bridge (rev 33)
Driver: 'unknown'
Entertainment encryption device: Advanced Micro Devices [AMD] Geode LX AES Security Block
Driver: 'geode_rng'
Ethernet controller: VIA Technologies, Inc. VT6105M [Rhine-III] (rev 96)
Driver: 'via_rhine'
Ethernet controller: VIA Technologies, Inc. VT6105M [Rhine-III] (rev 96)
Driver: 'via_rhine'
Ethernet controller: VIA Technologies, Inc. VT6105M [Rhine-III] (rev 96)
Driver: 'via_rhine'
Ethernet controller: Atheros Communications, Inc. AR5413 802.11abg NIC (rev 01)
Driver: 'unknown'
ISA bridge: Advanced Micro Devices [AMD] CS5536 [Geode companion] ISA (rev 03)
Driver: 'unknown'
IDE interface: Advanced Micro Devices [AMD] CS5536 [Geode companion] IDE (rev 01)
Driver: 'amd74xx'
USB Controller: Advanced Micro Devices [AMD] CS5536 [Geode companion] OHC (rev 02)
Driver: 'ohci_hcd'
USB Controller: Advanced Micro Devices [AMD] CS5536 [Geode companion] EHC (rev 02)
Driver: 'ehci_hcd'
\end{alltt}
\end{example}

In essential 3 network cards are in this machine, driven by 'via\_rhine'
and an Atheros-WiFi card driven by madwifi (the driver name is not resoved
correctly here).

\begin{description}

  \config{OPT\_HW\_DETECT}{OPT\_HW\_DETECT}{OPTHWDETECT} This variable
  triggers that the files needed for hardware identification are copied to
  the router. Results can either be found on the console while booting if
  \var{HW\_DETECT\_AT\_BOOTTIME} is set to 'yes' or in the web interface
  if \jump{OPTHTTPD}{\var{OPT\_HTTPD}} was set to 'yes'. The content of
  '/bootmsg.txt' can of course be checked in the web interface as well
  if you have a working network connection to the router.

  \config{HW\_DETECT\_AT\_BOOTTIME}{HW\_DETECT\_AT\_BOOTTIME}{HWDETECTATBOOTTIME} Starts
  hardware identification start on boot. Recognition will take place in
  background (it will take some time) and then writes its output to console
  and '/bootmsg.txt'.

\config{OPT\_LSPCI}{OPT\_LSPCI}{OPTLSPCI} Listing of all PCI devices

\config{OPT\_I2CTOOLS}{OPT\_I2CTOOLS}{OPTI2CTOOLS} Tools for
I\textsuperscript{2}C access.

\config{OPT\_IWLEEPROM}{OPT\_IWLEEPROM}{OPTIWLEEPROM} Tool to access the
EEPROM of Intel and Atheros WLAN cards.

Needed e.g. to reprogram the regulatory domain on ath9k cards
(see \altlink{http://blog.asiantuntijakaveri.fi/2014/08/one-of-my-atheros-ar9280-minipcie-cards.html}).

\config{OPT\_ATH\_INFO}{OPT\_ATH\_INFO}{OPTATHINFO} Hardware diagnosis tool
for WLAN cards based on Atheros chipset.

This tool can extract detailed information about the hardware used in Atheros
wireless cards, e.g. ath5k. These include, amongst others, the chipset used or
calibration information.

\config{OPT\_FLASHROM}{OPT\_FLASHROM}{OPTFLASHROM} Tool to Flash Chipsets.

This tool may provide a new BIOS or a new Firmware for example for PC Engines mainboards.
Details can be found at \altlink{http://www.flashrom.org}.

\end{description}

\subsection{File Management Tools}

\begin{description}

\config{OPT\_E3}{OPT\_E3}{OPTE3} An editor for fli4l

    E3 is a very small editor written in Assembler. It mimics various editor
    modes known from other (,,full size'') editors. To choose a mode e3 only has
    to be started with the right command. A short key overview is given by e3
    starting without parameter or by pressing Alt+H (except in VI-mode,
    press ,,:h'' in CMD mode then). Caret (\^{ }) stands for the Ctrl-/Strg key.

    \begin{tabular}{ll}
      Command & Mode \\
      \hline
      e3 / e3ws & WordStar, JOE \\
      e3vi      & VI, VIM \\
      e3em      & Emacs \\
      e3pi      & Pico \\
      e3ne      & NEdit
    \end{tabular}

\config{OPT\_MTOOLS}{OPT\_MTOOLS}{OPTMTOOLS} mtools provide some DOS-like commands
     for simpler handling of DOS media (copying, formatting, a.s.o.).

     Exact syntax of the commands can be found in the mtools documentation:\\
     \altlink{http://www.gnu.org/software/mtools/manual/mtools.html}

\config{OPT\_SHRED}{OPT\_SHRED}{OPTSHRED}

        Installs the program \emph{shred} on the router which is used
        for secure erasing of block devices.

\config{OPT\_YTREE}{OPT\_YTREE}{OPTYTREE} File Manager

    Installs the File Manager Ytree on the router.

\end{description}

\subsection {Developer-Tools}

\begin{description}

\config{OPT\_OPENSSL}{OPT\_OPENSSL}{OPTOPENSSL}

    The tool openssl can i.e. be used to test crypto accelerator devices.

\begin{example}
\begin{alltt}
openssl speed -evp des -elapsed
openssl speed -evp des3 -elapsed
openssl speed -evp aes128 -elapsed
\end{alltt}
\end{example}

\config{OPT\_STRACE}{OPT\_STRACE}{OPTSTRACE} debug

    By using the program strace you can trace system calls and signals or watch
    function calls and runtime behavior of a program.

    strace $<$program$>$

\config{OPT\_REAVER}{OPT\_REAVER}{OPTREAVER} Brute force attacks on Wifi WPS PINs

    Tests all possible WPS PINS to find the WPA password. For details on usage 
    see: \\
    \altlink{http://code.google.com/p/reaver-wps/}.

\config{OPT\_VALGRIND}{OPT\_VALGRIND}{OPTVALGRIND}
        Installs valgrind on the router.

\end{description}
