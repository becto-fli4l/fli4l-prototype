% Synchronized to r44499
\section {APCUPSD -- Démon pour onduleur APC}

\subsection{Introduction}

  Ce paquetage contient le démon apcupsd {[\ref{apcupsd}]} pour la surveillance
  des onduleurs APC sur fli4l. Tous les paramétres ont été pris dans le paquetage
  original {[\ref{apcupsd_manual}]}.

\subsection{Configuration du paquetage APCUPSD}

  La configuration est créée, comme tous les autre paquetage fli4l, en paramètrant
  le fichier\\
  \var{path/fli4l-\version/$<$config$>$/apcupsd.txt} il répondra selon vos propres
  exigences.

\begin{description}

\config {OPT\_APCUPSD}{OPT\_APCUPSD}{OPTAPCUPSD}

  Avec le paramètre \var{'no'} vous déactivez complètement le paquetage OPT\_APCUPSD.
  Il n'y aura aucun changement sur le support de boot de fli4l \var{rootfs.img} ou dans
  l'archive \var{opt.img}. De plus, OPT\_APCUPSD ne remplace aucune autre partie
  de l'installation de fli4l.\\
  Pour activer OPT\_APCUPSD vous devez paramétrer la variable \var{OPT\_APCUPSD} sur \var{'yes'}.

\end{description}

\subsection{Paramètres de communication}
\begin{description}

\config {APCUPSD\_UPSNAME}{APCUPSD\_UPSNAME}{APCUPSDUPSNAME}

  Utilisez cette variable pour donner un nom à votre onduleur, pour le fichier
  log (ou journal) et autres.
  C'est particulièrement utile si vous avez plusieurs onduleurs. Le nom n'est
  pas écrit dans l'EEPROM de l'onduleur. Le nom doit être de 8 caractères maximum.

\config {APCUPSD\_UPSCABLE}{APCUPSD\_UPSCABLE}{APCUPSDUPSCABLE}

 Dans cette variable vous définissez le type de câble de raccordement de l'onduleur
 à fli4l.

   \var{'simple'}, \var{'smart'}, \var{'ether'} ou \var{'usb'}

  Vous pouvez aussi utiliser l'un des câbles spécifiques~:

  \var{'940-0119A'}, \var{'940-0127A'}, \var{'940-0128A'}, \var{'940-0020B'},
  \var{'940-0020C'}, \var{'940-0023A'}, \var{'940-0024B'}, \var{'940-0024C'},
  \var{'940-1524C'}, \var{'940-0024G'}, \var{'940-0095A'}, \var{'940-0095B'},
  \var{'940-0095C'} ou \var{'M-04-02-2000'}


\config {APCUPSD\_UPSTYPE} {APCUPSD\_UPSTYPE}{APCUPSDUPSTYPE}

  Dans cette variable vous définissez le mode de transmission avec laquelle
  l'onduleur est connecté à fli4l. Le mode de transmission doit être associé
  au type de périphérique avec la variable \smalljump{APCUPSDUPSDEVICE}{\var{APCUPSD\_UPSDEVICE}}.


\config {APCUPSD\_UPSDEVICE} {APCUPSD\_UPSDEVICE}{APCUPSDUPSDEVICE}

  En plus du mode de transmission avec \smalljump{APCUPSDUPSTYPE}{\var{APCUPSD\_UPSTYPE}},
  le type de périphérique, doit être défini, il sera utilisé pour la
  communication avec l'onduleur. Le tableau suivant décrit les types de
  périphérique qui peuvent être spécifiés pour l'onduleur.

  Avec une connexion USB à l'onduleur la variable \var{APCUPSD\_UPSDEVICE}
  doit être vide.

\begin{tabular}{p{20mm}p{120mm}}
  Type d'onduleur & Périphérique,  \\ & Description
  \\\\

  \var{'apcsmart'} & \var{'/dev/tty*'} \\ &
  Dispositif série pour les modèles SmartUPS, ils utilisent un câble série
  (pas d'USB).
  \\\\

  \var{'usb'} & \var{''} \\ &
  La plupart des onduleurs les plus récents sont connectés via l'USB. 
  Si la variable \var{APCUPSD\_UPSDEVICE} est vide, cela permettra la détection
  automatique, c'est le meilleur choix pour la plupart des installations.
  \\\\

  \var{'net'} & \var{'hostname:port'} \\ & 
  Connexion réseau à un maître apcupsd par le protocole Network Information
  Server apcupsd. C'est utile pour un onduleur alimentant fli4l et qui est connecté
  à un autre ordinateur pour la surveillance.
  \\\\

%   snmp & hostname:port:vendor:community \\ & 
%   SNMP network link to an SNMP-enabled UPS device.
%   Hostname is the ip address or hostname of the UPS 
%   on the network. Vendor can be can be 'APC' or 
%   'APC\_NOTRAP'. 'APC\_NOTRAP' will disable SNMP trap 
%   catching; you usually want 'APC'. Port is usually 
%   161. Community is usually 'private'.
%   (disabled in build from apcupsd binary) \\
% 
%   netsnmp & :port:vendor:community \\ &
%   OBSOLETE \\ &
%   Same as SNMP above but requires use of the 
%   net-snmp library. Unless you have a specific need
%   for this old driver, you should use 'snmp' instead.
%   (disabled in build from apcupsd binary) \\
% 
%   dumb  & /dev/tty** \\ &
%   Old serial character device for use with simple-signaling UPSes. \\

  \var{'pcnet'} & \var{'ipaddr:username:passphrase[:port]'} \\ &

  Protocole PowerChute Network Shutdown, il peut être utilisé comme
  une alternative à SNMP avec la famille de carte Smart-Slot AP9617.

  ipaddr~: est l'adresse IP de la carte de gestion de l'onduleur.

  username et passphrase~: sont les informations de connexion pour la carte
  de gestion.

  port~: est le port TCP sur lequel on écoute les messages de l'onduleur,
  normalement le numéro de port est 3052. Si ce paramètre est manquant
  ou vide, la valeur par défaut utilisée est 3052.
  \\
\end{tabular}


\config {APCUPSD\_POLLTIME} {APCUPSD\_POLLTIME}{APCUPSDPOLLTIME}

  Vous indiquez dans cette variable l'intervalle en secondes avec lequel
  apcupsd interroge le statut de l'onduleur.

  Ce paramètre affecte directement les deux connexions aux onduleurs
  (\smalljump{APCUPSDUPSTYPE}{\var{APCUPSD\_UPSTYPE}} \var{'apcsmart'}, \var{'usb'},) 
  ainsi que les onduleurs connectés au réseau (\smalljump{APCUPSDUPSTYPE}{\var{APCUPSD\_UPSTYPE}} 
  \var{'net'}, \var{'pcnet'}).
  Une réduction de la valeur d'intervalle, augmente la sensibilité du apcupsd au prix
  d'une charge plus élevée du CPU (par défaut \var{'60'}).
  Le réglage par défaut de 60 secondes est approprié dans la plupart des situations.

\end {description}

\subsection{Paramètres des répertoires}
\begin {description}

\config {APCUPSD\_LOCKFILE} {APCUPSD\_LOCKFILE}{APCUPSDLOCKFILE}

  Vous indiquez dans cette variable le chemin vers le répertoire pour le fichier
  de verrouillage de l'appareil (par défaut \var{'/var/lock'}).


\config {APCUPSD\_SCRIPTDIR} {APCUPSD\_SCRIPTDIR}{APCUPSDSCRIPTDIR}

  Vous indiquez dans cette variable le chemin vers le répertoire pour le script
  apccontrol et des scripts événements, dans lequel ils sont situés
  (par défaut \var{'/etc'}).


\config {APCUPSD\_PWRFAILDIR} {APCUPSD\_PWRFAILDIR}{APCUPSDPWRFAILDIR}

  Vous indiquez dans cette variable le chemin vers le répertoire pour le fichier
  powerfail, ce fichier contient un marquage (ou un flag) à chaque coupure électrique.

  Ce fichier est créé lorsque apcupsd initie un arrêt du système et vérifie
  dans le script d'arrêt s'il est nécessaire d'éteindre l'onduleur
  (par défaut \var{'/etc'}).


\config {APCUPSD\_NOLOGINDIR} {APCUPSD\_NOLOGINDIR}{APCUPSDNOLOGINDIR}

  Vous indiquez dans cette variable le chemin vers le répertoire pour le fichier
  nologin. Si le fichier existe il désactivera les nouvelles connexions
  (par défaut \var{'/etc'}).

\end {description}

\subsection{Paramètres pour les coupures de courant}
\begin {description}

\config {APCUPSD\_ONBATTERYDELAY} {APCUPSD\_ONBATTERYDELAY}{APCUPSDONBATTERYDELAY}

  Vous indiquez dans cette variable le temps en secondes entre la détection
  d'une coupure de courant et la réaction avec l'argument OnBattery
  (par défaut \var{'6'}).

  Le script apccontrol sera appelé avec l'argument PowerOut immédiatement après
  une panne de courant détectée. Avec l'argument OnBattery le script apccontrol sera
  appelé seulement après un certain temps avec \var{APCUPSD\_ONBATTERYDELAY}


\config {APCUPSD\_BATTERYLEVEL} {APCUPSD\_BATTERYLEVEL}{APCUPSDBATTERYLEVEL}

  Vous indiquez dans cette variable le pourcentage de charge restant, lors
  d'une panne de courant si le pourcentage (tel que mesuré par l'onduleur) est
  inférieure ou égale à \var{APCUPSD\_BATTERYLEVEL}, apcupsd entamera un
  arrêt du système (par défaut \var{'5'}).


\config {APCUPSD\_MINUTES} {APCUPSD\_MINUTES}{APCUPSDMINUTES}

  Vous indiquez dans cette variable le temps restant en minutes, lors
  d'une panne de courant si le temps (tel que calculé par l'onduleur) est
  inférieure ou égale à \var{APCUPSD\_MINUTES}, apcupsd entamera un
  arrêt du système (par défaut \var{'3'}).


\config {APCUPSD\_TIMEOUT} {APCUPSD\_TIMEOUT}{APCUPSDTIMEOUT}

  Vous indiquez dans cette variable le temps restant en seconde, pendant
  une panne de courant l'onduleur fonctionne sur la batterie si le temps est
  inférieure ou égale à \var{APCUPSD\_TIMEOUT} apcupsd entamera un arrêt du
  système. Cela peut indiquer (un temps total de panne de courant à ne pas
  dépasser) avant l'arrêt du système. (par défaut \var{'0'}).
  La valeur \var{'0'} désactive cette temporisation.

  Remarque, si vous avez un onduleur Smart, vous allez probablement désactiver
  cette temporisation en mettant la valeur à zéro. De cette façon, votre onduleur
  continuera de fonctionner sur la batterie jusqu'à ce que la charge restante
  de la batterie soit égal à la valeur de \var{APCUPSD\_BATTERYLEVEL} ou que le temps
  indiqué dans \var{APCUPSD\_MINUTES}. Si vous voulez faire un test, avec la valeur
  à \var{'60'} et si vous débranchez le cordon d'alimentation cela provoquera un arrêt
  rapide du système.

  Autre remarque, les variables \var{APCUPSD\_BATTERYLEVEL}, \var{APCUPSD\_MINUTES}
  et \var{APCUPSD\_TIMEOUT} travail ensemble, de sorte que la première détection
  d'une valeur, déclenchera un arrêt du système.


\config {APCUPSD\_ANNOY} {APCUPSD\_ANNOY}{APCUPSDANNOY}

  Vous indiquez dans cette variable le temps en seconde, pour l'envoi de message
  aux utilisateurs pour se déconnecter avant l'arrêt du système. en appelant le script
  apccontrol apcupsd envoie un message énervent (par défaut \var{'300'}).
  \var{'0'} pour désactiver.


\config {APCUPSD\_ANNOYDELAY} {APCUPSD\_ANNOYDELAY}{APCUPSDANNOYDELAY}

  Vous indiquez dans cette variable le temps en seconde, avant que apcupsd
  commence à demander aux utilisateurs de se déconnecter et de sortir du système
  (par défaut \var{'60'}).


\config {APCUPSD\_NOLOGON} {APCUPSD\_NOLOGON}{APCUPSDNOLOGON}

  Vous indiquez dans cette variable une condition qui empêchera les utilisateurs
  de se connectés pendant une panne de courant.
  La variable \var{APCUPSD\_NOLOGON} peut prendre l'une les valeurs suivantes
  \var{'disable'}, \var{'timeout'}, \var{'percent'}, \var{'minutes'} ou
  \var{'always'} (par défaut \var{'disable'}).


\config {APCUPSD\_KILLDELAY} {APCUPSD\_KILLDELAY}{APCUPSDKILLDELAY}

  Vous indiquez dans cette variable le temps en seconde, si la valeur n'est pas
  à zéro, apcupsd continuera à fonctionner après une demande d'arrêt et après le
  temps indiqué ici, ensuite il essaiera de couper la connexion

  Cette fonction est utilisée sur des systèmes où apcupsd ne peut pas reprendre
  le contrôle après un arrêt (par défaut \var{'0'}).
  \var{'0'} pour désactiver.

\end {description}

\subsection{Paramètres du serveur sur le réseau}

\begin {description}

\config {APCUPSD\_NETSERVER} {APCUPSD\_NETSERVER}{APCUPSDNETSERVER}

  Avec le paramètre \var{'yes'} vous activez et avec le paramètre \var{'no'}
  vous déactivez la variable pour le serveur d'information réseau. Si vous
  avez démarré le processus du serveur d'information réseau, vous pouvez voir
  l'état et les données d'événement sur le réseau. (Les données sont utilisées
  par le programme CGI) (par défaut \var{'no'}).


\config {APCUPSD\_NISIP} {APCUPSD\_NISIP}{APCUPSDNISIP}

  Vous indiquez dans cette variable l'adresse IP du serveur NIS qui écoute
  les connexions entrantes. Ce paramètre est utile, si fli4l a de multiples
  connexions réseau. La valeur par défaut est \var{'0.0.0.0'} cela signifie
  que toutes requêtes entrantes seront acceptées. Alternativement, vous pouvez
  paramétrer une adresse IP spécifique de votre serveur NIS et écoutez
  uniquement les connexions entrantes de cette interface. Vous pouvez
  utiliser l'adresse de bouclage (\var{'127.0.0.1'}) pour accepter les
  connexions locales.


\config {APCUPSD\_NISPORT} {APCUPSD\_NISPORT}{APCUPSDNISPORT}

  Vous indiquez dans cette variable le port TCP pour la transmission
  de le statut et les données d'événement sur le réseau. A utiliser
  seulement si la variable \smalljump{APCUPSDNETSERVER}{\var{APCUPSD\_NETSERVER}}
  est sur \var{'on'}. Si vous changez ce port, vous aurez besoin de changer
  la valeur correspondante dans le répertoire cgi et reconstruire le programme CGI.
  Par défaut \var{'3551'} est enregistré auprès de l'IANA.


\config {APCUPSD\_EVENTSFILE} {APCUPSD\_EVENTSFILE}{APCUPSDEVENTSFILE}

  Vous indiquez dans cette variable le fichier d'événement, pour les
  événements récents qui seront interrogés par le serveur d'information
  de réseau, (par défaut \var{'/var/log/apcupsd.events'}).


\config {APCUPSD\_EVENTSFILEMAX} {APCUPSD\_EVENTSFILEMAX}{APCUPSDEVENTSFILEMAX}

  Par défaut la taille du fichier d'évenement de la variable
  \smalljump{APCUPSDEVENTSFILE}{\var{APCUPSD\_EVENTSFILE}}
  ne doit pas dépasser 10 kilo-octets. Lorsque le fichier se développe au-delà
  de cette limite, les événements plus anciens seront supprimés au début du fichier
  (premier entré, premier sorti). Le paramètre dans la variable \var{APCUPSD\_EVENTSFILEMAX}
  peut être réglé sur une valeur différente en kilo-octets ou vous pouvez mettre
  cette valeur à zéro pour permettre au paramètre de la variable
  \smalljump{APCUPSDEVENTSFILE}{\var{APCUPSD\_EVENTSFILE}} de croître sans limite.

\end {description}

\subsection{Configuration des actions de l'onduleur}

\begin {description}

\config {APCUPSD\_UPSCLASS} {APCUPSD\_UPSCLASS}{APCUPSDUPSCLASS}


   Normalement, vous indiquez dans cette variable \var{'standalone'}
   sauf si l'onduleur est partagée pour plusieurs serveurs, avec une
   carte APC Share-Ups. Les valeurs acceptées dans la variable
   \var{APCUPSD\_UPSCLASS} sont \var{'standalone'}, \var{'shareslave'}
   ou \var{'sharemaster'} (par défaut \var{'standalone'}).


\config {APCUPSD\_UPSMODE} {APCUPSD\_UPSMODE}{APCUPSUPSMODE}

   Normalement, vous indiquez dans cette variable \var{'disable'}
   sauf si l'onduleur est partagée pour plusieurs serveurs, avec une
   carte APC Share-Ups. Les valeurs acceptées dans la variable
   \var{APCUPSD\_UPSMODE} sont \var{'disable'} ou \var{'share'}
   (par défaut \var{'disable'}).

\end {description}

\subsection{Enregistrement du journal système}

\begin {description}

\config {APCUPSD\_STATTIME} {APCUPSD\_STATTIME}{APCUPSSTATTIME}

  Dans cette variable vous indiquez l'intervalle en seconde pour écrire
  le fichier de statut (par défaut \var{'0'}). \var{'0'} désactive l'écriture.


\config {APCUPSD\_STATFILE} {APCUPSD\_STATFILE}{APCUPSSTATFILE}

 Dans cette variable vous indiquez le chemin du fichier de statut, (il sera écrit
 seulement si la variable \smalljump{APCUPSDSTATTIME}{\var{APCUPSD\_STATFILE}}
 n'est pas à zéro) (par défaut \var{'/var/log/apcupsd.status'}).


\config {APCUPSD\_LOGSTATS} {APCUPSD\_LOGSTATS}{APCUPSLOGSTATS}

  Dans cette variable vous indiquez les enregistrements du statut dans un fichier
  journal \var{'on'} activé, \var{'off'} déactivé.

  Remarque~! Cela génère beaucoup informations, si cette option est activer,
  assurez-vous que dans le fichier syslog.conf à la variable LOG\_NOTICE soit
  défini 'named pipe' (par défaut \var{'off'}).
  Normalement vous n'avez pas à placer cette valeur sur \var{'on'}.


\config {APCUPSD\_DATATIME} {APCUPSD\_DATATIME}{APCUPSDATATIME}

  Dans cette variable vous indiquez l'intervalle en secondes entre l'écriture du
  statut et enregistrement des données pour le fichier journal, (par défaut \var{'0'}).
  \var{'0'} déactive l'enregistrement du journal.


\config {APCUPSD\_FACILITY} {APCUPSD\_FACILITY}{APCUPSFACILITY}

 Dans cette variable vous indiquez le domaine de protocole (classe) pour syslog.
 (par défaut \var{'daemon'}).
 Il est logique de séparer les données enregistrées par apcupsd des autres programmes.

\end {description}

\subsection{Message d'événement par Mail}

\begin {description}

\config {OPT\_APCUPSD\_EVENTMAIL} {OPT\_APCUPSD\_EVENTMAIL}{OPTAPCUPSDEVENTMAIL}

   Si vous indiquez \var{'yes'} dans cette variable les messages d'événements seront envoyés
   à l'adresse configurée dans \smalljump{APCUPSDEVENTMAILTO}{\var{APCUPSD\_EVENTMAIL\_TO}}
   et via le compte configuré dans \smalljump{APCUPSDEVENTMAILACCOUNT}{\var{APCUPSD\_EVENTMAIL\_ACCOUNT}}.
   De plus le paquetage MAILSEND doit être activé avec \var{OPT\_MAILSEND='yes'}.

  (par défaut \var{'no'}). 


\config {APCUPSD\_EVENTMAIL\_ACCOUNT} {APCUPSD\_EVENTMAIL\_ACCOUNT}{APCUPSDEVENTMAILACCOUNT}

  Dans cette variable vous indiquez le nom du compte pour transmettre des messages d'événement,
  cette variable est optionnelle. Si le nom du compte n'est pas indiqué le compte par \var{'défaut'}
  sera utilisé.


\config {APCUPSD\_EVENTMAIL\_TO} {APCUPSD\_EVENTMAIL\_TO}{APCUPSDEVENTMAILTO}

  Dans cette variable vous indiquez l'adresse mail pour envoyer les messages d'événement.
  Une ou plusieurs adresses peuvent être configurées, elles doivent être séparées par une
  virgule.

\end {description}

% \subsection{Im moment nicht gesetzte Variablen von apcupsd}
% 
%  ========== Configuration statements used in updating the UPS EPROM =========
% 
% 
%  These statements are used only by apctest when choosing "Set EEPROM with conf
%  file values" from the EEPROM menu. THESE STATEMENTS HAVE NO EFFECT ON APCUPSD.
% 
%  UPS name, max 8 characters 
% UPSNAME UPS\_IDEN
% 
%  Battery date - 8 characters
% BATTDATE mm/dd/yy
% 
%  Sensitivity to line voltage quality (H cause faster transfer to batteries)  
%  SENSITIVITY H M L        (default = H)
% SENSITIVITY H
% 
%  UPS delay after power return (seconds)
%  WAKEUP 000 060 180 300   (default = 0)
% WAKEUP 60
% 
%  UPS Grace period after request to power off (seconds)
%  SLEEP 020 180 300 600    (default = 20)
% SLEEP 180
% 
%  Low line voltage causing transfer to batteries
%  The permitted values depend on your model as defined by last letter 
%   of FIRMWARE or APCMODEL. Some representative values are:
%     D 106 103 100 097
%     M 177 172 168 182
%     A 092 090 088 086
%     I 208 204 200 196     (default = 0 => not valid)
% LOTRANSFER  208
% 
%  High line voltage causing transfer to batteries
%  The permitted values depend on your model as defined by last letter 
%   of FIRMWARE or APCMODEL. Some representative values are:
%     D 127 130 133 136
%     M 229 234 239 224
%     A 108 110 112 114
%     I 253 257 261 265     (default = 0 => not valid)
% HITRANSFER 253
% 
%  Battery charge needed to restore power
%  RETURNCHARGE 00 15 50 90 (default = 15)
% RETURNCHARGE 15
% 
%  Alarm delay 
%  0 = zero delay after pwr fail, T = power fail + 30 sec, L = low battery, N = never
%  BEEPSTATE 0 T L N        (default = 0)
% BEEPSTATE T
% 
%  Low battery warning delay in minutes
%  LOWBATT 02 05 07 10      (default = 02)
% LOWBATT 2
% 
%  UPS Output voltage when running on batteries
%  The permitted values depend on your model as defined by last letter 
%   of FIRMWARE or APCMODEL. Some representative values are:
%     D 115
%     M 208
%     A 100
%     I 230 240 220 225     (default = 0 => not valid)
% OUTPUTVOLTS 230
% 
%  Self test interval in hours 336=2 weeks, 168=1 week, ON=at power on
%  SELFTEST 336 168 ON OFF  (default = 336)
% SELFTEST 336
