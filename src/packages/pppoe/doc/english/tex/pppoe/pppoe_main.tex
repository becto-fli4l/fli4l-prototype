% Synchronized to r29817
\section{DSL - DSL over PPPoE, Fritz!DSL and PPTP}

fli4l supports DSL in three different variants:

\begin{itemize}
\item PPPoE (external DSL-modems connected over ethernet using pppoe)
\item PPTP (external DSL-modems connected over ethernet using pptp)
\item Fritz!DSL (DSL over DSL-adapters manufactured by AVM)
\end{itemize}

You can choose only one of these options, simultaneous operation 
isn't possible yet.
The configuration for all variants is similar, so the
general parameters are described at first and then the
special options for the individual variants will be discussed. 
DSL-access is handled by imond as a circuit. Therefore 
it is necessary to activate imond by setting 
(\jump{OPTIMOND}{see \var{OPT\_IMOND}})
to \verb*?'yes'?.

\subsection {General Configuration Variables}

The packages all use the same configuration variables, they
differ only by the package name prefixes. As an example: 
in all packages the user name is required. The variable is named 
\verb*?PPPOE_USER?, \verb*?PPTP_USER? or \verb*?FRITZDSL_USER? 
depending on the package. The variables are described indicating
the missing prefix by an asterisk. In concrete examples PPPOE is 
assumed but these are valid with any other prefix too.

\begin{description}
\configlabel{PPPOE\_NAME}{PPPOENAME}
\configlabel{PPTP\_NAME}{PPTPNAME}
\configlabel{FRITZDSL\_NAME}{FRITZDSLNAME}
\item[*\_NAME]
 
Set a name for the circuit - max. 15 digits long. This will be 
shown in the imon client imonc. Spaces (blanks) are not allowed. 
 
Example: \verb*?PPPOE_NAME='DSL'?

\configlabel{PPPOE\_USEPEERDNS}{PPPOEUSEPEERDNS}
\configlabel{PPTP\_USEPEERDNS}{PPTPUSEPEERDNS}
\configlabel{FRITZDSL\_USEPEERDNS}{FRITZDSLUSEPEERDNS}
\item[*\_USEPEERDNS]
 
This specifies whether the address of the name servers given by 
the internet service provider at dial-in will be added to the
configuration file of the local name server for the duration of 
the online connection or not.

It only makes sense to use this option for internet access circuits. 
Almost all providers support these type of transfer by now.

After the name server IP addresses have been transferred, name 
servers in \verb*?DNS_FORWARDERS? will be used as 
forwarders. Afterwards the local nameserver is forced to 
reread its configuration. Previously resolved names will 
not be lost from the name server cache.

This option has the advantage of always using the nearest name 
servers (given the provider transmits the correct IP addresses) 
thus accelerating the name resolution process.

In case of failure of a provider's DNS server in most cases the 
incorrect DNS server addresses are rapidly corrected by the provider.

Nevertheless it is necessary to specify a valid name server in 
\verb*?DNS_FORWARDERS?. Otherwise the first name request can not 
be resolved correctly. In addition the original configuration of 
the local nameservers will be restored when terminating the connection.

Default-setting: \verb*?*_USEPEERDNS='yes'?

\configlabel{PPPOE\_DEBUG}{PPPOEDEBUG}
\configlabel{PPTP\_DEBUG}{PPTPDEBUG}
\configlabel{FRITZDSL\_DEBUG}{FRITZDSLDEBUG}
\item[*\_DEBUG]
 
If you want pppd to be more verbose set \verb*?*_DEBUG? to
\verb*?'yes'?. In this case pppd will write additional informations
to syslog.

Attention: In order to get this to work syslogd has to activated by 
setting \verb*?OPT_SYSLOGD? to \verb*?'yes'? in base.txt.

\configlabel{PPPOE\_USER}{PPPOEUSER}
\configlabel{PPTP\_USER}{PPTPUSER}
\configlabel{FRITZDSL\_USER}{FRITZDSLUSER}
\configlabel{PPPOE\_PASS}{PPPOEPASS}
\configlabel{PPTP\_PASS}{PPTPPASS}
\configlabel{FRITZDSL\_PASS}{FRITZDSLPASS}
\item[*\_USER, *\_PASS]
 
Provide user ID an passwort for the provider used. 
\verb*?*_USER? is used for the user id
\verb*?*_PASS? is the password.

\emph{Attention:} For T-Online-access consider this:

Username AAAAAAAAAAAATTTTTT\#MMMM is put together from a 12 digit 'Anschlußkennung', 
T-Online-number and 'Mitbenutzernummer'. Behind the T-Online-number follows a
'\#' (only if its length is shorter than 12 digits).

If this does not work in rare cases put another '\#' between 
'Anschlußkennung' and T-Online-number.

User ID for T-Online has to have an additional '@t-online.de' at the end!

Example:

\begin{example}
\begin{verbatim}
        PPPOE_USER='111111111111222222#0001@t-online.de'
\end{verbatim}
\end{example}

Infos on user ID's for other providers are found in the FAQ:
\begin{itemize}
\item \altlink{http://extern.fli4l.de/fli4l_faqengine/faq.php?list=category&catnr=3&prog=1}
\end{itemize}

\configlabel{PPPOE\_HUP\_TIMEOUT}{PPPOEHUPTIMEOUT}
\configlabel{PPTP\_HUP\_TIMEOUT}{PPTPHUPTIMEOUT}
\configlabel{FRITZDSL\_HUP\_TIMEOUT}{FRITZDSLHUPTIMEOUT}
\item[*\_HUP\_TIMEOUT]
 
Specify the time in seconds after which the connection will be terminated 
when no more traffic is detected over the DSL line. A timeout of '0' or 
'never' stands for no timeout. Using 'never' the router immedeately 
reconnects after a disconnection. Changing dialmodes is not 
possible then - it has to stay 'auto'. 'Never' is
currently only available for PPPOE and FritzDSL.

\configlabel{PPPOE\_CHARGEINT}{PPPOECHARGEINT}
\configlabel{PPTP\_CHARGEINT}{PPTPCHARGEINT}
\configlabel{FRITZDSL\_CHARGEINT}{FRITZDSLCHARGEINT}
\item[*\_CHARGEINT]

Charge-interval: Time interval in seconds which will be used 
for calculating online costs.
 
Most provider calculate their charges per minute. In this case put in '60'. 
For providers that charge per second set \verb*?*_CHARGEINT='1'?.

Unfortunately for DSL the timing is not fully utilized, as it is the 
case for ISDN. In our case a hangup will be triggered after the time 
specified in \verb*?*_HUP_TIMEOUT?.

Hence \verb*?*_Chargeint? is only significant for the calculation of
charges.

\configlabel{PPPOE\_TIMES}{PPPOETIMES}
\configlabel{PPTP\_TIMES}{PPTPTIMES}
\configlabel{FRITZDSL\_TIMES}{FRITZDSLTIMES}
\item[*\_TIMES]

This times determine when to enable the circuit and how much 
it will cost. This makes it possible to have different default 
routes at different times for a circuit (Least Cost Routing). 
The imond daemon controls route assignment then.

Composition of variables:

\begin{example}
\begin{verbatim}
        PPPOE_TIMES='times-1-info [times-2-info] ...'
\end{verbatim}
\end{example}

Each field times-?-info consists of 4 parts divided by colons (':').
\begin{enumerate}
\item Field: W1-W2
  
  Weekday-period, i.e. Mo-Fr or Sa-Su aso. English and german 
  notations are both valid.
  For a single weekday write W1-W1 (i.e. Su-Su).

 
\item Field: hh-hh
  
  Hours, i.e. 09-18 or 18-09 too. 18-09 is equivalent to 
  18-24 plus 00-09. 00-24 means the complete day.

        
\item Field: Charge
  
  Currency-values as costs per minute, i.e. 0.032 for 3.2 Cent 
  per minute. They are used to calculate the actual costs incurred, 
  taking into account the cycle time. They will be displayed in the imon client.
    
\item Field: LC-Default-Route
  
  Content can be Y or N, which means:

  \begin{tabular}[h!]{lp{11cm}}
    Y: & The specified time range will be used as default route
    for LC-routing.\\

    N: & The specified time range is only used for calculating
    costs but it won't be used for automatic LC-routing. \\
  \end{tabular}

\end{enumerate}

Example (read as one long line):

\begin{example}
\begin{verbatim}
        PPPOE_TIMES='Mo-Fr:09-18:0.049:N
                     Mo-Fr:18-09:0.044:Y
                     Sa-Su:00-24:0.039:Y'
\end{verbatim}
\end{example}

        \wichtig{Times used in \var{*\_TIMES} have to cover the whole week.
          If that is not the case a valid configuration can't be build.}

        \emph{If the time ranges off all LC-default-route-circuits
          (``Y'') togethter don't contain the complete week there 
          will be no default route in these gaps. There will be no
          internet access possible at these times!}


One more simple example:

\begin{example}
\begin{verbatim}
        PPPOE_TIMES='Mo-Su:00-24:0.0:Y'
\end{verbatim}
\end{example}

for those using a flatrate.

One last comment to LC-routing: \emph{holidays are
   treated as Sundays.}

\configlabel{PPPOE\_FILTER}{PPPOEFILTER}
\configlabel{PPTP\_FILTER}{PPTPFILTER}
\configlabel{FRITZDSL\_FILTER}{FRITZDSLFILTER}
\item[*\_FILTER]

fli4l hangs up automatically if there is no traffic over the pppoe 
interface in the time specified in hangup timeout. Unfortunately also 
traffic from the outside counts here, for example connection attempts 
by a P2P client such as eDonkey. Since you will be contacted almost 
permanently from outside nowadays, it can happen that fli4l never 
terminates the DSL connection.

Option \verb*?*_FILTER? comes to help here. Setting it to yes will only
consider traffic that is generated from your own machine and
external traffic will be completely ignored. Since incoming traffic 
usually means that the router or computers behind it respond by i.e 
rejecting requests additionally some outgoing packets are ignored. 
Read here how this exactly works:
\begin{itemize}
\item \altlink{http://www.fli4l.de/hilfe/howtos/basteleien/hangup-problem-loesen/} and
\item \altlink{http://web.archive.org/web/20061107225118/http://www.linux-bayreuth.de/dcforum/DCForumID2/46.html}.
\end{itemize}
A more detailed description of the expression and its usage is to be 
found in the appendix but is only interesting if you want to make 
changes.

\configlabel{PPPOE\_FILTER\_EXPR}{PPPOEFILTEREXPR}
\configlabel{PPTP\_FILTER\_EXPR}{PPTPFILTEREXPR}
\configlabel{FRITZDSL\_FILTER\_EXPR}{FRITZDSLFILTEREXPR}
\item[*\_FILTER\_EXPR]

Filter to use if \verb*?*_FILTER? is set to \verb*?'yes'?.

\configlabel{FRITZDSL\_MTU}{FRITZDSLMTU}
\configlabel{FRITZDSL\_MRU}{FRITZDSLMRU}
\configlabel{PPPOE\_MTU}{PPPOEMTU}
\config{*\_MTU *\_MRU}{PPPOE\_MRU}{PPPOEMRU}
  
  These variables are optional and can be completely omitted.
  
  With these optional variables the so-called \textbf{MTU} (maximum
  transmission unit) and the \textbf{MRU} (maximum receive unit) can be
  set. By default MTU and MRU are set to 1492. This setting should only be 
  changed in special cases!
  These variables don't exist for OPT\_PPTP.

\configlabel{FRITZDSL\_NF\_MSS}{FRITZDSLNFMSS}
\config{*\_NF\_MSS}{PPPOE\_NF\_MSS}{PPPOENFMSS}

  For some providern these effects can occur:
  \begin{itemize}
  \item Webbrowser gets a connection but is unresponsive after that,
  \item small mails can be sent but big mails can't,
  \item ssh works, scp hangs after initial connecting.
  \end{itemize}

  To work around this problems fli4l manipulates the MTU as a default.
  In some cases this is not enough so fli4l explicitely permits setting 
  of the MSS (message segment size) to a value given by the provider. 
  If the provider does not give any values 1412 is a good start to try.
  If a MTU is given by the provider, subtract 40 Byte here ($mss = mtu - 40$).
  These variables are optional and can be completely omitted.
  These variable doesn't exist for OPT\_PPTP.

\end{description}
\marklabel{sec:pppoe}
{
\subsection {OPT\_PPPOE - DSL over PPPoE}
}

For communication via a DSL connection usually the PPPoE package 
is necessary because the provider does not provide a full blown router,
but only a DSL modem. Between the fli4l router and the modem the 
PPP protocol is used, but in this case specifically over Ethernet.

One or two ethernet cards can be used in fli4l:
\begin{itemize}
\item only one card with IP for the LAN and PPP towards the DSL-modem
\item two cards: one with IP for the LAN, the other for PPP towards the DSL-modem
\end{itemize}

The better choice is to use two ethernet cards. This way both protocols 
- IP and PPPoE - are clearly separated from each other.

But the method with one ethernet cards works as well. In this case 
the DSL-modem has to be connected to the network hub like all other 
clients. The maximum transfer speed can be slightly affected this way.

If experiencing communication problems between the modem and the 
network card you can try to slow down the speed of the NIC, eventually
even switching it to half-duplex mode. All PCI cards but only a few 
PCI-ISA network cards can be configured to run in various speed modes.
Either use ethtool from the package advanced\_networking or create a 
DOS boot media and include the native configuration tool there. Start 
fli4l with this media and execute the card's native tool to choose 
and save the slower operation mode to the card. The configuration 
tool usually is available on the driver media or can be downloaded 
from the website of the card manufacturer.
You may also find it in a search in the wiki:
\begin{itemize}
\item \altlink{https://ssl.nettworks.org/wiki/display/f/Netzwerkkarten}
\end{itemize}

If using two network cards you should use the first for the LAN and 
the second for the connection to the DSL modem.

Only the first card does need an IP address.

This means:

\begin{example}
\begin{verbatim}
        IP_NET_N='1'                # Only *one* card with IP-address!
        IP_NET_1xxx='...'              # Usual parameters
\end{verbatim}
\end{example}

For \verb*?PPPOE_ETH? set the second ethernet card to 'eth1' and
define *no* \verb*?IP_NET_2-xxx?-variables.

\begin{description}
\config{OPT\_PPPOE}{OPT\_PPPOE}{OPTPPPOE} \sloppypar{activates support
  for PPPoE. Default setting: \verb*?OPT_PPPOE='no'?.}

\config{PPPOE\_ETH}{PPPOE\_ETH}{PPPOEETH}

Name of the ethernet interface

\begin{tabular}[h!]{ll}
  'eth0' & first ethernet card\\
  'eth1' & second ethernet card\\
  ...   &  ...\\
\end{tabular}
            
Default setting: \verb*?PPPOE_ETH='eth1'?

\config{PPPOE\_TYPE}{PPPOE\_TYPE}{PPPOETYPE}

\emph{PPPOE} stands for transmission of PPP-packets over
ethernet lines. Data to be transmitted is transformed to ppp-packets
in a first step and then in a second step wrapped in pppoe-packets 
to be transmitted over ethernet to the DSL-modem. The second 
step can be done by the pppoe-daemon or by the kernel. 
\var{PPPOE\_TYPE} defines the way of pppoe packet generation.

\begin{table}[h!]
  \centering
  \begin{tabular}{|l|p{10cm}|}
    \hline
    Values & Description \\
    \hline
    async & Packets are genereated by the pppoe-daemon; 
    asynchronous communication between \emph{pppd} and \emph{pppoed}. \\
    sync & Packets are genereated by the pppoe-daemon; 
    synchronous communication between \emph{pppd} and \emph{pppoed}. 
    This way communication is more efficient and thus leads to
    lower processor load. \\
    in\_kernel & Packets are genereated by the linux kernel. This way 
    communication with a second daemon is omitted saving a lot of
    in-memory copying thus leading to even lower processor load. \\
    \hline
  \end{tabular}
  \caption{Ways of generating pppoe packets}
  \label{tab:pppoe-type}
\end{table}

Somebody did a comparison between the various variants 
on a Fujitsu Siemens PCD-H, P75 the results are shown in table
\ref{tab:pppoe-type-load} \footnote{Values
  were taken from a posting on spline.fli4l and have not been evaluated
  further. Message ID of the article:
  $<$caf9fk\$ala\$1@bla.spline.inf.fu-berlin.de$>$.}.
\begin{table}[h!]
  \centering
  \begin{tabular}[h]{|l|l|l|l|}
    \hline
    fli4l & NIC & Bandwidth (downstream) & CPU-load \\
    \hline
    2.0.8 & rtl8029 + rtl8139 & 310 kB/s  &100\% \\
    2.0.8 & 2x 3Com Etherlink III & 305 kB/s & 100\% \\
    2.0.8 & SMC + 3Com Etherlink III & 300 kB/s & 100\% \\
    2.1.7 & SMC + 3Com Etherlink III & 375 kB/s & 40\%\\
    \hline
  \end{tabular}
  \caption{Bandwidth und CPU-load with pppoe}
  \label{tab:pppoe-type-load}
\end{table}

\config{PPPOE\_HUP\_TIMEOUT}{PPPOE\_HUP\_TIMEOUT}{PPPOEHUPTIMEOUT2}

Using PPPoE-type in\_kernel and dialmode auto, timeout can be set 
to 'never'. The router then no longer hangs up and after a forced 
disconnection by the provider dials again automatically. Subsequent
changing of the dialmode is not possible anymore.

\end{description}

\subsection {OPT\_FRITZDSL - DSL via Fritz!Card DSL}

Internet connection by Fritz! Card DSL is activated here. A 
Fritz! Card DSL by AVM is used for the internet connection. 
Since the drivers for these cards are not subject to the GPL
it is not possible to provide them with the DSL package. It is
essential to download these drivers before from 
\altlink{http://www.fli4l.de/download/stabile-version/avm-treiber/} 
and to extract them into the fli4l directory.\\
      

Circuit-support for Fritz!Card DSL was realised 
with friendly help from Stefan Uterhardt
(\email{zer0@onlinehome.de}).


\begin{description}
  
  \config{OPT\_FRITZDSL}{OPT\_FRITZDSL}{OPTFRITZDSL}
  \sloppypar{activates support for Fritz!DSL.
    Default setting: \verb*?OPT_FRITZDSL='no'?}.

\config{FRITZDSL\_TYPE}{FRITZDSL\_TYPE}{FRITZDSLTYPE}

Several Fritz!-cards exist for a DSL connection. 
The card used will be specified by
\var{FRITZDSL\_TYPE}, have a look at table
\ref{tab:fritz-karten} for enumerating the supported cards.

\begin{table}[htb]
  \centering
  \begin{tabular}{l|l}
    Card Type & Usage \\
    \hline
    fcdsl & Fritz!Card DSL \\
    fcdsl2 & Fritz!Card DSLv2\\
    fcdslsl & Fritz!Card DSL SL\\
    fcdslusb & Fritz!Card DSL USB\\
    fcdslslusb & Fritz!Card DSL SL USB\\
    fcdslusb2 & Fritz!Card DSL USBv2
  \end{tabular}
  \caption{Fritz-Cards}
  \label{tab:fritz-karten}
\end{table}

        Default setting:

\begin{example}
\begin{verbatim}
        FRITZDSL_TYPE='fcdsl'
\end{verbatim}
\end{example}

\config{FRITZDSL\_PROVIDER}{FRITZDSL\_PROVIDER}{FRITZDSLPROVIDER}

With this option the type of the remote station is set. Possible options are:\\
U-R2, ECI, Siemens, Netcologne, oldArcor, Switzerland, Belgium, Austria1, Austria2, Austria3, Austria4

In Germany almost always UR-2 i used. Siemens and ECI are only used with very old ports.\\
For Switzerland and Belgium, the options are self-explanatory and in Austria you 
have to try what works for you.\\
If anyone has a better description for the options in Austria please tell.

        Default setting:

\begin{example}
\begin{verbatim}
        FRITZDSL_PROVIDER='U-R2'
\end{verbatim}
\end{example}

\end{description}

\subsection {OPT\_PPTP - DSL over PPTP in Austria/the Netherlands}

In Austria (and other european countries) PPTP-protocol is used 
instead of PPPoE. A separate ethernet card is connected to a PPTP-Modem 
in this case.

As of version 2.0 connecting via PPTP is realised as a Circuit - with 
the friendly help of Rudolf Hämmerle
(\email{rudolf.haemmerle@aon.at}).

PPTP uses two cards. The first card should be used for connecting the 
LAN and the second for connecting to the DSL-modem.

Only the first card can have an IP-address.

This means:

\begin{example}
\begin{verbatim}
        IP_NET_N='1'                   # Only *one* card with IP-address!
        IP_NET_1xxx='...'              # the usual parameters
\end{verbatim}
\end{example}

\var{PPTP\_\-ETH} is set to 'eth1' for the second ethernet card and
*no* \var{IP\_\-NET\_\-2-xxx}-variables are defined.

\begin{description}
 
  \config{OPT\_PPTP}{OPT\_PPTP}{OPTPPTP} \sloppypar{activates 
    support for PPTP. Default setting:
    \verb*?OPT_PPTP='no'?.}

\config{PPTP\_ETH}{PPTP\_ETH}{PPTPETH}

Name of the ethernet interfaces

\begin{tabular}[h!]{ll}
  'eth0' & first ethernet card\\
  'eth1' & second ethernet card\\
  ...   &  ...\\
\end{tabular}
        
Default setting: \verb*?PPTP_ETH='eth1'?

\config{PPTP\_MODEM\_TYPE}{PPTP\_MODEM\_TYPE}{PPTPMODEMTYPE}

There are several PPTP modem types to realise a pptp connection. 
Define \verb*?PPTP_MODEM_TYPE? to define the modem type. Possible 
types are shown in table \ref{tab:pptp-modemtypen}.

\begin{table}[htb]
  \centering
  \begin{tabular}{l|l}
    Modem Type & Usage\\
    \hline
    telekom & Austria (Telekom Austria) \\
    xdsl & Austria (Inode xDSL@home) \\
    mxstream & the Netherlands, Danmark
  \end{tabular}
  \caption{PPTP Modem Types}
  \label{tab:pptp-modemtypen}
\end{table}

        Default setting:

\begin{example}
\begin{verbatim}
        PPTP_MODEM_TYPE='telekom'
\end{verbatim}
\end{example}

\end{description}

\subsubsection{Inode xDSL@home}

Support for Inode xDSL@home was implemented using information 
found on Inode's support pages\footnote{See
  \altlink{http://www6.inode.at/support/internetzugang/xdsl_home/konfiguration_ethernet_linux.html}}.

At the moment there are sometimes problems with renewing of leases for
the interface (the IP for the interface is delivered by dhcp and
has to be renewed in regular periods). Hanging up and reconnecting 
via imonc does not work well by now. Help by providing patches 
or as a tester is highly appreciated.

With xdsl two further options exist for pptp:

\begin{description}
\config{PPTP\_CLIENT\_REORDER\_TO}{PPTP\_CLIENT\_REORDER\_TO}{PPTPCLIENTREORDERTO}

The pptp-client which is used for xdsl under certain circumstances must 
rearrange and buffer packets. It usually waits 0.3 seconds for a packet. 
By setting this variable you can vary the timeout between 0.00 (no buffer) 
and 10.00. Times always must be provided with two decimales.

\config{PPTP\_CLIENT\_LOGLEVEL}{PPTP\_CLIENT\_LOGLEVEL}{PPTPCLIENTLOGLEVEL}

Here you can define how much debug output will be produced by the 
pptp-client. Values are 0 (little), 1 (default) und 2 (much).

\end{description}

\subsection {OPT\_POESTATUS - PPPoE-Status-Monitor On fli4l-Console}
\configlabel{OPT\_POESTATUS}{OPTPOESTATUS}

    PPPoE-Status-Monitor for DSL Connections was developed by Thorsten Pohlmann.

    With setting \verb*?OPT_POESTATUS='yes'? dsl status can be watched 
    on the third fli4l console at any time. Switch to the third console 
    by pressing ALT-F3 and back to the first console with ALT-F1.
