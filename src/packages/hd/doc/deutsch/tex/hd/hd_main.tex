% Last Update: $Id: hd_main.tex 38653 2015-04-24 23:09:16Z kristov $
\marklabel{sec:hdinstall}{
  \section{HD - Unterstützung von Festplatten, Flash-Karten, USB-Sticks usw.}
  }

\subsection {OPT\_HDINSTALL - Installation auf Festplatte/CompactFlash}
\configlabel{OPT\_HDINSTALL}{OPTHDINSTALL}

    fli4l unterstützt eine Vielzahl an Bootmedien (CD, HD, Netzwerk, 
    Compact-Flash,...). Die Diskette zählt aus Platzgründen ab Version 
    4.0 nicht mehr dazu.\\

    Im Folgenden werden die notwendigen Schritte zur Installation auf
    einer Festplatte erklärt.\\ 

    Der übliche Weg ist die Installation mit einem Bootmedium, es kann aber auch 
    über Netzwerk-Boot installiert werden. Das \var{OPT\_\-HDINSTALL} bereitet  
    die Festplatte vor. Ist beim Erstellen des Bootmediums sowohl dort  
    als auch beim Ziel der Installation der gewählte \verb*?BOOT_TYPE='hd'? 
    werden die Installationsdateien direkt übertragen. Sollte ein direktes 
    Kopieren nicht möglich sein werden diese später über scp oder über ein 
    Remote-Update per Imonc übertragen.
    
    Eine Einführung in die verschiedenen Festplatten-
    Installationsvarianten A oder B befindet sich am \jump{VARIANTEN}{Anfang
    der fli4l-Dokumentation}. Bitte unbedingt vorher lesen!

    \subsubsection{HD-Installation in sechs einfachen Schritten}

\begin{enumerate}
    \item lauffähiges fli4l-Bootmedium mit dem Paket base
    sowie \var{OPT\_\-HDINSTALL} erstellen. Zusätzlich muss dieses Bootmedium  
    ein Remote-Update ermöglichen. Es muss also entweder \var{OPT\_\-SSHD}
    aktiv sein oder \var{OPT\_\-IMOND} auf \verb*?'yes'? stehen. Wenn zum Ansprechen des 
    Datenträgers Treiber erforderlich sind, die in der Standardinstallation 
    nicht enthalten sind, müssen diese zusätzlich über \var{OPT\_\-HDDRV} aktiviert 
    werden.
    \item den Router mit diesem Bootmedium booten.
    \item am Router einloggen und den Befehl ``hdinstall.sh'' ausführen.
    \item wenn die Aufforderung dazu erscheint die Dateien syslinux.cfg, kernel, rootfs.img,
    opt.img und rc.cfg mittels scp oder Imonc auf den Router nach /boot kopieren.
    Es wird empfohlen, dazu mit zwei fli4l-Verzeichnissen zu arbeiten, eines für das Setup 
    und ein zweites für die spätere HD-Version. Bei der HD-Version stellen Sie die Variable
    \verb*?BOOT_TYPE='hd'? ein und beim Bootmedium dessen Typ entsprechend.
    
    \achtung{Beim Remote-Update müssen natürlich die Dateien der HD-Version auf den
    Router übertragen werden!}
    \item Bootmedium entfernen, Router herunterfahren und neu starten
      (unter Verwendung von halt/reboot/poweroff). Der Router bootet jetzt von der Festplatte
    \item bei Problemen den folgenden Abschnitt gut durchlesen.
\end{enumerate}


    \subsubsection{HD-Installation ausführlich erklärt (inklusive Beispielen)}

    Zuerst muss ein Router-Bootmedium erstellt werden, bei dem in der Datei
    config/hd.txt das \var{OPT\_\-HDINSTALL} mit den Installationsscripten und eventuell 
    das \var{OPT\_\-HDDRV} (falls zusätzliche Treiber benötigt werden) richtig 
    konfiguriert wurden.
    Bitte dazu auch den Abschnitt zu \var{OPT\_\-HDDRV} gründlich durchlesen!

    Die Variable \var{BOOT\_\-TYPE} in der base.txt wird entsprechend dem gewählten 
    Setup-Medium eingestellt, es soll ja schließlich ein Setup durchgeführt werden. 
    Die Variable \var{MOUNT\_\-BOOT} in der base.txt muss auf \verb*?'rw'? eingestellt werden,
    damit später ggf. neue Archive (*.img) über das Netzwerk aufgespielt werden können.

    Anschließend wird der Router von diesem Setup-Bootmedium gebootet.  
    Durch Eingabe von ``hdinstall.sh'' an der fli4l-Console wird dann
    das Installationsprogramm gestartet. Nach Beantwortung von ein paar Fragen 
    wird auf die Festplatte installiert. Eventuell erscheint am Ende noch die Aufforderung, 
    dass man die für den Router benötigten Dateien per Remote-Update aufspielen soll.

    \achtung{Dieses Remote-Update keinesfalls vergessen, der Router bootet sonst 
    nicht von der Festplatte. Zum Neustarten des Routers nach dem
    Remote-Update unbedingt reboot/halt/poweroff verwenden,
    andernfalls können die beim Remote-Update vorgenommenen Änderungen verloren gehen.}
    
    Das Installationsscript kann sowohl direkt am Router als auch über ssh von einem
    anderen PC aus gestartet werden. Im jedem Fall muss man sich vorher 
    durch Eingabe des Passwortes am Router anmelden. Als ssh-Client für Windows-Rechner
    kann z.B. die Freeware Putty verwendet werden.


\subsubsection{Konfiguration des Setup-Bootmediums}

\begin{table}[htb]
  \begin{center}
    \begin{small}
    \begin{tabular}[h!]{lp{9cm}}
    \verb*?BOOT_TYPE? & entsprechend dem Bootmedium für die Installation einstellen\\
    \verb*?MOUNT_BOOT='rw'? & notwendig, um später neue Archive (*.img) über Netzwerk
        auf die Platte kopieren zu können\\
    \verb*?OPT_HDINSTALL='yes'? & notwendig um das Setup-Skript und die Tools zum 
        Formatieren der Partitionen auf dem Bootmedium zu haben\\
    (\verb*?OPT_HDDRV='yes'?) & nur dann notwendig, wenn ohne spezielle Treiber nicht auf die
        Festplatte zugegriffen werden kann\\
    \verb*?OPT_SSHD='yes'? & nach dem Vorbereiten der Festplatte werden
        eventuell noch Dateien per remote Update übertragen. Dazu
        benötigt man entweder den sshd, imond (\verb*?IMOND='yes'?) oder ein
        anderes Paket, das einen Filetransfer erlaubt. 
    \end{tabular}
    \end{small}
    \caption{Beispiel für die Konfiguration des Setup-Mediums}
    \marklabel{tab:hd-setupdisk}{}
  \end{center}
\end{table}

    Bereits hier muss die Netzwerkkonfiguration richtig eingestellt sein damit man später 
    noch Dateien über das Netzwerk aufspielen kann. Es wird empfohlen, DNS\_DHCP zu diesem 
    Zeitpunkt noch nicht zu aktivieren, da dies regelmäßig zu Problemen führt (der DHCP-Server 
    hat vielleicht noch eine lease für den zu installierenden Router). Für ein Remote-Update mittels scp 
    (befindet sich im Paket SSHD) bitte \verb*?OPT_SSHD='yes'? einstellen. Alternativ dazu 
    kann man die Dateien per IMOND übertragen, dafür wird zusätzlich allerdings eine gültige 
    DSL oder ISDN-Konfiguration benötigt.
    Alle nicht unbedingt nötigen Pakete bitte weglassen, also kein DNS\_DHCP, SAMBA\_LPD, LCD,
    Portforwarding usw.\\

    Falls die Installation mit der Fehlermeldung
\begin{example}
\begin{verbatim}
    *** ERROR: can't create new partition table, see docu ***
\end{verbatim}
\end{example}
       abbricht, können mehrere Fehlerquellen in Frage kommen:
        \begin{itemize}
        \item die Festplatte ist in Benutzung, evtl. durch einen abgebrochenen
        Installationsversuch. Einfach neu booten und noch einmal versuchen.
        \item es werden zusätzliche Treiber benötigt, siehe \verb*?OPT_HDDRV?
        \item es gibt Hardwareprobleme, mehr dazu bitte im Anhang nachlesen.
        \end{itemize}
    Im letzten Schritt kann man nun die endgültige Fassung der
    Konfigurationsdateien erstellen und alle gewünschten Pakete hinzufügen.

\subsubsection{Beispiele für eine fertige Installation nach Typ A und Typ B:}

Ein Beispiel für jede Konfiguration finden Sie in Tabelle
\ref{tab:hd-typ-a}.

\begin{table}[htb]
  \begin{center}
    \begin{small}
    \begin{tabular}{lp{9cm}}
    \verb*?BOOT_TYPE='hd'? & notwendig, da sie ja jetzt von Festplatte starten\\
    \verb*?MOUNT_BOOT='rw|ro|no'? &
                        nach Wahl. Um später neue fli4l-Archive über Netzwerk
                        auf die Platte kopieren zu können ist 'rw' nötig.\\

    \verb*?OPT_HDINSTALL='no'? & nach der erfolgreichen Installation ist dieses Paket
                        nicht mehr notwendig.\\


    \verb*?OPT_MOUNT? & nur aktivieren, falls eine Datenpartition erstellt wurde\\

    (\verb*?OPT_HDDRV='yes'?) & nur notwendig, wenn ohne zusätzliche Treiber nicht auf die
        Festplatte zugegriffen werden kann.\\
    
    \end{tabular}
    \end{small}
    \caption{Beispiel für eine Installation nach Typ A oder B}
    \marklabel{tab:hd-typ-a}{}
  \end{center}
\end{table}

    Das Erstellen einer Swap-Partition wird nur angeboten, falls weniger als 32MB 
    RAM im Router stecken und die Installation NICHT auf ein Flash-Medium durchgeführt wird!

\subsection {OPT\_MOUNT - Automatisches Einhängen von Dateisystemen}
\configlabel{OPT\_MOUNT}{OPTMOUNT}

    OPT\_MOUNT hängt eine bei der Installation erstellte Datenpartition nach /data ein, 
    eine Prüfung der Partition auf Fehler wird bei Bedarf automatisch durchgeführt. 
    Ein evtl. vorhandenes CD-ROM wird nach /cdrom eingehängt, falls eine CD eingelegt ist.
    Für die swap-Partition wird das OPT\_MOUNT nicht mehr benötigt!

    \achtung{OPT\_MOUNT liest die Konfigurationsdatei hd.cfg auf der Boot-Partition 
    und hängt die dort angegebenen Partitionen ein. 
    Wenn das \var{OPT\_\-MOUNT} mit einem Remote-Update auf einen bereits installierten
    Router übertragen wurde, muss diese Konfigurationsdatei ggf. geändert werden.}
   
    \achtung{Auch bei einem Boot von CD-ROM kann das OPT\_MOUNT nicht genutzt werden. 
    Die CD kann in diesem Fall mit \var{MOUNT\_\-BOOT='ro'} eingehängt werden.}
    
    Die Datei hd.cfg auf der DOS-Partition hat für einen Router nach Typ B mit Swap und 
    Datenpartition den folgenden Inhalt (Beispiel):
    \begin{verbatim}
        hd_boot='sda1'
        hd_opt='sda2'
        hd_swap='sda3'
        hd_data='sda4'
        hd_boot_uuid='4A32-0C15'
        hd_opt_uuid='c1e2bfa4-3841-4d25-ae0d-f8e40a84534d'
        hd_swap_uuid='5f75874c-a82a-6294-c695-d301c3902844'
        hd_data_uuid='278a5d12-651b-41ad-a8e7-97ccbc00e38f'
    \end{verbatim}
    
    Nicht existierende Partitionen werden einfach weggelassen, bei einem Router Typ A 
    ohne weitere Partitionen sieht das also so aus:
    \begin{verbatim}
        hd_boot='sda1'
        hd_boot_uuid='4863-65EF'
     \end{verbatim}

\subsection {OPT\_EXTMOUNT - Manuelles Einhängen von Dateisystemen}
\configlabel{OPT\_EXTMOUNT}{OPTEXTMOUNT}

    OPT\_EXTMOUNT hängt Datenpartitionen an jedem beliebigen Mountpoint
    im Dateisystem ein. Damit ist es möglich von Hand erstellte Dateisysteme einzuhängen
    und beispielsweise für einen Rsync-Server zur Verfügung zu stellen.

\begin{description}
\config{EXTMOUNT\_N}{EXTMOUNT\_N}{EXTMOUNTN}

    Die Anzahl der Datenpartitionen die extra eingehängt werden sollen.

\config{EXTMOUNT\_x\_VOLUMEID}{EXTMOUNT\_x\_VOLUMEID}{EXTMOUNTxVOLUMEID}

    Device, Label oder UUID des Volumens, das eingehängt werden soll.
    Mit dem Befehl 'blkid' kann man sich Device, Label und UUID aller
    verfügbaren Volumen anzeigen lassen.

\config{EXTMOUNT\_x\_FILESYSTEM}{EXTMOUNT\_x\_FILESYSTEM}{EXTMOUNTxFILESYSTEM}

    Das verwendete Dateisystem der Partition. fli4l unterstützt zur Zeit 
    die Dateisysteme isofs, fat, vfat, ext2, ext3 und ext4.\\
    (Der Standardwert \verb*?EXTMOUNT_x_FILESYSTEM='auto'? versucht das
    verwendete Dateisystem automatisch festzustellen.)

\config{EXTMOUNT\_x\_MOUNTPOINT}{EXTMOUNT\_x\_MOUNTPOINT}{EXTMOUNTxMOUNTPOINT}

    Der Pfad (Mountpoint) im Dateisystem in dem das Device eingehängt wird. Der Pfad
    muss vorher nicht existieren, er wird automatisch erzeugt.

\config{EXTMOUNT\_x\_OPTIONS}{EXTMOUNT\_x\_OPTIONS}{EXTMOUNTxOPTIONS}

    Wenn spezielle Optionen an den 'mount' Aufruf übergeben werden sollen
    können diese hier angegeben werden.

\config{EXTMOUNT\_x\_HOTPLUG}{EXTMOUNT\_x\_HOTPLUG}{EXTMOUNTxHOTPLUG}

    Wenn diese Variable den Wert `yes' enthält, ist es kein Fehler, wenn zur
    Boot-Zeit die Datenpartition nicht existiert. In diesem Fall wird davon
    ausgegangen, dass der zugehörige Datenträger fehlt und ggf. später
    eingebunden wird (z.B. via SATA-Hotplugging oder als USB-Stick). Das
    Aktivieren dieser Option erfordert zwingend \var{OPT\_AUTOMOUNT='yes'}. Des
    Weiteren muss zur Identifikation der gewünschten Datenpartition die
    eindeutige Kennung (UUID) des Dateisystems in \var{EXTMOUNT\_x\_VOLUMEID}
    eingetragen werden; andere IDs wie Gerätename oder Label werden \emph{nicht}
    unterstützt.

\end{description}

Beispiel:
\begin{example}
\begin{verbatim}
       EXTMOUNT_1_VOLUMEID='sda2'        # device
       EXTMOUNT_1_FILESYSTEM='ext3'      # filesystem
       EXTMOUNT_1_MOUNTPOINT='/mnt/data' # mountpoint for device
       EXTMOUNT_1_OPTIONS=''             # extra mount options passed via mount -o
       EXTMOUNT_1_HOTPLUG='no'           # device must exist at boot time
\end{verbatim}
\end{example}


\subsection {OPT\_AUTOMOUNT -- automatisches Einhängen von Datenpartitionen}
\configlabel{OPT\_AUTOMOUNT}{OPTAUTOMOUNT}

    \var{OPT\_AUTOMOUNT='yes'} erlaubt es, Datenpartitionen automatisch und
    dynamisch während der Laufzeit einzuhängen. Es gibt zwei
    Konfigurationsvarianten. Die erste arbeitet mit \var{OPT\_EXTMOUNT}
    zusammen und hängt nur Datenpartitionen ein, die beim Booten gefehlt haben.
    Die zweite ist unabhängig von \var{OPT\_EXTMOUNT} und hängt \emph{alle}
    lesbaren Datenpartitionen ein, egal ob bereits während des Bootens oder
    erst später. Steuern lässt sich das Verhalten mit Hilfe der Variablen
    \var{AUTOMOUNT\_UNKNOWN}:

\begin{description}
\config{AUTOMOUNT\_UNKNOWN}{AUTOMOUNT\_UNKNOWN}{AUTOMOUNTUNKNOWN}

    Diese Variable steuert, ob unbekannte Datenpartitionen eingehängt werden.
    Mit \var{AUTOMOUNT\_UNKNOWN='no'} werden nur Datenpartitionen dynamisch
    während der Laufzeit eingehängt, die einem \var{EXTMOUNT\_x}-Eintrag
    entsprechen. Dazu muss zusätzlich \var{EXTMOUNT\_x\_HOTPLUG='yes'}
    definiert sein, damit \var{OPT\_EXTMOUNT} nicht meckert, wenn die
    Datenpartition beim Booten fehlen sollte. Mit \var{AUTOMOUNT\_UNKNOWN='yes'}
    werden auch unbekannte Datenpartitionen eingehängt. Dies funktioniert aber
    nur, wenn das Dateisystem auf der Datenpartition eine eindeutige Kennung
    (UUID) besitzt. In diesem Fall wird die Datenpartition in dem Verzeichnis
    \texttt{/media/<UUID>} eingehängt (dieses Verzeichnis wird bei Bedarf
    erzeugt).

    Standard-Einstellung: \var{AUTOMOUNT\_UNKNOWN='no'}

\config{AUTOMOUNT\_UNKNOWN\_OPTS}{AUTOMOUNT\_UNKNOWN\_OPTS}{AUTOMOUNTUNKNOWNOPTS}

    Diese Variable gibt die \texttt{mount}-Optionen an, die bei unbekannten
    Datenpartitionen beim Einhängen verwendet werden. Ist die Datenpartition
    über \var{OPT\_EXTMOUNT} in der \texttt{/etc/fstab} identifizierbar, dann
    werden die hier angegebenen Optionen \emph{nicht} benutzt; vielmehr werden
    die Optionen im passenden \var{EXTMOUNT\_x\_OPTIONS}-Eintrag genutzt.

    Standard-Einstellung: \var{AUTOMOUNT\_UNKNOWN\_OPTS='ro'} (damit werden
    Schreibzugriffe auf unbekannte Datenpartitionen standardmäßig verhindert)

\end{description}

    Jede Datenpartition wird vor dem Einhängen mit Hilfe des für das jeweilige
    Dateisystem verfügbaren Prüfprogramms auf Fehler überprüft (\texttt{e2fsck}
    für ext2/ext3/ext4-Dateisysteme und \texttt{fsck.fat} für
    (V)FAT-Dateisysteme). Schlägt die Prüfung oder die automatische Korrektur
    fehl, wird das Dateisystem \emph{nicht} eingehängt, um Datenkorruption zu
    vermeiden.

    Wird ein Gerät entfernt, auf dem ein Dateisystem eingehängt war, wird dies
    nachträglich via \texttt{umount} ausgehängt. Natürlich können dabei
    womöglich noch nicht geschriebene Daten nicht mehr gesichert werden
    (schließlich ist der Datenträger nicht mehr da), aber immerhin kann nicht
    mehr versucht werden, auf den nicht mehr vorhandenen Datenträger weiter
    zuzugreifen. Die korrekte Vorgehensweise ist natürlich \emph{erst} das
    Dateisystem auszuhängen und \emph{dann} den Datenträger zu entfernen. Weil
    nicht alle Gerätetypen ein Entfernen verhindern, wenn das Dateisystem
    eingehängt ist (beispielsweise funktioniert dies gut bei CD-Laufwerken),
    muss man sich unter Umständen selbst um die korrekte Reihenfolge der+
    Aktionen kümmern.

    Alle Aktivitäten von \var{OPT\_AUTOMOUNT} werden in der Datei
    \texttt{/var/log/automount.log} protokolliert. Ein beispielhafter
    Ausschnitt einer solchen Log-Datei wird im Folgenden gezeigt. Zuerst kommt
    der Abschnitt, der die Aktivitäten für Datenpartitionen aufzeigt, die
    bereits während des Bootens verfügbar sind (\texttt{ACTION=change}):

\begin{tiny}
\begin{verbatim}
[2015-04-25 00:33:35] [INFO   ] ACTION=change SUBSYSTEM=block DEVNAME=vda1 DEVPATH=/devices/pci0000:00/0000:00:08.0/virtio4/block/vda/vda1 MDEV=vda1
[2015-04-25 00:33:35] [INFO   ] TYPE: vfat
[2015-04-25 00:33:35] [INFO   ] UUID: 442e-93ba
[2015-04-25 00:33:35] [INFO   ] mount point: /media/442e-93ba
[2015-04-25 00:33:35] [ERROR  ] /dev/vda1 already mounted on /boot, giving up
[2015-04-25 00:33:35] [INFO   ] ACTION=change SUBSYSTEM=block DEVNAME=vda2 DEVPATH=/devices/pci0000:00/0000:00:08.0/virtio4/block/vda/vda2 MDEV=vda2
[2015-04-25 00:33:35] [INFO   ] TYPE: ext3
[2015-04-25 00:33:35] [INFO   ] UUID: 77ab35b3-029e-42c9-93a0-d197c01e6e89
[2015-04-25 00:33:35] [INFO   ] mount point: /media/77ab35b3-029e-42c9-93a0-d197c01e6e89
[2015-04-25 00:33:35] [INFO   ] /dev/vda2: clean, 671/26208 files, 57544/104420 blocks
[2015-04-25 00:33:35] [NOTICE ] /dev/vda2 mounted on /media/77ab35b3-029e-42c9-93a0-d197c01e6e89
[2015-04-25 00:33:36] [INFO   ] ACTION=change SUBSYSTEM=block DEVNAME=vda3 DEVPATH=/devices/pci0000:00/0000:00:08.0/virtio4/block/vda/vda3 MDEV=vda3
[2015-04-25 00:33:36] [INFO   ] TYPE: ext3
[2015-04-25 00:33:35] [INFO   ] UUID: 1580b80c-92b1-4492-abfa-92a12a7d2027
[2015-04-25 00:33:35] [INFO   ] mount point: /media/1580b80c-92b1-4492-abfa-92a12a7d2027
[2015-04-25 00:33:35] [ERROR  ] /dev/vda3 already mounted on /data, giving up
[2015-04-25 00:33:35] [INFO   ] ACTION=change SUBSYSTEM=block DEVNAME=vdb1 DEVPATH=/devices/pci0000:00/0000:00:0a.0/virtio5/block/vdb/vdb1 MDEV=vdb1
[2015-04-25 00:33:35] [INFO   ] TYPE: ext3
[2015-04-25 00:33:35] [INFO   ] UUID: 4c1a03e1-3a0c-4835-88dc-a51879def464
[2015-04-25 00:33:35] [INFO   ] mount point: /mnt/extra
[2015-04-25 00:33:35] [ERROR  ] /dev/vdb1 already mounted on /mnt/extra, giving up
[2015-04-25 00:33:35] [INFO   ] ACTION=change SUBSYSTEM=block DEVNAME=vdc1 DEVPATH=/devices/pci0000:00/0000:00:1f.0/virtio6/block/vdc/vdc1 MDEV=vdc1
[2015-04-25 00:33:35] [INFO   ] TYPE: vfat
[2015-04-25 00:33:35] [INFO   ] UUID: ba6e-9ebd
[2015-04-25 00:33:35] [INFO   ] mount point: /media/ba6e-9ebd
[2015-04-25 00:33:35] [INFO   ] fsck.fat 3.0.26 (2014-03-07)
[2015-04-25 00:33:35] [INFO   ] /dev/vdc1: 0 files, 0/32672 clusters
[2015-04-25 00:33:35] [NOTICE ] /dev/vdc1 mounted on /media/ba6e-9ebd
\end{verbatim}
\end{tiny}

    Zwei Datenpartitionen wurden eingehängt (\texttt{/dev/vda2} und
    \texttt{/dev/vdc1}), davon wurden beide nicht via \var{OPT\_EXTMOUNT}
    konfiguriert und somit unterhalb von \texttt{/media} eingehängt. Die
    verbliebenen drei Datenpartitionen \texttt{/dev/vda1}, \texttt{/dev/vda3}
    und \texttt{/dev/vdb1} wurden bereits von anderen Boot-Skripten eingehängt
    und entsprechen der Boot- und der Datenpartition sowie einer
    benutzerdefinierten \var{OPT\_EXTMOUNT}-Datenpartition.

    Jetzt werden \texttt{/dev/vdb1} und \texttt{/dev/vdc1} ausgehängt
    (\texttt{ACTION=remove}; die Warnung, dass \texttt{/dev/vdb1} beim
    Aushängen nicht in der Volumen-Datenbank gefunden wurde, ist harmlos und
    weist darauf hin, dass diese Datenpartition bereits während des Bootens
    von \var{OPT\_EXTMOUNT} und nicht von \var{OPT\_AUTOMOUNT} eingehängt
    wurde)...

\begin{tiny}
\begin{verbatim}
[2015-04-25 00:34:52] [INFO   ] ACTION=remove SUBSYSTEM=block DEVNAME=vdb1 DEVPATH=/devices/pci0000:00/0000:00:0a.0/virtio5/block/vdb/vdb1 MDEV=vdb1
[2015-04-25 00:34:52] [WARNING] /dev/vdb1 not found in volume database
[2015-04-25 00:34:52] [INFO   ] mount point: /mnt/extra
[2015-04-25 00:34:52] [NOTICE ] /dev/vdb1 unmounted from /mnt/extra
[2015-04-25 00:34:55] [INFO   ] ACTION=remove SUBSYSTEM=block DEVNAME=vdc1 DEVPATH=/devices/pci0000:00/0000:00:1f.0/virtio6/block/vdc/vdc1 MDEV=vdc1
[2015-04-25 00:34:55] [INFO   ] UUID: ba6e-9ebd
[2015-04-25 00:34:55] [INFO   ] mount point: /media/ba6e-9ebd
[2015-04-25 00:34:55] [NOTICE ] /dev/vdc1 unmounted from /media/ba6e-9ebd
\end{verbatim}
\end{tiny}

    ...und in umgekehrter Reihenfolge wieder eingehängt (\texttt{ACTION=add}):

\begin{tiny}
\begin{verbatim}
[2015-04-25 00:35:14] [INFO   ] ACTION=add SUBSYSTEM=block DEVNAME=vdb1 DEVPATH=/devices/pci0000:00/0000:00:0b.0/virtio5/block/vdb/vdb1 MDEV=vdb1
[2015-04-25 00:35:14] [INFO   ] TYPE: vfat
[2015-04-25 00:35:14] [INFO   ] UUID: ba6e-9ebd
[2015-04-25 00:35:14] [INFO   ] mount point: /media/ba6e-9ebd
[2015-04-25 00:35:15] [INFO   ] fsck.fat 3.0.26 (2014-03-07)
[2015-04-25 00:35:15] [INFO   ] /dev/vdb1: 0 files, 0/32672 clusters
[2015-04-25 00:35:15] [NOTICE ] /dev/vdb1 mounted on /media/ba6e-9ebd
[2015-04-25 00:35:18] [INFO   ] ACTION=add SUBSYSTEM=block DEVNAME=vdc1 DEVPATH=/devices/pci0000:00/0000:00:0c.0/virtio6/block/vdc/vdc1 MDEV=vdc1
[2015-04-25 00:35:18] [INFO   ] TYPE: ext3
[2015-04-25 00:35:18] [INFO   ] UUID: 4c1a03e1-3a0c-4835-88dc-a51879def464
[2015-04-25 00:35:18] [INFO   ] mount point: /mnt/extra
[2015-04-25 00:35:18] [INFO   ] /dev/vdc1: recovering journal
[2015-04-25 00:35:18] [INFO   ] /dev/vdc1: clean, 11/16384 files, 7477/65488 blocks
[2015-04-25 00:35:18] [NOTICE ] /dev/vdc1 mounted on /mnt/extra
\end{verbatim}
\end{tiny}

    Das ``unsaubere'' Aushängen des ext3-Dateisystems auf \texttt{/dev/vdc1}
    hat zu einer ``recovering journal''-Meldung beim Einhängen geführt, die
    aber nicht kritisch ist, da keine weiteren Fehler gefunden wurden.

\subsection {OPT\_HDSLEEP -- automatisches Abschalten für Festplatten einstellen}
\configlabel{OPT\_HDSLEEP}{OPTHDSLEEP}

    Eine Festplatte kann sich automatisch abschalten, wenn eine bestimmte
    Zeit ohne Aktivität verstreicht. Damit benötigt die Platte kaum noch Strom
    und macht keine Geräusche mehr. Wenn ein Zugriff auf die Festplatte
    erfolgt, läuft sie automatisch wieder an.

    \achtung{Nicht alle Festplatten vertragen häufiges Wiederanlaufen. Daher
    sollte man die Zeit nicht zu klein wählen. Ältere IDE-Platten bieten diese 
    Funktion erst gar nicht an. Bei Flash-Medien ist diese Einstellung nicht 
    sinnvoll und auch nicht notwendig.}

\begin{description}
\config{HDSLEEP\_TIMEOUT}{HDSLEEP\_TIMEOUT}{HDSLEEPTIMEOUT}

        Diese Variable legt fest, nach welcher Zeit ohne Zugriff die 
        Festplatte in den Power-Down-Modus gehen soll. Dann schaltet sie sich 
        automatisch nach der Wartezeit aus und beim nächsten Zugriff wieder ein.
        Hierbei sind Wartezeiten in Minutenabständen von einer bis 20 Minuten 
        sowie in Abständen von 30 Minuten von einer Halben bis zu fünf
        Stunden möglich. Eine Wartezeit von 21 oder 25 Minuten z.B. wird also 
        auf 30 Minuten aufgerundet.
        Manche Festplatten ignorieren zu hohe Werte und stoppen dann schon 
        nach einigen Minuten. Bitte unbedingt die korrekte Funktion durchtesten,
        da dies sehr von der jeweiligen Hardware abhängig ist!

\begin{example}
\begin{verbatim}
        HDSLEEP_TIMEOUT='2'              # wait 2 minutes until power down
\end{verbatim}
\end{example}

\end{description}

\subsection {OPT\_RECOVER -- Notfalloption}
\configlabel{OPT\_RECOVER}{OPTRECOVER}

        Diese Variable legt fest, ob Funktionen zur Erstellung
        einer Notfalloption verfügbar sind.
        Wenn die Option aktiviert ist wird der Befehl ``mkrecover.sh'' mit auf den
        Router übertragen. Mit diesem kann an der Kommandokonsole durch einfachen
        Aufruf die Notfallinstallation aktiviert werden. Beim installierten Paket ``HTTPD''
        kann die Übertragung der aktuell laufenden Installation in eine Notfallinstallation
        im Menü Recover durchgeführt werden.

        Um die Notfallinstallation zu nutzen, ist beim nächsten Reboot im Bootmenü die Auswahl
        Recover auszuwählen.

\begin{example}
\begin{verbatim}
        OPT_RECOVER='yes'
\end{verbatim}
\end{example}

\subsection {OPT\_HDDRV - Treiber für Festplattencontroller}
\configlabel{OPT\_HDDRV}{OPTHDDRV}

    Mit \var{OPT\_\-HDDRV}='yes' können eventuell benötigte zusätzliche Treiber
    aktiviert und installiert werden.
    Für IDE und SATA ist es in der Regel nicht nötig einen speziellen Treiber zu laden,
    da diese bereits vom Paket Base geladen werden.

\begin{description}
\config{HDDRV\_N}{HDDRV\_N}{HDDRVN}
{
	Die Anzahl der Treiber, die geladen werden sollen, wird hier eingestellt.
}

\config{HDDRV\_x}{HDDRV\_x}{HDDRVx}
{
        Mit \var{HDDRV\_\-1} usw. werden die entsprechenden Treiber für die
        verwendeten Host-Adapter ausgewählt. Eine Liste der unterstützten
        Hostadapter ist in der initialen Konfigurationsdatei enthalten.
}

\config{HDDRV\_x\_OPTION}{HDDRV\_x\_OPTION}{HDDRVxOPTION}
{
        Mit \var{HDDRV\_\-x\_\-OPTION} können Optionen übergeben werden, die einige Treiber
        zum laden benötigen. Dies kann z.B. eine IO-Adresse sein.
        Bei den meisten Treibern kann diese Variable einfach leer gelassen werden.
}

    Im \jump{sec:hd-errors}{Anhang} finden Sie eine Übersicht der Fehler, die
    bei Festplatten und CompactFlash am häufigsten auftreten.

    Beispiel 1: Zugriff auf SCSI-Festplatte an einem Adaptec 2940

\begin{example}
\begin{verbatim}
   OPT_HDDRV='yes'             # install Drivers for Harddisk: yes or no
   HDDRV_N='1'                 # number of HD drivers
   HDDRV_1='aic7xxx'           # various aic7xxx based Adaptec SCSI 
   HDDRV_1_OPTION=''           # no need for options yet
\end{verbatim}
\end{example}

    Beispiel 2: Beschleunigter IDE-Zugriff beim PC-Engines ALIX

\begin{example}
\begin{verbatim}
   OPT_HDDRV='yes'             # install Drivers for Harddisk: yes or no
   HDDRV_N='1'                 # number of HD drivers
   HDDRV_1='pata_amd'          # AMD PCI IDE/ATA driver (e.g. ALIX) 
   HDDRV_1_OPTION=''           # no need for options yet
\end{verbatim}
\end{example}

\end{description}
