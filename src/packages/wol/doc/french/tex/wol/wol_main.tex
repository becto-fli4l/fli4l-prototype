% Do not remove the next line
% Synchronized to r45129

\marklabel{sec:opt-wol}
{
\section {WOL~- Wake On LAN}
}
Le paquetage OPT\_WOL permet d'étendre le routeur fli4l, avec la possibilité
de démarrer des ordinateurs équipés d'une carte réseau compatible Wake On LAN,
vous pouvez utliser la commande 'wol.sh' sur la console ou via l'interface web
du routeur.

Pour que cela fonctionne, la carte réseau doit généralement être relié avec
un câble de trois petits fils à la carte mère, ainsi la carte réseau peut être
alimenté par un courant de l'alimentation ATX, même lorsque l'ordinateur est
en veille.

\subsection {Configuration}

\begin{description}

\config{OPT\_WOL}{OPT\_WOL}{OPTWOL}

  Configuration par défaut~:  \var{OPT\_WOL}='no'

  La valeur \var{'no'} dans cette variable désactive complètement le paquetage
  OPT\_WOL. Il n'y aura aucun changement sur le support de boot de l'archive
  \var{opt.img} de fli4l. Pour activer le paquetage OPT\_WOL vous devez placer
  la valeur \var{'yes'} dans cette variable.

  Les clients qui seront connectés à WOL doivent être engegistrés dans le
  fichier $<$config-dir$>$/dns\_dhcp.txt et avec leur adresse MAC dans la
  variable (HOST\_x\_MAC). Tout ordinateur sans adresse MAC spécifiée sera
  automatiquement exclus de WOL.

\config{WOL\_LIST}{WOL\_LIST}{WOLLIST}

    La configuration se fait à l'aide de liste noire ou blanche. les clients
    paramétrés dans la liste noir seront exclus de WOL, les clients paramétrés
    dans la liste blanche seront inclus de WOL.

  Configuration par défaut~:  \var{WOL\_LIST}='black'

  Les valeurs dans cette variable sont les suivantes~:
  \begin{itemize}
    \item black - Tous les clients indiqué dans cette liste ne seront pas réveillé
    \item white - Tous les clients indiqué dans cette liste seront réveillé
  \end{itemize}

\config{WOL\_LIST\_N}{WOL\_LIST\_N}{WOLLISTN}

  Configuration par défaut~:  \var{WOL\_LIST\_N}='0'

  Dans la configuration par défaut zéro client ne sera dans la liste noire,
  donc chaque client peut être réveillé par WOL.

\config{WOL\_LIST\_x}{WOL\_LIST\_x}{WOLLISTx}

  Configuration par défaut~:  \var{WOL\_LIST\_x}=''

  Les valeurs dans cette variable sont les suivantes~:
  \begin{itemize}
    \item IP\_NET\_1 - Tous les clients qui fond partie de \var{IP\_NET\_x}
      peuvent être réveillés (ici IP\_NET\_1)
    \item @client1 - Le nom du client qui fait partie de (\var{HOST\_x\_NAME})
      peut être réveillé, ici 'client1'
    \item Adresse IP - L'adresse IP du client qui fait partie de (\var{HOST\_x\_IP4}
      ou \var{HOST\_x\_IP6}) peut être réveillé
  \end{itemize}

Exemple~:
\begin{example}
\begin{verbatim}
      WOL_LIST='black'            # Liste black ou white
      WOL_LIST_N='3'              # Nombre d'entré dans la liste
      WOL_LIST_1='IP_NET_1'       # Tous les Clients du réseau IP_NET_1
      WOL_LIST_2='@client1'       # Nom du client dans HOST_1_x
      WOL_LIST_3='192.168.6.3'    # Adresse IP du client
\end{verbatim}
\end{example}

\end{description}

\subsection{Wake On Lan lors du boot du routeur}
\begin{description}

\config{WOL\_BOOT}{WOL\_BOOT}{WOLBOOT}

Ce paramètre doit être réglé sur 'yes' seulement si vous voulez démarrer un
ordinateur de votre réseau avec Wake On LAN après un démarrage du routeur fli4l.
Cette configuration est indépendante de la variable \var{WOL\_LIST}, c'est à
dire, que les clients qui seront enregistrés ne figurent pas dans la liste
\var{WOL\_LIST}.

\config{WOL\_BOOT\_N}{WOL\_BOOT\_N}{WOLBOOTN}

  Configuration par défaut~:  \var{WOL\_BOOT\_N}='0'

  Dans la configuration par défaut zéro client ne sera dans la liste, donc aucun
  client ne sera réveillé avec Wake-on-LAN lors du démarrage du routeur.

\config{WOL\_BOOT\_x}{WOL\_BOOT\_x}{WOLBOOTx}

  Configuration par défaut~:  \var{WOL\_BOOT\_x}=''

  Les valeurs dans cette variable sont les suivantes~:
  \begin{itemize}
    \item IP\_NET\_1 - Tous les clients qui fond partie de \var{IP\_NET\_x}
      peuvent être réveillés (ici IP\_NET\_1)
    \item @client1 - Le nom du client qui fait partie de (\var{HOST\_x\_NAME})
      peut être réveillé, ici 'client1'
    \item Adresse IP - L'adresse IP du client qui fait partie de (\var{HOST\_x\_IP4}
      ou \var{HOST\_x\_IP6}) peut être réveillé
  \end{itemize}

Exemple~:
\begin{example}
\begin{verbatim}
      WOL_BOOT='yes'               # Activer WOL au Boot: yes or no
      WOL_BOOT_N='2'               # Nombre d'ordinateurs
      WOL_BOOT_1='@client1'        # Premier client
      WOL_BOOT_2='192.162.6.17     # Deuxième client
\end{verbatim}
\end{example}

\end{description}

\subsection{Utilisation}

Vous pouvez vous connecter avec le SSH ou avec la console du routeur pour
démarrer un client (ou ordinateur) avec les commandes suivantes~:\\
'wol.sh~$<$Nom-Client$>$' ou 'wol.sh~$<$Adresse-IP$>$' ou 'wol.sh~$<$Adresse-MAC$>$'.

Les ordinateurs qui ne sont pas enregistrés dans le fichier $<$config-dir$>$/wol.txt,
peuvent aussi être démarrés avec la commande 'etherwake~$<$Adresse-MAC$>$'

\subsection {Web GUI (ou interface Web)}

    Ce paquetage met à disposition une interface Web qui fonctionne avec mini-httpd. 
    L'interface Web sera automatiquement activée si la variable \var{OPT\_HTTPD='yes'}
	du paquetage httpd est activée.

