% Synchronized to r29817

\marklabel{sec:addcrontabentry}{
\section{EASYCRON - Adding To Crontab While Booting}
}

\wichtig{The following statements are intended only for developers
of opt-packages for fli4l routers.}

As of version 2.1.7 Easycron's rc-script provides the function 
\var{add\_crontab\_entry()}. Other rc-scripts starting from rc101 to rc949 can 
add entries to crontab by calling this function. These additional entries get 
activated with the start of the cron daemon at the end of booting.\\

\texttt{add\_crontab\_entry} \var{time command [custom]}\\

Use \var{time} to set the execution time in cron notation  
(see manpage crontab(5) at \altlink{http://linux.die.net/man/5/crontab}).
\var{command} contains the command to be executed. The third parameter
\var{custom} is optional. By using it you can set environment variables needed 
for the command used. If more than one variable has to be set separate 
them by a \var{$\backslash\backslash$}.\\
\achtung{Please do \textbf{not} change the environment variable \var{PATH} because
otherwise the following crontab entries could not be executed correctly anymore.}

\begin{verbatim}
#
# example I: normal use, 2 parameters
#
     crontime="0 5 1 * *"
     croncmd="rotate_i_log.sh"

     add_crontab_entry "$crontime" "$croncmd"

#
# end of example I
#
\end{verbatim}

\begin{verbatim}
#
# example II: extended use, 3 parameters and 
#                             multiple environment values 
#
     croncustom="source=/var/log/current \\ dest=/mnt/data/log"
     croncmd='cp $source $dest-`date +\%Y\%m\%d`; > $source'
     crontime="59 23 * * *"

     add_crontab_entry "$crontime" "$croncmd" "$croncustom"

#
# end of example II
#
\end{verbatim}

The accuracy of the entries must be ensured by the calling script.