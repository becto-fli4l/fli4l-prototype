% Last Update: $Id: easycron_appendix.tex 43052 2015-11-27 11:40:56Z cspiess $
\marklabel{sec:addcrontabentry}{
\section{EASYCRON - Crontab in der Boot-Phase ergänzen}
}

\textbf{Hinweis:} \emph{Die folgenden Ausführungen richten sich nur an Entwickler
von Opt-Paketen für den fli4l Router.}

Ab Version 2.1.7 stellt das rc-Script von Easycron die Funktion 
\var{add\_crontab\_entry()} zur Verfügung. Durch Aufruf dieser Funktion können
andere rc-Scripts ab Startposition rc101 bis Startposition 949 Einträge an die
Crontab anhängen. Am Ende der Boot-Phase mit dem Starten des cron Daemon sind 
die zusätzlichen Einträge aktiv.

\texttt{add\_crontab\_entry} \var{time command [custom]}

Mit \var{time} wird die Ausführungszeit in cron Notation übergeben, siehe die
Manpage zu crontab(5) (\altlink{http://linux.die.net/man/5/crontab}).
\var{command} enthält den auszuführenden Befehl.  Der dritte Parameter
\var{custom} ist optional. Mit ihm können Umgebungsvariablen passend zum Befehl
gesetzt werden. Soll mehr als eine Variable gesetzt werden, sind die
Zuweisungen durch \var{$\backslash\backslash$} zu trennen. Bitte \textbf{nicht} die
Umgebungsvariable \var{PATH} verändern, da sonst nachfolgende crontab Einträge
nicht mehr korrekt abgarbeitet werden können.

\begin{verbatim}
#
# example I: normal use, 2 parameters
#
     crontime="0 5 1 * *"
     croncmd="rotate_i_log.sh"

     add_crontab_entry "$crontime" "$croncmd"

#
# end of example I
#
\end{verbatim}

\begin{verbatim}
#
# example II: extended use, 3 parameters and 
#                             multiple environment values 
#
     croncustom="source=/var/log/current \\ dest=/mnt/data/log"
     croncmd='cp $source $dest-`date +\%Y\%m\%d`; > $source'
     crontime="59 23 * * *"

     add_crontab_entry "$crontime" "$croncmd" "$croncustom"

#
# end of example II
#
\end{verbatim}

Die Richtigkeit der Einträge muß das aufrufende Script sicherstellen.
