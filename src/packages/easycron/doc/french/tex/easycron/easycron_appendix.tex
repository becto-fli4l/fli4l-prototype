% Do not remove the next line
% Synchronized to r29817

\marklabel{sec:addcrontabentry}{
\section{EASYCRON - Complément de Crontab pour la phase de boot}
}

\textbf{Remarque:} \emph{Les informations qui suivent s'adressent uniquement aux 
développeurs pour le paquetage opt-fli4l du routeur.}

À partir de la version 2.1.7, le script rc de EasyCron met à disposition
la fonction \var{add\_crontab\_entry()}. Lorsque vous utilisez cette fonction
vous pourrez enregistrer d'autres scripts rc, de la position rc101 jusqu'à
la position rc949. Ces entrées supplémentaires seront activées avec le démarrage
du démon cron et la fin phase de boot.

\texttt{add\_crontab\_entry} \var{time command [custom]}

Avec \var{time} vous attribuez le temps d'exécution de cron, consultez le manuel
crontab (5) (\altlink{http://linux.die.net/man/5/crontab}). Avec \var{command}
vous utilisez la commande à exécuter. Avec le troisième paramètre \var{custom} qui 
est facultatif. Avec ce paramétre on peut installer des extentions de commandes, 
selon votre convenance. Si vous placez plusieurs réglages, vous devez les séparer 
par \var{$\backslash\backslash$}. SVP ne changer \textbf{pas} la variable 
d'environnement \var{PATH}, car les entrées crontab ne fonctionnerons plus 
correctement.

\begin{verbatim}
#
# example I: normal use, 2 parameters
#
     crontime="0 5 1 * *"
     croncmd="rotate_i_log.sh"

     add_crontab_entry "$crontime" "$croncmd"

#
# end of example I
#
\end{verbatim}

\begin{verbatim}
#
# example II: extended use, 3 parameters and 
#                             multiple environment values 
#
     croncustom="source=/var/log/current \\ dest=/mnt/data/log"
     croncmd='cp $source $dest-`date +\%Y\%m\%d`; > $source'
     crontime="59 23 * * *"

     add_crontab_entry "$crontime" "$croncmd" "$croncustom"

#
# end of example II
#
\end{verbatim}

L'exactitude des paramètres doit être enregistrés dans le script qui est appelé.
