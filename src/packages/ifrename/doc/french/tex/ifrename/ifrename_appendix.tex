% Do not remove the next line
% Synchronized to r29817
% ---------------------------------------------------------------------------------
% ifrename_appendix.tex:
% ---------------------------------------------------------------------------------
% creation date: 2007/04/14 - <tobias@tb-home.de>
% last modified: 2011/03/18 - <tobias@tb-home.de>
% ---------------------------------------------------------------------------------


\section {Contexte de développement/Problème}

Author: Tobias Becker\\
Date: 2009-08-02\\
To: fli4l\\
Sujet~: [fli4l] persistance/cohérence des étiquettes réseaux~- Localisation à distance\\
\\
Salut NG,\\
\\
Peut-être que quelqu'un peut parler de ce sujet~- Il s'agit de la persistance
ou de la cohérence des étiquettes réseaux avec le Kernel actuel de fli4l, ou
une autre version.\\
\\
Pour les grosse distributions le contrôle est fait par Udev, par exemple après
la détection d'une nouvelle NIC une règle correspondante à la carte réseau est
générée (persistent-net-rules dans /etc/udev). Il est alors possible de modifier
le contrôle généré, et attribuer un nom approprié pour la NIC -> à partir de
eth0 nous avons alors eth0-uplink.\\
\\
Si l'on a besoin d'installer dans fli4l des cartes réseau avec des pilotes
différents (1 pilote pour chaque NIC), vous êtes en sécurité, parce que la
carte réseau est chargé au démarrage avec toujours la même adresse IP. Mais
si vous avez une carte réseau avec plusieurs ports physiques (ou plusieurs NIC
pro drivers/Quadport etc), la mise en place des adresses IP est soumis à une
configuration aléatoires, l'idéal serait un changement de séquence complète,
c'est à dire mettre eth4 sur eth0 et vice-versa.\\
\\
J'exploite un fli4l à plusieurs kilomètres de distance, après un redémarrage
vous pouvez est exclus, etc\\
\\
À ce jour, je n'ai pu trouver dans la configuration de base aucun réglage, dans 
lequel vous pouvez définir au moins la carte réseau par un paramètre unique
(l'idéal serait l'adresse MAC ou le bus pci, etc) (je ne parle pas de l'utilisation
de la variable éthernet du dhcp du fournisseur cela serait en effet idiot)
-> à partir de rien, j'ai créé le paquetage Ifrename avec des sources d'outils
wireless, mais Il y a un autre moyen de résoudre le problème, peut-être
directement dans le paquetage base de fli4l, par conséquent vous devez constamment
mettre à jour fli4l.\\

[...]

\section {Utilisation avec KVM/XEN}
  mémo~: création d'un HowTo sur KVM/XEN 
