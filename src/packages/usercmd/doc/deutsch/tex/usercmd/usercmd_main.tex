% Last Update: $Id: usercmd_main.tex 31462 2014-07-04 21:49:30Z kristov $
\marklabel{sec:opt-usercmd}
{
\section {OPT\_USERCMD - Benutzerdefinierte Ausführung von Befehlen zum Systemstart/Systemende}
}

\subsection{Konfiguration}

Mit opt\_usercmd können beliebige Befehle und Scripte nach dem Booten bzw. beim Herunterfahren
ausgeführt werden. Mögliche Anwendungsbeispiele wären:

\begin{itemize}
    \item Das zusätzliche Mounten von Filesystemen (insbesondere bei CD-Boot).
    \item Das Setzen zusätzlicher Routen ausserhalb der Basiseinstellung.
    \item Einen Ping zu einem Host absetzen (automatisches Onlinegehen bei der Option \var{DIALMODE}='auto').
\end{itemize}

Zusätzlich ist es möglich eigene Dateien aus dem config Verzeichnis in
das fli4l Image zu integrieren. Dabei werden folgende Einträge
benutzt:

\begin{description}

\config{OPT\_USERCMD}{OPT\_USERCMD}{OPTUSERCMD}

        Standard-Einstellung: \var{OPT\_USERCMD}='no'

    Mit \var{OPT\_USERCMD}='yes' wird das Paket aktiviert. Erst dann ist werden die zu
    konfigurierenden Befehle und Scripte beim Systemstart bzw. beim Systemende
    ausgeführt.


\config{USERCMD\_BOOT\_N}{USERCMD\_BOOT\_N}{USERCMDBOOTN}

        Standard-Einstellung: \var{USERCMD\_BOOT\_N}='0'

    Anzahl der Befehle/Scripte, welche nach dem Systemstart (Boot) ausgeführt werden
    sollen. Die Zeilen werden jeweils nacheinander abgearbeitet.

\config{USERCMD\_BOOT\_x}{USERCMD\_BOOT\_x}{USERCMDBOOTx}

        Standard-Einstellung: \var{USERCMD\_BOOT\_x}='echo user-defined boot-command here'

    Auszuführender Befehl beim Systemstart.

    \wichtig{Pro Eintrag darf normalerweise nur ein Befehl eingetragen werden. Sollen
    mehrere Befehle in einer Zeile untergebracht werden, müssen sie mit einem Semikolon
    getrennt werden. (Siehe Beispiele)}


\config{USERCMD\_HALT\_N}{USERCMD\_HALT\_N}{USERCMDHALTN}

        Standard-Einstellung: \var{USERCMD\_HALT\_N}='0'

    Anzahl der Befehle/Scripte, welche beim Herunterfahren des Systems (halt, shutdown, reboot)
    ausgeführt werden sollen. Die Zeilen werden jeweils nacheinander abgearbeitet.

\config{USERCMD\_HALT\_x}{USERCMD\_HALT\_x}{USERCMDHALTx}

        Standard-Einstellung: \var{USERCMD\_HALT\_x}='echo user-defined halt-command here'

    Auszuführender Befehl beim Beenden des Systems.

    \wichtig{Pro Eintrag darf normalerweise nur ein Befehl eingetragen werden. Sollen
    mehrere Befehle in einer Zeile untergebracht werden, müssen sie mit einem Semikolon
    getrennt werden. (Siehe Beispiele)}

\config{USERCMD\_FILE\_N}{USERCMD\_FILE\_N}{USERCMDFILEN}

        Standard-Einstellung: \var{USERCMD\_FILE\_N}='0'

   Manchmal kann es notwendig sein eigene Dateien in das fli4l Image
   opt.img (es ist aktuell nicht möglich Dateien in das rootfs.img
   einzubinden) einzubinden, ohne das ein eigenständiges opt-Paket
   Sinn macht. Für diese Fälle gibt es die Möglichkeit eine kleine
   Anzahl von Dateien direkt aus dem
   Verzeichnis \var{$<$config$>$/etc/usercmd} in das entsprechende
   fli4l Image einzubinden.

   Mit \var{USERCMD\_FILE\_N}='x' geben Sie die Anzahl der Dateien an,
   die in ein fli4l Image integriert werden sollen.

\config{USERCMD\_FILE\_x\_SRC}{USERCMD\_FILE\_x\_SRC}{USERCMDFILExSRC}

    Dateiname der Quelldatei im aktuellen
    Konfigurationsverzeichnis \var{$<$config$>$/etc/usercmd}. Es
    werden wirklich nur Dateien direkt aus diesem Verzeichnis
    eingebunden!

\config{USERCMD\_FILE\_x\_DST}{USERCMD\_FILE\_x\_DST}{USERCMDFILExDST}

    Absoluter Dateiname genau so wie er in das erzeugte fli4l Image
    eingebunden werden soll. Der so angegebene Dateiname kann dann
    z.B. beim Starten von fli4l durch ein
    entsprechendes \var{USERCMD\_BOOT\_x}='/usr/bin/mystuff.sh'
    aufgerufen werden.

\config{USERCMD\_FILE\_x\_MODE}{USERCMD\_FILE\_x\_MODE}{USERCMDFILExMODE}

    Der Filemodus der Datei im fli4l Image. Der Unixdateimode
    entsprechend den üblichen Unixkonventionen. Die Details können Sie
    in der Entwicklerdokumentation im Abschnitt 'Die Liste der zu
    kopierenden Dateien' (siehe auch \ref{sec:opt_txt})
    nachlesen oder bei
    Wikipedia \altlink{http://de.wikipedia.org/wiki/Unix-Dateirechte}.

\config{USERCMD\_FILE\_x\_FLAGS}{USERCMD\_FILE\_x\_FLAGS}{USERCMDFILExFLAGS}

    Entspricht der flags= Angabe aus der opt/<package>.txt
    Datei. Siehe auch \ref{sec:Dateiformate}.

    Es gibt folgende Möglichkeiten Textdateien vor der Aufnahme ins das fli4l Image zu konvertieren:
    \newline\newline
    \begin{tabular}{lp{6cm}}
        \emph{utxt} & Konvertierung ins Unix-Format\\
        \emph{dtxt} & Konvertierung ins DOS-Format\\
        \emph{sh}   & Shell-Skript: Konvertierung ins Unix-Format, Entfernen überflüssiger Zeichen,
        empfohlen für Shellscripte
    \end{tabular}

\end{description}

\subsection{Beispiele}

\begin{verbatim}
USERCMD_BOOT_x='fli4lctrl dial pppoe'
\end{verbatim}
Baut am Ende des Bootvorgangs eine DSL-Internetverbindung auf.


\begin{verbatim}

USERCMD_BOOT_x='sleep 60; ip link set tr0 down; ip link set tr0 up'
\end{verbatim}
Legt nach dem Systemstart eine Minute Pause ein, fährt dann das Interface tr0 runter
und wieder rauf. Kann dazu benutzt werden einem Tokenring-Switch die Zeit
zum Booten zu lassen und anschließend das Netzwerk neu zu starten.


\begin{verbatim}

USERCMD_BOOT_N='1'
USERCMD_BOOT_1='cp /data/log/imond.log /var/log/'
USERCMD_HALT_N='2'
USERCMD_HALT_1='cp /var/log/sys.log /data/log/sys`date +%Y%m%d`.log'
USERCMD_HALT_2='fli4lctrl hangup pppoe ; sleep 2 ; cp /var/log/imond.log /data/log/'
\end{verbatim}
Stellt am Ende des Bootvorgang eine vorher gesicherte imond-Logdatei wieder her.
Beim Herunterfahren wird das Syslog mit aktuellem Datum im Dateinamen gesichert,
die DSL-Internetverbindung getrennt und die imond-Logdatei gesichert.

\subsection{Support}
Support wird nur im Rahmen der fli4l Newsgroups geleistet.

\subsection{Anmerkungen}
\begin{itemize}
    \item Die Nutzung des Zeichens \glqq{}'\grqq{} ist in den Befehlszeilen nicht möglich.
    \item Eine Befehlszeile darf nicht mit dem Zeichen \glqq{}-\grqq{} beginnen.
\end{itemize}



% Diese Dokumentation wurde dankenswerterweise bereitgestellt von Oliver Musch (23 Feb. 2005).
% Überarbeitet und erweitert hat sie Michael Köhler (3. Jan. 2005).
