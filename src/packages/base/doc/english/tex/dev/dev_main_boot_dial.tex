% Synchronized to r38663

\section{Boot, Reboot, Dialin And Hangup Under fli4l}

\subsection{Boot Concept}

FLI4L 2.0 should offer a clean install on a hard disk or a
CompactFlash (TM) media, but also an installation on a
Zip medium or the creation of a bootable CD-ROM should be
possible. In addition, the hard drive version should not be
fundamentally different from the one on an installation disk\footnote{
Originally fli4l could be operated from a single floppy disk.
This is no longer supported since it became too big in file size.}.

These requirements have been implemented by making it possible to move
the files of the \texttt{opt.img} archive from the previous RAM disk to
another medium, be it a partition on a hard disk or a CF medium. This second
volume is mounted to \texttt{/opt} and only symbolic links are created from there
to the rootfs. The resulting layout in the root file system then corresponds
to the one unpacked in the  \texttt{opt} directory of the fli4l distribution with
one exception -- the \texttt{files} prefix is not applicable. The file 
\texttt{opt/etc/rc} is then found directly under \texttt{/etc/rc},
\texttt{opt/bin/busybox} under \texttt{/bin/busybox}. It can be
ignored that these files may be only links to a directory mounted read only
as long as you do not want to modify them. If you want to do this,
you have to make the files writable before by using \texttt{mk\_writable} (see below).

\subsection{Start And Stop Scripts}

Scripts intended to be executed on system boot are located in the
directories \texttt{opt/etc/boot.d/} and \texttt{opt/etc/rc.d/} and will also
get executed in this sequence. Furthermore, scripts executed on shutdown
are to be found in \texttt{opt/etc/rc0.d/}.

\wichtig{These script must not contain an ``exit'', because no separate
process is created for their execution. This command would lead to a premature
ending of the boot process!}

\subsubsection{Start Scripts in \texttt{opt/etc/boot.d/}}

Scripts located in this directory are executed at first. They mount the boot
volume, parse the config file \texttt{rc.cfg} located on the boot medium and
unpack the \texttt{opt} archive. Depending on the \jump{BOOTTYPE}{boot type} these scripts are
more or less complex and do the following things:

\begin{itemize}
\item Loading of hardware drivers (optional)
\item Mount the \texttt{boot} volume (optional)
\item Read the config file \texttt{rc.cfg} off the boot volume and
      write it to the file \texttt{/etc/rc.cfg}
\item Mount the \texttt{opt} volume (optional)
\item Unpack the \texttt{opt} archive (optional)
\end{itemize}

To make the scripts aware of the fli4l configuration,
the configuration file \texttt{/etc/rc.cfg} is also integrated 
in the Rootfs archive. The configuration variables
in this file are parsed by the start scripts in \texttt{opt/etc/boot.d/}.
After mounting the \texttt{boot} volume \texttt{/etc/rc.cfg} is replaced
by the configuration file there, so that the the current configuration
of the \texttt{boot} volume is available for startup scripts in
\texttt{opt/etc/rc.d/} (see below).
\footnote{Normally, these two files are identical. Discrepancies
are possible only if the configuration file on the \texttt{boot} volume
was edited manually, for example to modify the configuration later on
without the need to rebuild the fli4l archives.}

\subsubsection{Start Scripts in \texttt{opt/etc/rc.d/}}

Commands that are executed at every start of the router can be stored
in the directory \texttt{opt/etc/rc.d/}. The following conventions
apply:

\begin{enumerate}
  \item Name conventions:

\begin{example}
\begin{verbatim}
    rc<three-digit number>.<Name of the OPT>
\end{verbatim}
\end{example}
      
    The scripts are started in ascending order of the numbers.
    If multiple scripts have the same number assigned, they
    will be sorted alphabetically at that point. In case
    that the start of a package is dependant on another one,
    this is the determined by the number.

    Here's a general outline which numbers should be used for
    which tasks:    
    
    \begin{table}[htbp]
    \centering
    \begin{tabular}{ll}
            \hline
            Number        &       Task         \\
            \hline
            \hline
            000-099       &       Base system (hardware, time zone, file system) \\
            100-199       &       Kernel modules (drivers) \\
            200-299       &       External connections (PPPoE, ISDN4Linux, PPtP) \\
            300-399       &       Network (Routing, Interfaces, Packet filter) \\
            400-499       &       Server (DHCP, HTTPD, Proxy, a.s.o.) \\
            500-900       &       Any \\
            900-997       &       Anything causing a dialin \\
            998-999       &       reserved (please do not use!) \\
            \hline
    \end{tabular}
    \end{table}

  \item These scripts \emph{must} contain all functions changing the RootFS,
    ie. creating of a directory \texttt{/var/log/lpd}.

  \item These scripts shall \emph{not} contain writing to files that could be
    part of the \texttt{opt} archive, because these files could be located on
    a volume mounted in read-only mode. If you have to modify such a file, you
    have to make it writeable before by using the function \texttt{mk\_writable}
    (see below). This will create a writable copy of the file in the RootFS
    if needed. If the file is already writable the call of \texttt{mk\_writable}
    will have no effect.
    
    \wichtig{\texttt{mk\_writable} has to be applied directly to files in RootFS,
    not indirectly via the \texttt{opt} directory. If, for example, you want to
    modify \texttt{/usr/local/bin/foo}, the function \texttt{mk\_writable} has to be called
    with the argument \texttt{/usr/local/bin/foo}.}

  \item Before executing the actual commands these scripts have to check for the
    associated OPT really being active. This is usually done by a simple if-case:

\begin{example}
\begin{verbatim}
    if [ "$OPT_<OPT-Name>" = "yes" ]
    then
        ...
        # Start OPT here!
        ...
    fi
\end{verbatim}
\end{example}

  \item For easier debugging the scripts should be enclosed in
    \texttt{begin\_script} and \texttt{end\_script}:

\begin{example}
\begin{verbatim}
    if [ "$OPT_<OPT-Name>" = "yes" ]
    then
        begin_script FOO "configuring foo ..."
        ...
        end_script
    fi
\end{verbatim}
\end{example}

    Debugging of start-scripts may be activated simply via
    \verb+FOO_DO_DEBUG='yes'+.

  \item All configuration variables are available to the scripts in direct.
    Explanations how to access configuration variables in scripts can be found
    in the section \jump{dev:sec:config-variables}{``Working with configuration
    variables''}.

  \item The path \texttt{/opt} may not be used for storing OPT data. If in need
    of additional file space you should enable the user to define a suitable
    location by using a configuration variable. Depending on the type of data
    to be stored (persistent or transient data) different default assignments
    should be used. A path under \texttt{/var/run/} makes sense for transient
    data, while for persistent data it is advised to use the function
    \jump{dev:sec:persistent-data}{map2persistent} combined with a suitable
    configuration variable.

\end{enumerate}

\subsubsection{Stop Scripts in \texttt{opt/etc/rc0.d/}}

Each machine must be shut down or restarted from time to time. It is
perfectly possible that you have to perform operations before the computer
is shut down or restarted. To shut down and restart the commands ``halt''
or ``reboot'' are used. These commands are also invoked when the corresponding
buttons in IMONC or the Web GUI are clicked.

All stop scripts can be found in the \texttt{opt/etc/rc0.d/}. The file names
have to be created using the same rules as for the scripts. They are as well
executed in \emph{ascending} order of numbers.

\subsection{Helper Functions}

{/etc/boot.d/base-helper} provides a number of different functions that can
be used in Start- and other scripts. They contian support for debugging, loading
of kernel modules, or message output. The functions
are listed and explained in short below

\subsubsection{Script Control}

\begin{description}

\item[\texttt{begin\_script <Symbol> <Message>}:]
Output of a message and activation of script debugging by calling
\texttt{set -x}, if \var{<Symbol>\_DO\_DEBUG} is set to ``yes''.

\item[\texttt{end\_script}:]
Output of an end-message and deactivation of debugging if it was activated
with \texttt{begin\_script}. For each \texttt{begin\_script} call a corresponding
\texttt{end\_script} call has to exist (and vice versa).

\end{description}

\subsubsection{Loading Of Kernel Modules}

\begin{description}

\item[\texttt{do\_modprobe [-q] <Modul> <Parameter>*}:]
Loads a kernel module including its parameters (if needed) while resolving
its module dependencies. The parameter ``-q'' prevents error messages to be
written to the console and to the boot log in case of failure.
The function returns 0 for success and another value in case of
error. This enables you to create code for handling failures while loading
kernel modules:

\begin{example}
\begin{verbatim}
    if do_modprobe -q acpi-cpufreq
    then
        # no CPU frequency scaling via ACPI
        log_error "CPU frequency scaling via ACPI not available!"
        # [...]
    else
        log_info "CPU frequency scaling via ACPI activated."
        # [...]
    fi
\end{verbatim}
\end{example}

\item[\texttt{do\_modrobe\_if\_exists [-q] <Module path> <Module> <Parameter>*}:]\mbox{}\\
Checks if the module \texttt{/lib/modules/<Kernel-Version>/<Module path>/<Module>}
exists and, if so, invokes \texttt{do\_modprobe}.

\wichtig{The module has to exist exactly by this name, no aliases may be used.
When using an alias \texttt{do\_modprobe} will be called immediately.}

\end{description}

\subsubsection{Messages And Error Handling}

\begin{description}

\item[\texttt{log\_info <Message>}:] Logs a message to the console and to
\texttt{/bootmsg.txt}. If no message is passed as a parameter \texttt{log\_info}
reads the default input. The function always returns 0.

\item[\texttt{log\_warn <Message>}:] Logs a warning message to the console
and to \texttt{/bootmsg.txt}, using the string \texttt{WARN:} as a prefix. If no
message is passed as a parameter \texttt{log\_warn} reads the default input.
The function always returns 0.

\item[\texttt{log\_error <Message>}:] Logs an error message to the console
and to \texttt{/bootmsg.txt}, using the string \texttt{ERR:} as a prefix. If no
message is passed as a parameter \texttt{log\_warn} reads the default input.
The function always returns a non-zero value.

\item[\texttt{set\_error <Message>}:] Output of an error message and setting
of an internal error variable which can be checked later via \texttt{is\_error}.

\item[\texttt{is\_error}:] Clears the internal error variable and returns
true if it was set before via \texttt{set\_error}.

\end{description}

\subsubsection{Network Functions}

\begin{description}

\item[\texttt{translate\_ip\_net <Value> <Variable name> [<Result variable>]}:]\mbox{}\\
Replaces symbolic references in parameters. At the moment the following translations
are supported:
\begin{description}
\item[*.*.*.*, \texttt{none}, \texttt{default}, \texttt{pppoe}] will not be translated
\item[\texttt{any}] will be replaced by \texttt{0.0.0.0/0}
\item[\texttt{dynamic}] will be replaced by the IP address of the router which represents
the Internet connection
\item[\var{IP\_NET\_x}] will be replaced by the network found in the configuration
\item[\var{IP\_NET\_x\_IPADDR}] will be replaced by the IP address found in the configuration
\item[\var{IP\_ROUTE\_x}] will be replaced by the routed network found in the configuration
\item[\texttt{@<Hostname>}] will be replaced by the Hosts IP address specified in the configuration
\end{description}

The result of the translation is stored in the variable whose name is passed in the
third parameter, if this parameter is missing, the result is stored in the variable
\var{res}. The variable name that is passed in the second parameter is used only
for error messages if the translation fails, to enable the caller to pass the source
of the value to be translated. In case of failure a message like

\begin{example}
\begin{verbatim}
    Unable to translate value '<Value>' contained in <Variable name>.
\end{verbatim}
\end{example}

will be printed.

The return value is 0 in case of success, and unequal to zero in case of errors.

\end{description}

\subsubsection{Miscellaneous}

\begin{description}

\item[\texttt{mk\_writable <File>}:]
Ensures that the given file is writable. If the file is located on a volume
mounted in read-only mode and is only linked to the file system via a
symbolic link, a local copy will be created which is then written to.

\item[\texttt{list\_unique <List>}:]
Removes duplicates from a list passed. The result is written to standard output.
\var{list}.

\end{description}

\subsection{mdev-Rules}

OPTs may establish additional mdev-rules that trigger defined actions in
appearance or disappearance of certain devices. \var{OPT\_AUTOMOUNT} from
package hd, for example, uses such a rule to mount storage devices automatically.
If you want to integrate an additional mdev rule, you have to embed a script of
the form

\begin{verbatim}
    /etc/mdev.d/mdev<Number>.<Name>
\end{verbatim}

into the RootFS whereas the number, similar to the start- and stop scripts, has
to consist of three digits while the name can be chosen arbitrarily. In the script
all output to the standard output is integrated in the resulting \texttt{/etc/mdev.conf}.
An example from the aforementioned \var{OPT\_AUTOMOUNT}:

\begin{small}
\begin{verbatim}
#!/bin/sh
#----------------------------------------------------------------------------
# /etc/mdev.d/mdev500.automount - mdev HD automounting rules     __FLI4LVER__
#
#
# Last Update:  $Id: dev_main_boot_dial.tex 51849 2018-03-06 14:55:21Z kristov $
#----------------------------------------------------------------------------

cat <<"EOF"
#
# mdev500.automount
#

-SUBSYSTEM=block;DEVTYPE=partition;.+       0:0 660 */lib/mdev/automount

EOF
\end{verbatim}
\end{small}

Reference the rule's syntax from the file header of \texttt{/etc/mdev.conf}
and from the mdev documentation at
\altlink{http://git.busybox.net/busybox/plain/docs/mdev.txt}. If a rule
calls a script (such as \texttt{/lib/mdev/automount} in the example above)
it has access to all variables of the triggering kernel-``uevent'', in particular:

\begin{itemize}
\item \var{ACTION} (typically \texttt{add} or \texttt{remove}, less common
\texttt{change})
\item \var{DEVPATH} (sysfs path to the affected component)
\item \var{SUBSYSTEM} (the affected kernel subsystem, see below)
\item \var{DEVNAME} (the affected device file under \texttt{/dev}; missing if
no devices have to be created or deleted, but for example, modules)
\item \var{MDEV} (is set by mdev to the name of the finally created device)
\end{itemize}

Example for kernel subsystems:

\begin{description}
\item[block] Block devices (storage media) like \texttt{sda} (first harddisk),
             \texttt{sr0} (first CD drive) or \texttt{ram1} (second
             RAM disk)
\item[input] Input devices for keyboard, mouse a.s.o. like
             \texttt{input/event0}; which device file is linked to which device
             is not set and has to be determined via sysfs
\item[mem]   Devices to access the memory and hardware ports as \texttt{mem} and 
	     \texttt{ports}; this encloses also pseudo devices like \texttt{zero}
	     (continuously delivers the ASCII character with value zero) and 
	     \texttt{null} (returns nothing, swallows everything) among them
\item[sound] various devices for sound output, no naming convention
\item[tty]   devices for accessing physical and virtual consoles like
             \texttt{tty1} (first virtual console) or \texttt{ttyS0} (first
             serial console)\end{description}

An example for the first two serial ports:

\begin{scriptsize}
\begin{verbatim}
mdev[42]: 30.050644 add@/devices/pnp0/00:04/tty/ttyS0
mdev[42]:   ACTION=add
mdev[42]:   DEVPATH=/devices/pnp0/00:04/tty/ttyS0
mdev[42]:   SUBSYSTEM=tty
mdev[42]:   MAJOR=4
mdev[42]:   MINOR=64
mdev[42]:   DEVNAME=ttyS0
mdev[42]:   SEQNUM=613

mdev[42]: 30.051477 add@/devices/platform/serial8250/tty/ttyS1
mdev[42]:   ACTION=add
mdev[42]:   DEVPATH=/devices/platform/serial8250/tty/ttyS1
mdev[42]:   SUBSYSTEM=tty
mdev[42]:   MAJOR=4
mdev[42]:   MINOR=65
mdev[42]:   DEVNAME=ttyS1
mdev[42]:   SEQNUM=614
\end{verbatim}
\end{scriptsize}

An example of a connected MF II keyboard:

\begin{scriptsize}
\begin{verbatim}
mdev[41]: 4.030653 add@/devices/platform/i8042/serio0/input/input0
mdev[41]:   ACTION=add
mdev[41]:   DEVPATH=/devices/platform/i8042/serio0/input/input0
mdev[41]:   SUBSYSTEM=input
mdev[41]:   PRODUCT=11/1/1/ab41
mdev[41]:   NAME="AT Translated Set 2 keyboard"
mdev[41]:   PHYS="isa0060/serio0/input0"
mdev[41]:   PROP=0
mdev[41]:   EV=120013
mdev[41]:   KEY=4 2000000 3803078 f800d001 feffffdf ffefffff ffffffff fffffffe
mdev[41]:   MSC=10
mdev[41]:   LED=7
mdev[41]:   MODALIAS=input:b0011v0001p0001eAB41-e0,1,4,11,14,k71,72,73,74,75,76,77,79,
7A,7B,7C,7D,7E,7F,80,8C,8E,8F,9B,9C,9D,9E,9F,A3,A4,A5,A6,AC,AD,B7,B8,B9,D9,E2,ram4,l0,
1,2,sfw
mdev[41]:   SEQNUM=604
\end{verbatim}
\end{scriptsize}

An example of a loaded USB kernel module (\texttt{uhci\_hcd}):

\begin{scriptsize}
\begin{verbatim}
mdev[41]: 6.537506 add@/module/uhci_hcd
mdev[41]:   ACTION=add
mdev[41]:   DEVPATH=/module/uhci_hcd
mdev[41]:   SUBSYSTEM=module
mdev[41]:   SEQNUM=633
\end{verbatim}
\end{scriptsize}

An example of a connected hard disk:

\begin{scriptsize}
\begin{verbatim}
mdev[41]: 7.267527 add@/devices/pci0000:00/0000:00:07.1/ata1/host0/target0:0:0/0:0:0:0/block/sda
mdev[41]:   ACTION=add
mdev[41]:   DEVPATH=/devices/pci0000:00/0000:00:07.1/ata1/host0/target0:0:0/0:0:0:0/block/sda
mdev[41]:   SUBSYSTEM=block
mdev[41]:   MAJOR=8
mdev[41]:   MINOR=0
mdev[41]:   DEVNAME=sda
mdev[41]:   DEVTYPE=disk
mdev[41]:   SEQNUM=688
\end{verbatim}
\end{scriptsize}

This is an ATA/IDE hard drive (ata1) that should be accessed via the device name
\texttt{sda}.

An example of a remote block device (assigning an image file to
a fli4l VM was dissolved via \texttt{virsh detach-device}):

\begin{scriptsize}
\begin{verbatim}
mdev[42]: 52.600646 remove@/devices/pci0000:00/0000:00:0a.0/virtio5/block/vdb/vdb1
mdev[42]:   ACTION=remove
mdev[42]:   DEVPATH=/devices/pci0000:00/0000:00:0a.0/virtio5/block/vdb/vdb1
mdev[42]:   SUBSYSTEM=block
mdev[42]:   MAJOR=254
mdev[42]:   MINOR=17
mdev[42]:   DEVNAME=vdb1
mdev[42]:   DEVTYPE=partition
mdev[42]:   SEQNUM=776

mdev[42]: 52.644642 remove@/devices/virtual/bdi/254:16
mdev[42]:   ACTION=remove
mdev[42]:   DEVPATH=/devices/virtual/bdi/254:16
mdev[42]:   SUBSYSTEM=bdi
mdev[42]:   SEQNUM=777

mdev[42]: 52.644718 remove@/devices/pci0000:00/0000:00:0a.0/virtio5/block/vdb
mdev[42]:   ACTION=remove
mdev[42]:   DEVPATH=/devices/pci0000:00/0000:00:0a.0/virtio5/block/vdb
mdev[42]:   SUBSYSTEM=block
mdev[42]:   MAJOR=254
mdev[42]:   MINOR=16
mdev[42]:   DEVNAME=vdb
mdev[42]:   DEVTYPE=disk
mdev[42]:   SEQNUM=778

mdev[42]: 52.644777 remove@/devices/pci0000:00/0000:00:0a.0/virtio5
mdev[42]:   ACTION=remove
mdev[42]:   DEVPATH=/devices/pci0000:00/0000:00:0a.0/virtio5
mdev[42]:   SUBSYSTEM=virtio
mdev[42]:   MODALIAS=virtio:d00000002v00001AF4
mdev[42]:   SEQNUM=779

mdev[42]: 52.644973 remove@/devices/pci0000:00/0000:00:0a.0
mdev[42]:   ACTION=remove
mdev[42]:   DEVPATH=/devices/pci0000:00/0000:00:0a.0
mdev[42]:   SUBSYSTEM=pci
mdev[42]:   PCI_CLASS=10000
mdev[42]:   PCI_ID=1AF4:1001
mdev[42]:   PCI_SUBSYS_ID=1AF4:0002
mdev[42]:   PCI_SLOT_NAME=0000:00:0a.0
mdev[42]:   MODALIAS=pci:v00001AF4d00001001sv00001AF4sd00000002bc01sc00i00
mdev[42]:   SEQNUM=780
\end{verbatim}
\end{scriptsize}

As you can see, at such a removal various kernel subsystems are 
involved (here \texttt{block}, \texttt{bdi}, \texttt{virtio} and
\texttt{pci}).

\subsection{ttyI Devices}

For ttyI devices (\texttt{/dev/ttyI0} \ldots \texttt{/dev/ttyI15}) used by
the ``modem emulation'' of the ISDN  card a counter exists to avoid conflicts
between multiple packages using these devices.
For this purpose the file \texttt{/var/run/next\_ttyI} is created on router start
which can be queried and incremented by the OPTs. The following example script can
query this value, increment it by one and export it again for the next OPT.

\begin{example}
\begin{verbatim}
    ttydev_error=
    ttydev=$(cat /var/run/next_ttyI)
    if [ $ttydev -le 16 ]
    then                                    # ttyI device available? yes
        ttydev=$((ttydev + 1))              # ttyI device + 1
        echo $ttydev >/var/run/next_ttyI    # save it
    else                                    # ttyI device available? no
        log_error "No ttyI device for <Name of your OPT> available!"
        ttydev_error=true                   # set error for later use
    fi

    if [ -z "$ttydev_error" ]               # start OPT only if next tty device
    then                                    # was available to minimize error
        ...                                 # messages and minimize the
                                            # risk of uncomplete boot
    fi
\end{verbatim}
\end{example}

\subsection{Dialin And Hangup Scripts}

\subsubsection{General}

After dialin resp.\ hangup of a dial-up connection the scripts
placed in \texttt{/etc/ppp/} are executed. OPTs may store actions here
that have to be executed after connecting resp.\ hanging up of a
connection. The name scheme for the files is as follows:

\begin{table}[htbp]
\centering
\begin{tabular}{l}
    \texttt{ip-up<three-digit number>.<OPT-Name>}\\
    \texttt{ip-down<three-digit number>.<OPT-Name>}\\
\end{tabular}
\end{table}

\texttt{ip-up} scripts will be excuted after \emph{establishing} and
\texttt{ip-down} scripts after \emph{hangig up} of the connection.

\wichtig{In \texttt{ip-down} scripts no actions may be taken that lead
to another dialin because this would create a permanent-online condition
not desired for users without a flatrate.}

\wichtig{Since no separate process is created for these scripts, they may
\emph{not} invoke ``exit'' as well!}

\textbf{Hint:} If a script wants to check for \texttt{ip-up} scripts being 
executed the variable \var{ip\_up\_events} may be sourced from \texttt{rc400}
and up. If it is set to ``yes'' dialup-connections exist and \texttt{ip-up}
scripts will be executed. No dialup-connections are configured if it is set
to ``no'' and \texttt{ip-up} scripts will not get executed. There is an exeception
to this rule: If an Ethernet router is configured without dialup-connections but
a default-Route (\texttt{0.0.0.0/0}) exists, \texttt{ip-up} scripts will get
executed only once at the end of the boot process. (And as well the \texttt{ip-down}
scripts on rooter shutdown.)

\subsubsection{Variables}

Due to the special call concept of the \texttt{ip-up} and \texttt{ip-down} scripts
the following variables apply:

\begin{table}[htbp]
\centering
\begin{tabular}{lp{10cm}}

    \var{real\_interface}    & the real interface, ie.\
                               \texttt{ppp0}, \texttt{ippp0}, \ldots\\
    \var{interface}          & the IMOND interface, ie. \texttt{pppoe},
                               \texttt{ippp0}, \ldots\\
    \var{tty}                & terminal connected, may be empty!\\
    \var{speed}              & connection speed, for ISDN ie.\
                               64000\\
    \var{local}              & own IP address\\
    \var{remote}             & IP address of the Point-To-Point partner\\
    \var{is\_default\_route} & specifies if the current
                               \texttt{ip-up}/\texttt{ip-down} is for
                               the interface of the default route
                               (may be ``yes'' or ``no'')\\
\end{tabular}
\end{table}

\subsubsection{Default Route}

As of version 2.1.0 \texttt{ip-up}/\texttt{ip-down} scripts are executed for
all connections, not only for the interface of the default route. To emulate the
old behaviour you have to include the following in \texttt{ip-up} and \texttt{ip-down}
scripts:

\begin{example}
\begin{verbatim}
    # is a default-route-interface going up?
    if [ "$is_default_route" = "yes" ]
    then
        # actions to be taken
    fi
\end{verbatim}
\end{example}


Of course, the new behaviour can also be used for specific actions.
