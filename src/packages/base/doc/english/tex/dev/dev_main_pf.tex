% Synchronized to r49912

\providecommand{\fwaction}[1]{{\small\textsf{#1}}}
\providecommand{\fwchain}[1]{\texttt{#1}}
\providecommand{\fwtable}[1]{\textsc{#1}}
\providecommand{\fwmatch}[1]{\texttt{#1}}
\providecommand{\fwpktstate}[1]{\texttt{#1}}
\providecommand{\fwloglevel}[1]{\texttt{#1}}

\section{Using The Packet Filter}
\subsection{Adding Own Chains And Rules}

A set of routines is provided to manipulate the packet filter to add or delete
so-called ``chains'' and ``rules''. A chain is a named list of ordered rules.
There is a set of predefined chains (\fwchain{PREROUTING}, \fwchain{INPUT},
\fwchain{FORWARD}, \fwchain{OUTPUT}, \fwchain{POSTROUTING}), using this set of
routines more chains can be created as needed.

\begin{description}
\item [\texttt{add\_chain/add\_nat\_chain <chain>}:]
  Adds a chain to the ``filter-'' or ``nat-'' table.
\item [\texttt{flush\_chain/flush\_nat\_chain <chain>}:]
  Deletes all rules from a chain of the ``filter-'' or ``nat-'' table.
\item [\texttt{del\_chain/del\_nat\_chain <chain>}:]
  Deletes a chain from the ``filter-'' or ``nat-'' table. Chains must be empty
  prior to deleting and all references to them have to be deleted as well before.
  Such a reference i.e. can be a \fwaction{JUMP}-action with the chain defined
  as its target.
\item[\texttt{add\_rule/ins\_rule/del\_rule}:] Adds rules to the end
  (\texttt{add\_rule}) resp. at any place of a chain (\texttt{ins\_rule})
  or deletes rules from a chain (\texttt{del\_rule}). Use the syntax like here:

\begin{example}
\begin{verbatim}
    add_rule <table> <chain> <rule> <comment>
    ins_rule <table> <chain> <rule> <position> <comment>
    del_rule <table> <chain> <rule> <comment>
\end{verbatim}
\end{example}

  \noindent where the parameters have the following meaning:
  \begin{description}
  \item[table] The table in which the chain is
  \item[chain] The chain, in which the rule is to be inserted
  \item[rule] The rule which is to be inserted, the format
     corresponds to that used in the configuration file
  \item[position] The position at which the rule will be added (only
     in \texttt{ins\_rule})
  \item[comment] A comment that appears with the rule
     when somebody looks at the packet filter.
  \end{description}
\end{description}


\subsection{Integrating Into Existing Rules}

fli4l configures the packet filter with a certain default rule set.
If you want to add your own rules, you will usually want to insert
them after the default rule set. You will also need to know what the
action is desired by the user when dropping a packet. This information
can be obtained for
\fwchain{FORWARD}- and \fwchain{INPUT} chains by calling two functions,
\texttt{get\_defaults} and \texttt{get\_count}. After calling

\begin{example}
\begin{verbatim}
    get_defaults <chain>
\end{verbatim}
\end{example}

the following results are obtained:

\begin{description}
\item[\var{drop}:] This variable contains the chain to which is branched
   when a packet is discarded.
\item[\var{reject}:] This variable contains the chain to which is branched
   when a packet is rejected.
\end{description}

After calling

\begin{example}
\begin{verbatim}
    get_count <chain>
\end{verbatim}
\end{example}

the variable \var{res} contains the number of rules in the chain
\texttt{<chain>}. This position is of importance because you can \emph{not}
simply use \texttt{add\_rule} to add a rule at the end of the predefined
``filter''-chains \fwchain{INPUT}, \fwchain{FORWARD} and \fwchain{OUTPUT}.
This is because these chains are completed with a default rule valid for
all remaining packets depending on the content of the \var{PF\_<chain>\_POLICY}-variable.
Adding a rule \emph{after} this last rule hence has no effect. The function
\texttt{get\_count} instead allows to detect the position right \emph{in front of}
this last rule and to pass this position to the \texttt{ins\_rule}-function as a
parameter \texttt{<position>} in order to add the rule in the place at the end of the
appropriate chain, but right in front of this last default rule targeting
all remaining packets.

An example from the script \texttt{opt/etc/rc.d/rc390.dns\_dhcp} from the
package ``dns\_dhcp'' shall make this clear:

\begin{example}
\begin{verbatim}
    case $OPT_DHCPRELAY in
        yes)
            begin_script DHCRELAY "starting dhcprelay ..."

            idx=1
            interfaces=""
            while [ $idx -le $DHCPRELAY_IF_N ]
            do
                eval iface='$DHCPRELAY_IF_'$idx

                get_count INPUT
                ins_rule filter INPUT "prot:udp  if:$iface:any 68 67 ACCEPT" \
                    $res "dhcprelay access"

                interfaces=$interfaces' -i '$iface
                idx=`expr $idx + 1`
            done
            dhcrelay $interfaces $DHCPRELAY_SERVER

            end_script
        ;;
  esac
\end{verbatim}
\end{example}

Here you can see in the middle of the loop a call to \texttt{get\_count}
followed by a call to the \texttt{ins\_rule} function and, among other things,
the \var{res} variable is passed as \texttt{position} parameter.


\subsection{Extending The Packet Filter Tests}

fli4l uses the syntax \fwmatch{match:params} in packet filter rules
to add additional conditions for packet matching (see \fwmatch{mac:},
\fwmatch{limit:}, \fwmatch{length:}, \fwmatch{prot:}, \ldots). If you
want to add tests you have to do this as follows:

\begin{enumerate}
\item Define a suitable name. The first character of this name has
to be lower case a-z. The rest of the name can consist of any character
or digit.

\achtung{If the packet filter test is used within IPv6 rules it is
to make sure that the name is not a valid IPv6 address component!}

\item Creating a file \texttt{opt/etc/rc.d/fwrules-<name>.ext}.
The content of this file is something like this:

\begin{example}
\begin{verbatim}
    # IPv4 extension is available
    foo_p=yes

    # the actual IPv4 extension, adding matches to match_opt
    do_foo()
    {
        param=$1
        get_negation $param
        match_opt="$match_opt -m foo $neg_opt --fooval $param"
    }

    # IPv6 extension is available
    foo6_p=yes

    # the actual IPv6 extension, adding matches to match_opt
    do6_foo()
    {
        param=$1
        get_negation6 $param
        match_opt="$match_opt -m foo $neg_opt --fooval $param"
    }
\end{verbatim}
\end{example}

\mtr{The packet filter test does not have to be implemented for both IPv4 and
IPv6 (though this would be preferred if reasonable for both layer 3 protocols).}

\item Testing the extension:

\begin{example}
\begin{verbatim}
    $ cd opt/etc/rc.d
    $ sh test-rules.sh 'foo:bar ACCEPT'
    add_rule filter FORWARD 'foo:bar ACCEPT'
    iptables -t filter -A FORWARD -m foo --fooval bar -s 0.0.0.0/0 \
        -d 0.0.0.0/0 -m comment --comment foo:bar ACCEPT -j ACCEPT
\end{verbatim}
\end{example}

\item Adding the extension and all other needed files
(\texttt{iptables} components) to the archive using
the known mechanisms.
\item Allowing the extension in the configuration by extending
of \var{FW\_GENERIC\_MATCH} and/or \var{FW\_GENERIC\_MATCH6} in an exp-file,
for example:

\begin{example}
\begin{verbatim}
    +FW_GENERIC_MATCH(OPT_FOO) = 'foo:bar' : ''
    +FW_GENERIC_MATCH6(OPT_FOO) = 'foo:bar' : ''
\end{verbatim}
\end{example}
\end{enumerate}
