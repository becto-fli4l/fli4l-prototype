% Synchronized to r29883

\section{Networks}

\begin{description}
  \config{IP\_NET\_N}{IP\_NET\_N}{IPNETN}

  Default Setting: \var{IP\_NET\_N='1'}
  
  {Number of networks to bound to the IP procotol, normally one (`1').
  If you set \var{IP\_NET\_N} to zero because you don't have any IP networks or
  because you configure them in a different way, mkfli4l will emit a warning
  when building the archives. You can disable this warning by using
  \marklabel{IGNOREIPNETWARNING}{\var{IGNOREIPNETWARNING}='yes'}.}

  \config{IP\_NET\_x}{IP\_NET\_x}{IPNETx}

  Default Setting: \var{IP\_NET\_1='192.168.6.1/24'}

  {The IP address and the net mask of the router's n-th device using the
   CIDR\footnote{Classless Inter-Domain Routing} notation. If you want the
   router to receive its IP address dynamically via a DHCP-client it is possible
   to set this variable to 'dhcp'.

   The following table shows how the CIDR notation and the dot notation for
   net masks are connected.

   \marklabel{tab:cidr}{
     \begin{tabular}[h!]{rcc}
       CIDR &     Net mask   & Number of IP addresses \\
       \hline
       \hline
       /8   & 255.0.0.0       & 16777216   \\
       /16  & 255.255.0.0     & 65536      \\
       /23  & 255.255.254.0   & 512        \\
       /24  & 255.255.255.0   & 256        \\
       /25  & 255.255.255.128 & 128        \\
       /26  & 255.255.255.192 & 64         \\
       /27  & 255.255.255.224 & 32         \\
       /28  & 255.255.255.240 & 16         \\
       /29  & 255.255.255.248 & 8          \\
       /30  & 255.255.255.252 & 4          \\
       /31  & 255.255.255.254 & 2          \\
       /32  & 255.255.255.255 & 1
     \end{tabular}
   }

   \textbf{Note: } As one address is reserved for the network and one for
   broadcasting, the maximum number of hosts in the network is computed by:
   \texttt{Number\_Hosts = Number\_IPs - 2}. Consequently, the smallest possible
   net mask is \texttt{/30} which corresponds to four IP addresses and hence to
   two possible hosts.
   }

  \config{IP\_NET\_x\_DEV}{IP\_NET\_x\_DEV}{IPNETxDEV}

  Default Setting: \var{IP\_\-NET\_\-1\_\-DEV}='eth0'
  
  {Required: device name of the network adapter.

    Starting with version 2.1.8, the name of the device used has to be
    supplied! Names of network devices typically start with \texttt{'eth'} 
    followed by a number. The first network adapter recognized by the
    system receives the name \texttt{'eth0'}, the second one \texttt{'eth1'} and so on.\\

    Example:

  \begin{example}
  \begin{verbatim}
        IP_NET_1_DEV='eth0'
  \end{verbatim}
  \end{example}

    The fli4l router is also able to do IP aliasing, i.e. to assign multiple
    IP addresses to a single network adapter. Additional IP addresses are
    simply specified by linking another network to the same device. When
    mkfli4l checks the configuration you are informed that you define such an
    alias---you can ignore the warning in this case.

    Example:

  \begin{example}
  \begin{verbatim}
        IP_NET_1='192.168.6.1/24'
        IP_NET_1_DEV='eth0'
        IP_NET_2='192.168.7.1/24'
        IP_NET_2_DEV='eth0'
  \end{verbatim}
  \end{example}
  }

  \config{IP\_NET\_x\_MAC}{IP\_NET\_x\_MAC}{IPNETxMAC}
  
  Default Setting: \var{IP\_\-NET\_\-1\_\-MAC}=''

  {Optional: MAC address of the network adapter.

    With this variable you are able to change the hardware address (MAC) of
    your network adapter. This is useful if you want to use a DHCP provider
    expecting a certain MAC address.
    If you leave \var{IP\_NET\_x\_MAC} empty or remove the variable definition
    completely, the original MAC address of your network adapter will be used.
    Most users will never need to use this variable.

    Example:

  \begin{example}
  \begin{verbatim}
        IP_NET_1_MAC='01:81:42:C2:C3:10'
  \end{verbatim}
  \end{example}
  }

  \config{IP\_NET\_x\_NAME}{IP\_NET\_x\_NAME}{IPNETxNAME}

  Default Setting: \var{IP\_\-NET\_\-x\_\-NAME}=''

  {Optional: Assigning a name to the IP address of a network adapter.

    If you perform reverse DNS lookup of the network adapter's IP address, the
    result is typically a name like 'fli4l-ethx.$<$domain$>$'. You can use
    the variable \var{IP\_NET\_x\_NAME} in order to change that name which will
    be returned when performing reverse DNS lookup. If the IP address is
    globally accessible, you can use this setting to enforce that reverse
    DNS lookups always return a globally accessible name.

    Example:

  \begin{example}
  \begin{verbatim}
        IP_NET_2='80.126.238.229/32'
        IP_NET_2_NAME='ajv.xs4all.nl'
  \end{verbatim}
  \end{example}
  }

  \config{IP\_NET\_x\_TYPE}{IP\_NET\_x\_TYPE}{}

  \config{IP\_NET\_x\_COMMENT}{IP\_NET\_x\_COMMENT}{IPNETxCOMMENT}

  Default Setting: \var{IP\_NET\_x\_COMMENT}=''
  
  {Optional: You can use this setting to assign a `meaningful' name to a network
  device. This name can then be used in packages like rrdtool for identifying
  the network.
  }
  
\end{description}
