% Synchronized to r34917

\marklabel{chap:bootmedien}{
  \chapter{Creating the fli4l Archives/Boot media}
  }

  If all configuration is completed, the fli4l archives/boot media may be created
  as either bootable Compact-Flash, a bootable ISO image, or only the files needed
  for a remote update.

\marklabel{sec:bootmedien_linux}{
  \section{Creating the fli4l Archives/Boot media under Linux or other Unix derivatives and Mac OS X}
  }

  This is done by using scripts (\texttt{.sh}), which can be found in the
  fli4l root directory.

  \begin{description}
    \item \texttt{mkfli4l.sh}
  \end{description}

  The Build Script recognizes the different \jump{BOOTTYPE}{boot types}.

  The simplest call on Linux looks like this:
   \begin{verbatim}
    sh mkfli4l.sh
  \end{verbatim}

  The actions of the build scripts are controlled by three mechanisms:
   \begin{itemize}
    \item Configuration variable \var{BOOT\_TYPE} from
          \texttt{$<$config$>$/base.txt}
    \item Configuration file \texttt{$<$config$>$/mkfli4l.txt}
    \item Build-Script Parameters
  \end{itemize}

  The variable \jump{BOOTTYPE}{\var{BOOT\_TYPE}}
  decides which action of the Build Scripts is executed:
  \begin{itemize}
    \item Create a bootable fli4l CD-ISO-Image
    \item Generating the fli4l files needed for a remote update
    \item Generating the fli4l files and directly do a remote update via SCP
    \item a.s.o.
  \end{itemize}

  The description of the variables in the configuration file
  \texttt{$<$config$>$/mkfli4l.txt} can be found in Chapter
  \jump{sec:mkfli4lconf}{Control file mkfli4l.txt}.

  \subsection{Command line options}
  The last control mechanism is appending option parameters
  to the call of the Build Script on the command line.
  The control options correspond to those in the file \texttt{mkfli4l.txt}.
  Option parameters override the values from the control file.
  Out of convenience, the names of the option parameters differ from the names of
  the variables from the control file. There is a long and, to some extent, a
  short form:

  \begin{verbatim}
Usage: mkfli4l.sh [options] [config-dir]

-c, --clean         cleanup the build-directory
-b, --build <dir>   set build-directory to <dir> for the fli4l-files
-h, --help          display this usage
--batch             don't ask for user input

config-dir          set other config-directory - default is "config"

--hdinstallpath <dir> install a pre-install environment directly to
                    usb/compact flash device mounted or mountable to
                    directory <dir> in order to start the real installation
                    process directly from that device
                    device either has to be mounted and to be writable
                    for the user or it has to be mountable by the user
                    Do not use this for regular updates!

*** Remote-Update options
--remoteupdate        remote-update via scp, implies "--filesonly"
--remoteremount       make /boot writable before copying files and
                      read only afterwards
--remoteuser <name>   user name for remote-update - default is "fli4l"
--remotehost <host>   hostname or IP of remote machine - default
                      is HOSTNAME set in [config-dir]/base.txt
--remotepath <path>   pathname on remote maschine - default is "/boot"
--remoteport <portnr> portnumber of the sshd on remote maschine

*** Netboot options (only on Unix/Linux)
--tftpbootpath <path>   pathname to tftpboot directory
--tftpbootimage <name>  name of the generated bootimage file
--pxesubdir <path>      subdirectory for pxe files relative to tftpbootpath

*** Developer options
-u, --update-ver    set version to <fli4l_version>-rev<svn revision>
-v, --verbose       verbose - some debug-output
-k, --kernel-pkg    create a package containing all available kernel
                    modules and terminate afterwards.
                    set COMPLETE_KERNEL='yes' in config-directory/_kernel.txt
                    and run mkfli4l.sh again without -k to finish
    --filesonly     create only fli4l-files - do not create a boot-media
    --no-squeeze    don't compress shell scripts
    --rebuild       rebuild mkfli4l and related tools; needs make, gcc

    \end{verbatim}

   An HD pre-installation of a suitably formatted (FAT16/FAT32) CompactFlash
   (in a USB cardreader) or an USB Stick can be done by using the option
   \verb+--hdinstallpath <dir>+.
   You are using this script \emph{at your own risk}.
   The necessary fli4l files will be copied onto the specified partition.
   At first, run in the fli4l directory:

  \begin{verbatim}
     sh mkfli4l.sh --hdinstallpath <dir>
  \end{verbatim}
  \vspace{-2ex}
  This will generate the fli4l files and copy them to the CF-Card or USB Stick.

  To run the next steps, you have to make sure:

   \begin{itemize}
        \item \verb+chmod 777 /dev/brain+
        \item superuser rights
        \item installed \verb+syslinux+
        \item installed \verb+fdisk+
   \end{itemize}

 The script will ensure that this storage device is a FAT-partitioned USB-Drive.
 After that the boot loader and the files needed will be copied to the disk.
 You will get notified about success or failure.

 After the build you have to execute the following:

  \begin{verbatim}
    syslinux --mbr /dev/brain

    # make partition bootable using fdisk
    #     p - print partitions
    #     a - toggle bootable flag, specify number of fli4l partition
    #         usually '1'
    #     w - write changes and quit
    fdisk /dev/brain

    # install boot loader
    syslinux -i /dev/brain
 \end{verbatim}
 \vspace{-2ex}

 Now the CF resp. USB-drive should be bootable.
 Don't forget to unmount the device (via \texttt{umount}).

  \bigskip

  An alternative configuration directory can be specified by appending its name
  to the end of the command line. The normal configuration directory is called
  \texttt{config} and cn be found under the fli4l root directory. This is where all
  fli4l packages place their configuration files.
  If you want to maintain more than one configuration, create another directory, i.e. \texttt{hd.conf},
  place a copy of the configuration files there and change them according to the requirements.
  Here are some examples:

  \begin{verbatim}
     sh mkfli4l.sh --filesonly hd.conf
     sh mkfli4l.sh --no-squeeze config.test
  \end{verbatim}



\marklabel{sec:bootmedien_windows}{
  \section{Creating the fli4l Archives/Boot media under Windows}
  }

  Utilize the tool `AutoIt3' (\altlink{http://www.autoitscript.com/site/autoit/}). This enables a
  `graphical' edition, as well as dialogues which allow to change the variables described in the following
   sections.


  \begin{description}
    \item \texttt{mkfli4l.bat}
  \end{description}


  The Build program automatically recognizes the different \jump{BOOTTYPE}{boot types}.

  The `mkfli4l.bat' can be invoked directly from Windows Explorer, if you need no
  optional parameters.

   The actions of the Build program are controlled by different mechanisms:
  \begin{itemize}
    \item Configuration variable \var{BOOT\_TYPE} from the
          \texttt{$<$config$>$/base.txt}
    \item Configuration file \texttt{$<$config$>$/mkfli4l.txt}
    \item Parameter of the build program
    \item Interactive settings in the GUI
  \end{itemize}

  The variable \jump{BOOTTYPE}{\var{BOOT\_TYPE}} decides which action
  the Build program executes:
  \begin{itemize}
    \item Create a bootable fli4l CD-ISO-Image
    \item Making the fli4l files available, for remote update
    \item Generating the fli4l files and direct remote update via SCP
    \item Hard drive pre-install of a suitably formatted CF in the Cardreader
    \item a.s.o.
  \end{itemize}

  The description of the variables in the configuration file
  \texttt{$<$config$>$/mkfli4l.txt} can be found in Chapter
  \jump{sec:mkfli4lconf}{Control file mkfli4l.txt}.

  \subsection{Command line options}
  The last control mechanism is appending of option parameters to the call of the Build program
  on the command line. The control options correspond to those in the control file \texttt{mkfli4l.txt}. 
  Option parameters override the values from the control file.
  Out of convenience, the names of the option parameters differ from the names of
  the variables from the control file. There is a long and, to some extent, a
  short form:

  \begin{verbatim}
Usage: mkfli4l.bat [options] [config-dir]

-c, --clean             cleanup the build-directory
-b, --build <dir>       sets build-directory to <dir> for the fli4l-files
-v, --verbose           verbose - some debug-output
    --filesonly         creates only fli4l-files - does not create a disk
    --no-squeeze        don't compress shell scripts
-h, --help              display this usage

config-dir              sets other config-directory - default is "config"

*** Remote-Update options
--remoteupdate          remote-update via scp, implies "--filesonly"
--remoteuser <name>     user name for remote-update - default is "fli4l"
--remotehost <host>     hostname or IP of remote machine - default
                        is HOSTNAME set in [config-dir]/base.txt
--remotepath <path>     pathname on remote machine - default is "/boot"
--remoteport <portnr>   portnumber of the sshd on remote machine

*** GUI-Options
--nogui                 disable the config-GUI
--lang                  change language
                        [deutsch|english|espanol|french|magyar|nederlands]

  \end{verbatim}

  An alternative configuration directory can be passed by appending its name to the end of the command line.
  The normal configuration directory is called \texttt{config} and can be found under the fli4l
  root directory. This is where all fli4l packages place their configuration files.
  If you want to maintain more than one configuration, create another directory, e.g. \texttt{hd.conf},
  place a copy of the configuration files there and change it according to the requirements.
  Here are some examples:

  \begin{verbatim}
     mkfli4l.bat hd.conf
     mkfli4l.bat -v
     mkfli4l.bat --no-gui config.hd
  \end{verbatim}

  \subsection{Configuration dialog~-- Setting the configuration directory}

  In the main window the configuration directory setting is indicated and a window can be opened for
  the selection of the configuration directory.\\

  It should be noted that any change in the 'Config-Dir' causes all options to be set to the values contained in the
  \jump{sec: mkfli4lconf}{control file 'mkfli4l.txt'} placed in that directory, or to the values
  given as command-line parameters, respectively.\\

  If mkfli4l.bat does not find a directory fli4l-x.y.z$\backslash$config or if there is no file in that directory
  named `base.txt', a window is immediately opened for the selection of the configuration directory.
  This makes it possible to easily manage several fli4l configuration directories in a simple manner.\\

  Example:

\begin{example}
\begin{verbatim}
          fli4l-x.y.z\config
          fli4l-x.y.z\config.fd
          fli4l-x.y.z\config.cd
          fli4l-x.y.z\config.hd
          fli4l-x.y.z\config.hd-create
\end{verbatim}
\end{example}

  \subsection{Configuration dialog~-- General Preferences}
  \begin{figure}[ht!]
  \centering
  \includegraphics[width=\columnwidth]{win_build_build}
  \caption{Preferences}
  \label{fig:win_build_build}
  \end{figure}

  In this dialogue the settings are specified for the archive/boot-media
  creation:
  \begin{itemize}
    \item Build-Dir~-- Directory for the Archives/CD-Images/...
    \item \var{BOOT\_TYPE}~-- Display of the utilized/settings \var{BOOT\_TYPE}~-- unchangeable
    \item Verbose~-- Activation of additional output during the creation
    \item Filesonly~-- Only the archives are created~-- no bootmedia/no image
    \item Remoteupdate~-- Activation of the remote update via SCP
  \end{itemize}

  Using the button \textbf{Current settings in mkfli4l.txt buffer}
  the current settings can be stored in mkfli4l.txt.

  \subsection{Configuration dialog~-- Settings for Remote update}
  \begin{figure}[ht!]
  \centering
  \includegraphics[width=\columnwidth]{win_build_remoteupdate}
  \caption{Settings for Remote update}
  \label{fig:win_build_remoteupdate}
  \end{figure}

  In this dialogue the settings for Remote update are specified:
  \begin{itemize}
    \item IP address or Hostname
    \item User name on the Remote host
    \item Remote path (default: /boot)
    \item Remote port (default: 22)
    \item SSH keyfile to use (format ppk from Putty)
  \end{itemize}

  \subsection{Configuration dialog~-- Settings for HD pre-install}
  \begin{figure}[ht!]
  \centering
  \includegraphics[width=\columnwidth]{win_build_hd_install}
  \caption{Settings for HD pre-install}
  \label{fig:win_build_hd_install}
  \end{figure}

   In this dialogue the options are set for HD pre-install on an
   accordingly partitioned and formatted Compact Flash card in a USB reader.

   Possible Options:
   \begin{itemize}
     \item Activate HD pre-install
     \item Drive letter to be used to access the CF card
  \end{itemize}

  Regarding the partitioning and formatting of the CF:
  A Type-A HD installation (see package HD) must be based on a primary, active,
  and formatted FAT partition on the CF card. If you would like to use a data partition
  additionally, a Linux partition which is formatted with the ext3 file system, as well as
  the file \texttt{hd.cfg} are also needed on the FAT Partiton (in this case please
  make sure to read the documentation of the HD package).

\marklabel{sec:mkfli4lconf}{
  \section{Control file mkfli4l.txt}}
  Since fli4l-Version 2.1.9 the control file
  \texttt{$<$config$>$/mkfli4l.txt} exists. This file can e.g. be used to specify
  directories which differ from the standard settings.
  The control file has a similar structure as the normal fli4l configuration files.
  All configuration variables here are optional, i.e. they need not exist or they can be commented out.
  \begin{description}

  \config {BUILDDIR}{BUILDDIR}{BUILDDIR}

  Default: `build'

  Specifies the directory where fli4l files will be created. If the variable is undefined,
  the Windows mkfli4l sets it to `build' relative to the fli4l root directory, resulting in
  the directory.
  \texttt{build} in the fli4l root directory:
  \begin{verbatim}
    Path/fli4l-x.y.z/build
  \end{verbatim}
  \vspace{-2ex}
  Under *nix mkfli4l is using \texttt{$<$config$>$/build} and is thus filing the
  generated files together with the configuration.

  The path for \var{BUILDDIR} must use the conventions of the Operating Systems
  Windows oder *nix. If relative paths configured there are converted by the build
  to the syntax of windows or *nix.

  \config {VERBOSE}{VERBOSE}{VERBOSE}

  Default: \var{VERBOSE='no'}

  Possible values are \var{'yes'} or \var{'no'}. Controls the \emph{Verbosity}
  of the Build Processes.

  \config {FILESONLY}{FILESONLY}{FILESONLY}

  Default: \var{FILESONLY='no'}

  Possible values are \var{'yes'} or \var{'no'}. This will actually
  turn off the creation of the boot-media and only the files will be created~--

  \config {REMOTEUPDATE}{REMOTEUPDATE}{REMOTEUPDATE}

  Default: \var{REMOTEUPDATE='no'}

  Possible values are \var{'yes'} or \var{'no'}. Enables automatic
  transferring of files by means of SCP to the router. This requires
  the package \jump{OPTSSHD}{SSHD} with activated \texttt{scp}.
  See also the following variables.

  \config {REMOTEHOSTNAME}{REMOTEHOSTNAME}{REMOTEHOSTNAME}

  Default: \var{REMOTEHOSTNAME=''}

  The target host name for the SCP data transfer.
  If no name is set, the variable
  \jump{HOSTNAME}{\var{HOSTNAME}} is used.

  \config {REMOTEUSERNAME}{REMOTEUSERNAME}{REMOTEUSERNAME}

  Default: \var{REMOTEUSERNAME='fli4l'}

  User name for the SCP data transfer.

  \config {REMOTEPATHNAME}{REMOTEPATHNAME}{REMOTEPATHNAME}

  Default: \var{REMOTEPATHNAME='/boot'}

  Destination path for the SCP data transfer.

  \config {REMOTEPORT}{REMOTEPORT}{REMOTEPORT}

  Default: \var{REMOTEPORT='22'}

  Destination port for the SCP data transfer.

  \config {SSHKEYFILE}{SSHKEYFILE}{SSHKEYFILE}

  Default: \var{SSHKEYFILE=''}

  Here you can specify a SSH key file for the SCP Remote update.
  Thus, an update can be made without specifying a password.

  \config {REMOTEREMOUNT}{REMOTEREMOUNT}{REMOTEREMOUNT}

  Default: \var{REMOTEREMOUNT='no'}

  Possible values are \var{'yes'} or \var{'no'}. If \var{'yes'}
  is set, a boot device "/boot" mounted read-only will be remounted
  read-write to allow remote updates of the boot files.

  \config {TFTPBOOTPATH}{TFTPBOOTPATH}{TFTPBOOTPATH}

  Path where the remote Netboot image is saved to.

  \config {TFTPBOOTIMAGE}{TFTPBOOTIMAGE}{TFTPBOOTIMAGE}

  Name of the Netboot image.

  \config {PXESUBDIR}{PXESUBDIR}{PXESUBDIR}

  Subdirectory for the PXE files relative to TFTPBOOTPATH.


  \config {SQUEEZE\_SCRIPTS}{SQUEEZE\_SCRIPTS}{SQUEEZESCRIPTS}

   Enable or disable the Squeezing (Compression) scripts.
   Compressing a script with Squeeze removes all comments and line indentations.
   Under normal conditions the default value of \var{'yes'} can be used.

  \config {MKFLI4L\_DEBUG\_OPTION}{MKFLI4L\_DEBUG\_OPTION}{MKFLI4LDEBUGOPTION}

   Additional debugging options can be handed over to the\jump{mkfli4l}{mkfli4l-Programm}.

  \end{description}

  \chapter{Connecting PCs in the LAN}

  For every host in the LAN you will have to set up:

  \begin{enumerate}
  \item IP address (see \smalljump{sec:pc-lan-ip}{IP address})
  \item Name of the host plus desired domain name
    (see \smalljump{sec:pc-lan-name}{Host and domain name})
  \item Default gateway (see \smalljump{sec:pc-lan-gateway}{Gateway})
  \item IP address of the DNS server (see \smalljump{sec:pc-lan-dns}{DNS server})
  \end{enumerate}

  \marklabel{sec:pc-lan-ip}{\section{IP address}}
  The IP address of the host has to belong to the same network as the IP
  address of the fli4l router (on the Ethernet interface), for example
  192.168.6.2 in case the router has the IP address 192.168.6.1.
  IP addresses have to be unique throughout the network, so it's a good idea to
  change (only) the last number. You will also have to make sure you specify
  the same IP address as specified in the file config/base.txt.

  \marklabel{sec:pc-lan-name}{\section{Host and domain name}}
  The name of the host is for example ``my-pc'', the domain ``lan.fli4l''.

  \wichtig{The domain set up on the host has to be identical to the domain
  set up on the fli4l if you want to use fli4l as a DNS server. Otherwise
  it could cause massive problems in the network.}

  The reason: Windows hosts regularly search for hosts within their
  workgroup, trying to resolve the name WORKGROUP.my-domain.fli4l. If the domain (here:
  my-domain.fli4l) doesn't match the one set up on the router, fli4l will try
  to answer the query by forwarding it to the Internet \ldots

  The domain has to be entered in the TCP/IP settings of the host.

  \subsection{Windows 2000}

  On Windows 2000 the settings can be found under:

  \noindent Start \pfeil\\
  \hspace*{2ex}Settings \pfeil\\
  \hspace*{4ex}Control Center \pfeil\\
  \hspace*{6ex}Network and Dial-up Connections \pfeil\\
  \hspace*{8ex}LAN Connection \pfeil\\
  \hspace*{10ex}Properties \pfeil\\
  \hspace*{12ex}Internet protocol (TCP/IP) \pfeil\\
  \hspace*{14ex}Properties \pfeil\\
  \hspace*{16ex}Extended\ldots \pfeil\\
  \hspace*{18ex}DNS \pfeil\\
  \hspace*{20ex}Add DNS-Suffix \pfeil\\

  Type ``lan.fli4l'' (or the domain set up~-- without ``''!)
  \pfeil Click OK.


\subsection{NT 4.0}

  Start \pfeil\\
  \hspace*{2ex}Settings \pfeil\\
  \hspace*{4ex}Control Center \pfeil\\
  \hspace*{6ex}Network \pfeil\\
  \hspace*{8ex}Protocols \pfeil\\
  \hspace*{10ex}TCP/IP \pfeil\\
  \hspace*{12ex}Properties \pfeil\\
  \hspace*{14ex}DNS \pfeil\\
  \hspace{16ex}\begin{itemize}
  \item Enter hostname (of the client)
  \item Enter domain (same as in config/base.txt)
  \item Add IP address of fli4l router
  \item Add DNS suffix (add domain~-- see two lines above)
  \end{itemize}

\subsection{Win95/98}

  Start \pfeil\\
  \hspace*{2ex}Settings \pfeil\\
  \hspace*{4ex}Control Center \pfeil\\
  \hspace*{6ex}Network \pfeil\\
  \hspace*{8ex}Configuration \pfeil\\
  \hspace*{10ex}TCP/IP (the one that is bound to the network\\
  \hspace*{10ex}interface to the router) \pfeil\\
  \hspace*{12ex}Properties \pfeil\\
  \hspace*{14ex}DNS Configuration:

  Activate DNS and under ``Domain:'' enter ``lan.fli4l'' (or the domain
  set up~-- without ``''!)

\subsection{Windows XP}

  On Windows XP the settings can be found at:

  \noindent Start \pfeil\\
  \hspace*{2ex}Settings \pfeil\\
  \hspace*{4ex}System Settings \pfeil\\
  \hspace*{6ex}Network Connections \pfeil\\
  \hspace*{8ex}LAN-Connection\pfeil\\
  \hspace*{10ex}Properties \pfeil\\
  \hspace*{12ex}Internetprotocol (TCP/IP) \pfeil\\
  \hspace*{14ex}Properties \pfeil\\
  \hspace*{16ex}Advanced\ldots \pfeil\\
  \hspace*{18ex}DNS \pfeil\\
  \hspace*{20ex}DNS-Suffix for this connection \pfeil\\

  Specify ``lan.fli4l'' (resp. the domain you use) (without ``''!)
  \pfeil Press OK.

\subsection{Windows 7}

  On Windows 7 the settings can be found at:

  \noindent Windows Button (ex. Start) \pfeil\\
  \hspace*{2ex}System settings \pfeil\\
  \hspace*{4ex}Network and Internet \pfeil\\
  \hspace*{6ex}Network- and Sharecenter \pfeil\\
  \hspace*{8ex}LAN-Connection\pfeil\\
  \hspace*{10ex}Properties \pfeil\\
  \hspace*{12ex}Internetprotocol Version 4 (TCP/IPv4) \pfeil\\
  \hspace*{14ex}Properties \pfeil\\
  \hspace*{16ex}Advanced \ldots \pfeil\\
  \hspace*{18ex}DNS \pfeil\\
  \hspace*{20ex}DNS-Suffix for this connection \pfeil\\

  Specify ``lan.fli4l'' (resp. the domain you use) (without ``''!)
  \pfeil Press OK.

\subsection{Windows 8}

  On Windows 8 the settings can be found at:

  \noindent Press Windows- and X-key simultaneously \pfeil\\
  \hspace*{2ex}System settings \pfeil\\
  \hspace*{4ex}Network and Internet \pfeil\\
  \hspace*{6ex}Network- and Sharecenter \pfeil\\
  \hspace*{8ex}Choose your net (Ehternet or WLAN) \pfeil\\
  \hspace*{10ex}Properties \pfeil\\
  \hspace*{12ex}Internetprotocol Version 4 (TCP/IPv4) \pfeil\\
  \hspace*{14ex}Properties \pfeil\\
  \hspace*{16ex}Advanced \ldots \pfeil\\
  \hspace*{18ex}DNS \pfeil\\
  \hspace*{20ex}DNS-Suffix for this connection \pfeil\\

  ``lan.fli4l'' (bzw. die eingestellte domain) eingeben (ohne ``''!)
  \pfeil OK drücken.
  \marklabel{sec:pc-lan-gateway}{\section{Gateway}}

  It is absolutely necessary to specify a default gateway, because without
  the correct IP address provided here nothing will work.
  So you will have to specify the IP address of the fli4l router here (the
  Ethernet interface's one)~-- for example 192.168.6.4, depending on the IP address
  that has been specified in the file config/base.txt for the router.

  It is wrong to enter fli4l as a proxy in the Windows or browser configuration
  unless you use a proxy on the router. Normally fli4l is not a proxy, thus
  please do \emph{not} specify fli4l as a proxy!

\marklabel{sec:pc-lan-dns}{\section{DNS server}}

  As for the IP address, you should not specify the IP address of the provider's
  DNS server, but the address of the router (Ethernet interface), as it
  will anwser queries itself or forward them to the Internet if needed.

  When fli4l is used as a DNS server, many queries from Windows hosts are not
  forwarded to the Internet but answered by fli4l itself.

\marklabel{sec:pc-lan-misc}{\section{Miscellaneous}}

  The items 1 to 4 do not have to be specified when a DHCP server is
  configured as fli4l transmits the needed data automatically.

  \textbf{Internet options:} Under connections you will have to select ``do not dial''.
  Under settings for local network (LAN): Do NOT enter anything (unless you use
  \var{OPT\_\-P}roxy). Both are default settings and should already exist.

\marklabel{IMONDSCHNITTSTELLE}{
  \chapter{Client/Server interface imond}
  }

  \marklabel{sec:imond}{
    \section{imon-Server imond}}

  imond is a network-capable server program that responds to certain queries
  or accepts commands that can control the router.

  imond also controls the Least-Cost-Routing. It uses the configuration
  file /etc/imond.conf, that is created automatically from the variables
  \var{ISDN\_\-CIRC\_\-x\_\-XXX} from the file config/isdn.txt and other
  at boot time by a shell script.

  imond runs permanentely as daemon and listens on TCP/IP port 5000 and the
  device /dev/isdninfo.


  All possible commands that can be sent to TCP/IP port 5000:
  \begin{table}
    \textbf{Admin commands}

    \vspace{1ex}
    \begin{tabular}{lp{9cm}}

      addlink ci-index              & Add channel to the circuit
                                      (channel bundling) \\
      adjust-time seconds           & Increments the date on the router by
                                      the number of seconds specified \\
      delete filename pw            & Deletes the file on the router \\
      hup-timeout \#ci-index [value]& Show or set the HUP timeout for
                                      ISDN circuits \\
      removelink ci-index           & Remove additional channel \\
      reset-telmond-log-file        & Deletes the telmond log file \\
      reset-imond-log-file          & Deletes the imond log file \\
      receive filename \#bytes pw   & Transfer a file to the router.
                                      Imond acknowledges the command using
                                      an ACK (0x06). After that, the file is
                                      transfered in blocks of 1024 bytes
                                      that are also acknowledged with an ACK.
                                      Finally, imond replies with an OK. \\
      send filename pw              & If the password is correct and the file
                                      exists, imond replies with OK \#bytes.
                                      Then, imond transfers the file in blocks
                                      of 1024 bytes that have to be
                                      acknowledged with an ACK (0x06).
                                      Finally, imond replies with an OK. \\
      support pw                    & Shows the status/configuration of the
                                      router \\
      sync                          & Syncs the cache of mounted drives \\
    \end{tabular}
  \end{table}

  \begin{table}
    \textbf{Admin or User commands}

    \vspace{1ex}
    \begin{tabular}{lp{9cm}}

      dial                      &    Dials the provider
                                     (Default-Route-Circuit) \\
      dialmode [auto|manual|off]&    Shows or sets the dialmode \\
      disable                   &    Hangs up and sets the dialmode to
                                     ``off'' \\
      enable                    &    Sets the dialmode to ``auto'' \\
      halt                      &    Cleanly shuts down the router \\
      hangup [\#channel-id]     &    Hangs up \\
      poweroff                  &    Shuts down the router and powers it off \\
      reboot                    &    Reboots the router \\
      route [ci-index]          &    Set the default route to circuit X
                                     (0=automatically) \\
    \end{tabular}
  \end{table}


  \begin{table}
    \textbf{User commands}

    \vspace{1ex}
    \begin{tabular}{lp{9cm}}
      channels                  & Shows the number of available ISDN
                                  channels\\
      charge \#channel-id       & Shows the online fee for a specific
                                  channel\\
      chargetime \#channel-id   & Time charged in consideration of the
                                  charge interval\\
      circuit [ci-index]        & Shows a circuit name\\
      circuits                  & Shows number of default-route-circuits\\
      cpu                       & Shows the CPU load in percent\\
      date                      & Shows date/time\\
      device ci-index           & Shows the device of the circuit\\
      driverid \#channel-id     & Shows driver-id of the channel X\\
      help                      & Shows help\\
      inout \#channel-id        & Shows direction (incoming/outgoing)\\
      imond-log-file            & Shows imond log file\\
      ip \#channel-id           & Shows IP\\
      is-allowed command        & Shows, whether a command is
                                  configured/allowed\newline
                                  Possible commands:
                                    dial|dialmode|route|reboot|
                                    imond-log|telmond-log|mgetty-log \\
      is-enabled                & Shows, whether dialmode is off (0) or auto
                                  (1)\\
      links ci-index            & Show number of channels 0, 1 or
                                  2, 0 means: No channel bundling possible\\
      log-dir imond|telmond|mgetty& Shows the log directory\\
      mgetty-log-file           & Shows mgetty logfile\\
      online-time \#channel-id  & Shows online time of the current connection
                                  in hh:mm:ss\\
      pass [password]           & Show, whether it is necessary to enter a
                                  password or enter a password\newline
                                  1 Userpassword is set\newline
                                  2 Adminpassword is set\newline
                                  4 imond is in admin mode\\
      phone \#channel-id        & Show telephone number/name of the peer\\
      pppoe                     & Show the number of pppoe devices (i.e. 0
                                  or 1)\\
      quantity \#channel-id     & Show the data transferred in bytes\\
      quit                      & Terminates the connection to imond\\
      rate \#channel-id         & Show transfer rates (incoming/outgoing
                                  in B/sec)\\
      status \#channel-id       & Show status of channel X\\
      telmond-log-file          & Shows telmond log files\\
      time \#channel-id         & Show the sum of online times, format
                                  hh:mm:ss\\
      timetable [ci-index]      & Shows the time table for the LC-Routing\\
      uptime                    & Shows the uptime of the router in seconds\\
      usage \#channel-id        & Show the type of connection, that is available
                                  responses: Fax, Voice, Net, Modem, Raw\\
      version                   & Show the protocol and program version\\
    \end{tabular}
  \end{table}

The TCP/IP port 5000 is only reachable from the masqueraded LAN.
  Access from remote is blocked by the firewall configuration by default.

  Imond supports two user levels: the user and the admin mode. For both
  levels you can set a password using var{IMOND\_\-PASS} and/or
  \var{IMOND\_ADMIN\_\-PASS}. Then, clients are forced by imond to
  submit a password. As long as no password has been submitted, only the
  commands ``pass'' and``quit'' are accepted. Others are rejected.

  If you want to further restrict access, e.g. only allow access
  from a single computer, the firewall configuration has to be changed.

  At present this is not possible using the standard configuration files
  config/base.txt. You will have to change the file /etc/rc.d/rc322.masq.

  The commands

\begin{example}
\begin{verbatim}
         enable/disable/dialmode   dial/hangup   route   reboot/halt
\end{verbatim}
\end{example}

  Can be globally enabled/disabled using the configuration variables
  \var{IMOND\_\-XXX} (see ``Configuration'').

  From a Unix/Linux computer (or a Windows computer in a DOS box) you can
  easily try it out:
  Type

\begin{example}
\begin{verbatim}
        telnet fli4l 5000        \# or the appropriate name of the fli4l-Routers
\end{verbatim}
\end{example}

  and you will be able to directly enter the listed commands and look at
  the output.

  For example after entering ``help'' the help is shown, after
  ``quit'' the connection to imond is terminated.

\marklabel{sec:leastcostrouting}{
  \subsection{Least-Cost-Routing~-- how it works}
  }

  imond contructs a table (time table) from the configuration file
  /etc/imond.conf (which is created on bootup from the config variables
  \var{ISDN\_\-CIRC\_\-x\_\-TIMES} and others). It contains a complete
  calendar week in a raster of 1 hour (168 hours = 168 Bytes). But the
  table only contains the circuits that have a default route defined.

  Using the imond command ``timetable'' you can have a look at it.

  Here an example:

  Supposing 3 circuits are defined:

\begin{example}
\begin{verbatim}
        CIRCUIT_1_NAME='Addcom'
        CIRCUIT_2_NAME='AOL'
        CIRCUIT_3_NAME='Firma'
\end{verbatim}
\end{example}

   Only the first two circuits have a default circuit defined, i.e. the
  corresponding variables ISDN\_CIRC\_x\_ROUTE have the value '0.0.0.0'.

  If the variables \var{ISDN\_\-CIRC\_\-x\_\-TIMES} look like this:

\begin{example}
\begin{verbatim}
        ISDN_CIRC_1_TIMES='Mo-Fr:09-18:0.0388:N Mo-Fr:18-09:0.0248:Y
                      Sa-Su:00-24:0.0248:Y'

        ISDN_CIRC_2_TIMES='Mo-Fr:09-18:0.019:Y Mo-Fr:18-09:0.049:N
                      Sa-Su:09-18:0.019:N Sa-Su:18-09:0.049:N'

        ISDN_CIRC_3_TIMES='Mo-Fr:09-18:0.08:N Mo-Fr:18-09:0.03:N
                      Sa-Su:00-24:0.03:N'
\end{verbatim}
\end{example}

  it results in the following /etc/imond.conf being created:

\begin{example}
\begin{verbatim}
        #day  hour  device  defroute  phone        name        charge  ch-int
        Mo-Fr 09-18 ippp0   no        010280192306 Addcom      0.0388   60
        Mo-Fr 18-09 ippp0   yes       010280192306 Addcom      0.0248   60
        Sa-Su 00-24 ippp0   yes       010280192306 Addcom      0.0248   60
        Mo-Fr 09-18 ippp1   yes       019160       AOL  0.019   180
        Mo-Fr 18-09 ippp1   no        019160       AOL  0.049   180
        Sa-Su 09-18 ippp1   no        019160       AOL  0.019   180
        Sa-Su 18-09 ippp1   no        019160       AOL  0.049   180
        Mo-Fr 09-18 isdn2   no        0221xxxxxxx  Firma       0.08     90
        Mo-Fr 18-09 isdn2   no        0221xxxxxxx  Firma       0.03     90
        Sa-Su 00-24 isdn2   no        0221xxxxxxx  Firma       0.03     90
\end{verbatim}
\end{example}

  imond creates the following time table in memory~-- here the output
  of the imond command ``timetable'':

\begin{example}
\begin{verbatim}
         0  1  2  3  4  5  6  7  8  9 10 11 12 13 14 15 16 17 18 19 20 21 22 23
     --------------------------------------------------------------------------
     Su  3  3  3  3  3  3  3  3  3  3  3  3  3  3  3  3  3  3  3  3  3  3  3  3
     Mo  2  2  2  2  2  2  2  2  2  4  4  4  4  4  4  4  4  4  2  2  2  2  2  2
     Tu  2  2  2  2  2  2  2  2  2  4  4  4  4  4  4  4  4  4  2  2  2  2  2  2
     We  2  2  2  2  2  2  2  2  2  4  4  4  4  4  4  4  4  4  2  2  2  2  2  2
     Th  2  2  2  2  2  2  2  2  2  4  4  4  4  4  4  4  4  4  2  2  2  2  2  2
     Fr  2  2  2  2  2  2  2  2  2  4  4  4  4  4  4  4  4  4  2  2  2  2  2  2
     Sa  3  3  3  3  3  3  3  3  3  3  3  3  3  3  3  3  3  3  3  3  3  3  3  3

     No.  Name                   DefRoute  Device  Ch/Min   ChInt
      1   Addcom                   no      ippp0   0.0388     60
      2   Addcom                   yes     ippp0   0.0248     60
      3   Addcom                   yes     ippp0   0.0248     60
      4   AOL               yes     ippp1   0.0190    180
      5   AOL               no      ippp1   0.0490    180
      6   AOL               no      ippp1   0.0190    180
      7   AOL               no      ippp1   0.0490    180
      8   Firma                    no      isdn2   0.0800     90
      9   Firma                    no      isdn2   0.0300     90
     10   Firma                    no      isdn2   0.0300     90
\end{verbatim}
\end{example}

  For circuit 1 (Addcom) there are three time ranges (1-3) defined.
  For circuit 2 (AOL) there are four time ranges (4-7) and
  for the last one there are three time ranges (8-10).

  In the time table, the indices are printed that are valid in the
  corresponding hour. Only the indices 2-4 show up here, as the others
  are not default routes.

  If there are zeros in the table, there are gaps in the values of the
  \var{ISDN\_\-CIRC\_\-X\_\-TIMES} variables. At this point there is no
  default route, no internet access is possible!

  On program start, imond checks for the weekday and the hour. Then, the
  index from the time table is picked out and the corresponding circuit.
  The default route is then set to this circuit.

  If the status of a channel changes (e.g. offline~-- online) or at least
  after one minute, this procedure is repeated: check time, lookup
  in table, pick default route circuit.

  If the used circuit changes, e.g. on mondays, 18:00, the default route
  is deleted, existing connections are terminated (sorry\ldots) and after
  that the default route is set to the new circuit. Imond may notice this
  up to 60 seconds too late~-- so at least at 18:00:59 the route is
  changed.

  If a circuit does not have a default route, nothing will change. The value
  of \var{ISDN\_\-CIRC\_\-x\_\-TIMES} is only used to calculate the fee.
  This can be important if the LC routing is disabled temporarily, e.g. using
  the client imonc, and a circuit is dialed manually.

  But you can have a look at the tables for other time-range-indices
  (in our example 1 to 10), also at the ones of the
  ``Non-LC-Default-Route-Circuits''.

  Command:

\begin{example}
\begin{verbatim}
                    timetable index
\end{verbatim}
\end{example}

  Example:

\begin{example}
\begin{verbatim}
                    telnet fli4l 5000
                    timetable 5
                    quit
\end{verbatim}
\end{example}

  The output will look like:

\begin{example}
\begin{verbatim}
         0  1  2  3  4  5  6  7  8  9 10 11 12 13 14 15 16 17 18 19 20 21 22 23
     --------------------------------------------------------------------------
     Su  0  0  0  0  0  0  0  0  0  0  0  0  0  0  0  0  0  0  0  0  0  0  0  0
     Mo  5  5  5  5  5  5  5  5  5  0  0  0  0  0  0  0  0  0  5  5  5  5  5  5
     Tu  5  5  5  5  5  5  5  5  5  0  0  0  0  0  0  0  0  0  5  5  5  5  5  5
     We  5  5  5  5  5  5  5  5  5  0  0  0  0  0  0  0  0  0  5  5  5  5  5  5
     Th  5  5  5  5  5  5  5  5  5  0  0  0  0  0  0  0  0  0  5  5  5  5  5  5
     Fr  5  5  5  5  5  5  5  5  5  0  0  0  0  0  0  0  0  0  5  5  5  5  5  5
     Sa  0  0  0  0  0  0  0  0  0  0  0  0  0  0  0  0  0  0  0  0  0  0  0  0

     No.  Name                   DefRoute  Device  Ch/Min   ChInt
      5   AOL               no      ippp1   0.0490    180
\end{verbatim}
\end{example}

  Got everything?

  Using the command ``route'', the LC routing can be enabled or disabled.
  If a positive circuit index is specified (1\ldots N) the default route
  is changed to the circuit specified. If the index is 0, LC routing will be
  activated again and the active circuit is chosen automatically.


  \subsection{Annotations to the calculation of the online changes}

  The whole model how the online charges are calculated will only work
  correctly, if the charge interval for a single circuit (variable
  \var{ISDN\_\-CIRC\_\-x\_\-CHARGEINT}) remains constant throughout the whole
  week.

  Normally, this is correct for most of the internet providers. But if you
  dial in, e.g. to your companies network, using the (German) Telecom (not
  the internet provider T-Online), the change interval changes at 18:00 from
  90 seconds to 4 minutes (information from June 2000). Because of that, the
  definition

\begin{example}
\begin{verbatim}
        ISDN_CIRC_3_CHARGEINT='90'
        ISDN_CIRC_3_TIMES='Mo-Fr:09-18:0.08:N Mo-Fr:18-09:0.03:N Sa-Su:00-24:0.03:N'
\end{verbatim}
\end{example}

  is not absolutely correct. After 18:00 the price is converted to 3
  cents (4 minutes cost 12 cents), but the charge interval is wrong.
  Because of that, the displayed charge could differ from the actual one.

  Here is a tip, how different charge intervals can be handled correctly,
  anyhow (also important for \var{ISDN\_\-CIRC\_\-x\_\-CHARGEINT}):
  Just define 2 cicuits~-- one for each charge interval.
  Of course you will have to adjust \var{ISDN\_\-CIRC\_\-x\_\-TIMES}
  so that the valid circuit is always dialed, depending on the charge
  interval.

  Once again: If you connect to an ISP you most likely will not have this
  problem, because the charge interval is constant all the time and
  only the prices per minute change (or doesn't it? I guess the German
  provider T-* could even introduce such a price model :-).

  % Do not remove the next line
% Synchronized to r30003

  \marklabel{sec:winimonc}{
    \section{Client Windows imonc.exe}}

  \subsection{Introduction}

  Le démon Imond sur le routeur fli4l gère deux modes d'utilisations différents~:
  le mode Administrateur (Admin) et le mode Utilisateur. Dans le mode Admin
  toutes les commandes sont activées automatiquement. Dans le mode Utilisateur
  vous devez activer les variables \jump{IMONDENABLE}{\var{IMOND\_ENABLE}},
  \jump{IMONDDIAL}{\var{IMOND\_DIAL}}, \jump{IMONDROUTE}{\var{IMOND\_ROUTE}} et
  \jump{IMONDREBOOT}{\var{IMOND\_REBOOT}}, dans le fichier /config/base.txt pour avoir
  les commandes. Si les variables sont sur `no' les commandes ne seront pas activées,
  même les commandes Exit et mode Admin ne seront pas activées dans le client imonc.
  Le choix de l'utilisation entre le mode Utilisateur et le mode Admin se fait
  par l'intermédiaire d'un Mot de Passe qui sera transféré au routeur. Vous pouvez
  activer le mode Admin ou Utilisateur, en cliquant sur l'icone située dans
  la barre de taches et entrer le Mot de Passe n'oubliez pas de redémarrer
  le client imonc.

  Lorsque imonc a démarré, une icône supplémentaire apparait dans la barre de taches,
  il indique le statut des canaux de la connexion Internet pour le (numéris).

  Les couleurs de l'icône ignifient~:

  \begin{description}
    \item[Rouge]~: offline (déconnecté)
    \item[Jaune]~: en cours de connexion 
    \item[Vert clair]~: online (en ligne il y a du trafic sur le canal)
    \item[Vert foncé]~: online (en ligne il n'y a pas de trafic sur le canal)
  \end{description}

  \noindent Suivant le Windows que vous utilisez le comportement d'imonc diverge,
  il peut être réduit à une icone dans la barre des taches près de l'heure. pour
  ouvrir la fenêtre il suffit de faire un double clic avec le bouton gauche de
  la souris sur l'icone. Pour ouvrir le menu contextuel vous utilisez le bouton
  droit, delà vous pouvez choisir directement les commandes imonc.

  Un (grand nombre de paramètres) peuvent être adaptés selon vos propres besoins,
  ils seront enregistrés et sauvegardés dans la base de registre de Windows à cet
  endroit HKCU{\textbackslash}Software{\textbackslash}fli4l.

  Il y a toujours quelques erreurs dans la documentation d'imonc et du routeur
  fli4l, malgré des relectures. Si vous rencontrez des problèmes, allez dans la
  page "A propos" cliquer sur le bouton systeminfo puis sur le bouton support info,
  ensuite le mot de passe du routeur vous sera demandé (pas le mot de passe d'imond~!).
  Imonc produira un fichier fli4lsup.txt, qui inclura toutes les informations
  importantes sur le routeur fli4l et sur imonc. Ce fichier peut être ajouté
  dans le Newsgroup pour demander de l'aide. Cela maximisera les chances
  d'avoir de l'aide plus rapidement.

  Vous pouvez trouve des détails concernant le développement du client imonc pour
  Windows sur le site \altlink{http://www.imonc.de/}, vous trouverez des informations
  sur les nouveaux dispositifs les futures versions d'imonc, les résolutions de bug
  et aussi la dernière version à télécharger (si elle n'est pas déjà inclue dans
  la distribution fli4l).

  \subsection{Paramètre de démarrage}

  Le client imonc à besoin du Nom ou de Adresse IP du routeur fli4l pour pouvoir
  établir une connexion avec celui-ci "l'ordinateur fli4l". Si l'ordinateur du client
  imonc est enregistré correctement dans le DNS, il devrait fonctionner sans problème.
  Voici les paramètres que l'on peut transmettre~: 

  \begin{itemize}
    \item /Server:IP ou Nom d'Hôte du routeur (Forme abrégée~: /S:IP ou Nom d'Hôte)
    \item /Password:Mot de Passe (Forme abrégée: /P:Mot de Passe)
    \item /log Active le protocole de communication entre imonc et imond, lorsque
      cette option est activée un fichier imonc.log est créé. Ce fichier enregistre
      toutes les communications, il peut être très volumineux. C'est pour cette
      raison que l'on active ce paramètre uniquement si il y a des problèmes de
      configurations.
    \item /iport:N$^\circ$ port Par défaut imond écoute sur le Port~: 5000
    \item /tport:N$^\circ$ port Par défaut telmond écoute sur le Port~: 5001
    \item /rc:"Commande" Les commandes écrites ici sont transmis au routeur sans
      aucun contrôle supplémentaire. Si plusieurs commandes sont exportées
      simultanément elles doivent être séparées par un point virgule. Pour être
      sur du fonctionnement de imonc vous devez retaper le Mot de Passe
      (si configuré?) car il n'y aura aucune redemande de Mot de Passe.
      les commandes possibles sont documentées dans le Chapitre 8.1. La commande
      dialtimesync n'ai plus utilisée elle est remplacée par \flqq{}dial; timesync\frqq{},
      qui force le routeur à synchroniser l'heure avec le client.
    \item /d:"Répertoire-fli4l" Cette option permet d'écrire le répertoire du
      dossier fli4l avec des paramètres de démarrage, c'est intéressant pour ceux
      qui utilisent plusieurs versions de fli4l.
    \item /wait Si le Nom d'Hôte ne peut pas être résolu, imonc se bloque,
      il faut redémarrer imonc par un double clic sur l'icône de celui-ci.
    \item /nostartcheck Cela coupe le contrôle d'imonc, s'il est en fonction.
      c'est uniquement nécessaire si vous avez plusieurs routeurs fli4l différents
      à surveiller dans votre Réseau. Si des fonctions supplémentaires étaient
      connectées comme syslog ou e-mail ils resteront désactivées.
  \end{itemize}

  Utilisation (enregistrement de lien)~:

\begin{example}
\begin{verbatim}
X:\...imonc.exe [/Server:Nom d'Hôte] [/Password:Mot de passe] [/iport:Numéro port]
            [/log] [/tport:Numéro port] [/rc:"Commande"]
\end{verbatim}
\end{example}

  Exemple d'enregistrement avec une adresse-IP~:

\begin{example}
\begin{verbatim}
        C:\wintools\imonc /Server:192.168.6.4
\end{verbatim}
\end{example}

  Ou avec le nom et le Mot de Passe~:

\begin{example}
\begin{verbatim}
        C:\wintools\imonc /S:fli4l /P:secret
\end{verbatim}
\end{example}

  Ou avec le nom, le Mot de Passe et une commande au routeur~:

\begin{example}
\begin{verbatim}
        C:\wintools\imonc /S:fli4l /P:secret /rc:"dialmode manual"
\end{verbatim}
\end{example}

  \subsection{Concernant l'aperçu de imonc}

  Imonc client Windows interroge imond pour avoir les informations sur les
  connexions Internet existantes, il les affichent dans un tableau. Sur cette
  page il y a aussi le statut général du routeur, l'heure, la date, le bouton
  synchronisation, etc. Voici les descriptions de ces fenêtres~:

  \begin{tabular}{lp{9cm}}
    Statut             &Calling/Online/Offline (appel/en ligne/raccrocher)\\
    Nom                &Le numéro de Tél ou le Nom du circuit\\
    Direction          &On voie si c'est une connexion entrante ou sortante\\
    IP                 &Adresse IP qui à été assignée\\
    I/Octets           &Octets Entrants\\
    O/Octets           &Octets Sortants\\
    T/enligne          &Temps en ligne\\
    T/Total            &Temps total en ligne\\
    Prix/Unit          &Prix de l'unité par connexion\\
    Prix               &Prix total de la connexion\\
  \end{tabular}

  \medskip

  Les données seront actualisées toutes les deux secondes. (Maintenant) cette
  intervalle peut être changé. Dans le menu on est en mesure de voir le
  canal sur lequel le routeur est en ligne en temps réel. Copiez l'Adresse IP
  réelle dans le presse-papier et installez le canal indiqué explicitement.
  Ceci peut être intéressant s'il y a plusieurs connexions différents par ex.
  une pour naviguer sur Internet et l'autre connectée à votre entreprise,
  de cette façon vous pouvez débrancher l'une ou l'autre connexion.

  En plus si vous avez activé telmond sur votre routeur fli4l, imonc sera en
  mesure d'afficher les informations sur les appels téléphoniques entrants
  (le nom et le numéro de Tél du correspondant). Le dernier appel téléphonique
  reçu sera vu au-dessus des boutons de commande. Un protocole des appels
  téléphoniques entrants peut être vu en utilisant les pages d'appels.

  Les six boutons mentionnés ci-dessous vous permettront de choisir les commandes
  suivantes~:

  \begin{tabular}{clp{9cm}}
    Bouton & Description      & Fonction\\
    1& Connecter/Raccrocher   & Connecter ou raccrocher la ligne\\
    2& Ajout Canal/Supp Canal & Ajoute ou supprime un canal, cette caractéristique
                                n'est disponible que dans le Mode Admin\\
    3& Redémarrer             & Redémarre fli4l!\\
    4& Éteindre               & Arrête fli4l proprement et met le routeur hors tension\\
    5& Arrêter                & Arrête fli4l proprement, pour éteindre le routeur 
                                en toute sécurité\\
    6& Sortir                 & Sort du programme client imonc\\
  \end{tabular}

  \medskip

  \noindent Les cinq premières commandes en mode Utilisateur peuvent être activées
  ou désactivées dans le fichier de configuration /config/base.txt pour le
  routeur fli4l. En mode administrateur toutes les commandes sont toujours
  activées.
  Le choix de la commande Dialmode modifie le comportement du routeur~:

  \begin{tabular}{lp{9cm}}
    Auto    & Le routeur établira automatiquement une connexion
              Internet s'il y a une demande dans réseau local.\\
    Manuel  & L'utilisateur doit établir la connexion manuellement.\\
    Couper  & Il n'y a aucune connexion possible, ni manuellement ni
              automatiquement. La selection du bouton de "connexion" est
              désactivée.\\
  \end{tabular}

  \noindent La volonté de fli4l par défaut c'est d'établir automatiquement une
  connexion Internet sur une demande de requéte Internet par n'importe quel
  Hôte du réseau local. En principe on ne doit jamais modifier la commande pour
  se connecter \ldots

  Il y a également la possibilité de changer manuellement le Circuit-Défaut-Route,
  c.à d. commuter marche/arrêt ou automatique, c'est pourquoi la liste de sélection
  de "Default route" (choix du FAI) est prévu dans la version de Windows d'imonc.
  En outre, on peut maintenant configurer directement dans imonc l'heure de déconnexion.
  Utiliser le Bouton "config" en dessous de Défaut-Route ici la configuration de
  tous les circuits pour le routeur sont indiqués. La valeur de la variable
  Hup-timeout peut être éditée directement dans le fichier isdn.txt du Circuit ISDN
  (ne fonctionne pas pour le moment avec la DSL).

  Un aperçu de LCR-Routing se trouve sur la page Admin/Plage Horaire. Là,
  vous pouvez voir, le Circuit qui sera démarré automatiquement.

  \subsection{Paramètres de configuration}

  On peut accéder à la configuration par le bouton "config" dans la barre d'état.
  La fenêtre qui s'ouvre est divisée en deux, dans le tableau de gauche vous
  avez les répertoires et sous-répertoires, dans celui de droite la configuration
  de imonc. Voici les répertoires en détail~:

  \begin{itemize}
  \item Répertoire général~:
    \begin{itemize}
    \item Synchroniser tous les~: on ajuste ici le nombre de rafraîchissement
      en seconde de la page d'accueil.
    \item Synchroniser au démarrage~: synchronise l'heure et la date du routeur
      avec le client au démarrage. on peut activer cette fonction manuellement
      avec le bouton "Synchroniser" sur la page d'accueil. 
    \item Réduire au démarrage~: au démarrage le programme sera réduit en icône.
      Vous verrez seulement l'icône à côté de l'heure.
    \item Lancer imonc au démarrage Windows~: ici le client imonc démarre
      automatiquement aprés le démarrage de Windows. on peut entrer dans la
      fenêtre "paramètre" des commandes supplémentaires.
    \item Voir l'actualité de fli4l.de~: ici on peut recevoir les (News) du site
      fli4l.de chargées automatiquement par imonc, les titres sont alors montrés
      dans la fenêtre "Nouvelles" et qui pourront être lus.
    \item Appel du fichier log~: on indique ici le nom du fichier pour
      enregistrer la liste des appels locaux.
    \item Attendre la réponse du routeur~: temps d'attente d'une réponse du
      routeur en seconde, avant que la connexion soit perdue.
    \item Langue~: on choisit ici la langue pour imonc.
    \item Confirmer les commandes du routeur~: si la case est cochée, toutes
      les commandes envoyées au routeur demande une confirmation,
      ex. redémarrage, déconnexion etc \ldots
    \item Arrêt même avec trafic~: si aucune réponse n'aboutit, la connexion
      s'arrête même si il y a toujours du trafic sur cette connexion.
    \item Reconnexion automatique au routeur~: une reconnexion du routeur est
      faite automatiquement, si une coupure de la connexion a eu lieu,
      (p. ex. un redémarrage du routeur).
    \item Reduire la fenêtre système~: si activée, en cliquant sur le bouton
      "Sortir" imonc se réduit en icône vers la barre des taches à côté de
      l'heure, au lieu de s'arrêter.
    \end{itemize}

  \item Sous-répertoire proxy~: ici on enregistre le proxy pour les demandes http.
    Celui-ci est utilisé à présent pour l'actualisation des fenêtres, time-table
    et news.
    \begin{itemize}
    \item Active le proxy pour le protocole http~: ici on active Proxy
          \begin{itemize}
            \item Adresse~: ici l'adresse du serveur proxy
            \item Port~: ici le numéro de port du serveur proxy (defaut: 8080)
          \end{itemize}
    \end{itemize}

  \item Sous-répertoire icône~: ici on peut personnaliser les couleurs des icônes.
    Dans l'avenir on pourra choisir les couleurs de fond de l'icône pour dialmode
    (mode de connexion) qui sera placé dans la barre de tache.

  \item Répertoire d'appel, le réglage de la position de la fenêtre avis d'appel
    sur l'écran, sera stockée et sauvegardée dans la base de Registre. Vous pouvez
    déplacer la fenêtre à l'endroit de votre choix. Après ce réglage, la fenêtre
    apparaitra exactement cet endroit à chaque fois.
    \begin{itemize}
      \item Mise à jour~: on peut choisir ici, comment imonc reçois les
        informations des nouveaux appels tél, Il y a trois possibilités différentes.
        La premier consiste à interroger régulièrement de service telmond sur le
        routeur. Une autre possibilité consiste à interroger les annonces de
        Syslog, cette variante est la préférer~-- On doit Naturellement activer
        Syslog dans le client imonc. Imonc doit être connecté à une direction approprié,
        la troisième possibilité proposé est d'utiliser le paquetage Capi2Text pour
        la signalisation d'appel.
      \item Effacer premier zéro~: parfois devant le numéro de téléphonique est
        placé un zéro supplémentaire. Celui-ci peut être supprimé avec cette option.
      \item Indicatif régional~: la présélection personnelle du numéro de tél
        peut être écrite ici. Quand un appel arrive avec la même présélection.
        La présélection ne sera pas visible.
      \item Annuaire~: ici on indique le fichier dans lequel l'annuaire
        téléphonique local sera sauvegardé pour les numéros de téléphones.
        Si le fichier n'existe pas, il est automatiquement installé.
      \item Fichier log~: on indique ici le nom du fichier, utile pour enregistrer
        la liste des appels sur l'ordinateur local. Ce paramètre est visible
        uniquement si la variable \var{TELMOND\_\-LOG} être sur `yes', c'est
        également valable pour la liste des appels réelle.
      \item Recherche externe~: un programme peut être indiqué dans cette fenêtre,
        que l'on appelle si un numéro de téléphone ne peut pas être résolu au moyen
        de l'annuaire téléphonique local. Des infos plus précises devraient être
        jointes aux programmes correspondants. Il y a jusqu'à présent un CD
        d'annuaire téléphonique de Marcel Wappler KlickTel ainsi qu'un lien vers
        une base de données.
    \end{itemize}

  \item Sous-Répertoire des appels tél~:
    ces options sont destinées, à détailler des instructions des appels téléphoniques
    et de les afficher, voir les illustrations ci-dessous.
    \begin{itemize}
      \item Notification d'appel actif~: détermine si des appels doivent être signalés.
      \item Indication des notifications d'appels~: lors d'un appels tél une fenêtre d'
        apparait, elle détaille les Infos suivantes~: l'appel MSN, le numéro de tél du
        correspondant et la date/heure de l'appel. Pour cela il est nécessaire que la
        variable \var{OPT\_\-TELMOND} soit placée sur `yes' dans le fichier
        config/isdn.txt
        \begin{itemize}
          \item Ne pas enregister les numéros non transmis~: les appels ne doivent pas
            être écrit dans la fenêtre d'appel, si aucun numéro de Tél n'a été transféré.
          \item Temps d'affichage~: cette indication influe sur la durée de fermeture de
            la fenêtre avis d'appel, la fenêtre doit rester ouverte un certain temps.
            Si on indique "0" la fenêtre ne se fermera pas automatiquement.
          \item Fontsize (ou police)~: ici on choisit la taille des caractères pour la
            fenêtre. Celle-ci affecte la taille de la fenêtre, puisque la taille de la
            fenêtre sera calculée par rapport à la taille du message.
          \item Couleur~: ici on choisit la couleur des textes dans la fenêtre d'appel.
            J'emploie le rouge pour l'identification des messages.
      \end{itemize}
    \end{itemize}

  \item Sous-répertoire annuaire~: la fenêtre contient l'annuaire téléphonique
    qui est utilisé pour la définition des numéros de téléphones des appels
    entrants et aussi si vous possédez un MSN. Cette fenêtre apparait même si la
    variable \var{TELMOND\_\-LOG} est sur `no' parce que cette fenêtre est utilisée
    aussi pour le dernier appel entrant vu dans la fenêtre principale. On peut choisir
    un fichier qui sera placé sur le routeur.

    Construction d'un appel entrant~:

\begin{example}
\begin{verbatim}
  # Format:
  # Telefonnummer=anzuzeigender Name[, Wavefilename]
  # 0241123456789=Testuser
  00=unbekannt
  508402=Fax
  0241606*=Elsa AG Aachen
\end{verbatim}
\end{example}

    Les trois premières lignes sont des commantaires. La quatrième
    ligne est créée si aucun numéro n'est transmis, "unbekannt" (ou inconnu)
    sera affiché. La cinquième ligne indique le numéro de tél "508402" et le
    Nom "Fax" , dans tous les cas le format sera toujours le même, Numéro de
    Tél=Nom. La sixième ligne détermine l'ensemble des numéros de Tél, pour
    toutes appel ex. 0241606 le Nom sera affiché. Souvenez-vous que le dernier
    numéro d'appel du correspondant est indiqué sur la première fenêtre
    principale. Optionnel, un fichier son peut être défini et sera joué lors
    d'un appel Tél.

    Dés la Version 1.5.2, il est possible d'installer un annuaire Téléphonique
    sur le routeur sous la forme d'un fichier il sera enregistré et synchronisé
    dans (/etc/phonebook). Si un même numéro de téléphone avec un Nom différent
    sont enregistrés dans l'annuaire Du routeur et dans l'annuaire de imonc,
    il sera demandé à l'utilisateur qu'elle est l'entrée valide. les appels ne
    sont pas juste recopiés mais sont enregistrés sur les deux annuaires. La
    synchronisation du fichier d'annuaire est faite dans la mémoire RAM, cela
    veut dire, lorsque l'on reboot (redémarre) le routeur, le fichier sera perdu.

  \item Répertoire son, les fichiers son qui seront installés ici seront joués,
    si l'événement indiqué se produit.
    \begin{itemize}
      \item Courriel~: le fichier son sera joué, si un nouveau courriel se trouve
        sur votre Serveur POP3. 
      \item Erreur courriel~: le fichier son sera joué, si une erreur se produit
        lors de la réception du Courriel.
      \item Connexion perdu~: le fichier son sera joué, si la connexion avec
        le routeur est perdue (ex. redémarrage du routeur). Si l'option
        "reconnexion automatique au routeur" n'est pas activée, un messagebox
        s'ouvrira pour demander une nouvelle connexion au routeur.
      \item Connexion~: le fichier son sera joué, si le routeur établit
        une connexion Internet.
      \item Déconnexion~: le fichier son sera joué, lorsque le routeur déactive
        la connexion Internet.
      \item Avis appel~: le fichier son sera joué, si l'annonce des appels est
        activée et si un nouvel appel est reçu.
      \item Annonce de fax~: le fichier son sera joué, après la réception de
        nouveaux fax.
    \end{itemize}

  \item Répertoire courriel
    \begin{itemize}
      \item Comptes~: cette fenêtre sert à configurer les comptes POP3.
      \item Activer le contrôle courriel~: si vous avez un compte courriel il
        recherchera automatiquement les nouveaux courriels.
        \begin{itemize}
          \item Vérifier x/Min~: cette option définit un intervalle temps entre
            chaque contrôle Courriel sur le compte. Attention~: en définissant un 
            intervalle trop court, le routeur peut rester constamment en ligne!
            Ceci se produit lorsque l'intervalle est plus court que "Delai attente"
            du circuit utilisé.
          \item Temps d'attente x/Sec~: temps d'attente d'une réponse du Serveur POP3
            avant l'arrêt de celui-ci, si la valeur est à "0" cela signifie qu'aucun
            TimeOut (temps d'attente) n'est installé.
          \item routeur déconnecté~: cette option permet au routeur de se connecter
            automatiquement pour rechercher les nouveaux courriels sur le Serveur POP3.
            Aprés le téléchargement des courriels le routeur se déconnecte. pour pouvoir
            utiliser ce dispositif on doit mettre Dialmode sur 'auto'. Attention~:
            cela occasionne des frais supplémentaires de connexion si aucun tarif
            unitaire est utilisé~!
          \item Circuit à utiliser~: cette option définit le circuit qui sera utilisé
            pour la connexion aux courriels.
          \item Rester en ligne aprés contrôle~: la déconnexion doit être faire
            manuellement ou l'arrêt doit être réalisé automatiquement par l'option
            Délai attente.
          \item Charger en-têtes des courriels~: télécharger les en-têtes des courriels ou
            uniquement le nombre de courriel disponible? Cette option doit être activée
            pour supprimer les courriels directement sur le serveur POP3.
         \item M'avertir de nouveaux courriels~: faut-il un message sonore et une icône
            dans la barre de tâche pour m'annoncer de nouveaux courriels.
         \item Exécuter le programme de messagerie~: démarrer automatiquement le
            programme de messagerie pour lire les nouveaux courriels disponibles.
         \item Programme~: indiquer ici le programme de messagerie.
         \item Paramètre~: entrer les paramètres additionnels qui seront transférés
            au démarrage du programme de messagerie. Si Outlook est utilisé comme
            programme courriel (pas Outlook Express!) vous pouvez entrer comme paramètre
            "/recycle" empêche de lancer Outlook dans une nouvelle fenêtre s'il est
            déjà ouvert.
      \end{itemize}
    \end{itemize}

  \item Repertoire Admin
    \begin{itemize}
      \item Mot de passe Root~: ici on entre le mot de passe du routeur qui est
        dans le fichier (/config/base.txt dans la variable \verb+PASSWORD+) pour pouvoir
        par exemple configurer Portforwarding sur votre ordinateur et l'envoyer
        sur le routeur.
      \item Voir les fichiers sur le routeur~: tous les fichiers log (ou journal)
        qui se trouvent sur le routeur sont à indiquer ici, ils peuvent être lus,
        avec un simple clic de la souris dans la page Admin/fichier, ainsi on peut
        afficher les fichiers log du routeur directement dans imonc.
      \item Fichier de configuration~: ici on peut choisir, si tous les fichiers
        seront ouverts avec le programme éditeur de texte ou uniquement les fichiers
        *.txt pour étudier et travailler dessus. On peut égalment ouvrir un ensemble
        de fichiers.
      \item DynEisfairLog~: si vous avez créé un compte sur DynEisfair, vous pouvez
        enregistrer ici les données d'accés et de voir avec le fichier Log les mises
        à jours des fonctions sur la page Admin/DynEisfairLog.
    \end{itemize}

  \item Répertoire démarrage auto, sert à configurer une liste de programmes qui
    sera lancée automatiquement. Celle-ci est exportée après une connexion réussie
    si l'option "Activer la liste des programmes" est cochée.
    \begin{itemize}
      \item Programme~: tous les programmes installés ici seront lancés
        automatiquement, si le routeur est connecté et que La Liste des
        programmes est cochée.
      \item Activer la liste des programmes~: la liste doit-elle être activé pour
        exécution des programmes aprés une connexion réussie~?
    \end{itemize}

  \item Répertoire trafic du réseau, est utilisé pour la configuration
    (personnalisée) de la fenêtre de Info trafic. Un utilisateur m'a averti qu'il
    y avait quelques erreurs sur la définition des données avec des versions
    anciennes de DirectX.
    \begin{itemize}
      \item Voir les informations sur le trafic~: voulez-vous afficher une utilisation
        graphique des canaux dans une fenêtre à part? Dans le menu contextuel vous
        pouvez choisir l'attribut StayOnTop, cette option provoque l'affichage de la
        fenêtre sur toutes les autres fenêtres. Cette option sera enregistrée dans
        la base de registre et sera en service aprés un redémarrage du programme.
      \item Voir les titres~: doit-on monter la barre de titre dans la fenêtre
        Traffic-Info? Cette fenêtre montrera les informations des circuits utilisés
        par le routeur.
        \begin{itemize}
          \item Voir l'utilisation CPU~: montrer l'utilisation du CPU dans la barre
            de titre?
          \item Voir le temps de communication~: le temps en ligne du canal doit-il
            aussi être indiqué dans la barre de titre~?
        \end{itemize}
      \item Fenêtre semi-transparente~: la fenêtre doit-elle être représentée en
        transparence? Cette fonction n'est disponible que sous
        Windows 2000 et Windows XP.
      \item Couleur~: les couleurs sont définies ici pour la fenêtre Traffic-Info.
        Maintenant le canal DSL et le premier canal ISND utiliseront les mêmes
        couleurs. 
      \item Limite~: entrer les valeurs maximales des taux de transmission xDSL~--
        pour T-Online~: Débit Montant (upload) 128 Ko/s et Débit Descendant
       (download) 1024 Ko/s.
    \end{itemize}

  \item Répertoire Syslog, est utilisé pour la configuration de l'affichage
    des messages Syslog.
    \begin{itemize}
      \item Activer le client Syslog~: montrez les messages Syslog dans imonc~?
        Cette option doit être arrêtée, si vous utilisez un autre client Syslog
        externe, par exemple le client Kiwi's Syslog.
      \item Indiquer les messages Syslog~: monter les messages Syslog avec un
        niveau de prioritaire~? Vous pouvez indiquer ici les niveaux prioritaires
        des messages Syslog, par defaut le message débug est coché, vous pouvez
        cocher le niveau selon vos besoins.
      \item Enregistrer les messages Syslog~: les messages lus doivent-ils être
        sauvegardés? Dans la fenêtre on peut choisir les messages que l'on veut
        sauvegarder. On peut insérer des caractères supplémentaires avec nom de
        fichier à sauvegarder, les voici~:
        \begin{description}
          \item[\%y]~-- On l'ajoute pour avoir l'année actuel
          \item[\%m]~-- On l'ajoute pour avoir le mois actuel
          \item[\%d]~-- On l'ajoute pour avoir le jour actuel
        \end{description}
      \item Voir le nom des ports~: doit on afficher la description du port au
        lieu du numéro de port~?
      \item Voir les messages pare-feu~: ici, on indique les messages du firewall
        (ou pare-feu), il seront aussi indiqués en mode utilisateur.
    \end{itemize}

  \item Répertoire fax, sert à configurer les fax (ou télécopie) dans imonc. Pour
    que ce dossier soit visible vous devez installer sur le routeur le paquetage
    mgetty et/ou faxrcv, (vous pouvez les trouver sur le site de fli4l).
    \begin{itemize}
      \item Fichier Log pour fax~: ici on peut enregister les fax reçus sous forme
        de fichier dans un dossier de l'ordinateur.
      \item Répertoire local des fax~: configurer le répertoire pour stocker les
        fax reçus, avant de les consulter.
      \item Actualisation~: il y deux possibilités, lorsque imonc reçoit un
        nouveau fax. Soit c'est imonc Syslog qui reçoit les fax (naturellement
        le client imonc-Syslog doit être activé), soit imonc regarde régulièrement
        le fichier log. La première variante est la meilleure. Si vous utilisé la
        deuxième variante, vous pouvez indiquer combien de fois la page d'aperçu de
        fax doit être actualisée. Il faut faire attention cette valeur n'est pas
        une indication en seconde, mais c'est une indication en multiple,
        en général c'est une intervalle d'actualisation.
    \end{itemize}

  \item Répertoire tableau, sert à ajuster les colonnes des (tableaux) dans imonc
    par rapport à vos besoins. D'une part, pour chaque tableau on peut régler les
    en-têtes les colonnes qui doivent être affichées, d'autre part pour chaque
    service de communication il y a un tableau différent, appel Tél, fax, on peut
    régler le moment ou les Infos doivent être affichées.
  \end{itemize}

  \subsection{Concernant les appels tél}

  L'annuaire Téléphonique sera uniquement vu, que si la variable \var{TELMOND\_\-LOG}
  est placée sur 'yes', sinon aucun journal d'appels ne sera conservée. Dans cet
  annuaire sera enregistré tous les appels téléphoniques qui seront entrées sur
  le routeur. Vous pouvez commuter entre les appels enregistrés sur le PC local
  et les appels enregistrés sur le routeur, vous pouvez effacer le fichier sur
  le routeur avec le bouton-Réinitialisé.

  Dans l'annuaire téléphonique, vous pouvez cliquer avec le bouton droit
  de la souris sur le Numéro de Tél pour attribuer un Nom au numéro, comme
  cela le Nom apparaitra à la place du Numéro de Tél.

  \subsection{Concernant les connexions}

  L'affichage des connexions internet par le routeur dans une page est utilisé
  de puis la Version 1.4, elle donnera une bonne vue d'ensemble du comportement
  du routeur connecté à Internet. Pour voir cette page la variable \var{IMOND\_\-LOG}
  doit être placée sur `yes' dans le fichier /config/base.txt.

  De la même façon que l'annuaire-Tél, vous pouvez commuter les connexions
  enregistrées localement et celles enregistrées sur le routeur. Vous pouvez
  aussi effacer le fichier des données sur le routeur en cliquant sur
  le bouton-rafraîchir.

  Affichage du tableau de connexions.

  \begin{itemize}
  \item Nom du FAI
  \item Date et heure de départ
  \item Date et heure de fin
  \item Temps en ligne
  \item Prix de l'unité
  \item Prix Total
  \item Réception du signal
  \item Émission du signal
  \end{itemize}

  \subsection{Concernant les FAX}

  Pour que soit affichée la page FAX il faut installer le paquetage
  \var{OPT\_\-MGETTY} par M. Michael Heimbach sur le routeur ou
  \var{OPT\_\-MGETTY} par M. Felix Eckhofer. Sur le site Internet de
  fli4l, à la page d'accueil vous avez un raccourci pour les
  paquetages-OPT. Dans cette fenêtre tous les FAX reçus seront enregistrés,
  le menu contextuel offre plusieurs options de configuration qui seront
  uniquement disponibles en mode Administrateur~:

  \begin{itemize}
  \item Concernant les Fax reçus, il faut correctement configurer le chemin
  d'accès pour fli4l dans répertoire Admin/Remoteupdate, pour que les FAX
  reçus sur le routeur soient enregistrés et compressés avec le programme
  gzip, qui se trouve dans le paquetage fli4l, le programme gzip.exe et le
  fichier win32gnu.dll peuvent aussi être copié dans le répertoire imonc.
  Si gzip.exe n'est pas trouvé dans l'un des deux emplacements, et si le
  routeur est connecté à Internet, il recherchera le programme sur internet
  (directement sur le site CGIs).
  \item Supprimer un FAX reçu. Cela signifie que le FAX sera supprimé sur
    votre PC local et sur le routeur (le fichier FAX réel, et aussi dans
    le fichier log).
  \item Supprimer tous les FAX présents sur routeur. Ici tous les FAX
    sur le routeur dans le fichier log seront effacés. Les FAX ne seront
    pas effacés du fichier log de votre PC local.
  \end{itemize}

  Comme dans la page des appels Tél, vous pouvez commuter entre les Fax
  enregister localement et les Fax enregistrés sur le routeur.

  \subsection{Concernant les courriels}

  Cette page apparait, si dans le répertoire Config courriel il y a au moins
  un compte \mbox{courriel} avec serveur POP3 qui a été configuré et activé.

  Description de la page \mbox{courriel}. Maintenant on a intégré dans cette section
  le contrôleur de \mbox{courriel}. Si l'option "le routeur n'est pas en ligne" dans
  config \mbox{courriel} n'est pas activée, le contrôleur de Mail vérifiera tous les
  comptes \mbox{courriel}, ensuite il utilisera l'intervalle Temps pour vérifier le
  Serveur (le routeur doit être connecté, il utilise le circuit
  présélectionné). Si le routeur n'est pas connecté, activer l'option "le
  routeur n'est pas en ligne" et indiquer le circuit à utiliser, il établira
  une connexion en utilisant le circuit choisi et téléchargera les \mbox{courriels} de
  tous les comptes \mbox{courriels} configurés ensuite il fermera la connexion. Pour
  utiliser cette option vous devrez placer Dialmode sur "auto".

  Si des \mbox{courriels} sont disponible sur le serveur POP3, le programme
  \mbox{courriel} client sera démarré automatiquement ou une icône apparaitra
  prés de l'heure dans la barre de tache, il indiquera le nombre de \mbox{courriel}
  sur le serveur. En double cliquant dessus l'ensemble du \mbox{courriel} client
  sera lancés. Si une erreur se produit sur un compte \mbox{courriel}, d'une part,
  une note sur l'erreur sera écrit dans le dossier Histoire du \mbox{courriel},
  d'autre part, l'icone du \mbox{courriel} affichera dans le coin supérieur droit
  une couleur rouge.

  Dans la fenêtre \mbox{courriel}, on peut effacer directement les \mbox{courriels} sur le
  Serveur sans les avoir préalablement téléchargés. Il faut avoir téléchargé
  les en-têtes des courriels, vous devez marquer les cellules à supprimer, puis
  en cliquant sur le bouton droit de la souris le menu contextuel s'ouvre,
  et cliquer sur Delete MailMessage.

  \subsection{Admin}

  Cette partie est uniquement disponible si imonc est démarré en mode Admin.

  Premier point, cette page offre une vue d'ensemble des circuits utilisés,
  ~--les fournisseurs d'accès Internet~-- qui ont été choisis automatiquement
  par le routeur (par l'intermédiaire du LC Routing). En double cliquant sur un
  fournisseur d'accès dans l'aperçu fournisseur d'accès vous obtiendrez
  l'affichage des définitions des plages horaires pour ce fournisseur qui à été
  défini dans /config/base.txt.

  Deuxième point, cette page donne l'occasion d'installer les mises à jour à
  distance sur le routeur. Vous pouvez choisir l'un ou les cinq programmes
  (Kernel, fichier système, fichier OPT, rc.cfg et syslinux.cfg) qui seront
  copiés sur le routeur. Pour pouvoir faire la mise à jour à distance, vous
  devrez indiquer le répertoire de fli4l dans imonc et les fichiers nécessaires
  à copier. En plus, vous devez écrire le sous-répertoire des fichiers de
  configuration (par défaut: /config/*.txt) pour devez construire tous les
  fichiers systèmes de fli4l. Il est conseillé de Rebooter (ou redémarrer) après
  avoir envoyé les fichiers système sur le routeur pour que les modifications
  soient prises en compte. Si un mot de passe est demandé par le routeur, il
  est inscrit dans la variable PASSWORD dans /config/base.txt.

  Troisième point, cette page traite des contraintes du Port Forwarding,
  un port est connecté exactement et uniquement à un ordinateur client.
  Maintenant il est possible d'éditer et de configurer Port-Forwarding du
  routeur. aprés les modifications des ports ils seront activées, la
  connexion doit être active. Puisque les fichiers sont enregistés dans la
  mémoire virtuelle (Ramdisk), tous les changements seront uniquement
  sauvegardés jusqu'au prochain redémarrage du routeur. Pour sauvegarder des
  changements de manière permanente vous devez changer des Port Forward dans
  le fichier /config/base.txt et installer le nouveau fichier-OPT sur le routeur.

  Quatrième point, dans la fenêtre Admin, puis~-- fichier~-- vous pouvez utilisée
  et voir la configuration des fichiers Log du routeur, en cliquant
  simplement sur la souris. La liste de choix peut être configurée dans le
  dossier config-Admin d'imonc "voir les fichiers sur le routeur". Ensuite,
  vous pouvez simplement choisir les fichiers qui sont indiqués dans le menu
  déroulant.

  Cinquième point, cette fenêtre montre DynEisfair log, elle apparaît
  uniquement si dans le répertoire de configuration Config-Admin les
  enregistrements les données pour un accès à un compte DynEisfair a
  été configuré (pour simuler une IP fixe, lorsque l'on a une IP dynamique).
  Si cela est fait, le fichier log des services sera indiqué dans cette fenêtre.

  Dernier point, fenêtre hôtes, tous les ordinateurs enregistrés dans le fichier
  /etc/hosts sont indiqués ici, à l'avenir on essaiera de configurer chacun des
  ordinateurs enregistrés pour pouvoir les "pinger" (ou interroger)
  individuellement, ainsi on pourra rapidement vérifier l'ordinateur qui est 
  connecté au réseau local.

  \subsection{Concernant les erreurs syslog et firewall}

  Les pages erreur, syslog et Firewall (pare-feu), s'affiche uniquement
  s'il y a des événements enregistrés dans ce fichier, en plus il faut
  être en mode Admin pour que les pages soit affichées.

  Toutes les erreurs spécifiques à imonc/imond seront enregistrées dans la 
  fenêtre erreur. Si vous avez des problèmes vous pouvez aller vérifier dans
  cette liste pour voir les causes des erreurs que vous avez rencontrées.

  Dans la fenêtre Syslog les messages de syslog seront affichés, excepté des
  messages du pare-feu. Ceux-ci sont affichés dans une page indépendante
  (voir ci-dessous). Pour que la page syslog  fonctionne vous devrez placer
  la variable \var{OPT\_\-SYSLOGD} sur "yes" dans le fichier de configuration
  /config/base.txt En plus dans la variable \var{SYSLOGD\_\-DEST} on doit
  placer l'adresse IP du client qui bien entendu utilise imonc (par exemple~:
  \var{SYSLOGD\_\-DEST}='@ 100.100.100.100~-- adresse IP de votre client!).
  Il n'y aura pas que les messages syslog qui seront affichés, mais aussi
  la date, l'heure, l'IP et le niveau de priorité.

  Des messages du Firewall (pare-feu) seront affichés dans une page
  indépendante. Pour que la page fonctionne, vous devez placer la variable
  \var{OPT\_\-KLOGD} sur 'yes' dans le fichier de configuration /config/base.txt.

  \subsection{Concernant les News}

  Cette page News (ou d'actualité), doit être activée dans le répertoire
  config-Imonc. Les News mentionnés sur la page accueil du site fli4l, seront
  visibles directement dans Imonc à la page accueil. On peut directement aller
  sur le site http://www.fli4l.de/german/news.xml avec le bouton-plus. Vous
  avez une fenêtre à côté des titres des News, qui indique les 10 derniers
  paquetages-OPT enregistrés sur le site
  http://www.fli4l.de/german/imonc\_opt\_show.php, en double cliquant sur le
  paquetage choisi, vous allez directement sur le site. En plus, il est
  indiqué dans la barre de statut en bas de Imonc, les titres des News.


  \marklabel{sec:imonc}{
    \section{Unix/Linux-Client imonc}}

  There are 2 different versions for Linux: a text-based imonc) and a
  graphical user interface version(ximonc). The source of ximonc can be
  found under the directory src. The documentation for ximonc will only
  be available in the 1.5-Final-Version. An experienced Linux-User should
  have no problems with the source.

  Let's limit to the text-based version.
  This is a curses based program, thus it has no graphical interface.
  The source lies under the directory unix.


  Installation:

\begin{example}
\begin{verbatim}
        cd unix
        make install
\end{verbatim}
\end{example}

  imonc will be installed to /usr/local/bin.

  Command line parameters:

\begin{example}
\begin{verbatim}
        imonc hostname
\end{verbatim}
\end{example}

  hostname can be the name or the IP address of the fli4l
  router, e.g.

\begin{example}
\begin{verbatim}
        imonc fli4l
\end{verbatim}
\end{example}


  imonc shows the following:

  \begin{itemize}
  \item Date/Time of the fli4l router

  \item Momentarily configured route

  \item Default-Route-Circuits

  \item ISDN channels
    \begin{description}
    \item[Status]:         Calling/Online/Offline
    \item[Name]:           Phone number of the peer or the circuit-name
    \item[Time]:           Online time
    \item[Charge-Time]:    Online time considering the charge interval
    \item[Charge]:         The actual charge
    \end{description}
  \end{itemize}


  Possible commands:

  \begin{tabular}{lll}
    Nr  &Command             &Meaning\\
    0   &quit                &Quit program\\
    1   &enable              &Activate\\
    2   &disable             &Deactivate\\
    3   &dial                &Dial\\
    4   &hangup              &Hang up\\
    5   &reboot              &Reboot\\
    6   &timetable           &Output timetable\\
    7   &dflt route          &Set Default-Route-Circuit\\
    8   &add channel         &Add 2. channel\\
    9   &rem channel         &Remove 2. channel\\
  \end{tabular}

  \medskip

  \noindent Detailed information on every command:

  \begin{description}
  \item[0~-- quit] The connection to the imond server is terminated
    and the program quits.


  \item[1~-- enable] All circuits are set to dialmode ``auto''. This is
    the default state after boot. It results in fli4l dialing automatically
    on demand as soon as it receives a request by a host from the LAN.


  \item[2~-- disable] All circuits are set to dialmode ``off''. This means
    fli4l is virtually ``dead'' until it is revived by the enable command.


  \item[3~-- dial] Manual dial using the Default-Route-Circuit. You won't
    need this normally as fli4l normally dials automatically.


  \item[4~-- hangup] Manual hangup. You can make fli4l hangup before it does
    it automatically.


  \item[5~-- reboot] fli4l is rebooted. Pretty unnecessary command \ldots


  \item[6~-- timetable] The timetable for the Default-Route-Circuits is
    printed out. Example: see above.


  \item[7~-- default route circuit] Manually changing the default route
    circuit can make sense, if you want to disable the automatic
    LC routing of fli4l for a while, as some providers will only let you
    access your email if you are dialed in to their servers.

  \item[8~-- add channel] The second ISDN channel is manually added.
    Prerequisite: \var{ISDN\_\-CIRC\_\-x\_\-BUNDLING} is set to `yes'.


  \item[9~-- remove channel] Removes the second ISDN channel. See also
    ``add channel''.

  \end{description}

  \noindent Apart from that, the same annotations as for the windows client
  \verb+imonc.exe+ apply.

  A little remark: From fli4l-1.4 on, it is possible, to install a
  ``minimalistic'' imon client on the fli4l router itself using
  \smalljump{OPTIMONC}{\var{OPT\_\-IMONC}}='yes' in package
  \smalljump{sec:tools}{\var{TOOLS}}.

  You will be able to change some settings, e.g. routing etc. on the fli4l
  console locally. But Beware: This mini-imonc will only work on the fli4l
  router itself! On a Linux or Unix client you should always use the
  ``big brother'' unix/imonc.
