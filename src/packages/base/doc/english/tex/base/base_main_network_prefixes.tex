% Synchronized to r50375
\section{Network prefix configuration}

\begin{description}
  \config{OPT\_NET\_PREFIX}{OPT\_NET\_PREFIX}{OPTNETPREFIX}{
  Enables support for custom network prefixes.

  A network prefix is technically nothing else than the address of a network, 
  but it usually stands for a network that shall be divided further. This is
  especially useful if a fli4l router should not manage the whole network, but
  leave subnets for other routers. By the definition (and thus naming) of the
  whole network it is possible to use the network address in several places
  without always having to write the prefix again.
  
  Concrete examples of how to define a network prefix can be found below
  for the different types of network prefixes.

  Defaul Setting:
  }
  \verb*?OPT_NET_PREFIX='yes'?

  \config{NET\_PREFIX\_x}{NET\_PREFIX\_x}{NETPREFIXx}{
  
  This array defines the various network prefixes.
   The individual components are explained below.
  }

  \config{NET\_PREFIX\_x\_NAME}{NET\_PREFIX\_x\_NAME}{NETPREFIXxNAME}{
  Name of the network prefix.

  This variable contains the name of the prefix. This name can then be used in 
  address informations in order to use the prefix. The name has to be set like
  circuit names, i.e. it must be specified in curly brackets.
  }

  \config{NET\_PREFIX\_x\_TYPE}{NET\_PREFIX\_x\_TYPE}{NETPREFIXxTYPE}{
  Type of the network prefix.

  This variable contains the type of the prefix. The supported types are
  explained in Tab.~\ref{base:net:prefix:types}.

  \begin{center}
      \begin{longtable}{|l|p{0.7\textwidth}|}
          \hline
          \multicolumn{1}{|l}{\textbf{Typw}} &
          \multicolumn{1}{|l|}{\textbf{Explanation}} \\
          \hline
          \endhead
          \hline
          \endfoot
          \endlastfoot
          static        & The network prefix is specified directly as a fixed address.
                          \\
          generated-ula & The network prefix is generated by fli4l as an
                          ULA\footnote{``Unique Local Address''} according to RFC
                          4193.\footnote{\altlink{https://tools.ietf.org/html/rfc4193}}
                          If the fli4l has access to persistent storage,
                          then the prefix is only generated once, so it
                          also remains intact during reboots of the router.
                          \\
          \hline
          \caption{Types of network prefixes}\label{base:net:prefix:types}
      \end{longtable}
  \end{center}
  }
\end{description}

\subsection{Network prefixes of type ``stable''}
For network prefixes of type ``stable'' the following settings apply:

\begin{description}
  \configlabel{NET\_PREFIX\_x\_STATIC\_IPV6}{NETPREFIXxSTATICIPV6}
  \config{NET\_PREFIX\_x\_STATIC\_IPV4 NET\_PREFIX\_x\_STATIC\_IPV6}{NET\_PREFIX\_x\_STATIC\_IPV4}{NETPREFIXxSTATICIPV4}{
  Adresse(s) of the network prefix.

  This setting can be used to specify the IPv4 and/or IPv6 address of the
  network prefix.

  Example:
  }
  \begin{example}
  \begin{verbatim}
    NET {
      PREFIX {
        [] {
          NAME='site'
          TYPE='static'
          STATIC {
            IPV4='10.1.0.0/16'
            IPV6='fdce:1c35:301f::/48'
          }
        }
      }
    }
  \end{verbatim}
  \end{example}

\end{description}

\subsection{Network prefixes of type``generated-ula''}
For Network prefixes of type ``generated-ula'' the following settings apply:

\begin{description}
  \config{NET\_PREFIX\_x\_ULA\_DEV}{NET\_PREFIX\_x\_ULA\_DEV}{NETPREFIXxULADEV}{
  Ethernet-Interface.

  This setting specifies the Ethernet interface whose
  MAC address is used to generate the ULA.

  Example:
  }
  \begin{example}
  \begin{verbatim}
    NET {
      PREFIX {
        [] {
          NAME='site'
          TYPE='generated-ula'
          ULA {
            DEV='eth0'
          }
        }
      }
    }
  \end{verbatim}
  \end{example}

\end{description}
