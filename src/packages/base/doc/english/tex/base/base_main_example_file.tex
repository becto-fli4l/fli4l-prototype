% Synchronized to r29818

\marklabel{beispielbase}{\section{Example file}}\index{Example file(base.txt)}\index{base.txt}

The content of the example file \verb+base.txt+ in directory \verb+config/+
is as follows:

\begin{example}
\verbatimfile{\basedir/config/base.txt}
\end{example}

\medskip

Please note that this file is stored with DOS line endings, i.e. each line
contains an additional carriage return (CR) at the end. Since most Unix editors
can handle such files it was decided to use this style, as Windows editors typically 
do have problems if no CR/LF line endings are used!

If your favourite Unix/Linux editor does not like editing some configuration
file due to the DOS line endings, you can convert the DOS line endings to Unix
ones with the following command before you start editing the file:

\begin{example}
\begin{verbatim}
        sh unix/dtou config/base.txt
\end{verbatim}
\end{example}

For the creation of the boot media it is irrelevant whether the file contains
DOS oder Unix line endings. They are always converted to Unix style when
being written to the boot media.

But let's proceed to the contents \ldots
