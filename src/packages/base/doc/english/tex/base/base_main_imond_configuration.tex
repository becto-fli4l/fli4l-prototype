% Synchronized to r29848

  \section{imond configuration}

  \begin{description}

    \config{OPT\_IMOND}{OPT\_IMOND}{OPTIMOND}

      Default setting: \var{OPT\_\-IMOND='no'}
    
    {\var{OPT\_\-IMOND} controls whether to start the imond server or not.
      The imond server is responsible for monitoring/controlling the fli4l
      router and for the so-called least cost routing. You can
      find a detailed description of the
      \jump{IMONDSCHNITTSTELLE}{Client/Server interface imond} in a separate
      appendix.

      Important: The least cost routing funtionality of fli4l can only be used
      when imond is running. Time-based switching of connections is impossible
      without imond!

      Starting with version 1.5, imond is mandatory for ISDN and DSL routing.
      In this case you have to set \var{OPT\_\-IMOND}='yes'. If you use fli4l
      as a router between LANs only, you should set \var{OPT\_\-IMOND}='no'.}

    \config{IMOND\_PORT}{IMOND\_PORT}{IMONDPORT}

    {The TCP/IP port where imond should wait for connections. You shouldn't
      change the default value `5000' unless in very exceptional cases.}


    \config{IMOND\_PASS}{IMOND\_PASS}{IMONDPASS}

      Default setting: \var{IMOND\_\-PASS=''}

    {This variable can be used to set a user password for imond.
      If a client connects to imond at port 5000, imond expects the client to
      provide this password before processing any requests, with the exception
      of the commands ``quit'', ``help'', and ``pass''. If you leave
      \var{IMOND\_\-PASS} empty, no password is necessary.

      The variables
      \begin{itemize}
      \item \jump{IMONDENABLE}{IMOND\_ENABLE},
      \item \jump{IMONDDIAL}{IMOND\_DIAL},
      \item \jump{IMONDROUTE}{IMOND\_ROUTE}, and
      \item \jump{IMONDREBOOT}{IMOND\_REBOOT}
      \end{itemize}
      control whether providing the user password is sufficient to execute the
      control commands like Dial, Hangup, Reboot, or Changing the Default Route, or
      whether you need a special admin password for these requests (see below).}

    \config{IMOND\_ADMIN\_PASS}{IMOND\_ADMIN\_PASS}{IMONDADMINPASS}

    Default setting: \var{IMOND\_ADMIN\_\-PASS=''}

    {Using the Admin Passwords the client receives all the rights and
    can thus use all control functions of the server imond~-- regardless 
    of the content of the variables \var{IMOND\_\-ENABLE}, \var{IMOND\_\-DIAL} etc.
    If you leave \var{IMOND\_\-ADMIN\_\-PASS} empty, the user password is
    sufficient to gain all rights!
    }

    \config{IMOND\_LED}{IMOND\_LED}{IMONDLED}

    {The imond server is able to display the router's online/offline state via
     a LED. This LED is connected to a serial port as follows:

      25 pin connector:

\begin{example}
\begin{verbatim}
        20 DTR  -------- 1kOhm ----- >| ---------- 7 GND
\end{verbatim}
\end{example}


      9 pin connector:
\begin{example}
\begin{verbatim}
         4 DTR  -------- 1kOhm ----- >| ---------- 5 GND
\end{verbatim}
\end{example}

      The LED is on if an ISDN or DSL connection is established, otherwise
      it is off. If you want this the other way round you have to reverse the polarity of
      the LED. You can reduce the dropper resistor down to 470 ohm if the LED
      is lit too dimly.

      It is also possible to use two different coloured LEDs. In this
      case you have to connect the second LED together with a dropper
      resistor between DTR and GND too, but with reversed polarity. Then either the
      first or the second LED will be lit depending on the router's state.
      Another possibility is to use a DUO LED (two-coloured, three pins).

      Currently, the serial port's RTS pin behaves exactly as the DTR pin.
      You could even attach a third LED for displaying the
      online/offline state. However, this behaviour may change in the future.

      The variable \var{IMOND\_\-LED} has to be set to the name of the serial
      port to where the LED is attached; possible values are `com1', `com2',
      `com3', and `com4'. Leave the variable empty if you don't use an LED.}


    \config{IMOND\_BEEP}{IMOND\_BEEP}{IMONDBEEP}

    {If setting \var{IMOND\_\-BEEP}='yes', imond will emit a two-tone sound
      over the PC speaker whenever the router's state changes from offline to
      online and the other way round. In the first case, the higher tone
      follows the lower one. In the second case, the higher tone is emitted
      before the lower one.}


    \config{IMOND\_LOG}{IMOND\_LOG}{IMONDLOG}

     Default setting: \var{IMOND\_\-LOG='no'}
    
    {You can set \var{IMOND\_LOG}='yes' in order to log connections in the
      file \verb+/var/log/imond.log+. This file can be copied i.e. by scp to
      another host e.g. for statistical purposes. However, using scp requires
      you to install and configure the sshd package appropriately.

      The structure of the log file entries is described in Table \ref{tab:imondlog}.
      \begin{table}[htbp]
        \small
        \centering
        \caption{Structure of Imond log files}\label{tab:imondlog}
        \begin{tabular}{lp{12cm}}
          \hline
          Entry & Meaning \\
          \hline
          Circuit & the name of the circuit for which the entry has been created \\
          Start time & the date and time of dialing this circuit \\
          Stop time & the date and time of hanging up this circuit \\
          Online time & the time this circuit was online \\
          Billed time & the time for which the provider will charge you (depends on
          the timing) \\
          Costs & the costs the provider will charge to your account \\
          Bandwidth & the bandwidth used, separated into ``in'' and ``out''
          (``in'' coming first), presented as two unsigned integer numbers
          for which the following applies: Bandwidth =\newline
          \emph{4GiB~$*<$first~number$>+<$second~number$>$} \\
          Device & the device used for communication \\
          Invoice pulse & the invoice pulse used by the provider for charging
          (taken from the circuit configuration)\\
          Call charge & the fee charged per invoice pulse (taken from the
          circuit configuration)\\
          \hline
        \end{tabular}
      \end{table}

      The costs are denoted in Euro. These values are only meaningful if you
      correctly define the corresponding circuit variables
      \jump{ISDNCIRCxTIMES}{\var{ISDN\_CIRC\_x\_TIMES}}}.

    \config{IMOND\_LOGDIR}{IMOND\_LOGDIR}{IMONDLOGDIR}

    {If the imond log is activated, this variable can be used to choose an
    alternative log directory instead of the default /var/log, e.g. /boot.
    This is useful in order to make the log persistent on the boot medium.
    However, this requires the boot medium to be mounted read/write.

    The default value is 'auto' which lets the fli4l router to determine the
    storage location automatically. Depending on further configuration, the
    storage path is /boot/persistent/base or some other path determined by
    the FLI4L\_UUID variable. If neither FLI4L\_UUID is set nor /boot is mounted
    read/write, the log file can be found under /var/run.}

    \configlabel{IMOND\_DIAL}{IMONDDIAL}
    \configlabel{IMOND\_ROUTE}{IMONDROUTE}
    \configlabel{IMOND\_REBOOT}{IMONDREBOOT}
    \config{IMOND\_ENABLE  IMOND\_DIAL  IMOND\_ROUTE  IMOND\_REBOOT}{IMOND\_ENABLE}{IMONDENABLE}

    {These variables make certain imond commands available in user mode
    (enabling/disabling the ISDN interface, dialing/hanging up, changing the
    default route, rebooting the router).

    Default settings:

\begin{example}
\begin{verbatim}
        IMOND_ENABLE='yes'
        IMOND_DIAL='yes'
        IMOND_ROUTE='yes'
        IMOND_REBOOT='yes'
\end{verbatim}
\end{example}

      All other features of imond's Client-Server interface are described in a
      \jump{IMONDSCHNITTSTELLE}{separate chapter}.}

  \end{description}
