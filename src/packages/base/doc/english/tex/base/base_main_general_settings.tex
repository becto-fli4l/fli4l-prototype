% Synchronized to r51346

\section{General settings}

\begin{description}

  \config{HOSTNAME}{HOSTNAME}{HOSTNAME}

  Default Setting: \var{HOSTNAME='fli4l'}
  
  At the very beginning you should choose a name for your fli4l router.


  \config{PASSWORD}{PASSWORD}{PASSWORD}

  Default Setting: \var{PASSWORD='fli4l'}
  
  This password is needed for logging on to the router---regardless whether
  you use a keyboard attached to the router or a remote SSH console
  (for the latter you will need the sshd package). The minimum password
  length is 1, the maximum 126 characters.

  \config{BOOT\_TYPE}{BOOT\_TYPE}{BOOTTYPE}

  Default setting: \var{BOOT\_TYPE='hd'}

  \var{BOOT\_TYPE} determines the boot medium in the broadest sense and
  affects the drivers (kernel modules) and start scripts being included in
  the RootFS. A short description of the boot process for
  better understanding:

  \begin{itemize}
  \item The BIOS of the computer loads and starts the boot loader on the boot
        medium.
  \item The boot loader (typically syslinux) extracts, loads, and starts the
        kernel.
  \item The kernel extracts the RootFS (= the basic file system containing
        tools and scripts needed for booting), mounts the RootFS and begins to
        execute the start scripts.
  \item Depending on \var{BOOT\_TYPE}, the start scripts loads the kernel modules
        for the boot medium, mounts the boot partition, and extracts the OPT
        archive (\texttt{opt.img}) containing additional programs.
  \item Subsequently, fli4l starts to configure the individual services.
  \end{itemize}

  The following values are valid for \var{BOOT\_TYPE} at the moment:

  \begin{description}
  \item[ls120] Choose this to boot from LS120/240 and ZIP disks.
  \item[hd] Choose this to boot from a hard disk. You will find more information in the
    \jump{sec:hdinstall}{Documentation} of the HD package.
  \item[cd] Choose this to boot from CD-ROM. With this setting, the ISO image fli4l.iso will be created which
    you have to burn onto CD with your favourite CD burning application.
    Please pay attention to choosing the right driver for your CD drive.
  \item[integrated] Choose this if you do not plan to use a conventional boot
    medium but e.g. want to boot over a network. This setting integrates the
    OPT archive into the RootFS so the kernel can extract everything at once
    and does not need not mount a boot medium.\\
    \textbf{Note: } You cannot perform a remote update of your fli4l router in
    this case.
  \item[attached]
    This setting is similar to \textbf{integrated} but it does not integrate the
    contents of the OPT archive into the RootFS; rather, the OPT archive is
    put ``as is'' into the \texttt{/boot} directory. From there it will be
    extracted during the boot process.\\
    Apart from that, the caveats described for \texttt{integrated} apply to this
    boot type as well.
  \item[netboot]
    This setting corresponds to \textbf{integrated}. However, the script
    \texttt{mknetboot.sh} is additionally run to create an image for booting
    over the local network. Please read the wiki
    \altlink{https://ssl.nettworks.org/wiki/display/f/fli4l+und+Netzboot}
    for further information.\\
  \item[pxeboot]
    Two images are generated, kernel and rootfs.img. These are the two files to be loaded
    by the PXE bootloader. During execution the local tftp directory may be specified and
    in addition a subdirectory in the tftp directory (--pxesubdir).
    Refer to the Wiki here as well:
    \altlink{https://ssl.nettworks.org/wiki/display/f/fli4l+und+Netzboot}.
  \end{description}

   \textbf{Note:} How to configure fli4l as a boot-server (pxe/tftp) you 
    can find in the documentation of opt dns\_dhcp!
  
  \config{LIBATA\_DMA}{LIBATA\_DMA}{LIBATADMA}
  
  Default Setting: \var{LIBATA\_DMA='disabled'}

  This options selects if DMA is used for libata based Devices. It is needed 
  for example for incompletely wired IDE to CompactFlash Adapters. Select
  'enabled' to use DMA.

  \config{MOUNT\_BOOT}{MOUNT\_BOOT}{MOUNTBOOT}
  
  Default Setting: \var{MOUNT\_\-BOOT='rw'}
  
  {This variable specifies how to mount the boot medium. There are three
  possibilities:

    \begin{description}
    \item[rw]~-- Read/Write~-- Writing and reading is possible
    \item[ro]~-- Read-Only~-- Only reading is possible
    \item[no]~-- None~-- Medium will be unmounted after booting and can then
      be removed if desired
    \end{description}

    Some configurations require mounting the boot medium read/write, e.g. if
    you want to run a DHCP server or if you want the imond log file to be
    stored on the boot medium.}

  \config{BOOTMENU\_TIME}{BOOTMENU\_TIME}{BOOTMENUTIME}

  Default setting: \var{BOOTMENU\_TIME='20'}

  {This variable controls how LONG the syslinux boot loader should wait
   until the default installation is booted automatically.

    The \var{OPT\_RECOVER} variable of the HD package allows you to activate
    a function which enables you to create a recovery installation from a
    working installation. This recovery installation can be activated in the
    boot menu by choosing the recovery version.

    If this variable contains the value '0', the syslinux boot loader will
    wait indefinitely until the user chooses either the default or the
    recovery installation!}

  \config{TIME\_INFO}{TIME\_INFO}{TIMEINFO}
  
  Default Setting: \var{TIME\_INFO='MEZ-1MESZ,M3.5.0,M10.5.0/3'}

  Normally, Unix operating systems use the UTC (Coordinated Universal Time)
  for clocks running under their control, and so does fli4l. The UTC is
  consistent around the world and has to be converted to local time before
  use. By using \var{TIME\_\-INFO}, you provide the necessary information for
  fli4l about your time zone, its difference to UTC, and about daylight
  saving time. Your local hardware clock must be set to UTC (corresponds to
  London Standard Time) in order to make these settinge effective.
  Alternatively, you may use the chrony Package which allows fli4l to
  synchronize its clock with an external time server (time servers always
  provide the current time in UTC).

  The meaning of the possible settings \var{TIME\_\-INFO} are as
  follows:
\begin{example}
\begin{verbatim}
        TIME_INFO='MEZ-1MESZ,M3.5.0,M10.5.0/3'
\end{verbatim}
\end{example}
  \begin{itemize}
  \item \emph{MEZ-1:} ``MEZ'' is the German abbreviation for ``Central European
  Time'' (you could also use ``CET'' here). The ``-1'' means that \emph{MEZ-1=UTC},
  i.e. the MEZ is one hour ahead of UTC.
  \item \emph{MESZ:} ``MESZ'' is the abbreviation for ``Central European Summer
  Time'' and means that fli4l has to handle daylight saving changes.
  Because no further information (``+x'' or ``-x'') is given, the clock is
  adjusted forward one hour when reaching the daylight saving time.
  \item \emph{M3.5.0,M10.5.0/3:} This means that the change to the daylight
  saving time occurs on the last Sunday in March at two o'clock and that the
  change to standard time occurs on the last Sunday in October at three
  o'clock.
  \end{itemize}

  Normally you do not have to touch these settings, unless your fli4l router
  resides in another time zone. In this case you have to adjust these settings 
  accordingly. In order to do this properly, it is helpful to take a look at the
  specification of the TZ environment variable which can be found at the
  following URL:

  \altlink{http://pubs.opengroup.org/onlinepubs/009695399/basedefs/xbd_chap08.html}

  \config{RTC\_SYNC}{RTC\_SYNC}{RTCSYNC}{
  Default Setting: \var{RTC\_SYNC='hwclock'}

  Many computers have a battery-backed hardware clock, which is also supplied with power
  while the system is powered off. This enables the system time to be running continuously
  even then in order to have a valid system time at the next system startup.
  At this point it is important to differentiate between the \emph{system time}
  and the \emph{hardware time}:
 
  \begin{itemize}
  \item \emph{Hardware time} is the time stored in the hardware clock
  and kept up to date by the device. It is usually read at system startup and
  becomes system time then.

  \item \emph{System time} is the time used by the linux system, i.e. if executing
  the command \texttt{date -u}. It is kept up to date by the Linux kernel, for example
  using hardware interrupts on a regular basis (timer interrupts) and always
  indicates a time in coordinated Universal Time (UTC). Hence it is not affected
  by any time zone settings.
 
  \item \emph{localized system time} is only the conversion of system time to
  another time zone. On the fli4l router it is configured by the environment
  variable \texttt{TZ} (see \jump{TIMEINFO}{\var{TIME\_INFO}}). This is not of
  any intertest for the further explanations below.
  \end{itemize}
  
  With the help of this variable you may configure how fli4l handles the adjustment
  between hardware time and system time, meaning if and how often hardware time
  should be set to the system time. Such an adjustment is necessary because even
  the best hardware clock is not 100 percent accurate and tends to systematic
  drifting, in the long run it will be a bit too slow or too fast.
  
  There are basically two ways of synchronization:
  
  \begin{itemize}
  \item ``kernel'' mode: An NTP client is used to get the real time from outside
  (usually via the Internet or an external (radio) clock) and keep the system 
  time of the fli4l router up to date. The Linux kernel takes care of updating
  the hardware time, so no further synchronization is needed. The update by the
  Linux kernel is a little less accurate than updating by \texttt{hwclock}
  (see `` hwclock '' mode below), however, the quality of the update is
  less important because errors are compensated by the NTP client.
  
  This mode must also be used if no hardware clock at all exists. The Linux kernel
  will not keep hardware time up-to-date in this case simply because there is none. In
  order to have a realistic system time at all, the use of a NTP client should
  be mandatory.
   
  \item Modus ``hwclock'': At shutdown of the system (when running the stop-script\\
  \texttt{/etc/rc0.d/rc950.hwclock}) and in regular intervals (every 24 hours) a
  synchronization using the \texttt{hwclock} program will take place. Not only the
  hardware time is set, but \texttt{hwclock} also measures to what extent the system
  time differs from the hardware time. When starting the system the system time is 
  not taken directly from the hardware time, the time drift is also taken into account
  in order to minimize drifting of the system time. The drift is stored in the file
  \texttt{/etc/adjtime}. If a writable persistent medium is available, the drift is
  stored in \texttt{/var/lib/persistent/base/adjtime}, in this case \texttt{/etc/adjtime}
  is a symbolic link pointing there.
  
  This mode is incompatible with updating the system time by the help of an NTP client.
  That's because an NTP client automatically enables updating the hardware clock by
  the Linux kernel. It is however, of little sense or problematic that both \texttt{hwclock}
  and the Linux kernel at the same time try to keep the hardware time up to date.
  \end{itemize}
  
  It should be noted that if a hardware clock is available, the time stored there
  is \emph{always} interpreted as coordinated world time (UTC). The time zone
  defined by the variable \var{TIME\_INFO} does not affect the time stored
  in the hardware clock. Saving a localized non-UTC time in the hardware clock
  is \emph{not} supported by fli4l.
  
  Loading the system time from the hardware time is done only once at system startup.
  The Linux kernel then reads the time stored in the hardware clock and sets system
  right at the beginning f the boot process. In ``hwclock'' mode the time will be
  set again later when running the boot script \texttt{/etc/rc.d/rc100.hwclock},
  this time taking into account the system's time drift.}

  \config{KERNEL\_VERSION}{KERNEL\_VERSION}{KERNELVERSION}

  Chooses the version of the kernel to be used. According to the contents of
  this variable, the kernel and the kernel modules are selected from
  \emph{img/kernel-$<$kernel version$>$.$<$compression extension$>$} and
  \emph{opt/lib/modules/$<$kernel version$>$}, respectively.

  \config{KERNEL\_BOOT\_OPTION}{KERNEL\_BOOT\_OPTION}{KERNELBOOTOPTION}
  
  Default Setting: \var{KERNEL\_BOOT\_OPTION=''}

  The contents of this variable is appended to the kernel's command line
  defined in syslinux.cfg. Some systems require 'reboot=bios' for proper 
  rebooting, i.e. WRAP systems.

  \config{COMP\_TYPE\_ROOTFS}{COMP\_TYPE\_ROOTFS}{COMPTYPEROOTFS}

  Default setting: \var{COMP\_TYPE\_ROOTFS='xz'}

  This variable selects the compression method to be used for the RootFS
  archive. Possible values are 'xz', 'lzma', and 'bzip2'.

  \config{COMP\_TYPE\_OPT}{COMP\_TYPE\_OPT}{COMPTYPEOPT}

  Default setting: \var{COMP\_TYPE\_OPT='xz'}

  This variable selects the compression method to be used for the OPT archive.
  Possible values are 'xz', 'lzma', and 'bzip2'.

  \config{POWERMANAGEMENT}{POWERMANAGEMENT}{POWERMANAGEMENT} 
  
  Default Setting: \var{POWERMANAGEMENT='acpi'}
  
  {
  The kernel supports different flavours of power management: the somewhat aged
  APM and the newer ACPI. This variable lets you choose which flavour is to be
  used. Possible values are 'none' (no power management), 'acpi', and the two
  APM variants 'apm' and 'apm\_rm'. The latter uses a special processor mode
  before switching the router off.}

  \config{FLI4L\_UUID}{FLI4L\_UUID}{FLI4LUUID}
  
  Default Setting: \var{FLI4L\_UUID=''}

    {This variable contains an universally unique identifier (UUID) which is used
    to point to a place where persistent data can be stored, e.g. on a USB
    stick. The UUID can be generated on any Linux system (e.g. on the fli4l
    router) by executing \verb*?'cat /proc/sys/kernel/random/uuid'?.
    Each execution of this command above produces a new UUID which you can
    use in \var{FLI4L\_UUID} variable. If you create a directory on
    a persistent medium by the name of this UUID, this directory will
    be used to store configuration changes as well as persistent run-time data
    (e.g. DHCP leases). However, the corresponding packages has to support
    this persistence mechanism (see the documentation to check this).
    Typically, use 'auto' for the according storage location, instead 
    of a hard-coded path.

    If fli4l already stored data using this mechanism before configuring an
    UUID and creating the directory, this data can be found under
    /boot/persistent. In this case, you will have to manually move the data to
    the new location. We advice that you generate and configure
    the UUID at the very beginning, avoiding the migration later on.

    Additionally, please note that \var{MOUNT\_BOOT}='rw' is needed
    if the storage directory is located on the /boot partition.

    We suggest using the /data partition (with the UUID-named directory being a
    top-level directory there) or an USB stick for the storage location of
    persistent configuration and run-time data. The file systems allowed are
    VFAT or, if you use OPT\_HD all read-writable filesystems supported there.}

  \config{IP\_CONNTRACK\_MAX}{IP\_CONNTRACK\_MAX}{IPCONNTRACKMAX}
  
  Default Setting: \var{IP\_CONNTRACK\_MAX=''}

  This variable enables you to change the maximum number of simultaneously existing
  connections. Normally, a sensible value for this setting is computed
  automatically, based on the amount of your router's physical RAM. Table
  \ref{tab:connectiontracking} shows the defaults used.

    \begin{table}[ht!]
        \centering
        \caption{Automtically generated maximum number of simultaneous connections}\marklabel{tab:connectiontracking}{}
        \begin{tabular}{p{6cm}p{6cm}}
            RAM in MiB               &    simultaneous connections \\\hline
            16                       &    1024 \\
            24                       &    1280 \\
            32                       &    2048 \\
            64                       &    4096 \\
            128                      &    8192 \\
        \end{tabular}
    \end{table}

   If you use file sharing programs behind or on the router and your router
   has only little RAM, you will hit the maximum number of simultaneous
   connections fastly. This will prevent further connections to be
   established.\\
   This causes error messages as

\begin{example}
\begin{verbatim}
        ip_conntrack: table full, dropping packet
\end{verbatim}
\end{example}

or

\begin{example}
\begin{verbatim}
        ip_conntrack: Maximum limit of XXX entries exceeded
\end{verbatim}
\end{example}

    The variable \var{IP\_\-CONNTRACK\_\-MAX} changes the maximum
    number of simultaneously existing connections to a fixed value.
    Each possible connection consumes 350 bytes of RAM, which
    cannot be used for other things. If you e.g. choose the value '10000',
    you reserve about 3,34 MB RAM that are lost for any other usage
    (kernel, RAM disks, programs).

    If your router has 32 MiB RAM, it should not be much of a problem to
    reserve 2 or 3 MiB for the ip\_conntrack table. If only 16 MiB RAM or less
    are available you should be more conservative to prevent your router from
    running out of RAM.

    The setting currently being used can be display on the console by executing

\begin{example}
\begin{verbatim}
        cat /proc/sys/net/ipv4/ip_conntrack_max
\end{verbatim}
\end{example}

    and can be set on-the-fly by executing

\begin{example}
\begin{verbatim}
        echo "XXX" > /proc/sys/net/ipv4/ip_conntrack_max
\end{verbatim}
\end{example}

    where XXX denotes the number of entries.
    The entries of the \var{IP\_CONNTRACK} table can be displayed on the
    console by executing

\begin{example}
\begin{verbatim}
        cat /proc/net/ip_conntrack
\end{verbatim}
\end{example}

    and can be counted by executing

\begin{example}
\begin{verbatim}
        cat /proc/net/ip_conntrack | grep -c use
\end{verbatim}
\end{example}

  \config{LOCALE}{LOCALE}{LOCALE}

  Default setting: \var{LOCALE}='de'

  Meanwhile, some fli4l components support multiple languages, for example
  the console menu and the Web GUI. This variable lets you choose your
  preferred language. In addition, some components support a private setting
  to override this global setting if necessary. English is used as a fallback
  if the language chosen is not supported for some component.

  \var{KEYBOARD\_LOCALE}='auto' tries to find a keyboard layout that is
  compatible with the \var{LOCALE} setting.

  By now, the following values are possible: de, en, fr.

\end{description}
