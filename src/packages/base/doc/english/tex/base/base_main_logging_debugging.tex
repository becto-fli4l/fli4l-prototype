% Synchronized to r38340

\section{Hints To Identify Problems And Errors}

fli4l logs all output produced while booting into the file
(\emph{/var/tmp/boot.log}). After the boot process has finished you can
review this file at the console or using the web interface.

Sometimes it is useful to generate a more detailed trace of the start sequence, 
e.g. to analyze the boot process in case of problems.
The variable \var{DEBUG\_STARTUP} exists for this very reason. Other settings
help developers to find bugs in certain situations; these settings are also
documented in this section.

\begin{description}

  \config{DEBUG\_STARTUP}{DEBUG\_STARTUP}{DEBUGSTARTUP}
  
    Default Setting: \var{DEBUG\_STARTUP='no'}

    If set to `yes', each command to be executed is written to the console
    while booting. As a change in syslinux.cfg is necessary for enabling
    this functionality, everything mentioned for \var{SER\_CONSOLE}
    also applies to this case. If you want to adapt syslinux.cfg by hand, you
    need to insert \verb+fli4ldebug=yes+ to it. Nevertheless,
    \var{DEBUG\_STARTUP} needs to be set to `yes'.

    \config{DEBUG\_MODULES}{DEBUG\_MODULES}{DEBUGMODULES} 
    
    Default Setting: \var{DEBUG\_MODULES='no'}
    
    Some modules are loaded automatically by the kernel without further
    notification. \var{DEBUG\_MODULES='yes'} activates a mode showing the
    sequence of all modules being loaded, regardless whether they are
    loaded explicitly by a script or automatically by the kernel.

    \config{DEBUG\_ENABLE\_CORE}{DEBUG\_ENABLE\_CORE}{DEBUGENABLECORE}
    
    Default Setting: \var{DEBUG\_ENABLE\_CORE='no'}
    
    If this setting is activated, every program crash causes the creation of
    a so-called ``core dump'', a memory image of the process just before
    the crash. These files are saved in the directory \texttt{/var/log/dumps}
    on the router and can be helpful in finding program errors. More details
    details can be found in the section \jump{sec:debugging}
    {``Debugging programs on the fli4l''} in the documentation of the SRC
    package.

    \config{DEBUG\_MDEV}{DEBUG\_MDEV}{DEBUGMDEV}
    
    Default Setting: \var{DEBUG\_MDEV='no'}
    
    With \var{DEBUG\_MDEV='yes'} all actions related to the \texttt{mdev}
    daemon will be logged, in detail all additions or removals of device
    nodes in \texttt{/dev} or the loading of firmware. Output is directed
    to the file \texttt{/dev/mdev.log}.

    \config{DEBUG\_IPTABLES}{DEBUG\_IPTABLES}{DEBUGIPTABLES}
    
    Default Setting: \var{DEBUG\_IPTABLES='no'}
    
    With \var{DEBUG\_IPTABLES='yes'} all \texttt{iptables} invocations are
    logged to \texttt{/var/log/iptables.log}, including the return values.

    \config{DEBUG\_IP}{DEBUG\_IP}{DEBUGIP}
    
    Default Setting: \var{DEBUG\_IP='no'}
    
    With \var{DEBUG\_IP='yes'} all invocations of the program \texttt{/sbin/ip}
    are logged to the file \texttt{/var/log/wrapper.log}.

\end{description}
