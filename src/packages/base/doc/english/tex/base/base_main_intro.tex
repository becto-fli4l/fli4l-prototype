% Synchronized to r34514
\chapter{Documentation of the base package}

\section{Introduction}

fli4l is a Linux-based router, capable of handling ISDN, DSL, UMTS, and
ethernet connections, with little hardware requirements: an USB stick used for
booting, an Intel Pentium MMX processor, 64 MiB RAM as well as (at least) one
ethernet network adapter are completely sufficient. The necessary boot medium
can be created under Linux, Mac OS~X or MS~Windows. You don't need any specific Linux
knowledge, but it is definitely helpful. However, you should possess basic
knowledge about networking, TCP/IP, DNS, and routing. For developing your own
extensions exceeding the basic configuration, you will need a working Linux
system as well as Linux skills.

fli4l supports various boot media, among them USB sticks, hard disks, CDs, and
last but not least booting over the network. An USB stick is in many respects
ideal:  Today, almost every PC can boot from it, it is relatively cheap, it is
big enough, and installing fli4l onto it is relatively easy under both
MS~Windows and Linux. In contrast to a CD it is writable and thus additionally
able to hold non-volatile configuration data (as e.g. DHCP leases).

\begin{itemize}

\item General features

\begin{itemize}
\item Creation of boot media under \jump{sec:bootmedien_linux}{Linux},
      \jump{sec:bootmedien_linux}{Mac OS~X}, and
      \jump{sec:bootmedien_windows}{MS~Windows}
\item Configuration through flat ASCII/UTF-8 files
\item Support for IP masquerading and port forwarding
\item Least Cost Routing (LCR): automatic provider selection based on daytime
\item Displaying/Computing/Logging of connection times and costs
\item MS~Windows/Linux client imonc talking to imond and telmond
\item Upload of updated configuration files via MS~Windows client imonc or via
      SCP under Linux
\item Boot media use the VFAT file system as permanent storage
\item Packet filter: External access to blocked ports is logged
\item Uniform mapping of WAN interfaces to so-called circuits
\item Running ISDN and DSL/UMTS circuits in parallel is possible
\end{itemize}

\item Router basics

\begin{itemize}
\item Linux kernel 3.18 or 3.19
\item Packet filter and IP masquerading
\item Local DNS server in order to reduce the number of DNS queries to external
      DNS servers
\item Remotely accessible imond server daemon for monitoring and controlling
      Least Cost Routing
\item Remotely accessible telmond server daemon logging incoming phone calls
\end{itemize}

\item Ethernet support

\begin{itemize}
\item Up-to-date network device drivers: Support for more than 140 adapter types
\end{itemize}

\item DSL support

\begin{itemize}
\item Roaring Penguin PPPoE driver supporting Dial-on-Demand (can be switched
      off)
\item PPTP for DSL providers in Austria and the Netherlands
\end{itemize}

\item ISDN support

\begin{itemize}
\item Support for some 60 adapter types
\item Multiple possibilities for ISDN connectons: incoming/outgoing/callback,
      raw/point-to-point (ppp)
\item Channel bundling: automatic band width adaptation or manual activation of
      the second channel using MS~Windows/Linux client software
\end{itemize}

\item Optional software packages

\begin{itemize}
\item DNS server
\item DHCP server
\item SSH server
\item Simple online/offline display using a LED
\item Serial console
\item Minimalistic Web server for ISDN and DSL monitoring as well as
      for reconfiguring and/or updating the router
\item Ability to let external hosts access LAN hosts in a controlled manner
\item Support for PCMCIA cards (called PC cards nowadays)
\item Logging of system messages
\item Configuration of ISAPnP cards by the use of isapnp tools
\item Additional tools for debugging
\item Configuration of the serial port
\item Rescue system for remote administration over ISDN
\item Software for displaying configurable information on an LCD, e.g.
      transmission rates, CPU load etc.
\item PPP server/router over the serial port
\item ISDN modem emulator over the serial port
\item Print server
\item Time synchronization with external time servers
\item Execution of user-defined commands on incoming phone calls (e.g.
      to perform Internet dial-up)
\item Support for IP aliasing (multiple IP addresses per network interface)
\item VPN support
\item IPv6 support
\item WLAN support: fli4l can be an access point as well as a client
\item RRD tool for monitoring the fli4l
\item and much more\ldots
\end{itemize}

\item Hardware requirements

\begin{itemize}
\item Intel Pentium processor with MMX support
\item 64 MiB RAM, better 128 MiB
\item Ethernet network adapter
\item ISDN: supported ISDN adapter
\item an USB stick, an ATA hard disk or a CF card (which is accessed the same
      way as an ATA hard disk); alternatively, booting from a CD is also
      possible
\end{itemize}

\item Software requirements

The following tools are required on Linux systems:

\begin{itemize}
\item GCC and GNU make
\item syslinux
\item mtools (mcopy)
\end{itemize}

No additional tools are required on MS~Windows systems, all necessary tools are
provided by fli4l.

\end{itemize}

Last but not least, the client utility imonc exists for controlling the router
and for displaying the router's state. This tool is available for MS~Windows
(windows/imonc.exe) and also for Linux (unix/gtk-imonc).

And now \ldots \bigskip

Have fun with fli4l!\bigskip

Frank Meyer and the fli4l team

\email{team@fli4l.de}
