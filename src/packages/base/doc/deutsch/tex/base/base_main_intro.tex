% Last Update: $Id$
\chapter{Dokumentation des Basispaketes}

\section{Einleitung}

fli4l ist ein auf Linux basierender ISDN-, DSL-, UMTS- und Ethernet-Router mit
geringen Anforderungen an die zugrunde liegende Hardware: Ein USB-Stick als
Bootmedium, ein Intel Pentium MMX-Prozessor, 64 MiB RAM sowie (mindestens) eine
Ethernet-Netzwerkkarte sind dafür vollkommen ausreichend. Das notwendige
Bootmedium kann unter Linux, Mac OS~X oder MS~Windows erstellt werden. Dabei sind keine
Linux-Kenntnisse erforderlich, aber durchaus hilfreich. Grundkenntnisse von
Netzwerken, TCP/IP, DNS und Routing sollten jedoch vorhanden sein. Für eigene
Erweiterungen/Entwicklungen, welche über die Standardkonfiguration hinausgehen,
sind ein lauffähiges Linux-System und Linux-Kenntnisse notwendig.

fli4l unterstützt verschiedene Bootmedien, darunter USB-Sticks, Festplatten,
CDs und nicht zuletzt das Booten über das Netzwerk. Ein USB-Stick ist in
vielerlei Hinsicht ideal: Heutzutage kann so gut wie jeder PC von einem
USB-Stick starten, er ist recht erschwinglich, er hat eine ausreichende Größe,
und man kann sowohl unter Linux als auch unter MS~Windows auf relativ einfache
Weise eine fli4l-Installation darauf ablegen. Auch ist er im Gegensatz zu einer
CD beschreibbar und kann somit nichtflüchtige Konfigurationsdaten (wie z.B.
DHCP-Leases) speichern.

\begin{itemize}

\item Allgemeine Features

\begin{itemize}
\item Erstellen von Bootmedien unter \jump{sec:bootmedien_linux}{Linux},
      \jump{sec:bootmedien_linux}{Mac OS~X} und
      \jump{sec:bootmedien_windows}{MS~Windows}
\item Konfiguration über normale ASCII/UTF-8-Dateien
\item Unterstützung von IP-Masquerading und Portweiterleitung
\item Least-Cost-Routing (LCR): automatische Auswahl des Providers, je nach
      Uhrzeit
\item Anzeige/Berechnung/Protokollierung von Verbindungszeiten und -kosten
\item MS~Windows/Linux-Client imonc mit Schnittstelle zu imond und telmond
\item Upload von aktualisierten Konfigurationsdateien über MS~Windows-Client
      imonc oder via SCP unter Linux
\item Bootmedien nutzen das VFAT-Dateisystem zum dauerhaften Speichern von
      Dateien
\item Paketfilter: Protokollieren von Zugriffen von außen auf gesperrte Ports
\item Einheitliche Abbildung von WAN-Schnittstellen auf sogenannte Circuits
\item Paralleler Betrieb von ISDN- und DSL/UMTS-Circuits ist möglich
\end{itemize}

\item Router-Basisfunktionalität

\begin{itemize}
\item Linux-Kernel 3.18 oder 3.19
\item Paketfilter und IP-Masquerading
\item DNS-Server, um die Anzahl von DNS-Abfragen an externe DNS-Server zu
      reduzieren
\item Netzwerkfähiger imond-Server mit Monitor- und LCR-Steuerfunktionen
\item Netzwerkfähiger telmond-Server zur Protokollierung von eingehenden
      Telefonanrufen
\end{itemize}

\item Ethernet-Unterstützung

\begin{itemize}
\item Aktuelle Netzwerkkartentreiber: Unterstützung von über 140 Kartentypen
\end{itemize}

\item DSL-Unterstützung

\begin{itemize}
\item Roaring Penguin PPPoE-Treiber, mit Dial-on-Demand (abschaltbar)
\item PPTP für DSL-Anbindungen in Österreich und den Niederlanden
\end{itemize}

\item ISDN-Unterstützung

\begin{itemize}
\item Unterstützung von knapp 60 ISDN-Kartentypen
\item Mehrere ISDN-Verbindungsmöglichkeiten: eingehend/ausgehend/Rückruf,
      "`roh"'/Punkt-zu-Punkt (ppp)
\item Kanalbündelung: automatische Bandbreitenanpassung oder manuelle
      Zuschaltung des zweiten Kanals über MS~Windows-/Linux-Client
\end{itemize}

\item Optionale Programmpakete

\begin{itemize}
\item DNS-Server
\item DHCP-Server
\item SSH-Server
\item Einfache Online/Offline-Anzeige über LED
\item Serielle Konsole
\item Mini-Webserver für ISDN- und DSL-Monitoring sowie zur Rekonfiguration
      und/oder Aktualisierung des Routers
\item Zugangserlaubnis für bestimmte konfigurierte Netzwerke von außen
\item Unterstützung für PCMCIA-Karten (heutzutage PC-Cards genannt)
\item Protokollierung von Systemmeldungen
\item Konfiguration von ISAPnP-Karten mit den isapnp-Werkzeugen
\item Zusätzliche Werkzeuge zum Debugging
\item Konfiguration der seriellen Schnittstelle
\item Notfallsystem zur Fernwartung über ISDN
\item Software zur Anzeige konfigurierbarer Informationen auf einem LCD, z.B.
      von Übertragungsraten, CPU-Auslastung etc.
\item PPP-Server/Router über serielle Schnittstelle
\item ISDN-Modem-Emulator über serielle Schnittstelle
\item Druckerserver
\item Synchronisierung der Uhrzeit mit externen Zeit-Servers
\item Ausführen von benutzerdefinierten Kommandos bei eingehenden Telefonanrufen
      (z.B. um ein Internet-Einwahl durchzuführen)
\item Unterstützung von IP-Aliasing (mehrere IP-Adressen pro
      Netzwerkschnittstelle)
\item VPN-Unterstützung
\item IPv6-Unterstützung
\item WLAN-Unterstützung: fli4l kann sowohl Zugangsknoten als auch Client sein
\item RRD-Tool zum Überwachen des fli4l
\item und vieles andere mehr\ldots
\end{itemize}

\item Hardwarevoraussetzungen

\begin{itemize}
\item Intel Pentium-Prozessor mit MMX Unterstützung
\item 64 MiB Speicher, besser 128 MiB
\item Ethernet-Netzwerkkarte
\item ISDN: unterstützte ISDN-Karte
\item ein USB-Stick, eine ATA-Festplatte oder eine CF-Karte (die genauso wie
      eine ATA-Festplatte angesprochen wird); alternativ ist auch der Start von
      CD möglich
\end{itemize}

\item Softwarevoraussetzungen

Unter Linux werden folgende Programme vorausgesetzt:

\begin{itemize}
\item GCC und GNU make
\item syslinux
\item mtools (mcopy)
\end{itemize}

Unter MS~Windows werden keine zusätzlichen Werkzeuge benötigt, fli4l bringt
alles Notwendige mit.

\end{itemize}

Zusätzlich gibt es zur Steuerung/Statusanzeige des fli4l-Routers noch
den Client imonc. Dieses Programm ist für MS~Windows (windows/imonc.exe)
und auch für Linux (unix/gtk-imonc) vorhanden.

Und nun \ldots \bigskip

Viel Spaß mit fli4l!\bigskip

Frank Meyer und das fli4l-Team 

\email{team@fli4l.de}
