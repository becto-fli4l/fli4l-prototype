% Last Update: $Id$
\section{Länderspezifische Tastaturlayouts}

\begin{description}

  \config{KEYBOARD\_LOCALE}{KEYBOARD\_LOCALE}{KEYBOARDLOCALE}

  Standard-Einstellung: \var{KEYBOARD\_LOCALE='auto'}
  
  Wenn man öfter direkt auf dem fli4l Router arbeitet ist ein lokales
  Tastaturlayout eine willkommene Hilfe. Mit
  \var{KEYBOARD\_LOCALE='auto'} wird versucht, ein Tastaturlayout passend zu
  der Einstellung von \var{LOCALE} zu benutzen. Mit der
  Einstellung \var{''} wird kein lokales Tastaturlayout auf dem
  fli4l--Router installiert; es wird dann das im Kernel anwesende
  Standardlayout verwendet. Alternativ kann man auch den Namen einer
  lokalen Tastaturlayoutmap direkt angeben. Wenn z.B. \var{'de-latin1'}
  eingestellt wird, prüft der Buildprozess ob in opt/etc eine Datei mit
  Namen de-latin1.map vorliegt. Wenn ja, wird die entsprechende .map-Datei
  als Tastaturlayout eingebunden.

  \config{OPT\_MAKEKBL}{OPT\_MAKEKBL}{OPTMAKEKBL}
  
  Standard-Einstellung: \var{OPT\_MAKEKBL='no'}
  
  Wenn man für seine Tastatur eine map Datei erstellen will muss man
  wie folgt vorgehen:

  \begin{itemize}
    \item \var{OPT\_MAKEKBL} auf `yes' setzen.
    \item Auf dem Router 'makekbl.sh' anrufen. Vorzugsweise macht man dies über
      eine ssh-Verbindung weil das Tastaturlayout sich ändert und das lästig sein
      kann.
    \item Die Anweisungen befolgen.
    \item Ihre neue $<$locale$>$.map Datei liegt in der /tmp.
    \item Die Arbeiten direkt auf dem Router sind jetzt abgeschlossen.
    \item Nun kopieren Sie die eben erzeugte Tastaturtabelle in Ihr fli4l"=Verzeichnis unter 
      opt/etc/$<$locale$>$.map. Wenn Sie jetzt \var{KEYBOARD\_LOCALE}='$<$locale$>$' setzen 
      wird beim nächsten Buildprozess Ihre neu erzeugtes Tastaturlayout benutzt.
    \item Vergessen Sie nicht \var{OPT\_MAKEKBL} wieder auf `no' zu setzen.
  \end{itemize}
\end{description}