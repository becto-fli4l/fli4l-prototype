% Last Update: $Id$
\section{Netzwerk-Konfiguration (IPv4)}

\begin{description}
  \config{OPT\_IPV4}{OPT\_IPV4}{OPTIPV4}{
  Aktiviert die IPv4-Unterstützung.
  
  \wichtig{Diese Einstellung muss zur Zeit den Wert 'yes' beinhalten! Ein
  reiner IPv6-Betrieb ist zur Zeit noch \emph{nicht} möglich!}

  Standard-Einstellung:
  }
  \verb*?OPT_IPV4='yes'?

  \config{IP\_NET\_N}{IP\_NET\_N}{IPNETN}
  
  Standardwert: \var{IP\_NET\_N='1'}
  
  {Anzahl der Netzwerke, die an das IPv4-Pro\-to\-koll gebunden werden sollen.
  Im Normalfall also `1'. Falls es keine Netzwerke geben sollte oder sie über einen 
  anderen Weg konfiguriert werden, und daher \var{IP\_NET\_N} auf 0 gesetzt wird, 
  wird beim Bauen der Archive eine Warnung ausgegeben. Diese kann man durch setzen 
  von \marklabel{IGNOREIPNETWARNING}{\var{IGNOREIPNETWARNING}='yes'} aussschalten.}

  \config{IP\_NET\_x}{IP\_NET\_x}{IPNETx}

  Standardwert: \var{IP\_NET\_1='192.168.6.1/24'}

  {Die IPv4-Adresse und die Netzmaske in CIDR\footnote{Classless Inter-Domain
   Routing}-Schreibweise des x'ten Devices im fli4l-Router. Wird die
   IPv4-Adresse dynamisch durch einen DHCP-Klienten zugewiesen, kann
   hier auch 'dhcp' als Wert eingetragen werden.

   Nachfolgende Tabelle zeigt CIDR und Punkt-Schreibweise (DOT
   Notation).

   \marklabel{tab:cidr}{
     \begin{tabular}[h!]{rcc}
       CIDR &     Netzmaske   & Anzahl IPs \\
       \hline
       \hline
       /8   & 255.0.0.0       & 16777216   \\
       /16  & 255.255.0.0     & 65536      \\
       /23  & 255.255.254.0   & 512        \\
       /24  & 255.255.255.0   & 256        \\
       /25  & 255.255.255.128 & 128        \\
       /26  & 255.255.255.192 & 64         \\
       /27  & 255.255.255.224 & 32         \\
       /28  & 255.255.255.240 & 16         \\
       /29  & 255.255.255.248 & 8          \\
       /30  & 255.255.255.252 & 4          \\
       /31  & 255.255.255.254 & 2          \\
       /32  & 255.255.255.255 & 1
     \end{tabular}
   }

   \textbf{Anmerkung: } Da jeweils eine IPv4 für die Netzwerk- und Broadcast-Adresse
   reserviert sind, errechnet sich die maximale Anzahl der Hosts im Netzwerk
   zu: \texttt{Anzahl\_Hosts = Anzahl\_IPs - 2}. Die kleinste mögliche Netzmaske
   wäre also \texttt{/30}, entspricht 4 IPs und somit 2 möglichen Hosts.
   }

  \config{IP\_NET\_x\_DEV}{IP\_NET\_x\_DEV}{IPNETxDEV}

  Standard-Einstellung: \var{IP\_\-NET\_\-1\_\-DEV}='eth0'
  
  {Erforderlich: Device-Name der Netzwerkkarte.

    Ab Version 2.1.8 ist die Angabe des verwendeten Device zwingend
    erforderlich! Die Namen der Devices beginnen in den meisten Fällen mit 
    \texttt{'eth'} gefolgt von einer Zahl. Die erste vom System erkannte Netzwerkkarte
    bekommt den Namen \texttt{'eth0'}, die zweite \texttt{'eth1'} usw.\\

    Beispiel:

    \begin{example}
    \begin{verbatim}
        IP_NET_1_DEV='eth0'
    \end{verbatim}
    \end{example}

    fli4l beherrscht auch IP-Aliasing, also die Zuweisung von mehreren IPs auf
    eine Netzwerkkarte. Zusätzliche IPs definiert man einfach mit einem weiteren
    Netzwerk auf dem selben Device. Beim Überprüfen der Konfiguration informiert
    mkfli4l darüber, dass man ein solches Alias definiert --- diese Warnung kann
    man dann ignorieren.

    Beispiel:

    \begin{example}
    \begin{verbatim}
        IP_NET_1='192.168.6.1/24'
        IP_NET_1_DEV='eth0'
        IP_NET_2='192.168.7.1/24'
        IP_NET_2_DEV='eth0'
    \end{verbatim}
    \end{example}
    }

  \config{IP\_NET\_x\_MAC}{IP\_NET\_x\_MAC}{IPNETxMAC}

  Standard-Einstellung: \var{IP\_\-NET\_\-1\_\-MAC}=''
   
  {Optional: MAC-Adresse der Netzwerkkarte.
    
    Hier kann die Hardwareadresse (MAC) der Netzwerkkarte angepasst werden.
    Dies ist zum Beispiel nützlich, wenn Sie einen DHCP-Provider nutzen wollen,
    der eine bestimmte MAC-Adresse erwartet.
    Wird \var{IP\_NET\_x\_MAC} leer gelassen oder gar nicht angegeben, so wird
    die voreingestellte MAC-Adresse der Netzwerkkarte verwendet.
    Die allermeisten Nutzer werden diese Variable nicht benötigen.

    Beispiel:

    \begin{example}
    \begin{verbatim}
        IP_NET_1_MAC='01:81:42:C2:C3:10'
    \end{verbatim}
    \end{example}
    }

  \config{IP\_NET\_x\_NAME}{IP\_NET\_x\_NAME}{IPNETxNAME}

  Standard-Einstellung: \var{IP\_\-NET\_\-x\_\-NAME}=''
  
  {Optional: Der IPv4-Adresse der Netzwerkkarte einen Namen geben.
    
    Bei umgekehrter Namensauflösung der IPv4-Adresse der Netzwerkkarte erscheint
    standardmässig ein Name der Form 'fli4l-ethx.$<$domain$>$'. In
    \var{IP\_NET\_x\_NAME} kann man selbst einen anderen Namen angeben.
    Dieser Name wird dann bei umgekehrter Namensauflösung angezeigt.
    Bei einer öffentlichen IPv4-Adresse kann man so erreichen, dass immer
    der öffentliche Name aufgelöst wird.

    Beispiel:

    \begin{example}
    \begin{verbatim}
        IP_NET_2='80.126.238.229/32'
        IP_NET_2_NAME='ajv.xs4all.nl'
    \end{verbatim}
    \end{example}
    }

  \config{IP\_NET\_x\_TYPE}{IP\_NET\_x\_TYPE}{}

  \config{IP\_NET\_x\_COMMENT}{IP\_NET\_x\_COMMENT}{IPNETxCOMMENT}
    
    Standard-Einstellung: \var{IP\_NET\_x\_COMMENT}=''
    
    {Optional:  Die Angabe dient dazu, einem Device einen 'aussagekräftigen' Namen zu 
    geben. Dieser kann dann in Paketen wie z.B. rrdtool zur Identifikation des Netzwerks 
    verwendet werden.
    }

\end{description}
