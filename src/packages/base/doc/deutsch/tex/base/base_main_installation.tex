% Last Update: $Id: base_main_installation.tex 51849 2018-03-06 14:55:21Z kristov $
\chapter{Installation und Konfiguration}

\section{Entpacken der Archive}

Unter Linux:

\begin{verse}\texttt{tar xvfz fli4l-\version.tar.gz}\end{verse}

\noindent Funktioniert dies nicht, geht's auch so:

\begin{verse}\texttt{gzip -d < fli4l-\version.tar.gz | tar xvf -}\end{verse}

Wer die aktuelle Version in einem bereits existierenden
fli4l-Verzeichnis installiert, sollte anschließend
\texttt{mkfli4l.sh -c} aufrufen, also:

\begin{verse}
    \texttt{cd fli4l-\version}\\
    \texttt{sh mkfli4l.sh -c}
\end{verse}

Es wird jedoch empfohlen, ein neues Verzeichnis für eine neue Version
zu benutzen~-- die Konfiguration kann durch ein entsprechendes Werkzeug zum
Dateivergleich sehr einfach übernommen werden.

Unter MS~Windows kann das komprimierte Tar-Archiv zum Beispiel mit WinZip
extrahiert werden. Dabei ist jedoch zu beachten, dass die Dateien
\emph{mit} Unterverzeichnissen (Einstellung in WinZip überprüfen!)
ausgepackt werden. Außerdem ist in \emph{Optionen \pfeil
  Konfiguration} die so genannte "`Smart TAR CR conversion"'
abzuschalten. Ist diese eingeschaltet, werden einige wichtige
Dateien von WinZip falsch extrahiert.

Alternativ ist das OpenSource-Programm 7-Zip (\altlink{http://www.7-zip.org/})
sehr zu empfehlen, welches ebenso mächtig wie WinZip ist.

Es werden folgende Dateien im Unterverzeichnis
\texttt{fli4l-\version/} installiert:

\begin{itemize}
\item Dokumentation:
  \begin{itemize}
  \item doc/deutsch/* Deutsche Dokumentation
  \item doc/english/* Englische Dokumentation
  \item doc/french/*  Französische Dokumentation
  \end{itemize}

\item Konfiguration:
  \begin{itemize}
  \item config/*.txt Konfigurationsdateien, diese müssen bearbeitet
    werden
  \end{itemize}

\item Skripte/Prozeduren:
  \begin{itemize}
  \item mkfli4l.sh Boot-Medium oder Dateien erzeugen: Linux/Unix-Version
  \item mkfli4l.bat Boot-Medium erzeugen: Windows-Version
  \end{itemize}

\item Kernel/Boot-Dateien:
  \begin{itemize}
  \item img/kernel Linux-Kernel
  \item img/boot*.msg Bootscreen Texte
  \end{itemize}

\item Zusatzpakete:
  \begin{itemize}
  \item opt/*.txt Diese Dateien beschreiben, was bei welchen Einstellungen in
  das Archiv opt.img gelangt.
  \item opt/... Optionale Kernel-Module, Dateien und Programme
  \end{itemize}

\item Quellcode:
  \begin{itemize}
  \item src/* Quellcode/Werkzeuge für Linux, siehe src/README
  \end{itemize}

\item Programme:
  \begin{itemize}
  \item unix/mkfli4l* Erzeugen des Bootmediums: Unix/Linux-Version
  \item windows/* Erzeugen des Bootmediums: Windows-Version
  \item unix/imonc* imond-Client für Unix/Linux
  \item windows/imonc/* imond-Client für Windows
  \end{itemize}
\end{itemize}

\section{Konfiguration}
\subsection{Editieren der Konfigurationsdateien}

Zur Konfiguration von fli4l müssen lediglich die Dateien config/*.txt
angepasst werden. Um im Nachhinein die eigenen Konfiguration mit der
ausgelieferten vergleichen zu können oder um mehrere Konfigurationen
verwalten zu können, empfiehlt es sich, eine Kopie des
config-Verzeichnisses anzulegen und die Konfiguration in dieser Kopie
durchzuführen. Ein Vergleich der Konfigurationen ist dann durch
Verwendung eines geeigneten Werkzeugs (z.B. ``diff'' unter *nix) relativ
einfach möglich. Nehmen wir einmal an, die eigene config liegt in einem
Verzeichnis mit Namen ``meine\_config'' ebenfalls im fli4l-Verzeichnis
dann wäre der Aufruf wie folgt:
\begin{example}
\begin{verbatim}
    ~/src/fli4l> diff -u {config,meine_config}/build/rc.cfg | grep '^[+-]'
    --- config/build/rc.cfg    2014-02-18 15:34:39.085103706 +0100
    +++ meine_config/build/rc.cfg        2014-02-18 15:34:31.094317441 +0100
    -PASSWORD='/P6h4iOIN5Bbc'
    +PASSWORD='3P8F3KbjYgzUc'
    -NET_DRV_1='ne2k-pci'
    +NET_DRV_1='pcnet32'
    -START_IMOND='no'
    +START_IMOND='yes'
    -OPT_PPPOE='no'
    +OPT_PPPOE='yes'
    -PPPOE_USER='anonymer'
    -PPPOE_PASS='surfer'
    +PPPOE_USER='ich'
    +PPPOE_PASS='mein-passwd'
    -OPT_SSHD='no'
    +OPT_SSHD='yes'
\end{verbatim}
\end{example}

Man sieht hier auch sehr schön, dass ein einfacher DSL-Router mit
wenigen Handgriffen konfiguriert ist, auch wenn einen die
Konfigurationsdateien auf den ersten Blick mit ihrer Fülle von
Einstellungsmöglichkeiten erschlagen.

\subsection{Konfiguration über eine spezielle Konfigurationsdatei}

Da sich die Konfiguration durch das Modul-Konzept auf verschiedene
Dateien verteilt, und das Bearbeiten dadurch unter Umständen etwas
mühsam wird, kann man die Konfiguration auch in einer einzelnen Datei
namens \emph{$<$config~verzeichnis$>$/\_fli4l.txt} ablegen, deren Inhalt
dann zusätzlich zu den normalen Konfigurationsdateien eingelesen wird
und deren Inhalt dominiert. Um beim obigen Beispiel zu bleiben: Um
einen einfachen DSL-Router zu konfigurieren, könnten wir einfach
folgendes in diese Datei schreiben:

\begin{example}
\begin{verbatim}
    PASSWORD='3P8F3KbjYgzUc'
    NET_DRV_N='1'
    NET_DRV_1='pcnet32'
    START_IMOND='yes'
    OPT_PPPOE='yes'
    PPPOE_USER='ich'
    PPPOE_PASS='mein-passwd'
    OPT_SSHD='yes'
\end{verbatim}
\end{example}

Man sollte vermeiden, beide Konfigurationsvarianten zu mischen.

\subsection{Variablen}

  Sie werden merken, dass einige Variablen auskommentiert sind. Wenn das der 
  Fall ist, erhält sie eine sinnvolle Standard-Belegung. Diese Standard-Belegung 
  ist für jede Variable dokumentiert. Wünschen Sie einen anderen Wert für diese 
  Variable, sollten Sie das Kommentarzeichen am Anfang der Variablendefinition ('\#') 
  entfernen und den entsprechenden Wert zwischen den Hochkommata einfügen.

\marklabel{VARIANTEN}{\section{Installationsvarianten}}

In den vorhergehenden Versionen von fli4l wurde lediglich das Booten
von einer Diskette unterstützt. Dies ist aus oben genannten Gründen nun
nicht mehr möglich, aber die Alternative mittels eines USB-Sticks ist 
gegeben.

Es sind auch eine Vielzahl anderer Bootmedien (CD, HD, Netzwerk,
Compact-Flash, DoC, \ldots) möglich und fli4l kann auch auf diversen
Medien installiert (HD, Compact-Flash, DoC) werden. Dazu kann fli4l auf
drei verschiedenen Wegen gebootet werden:

\begin{description}
\item [Single Image] Der Bootloader lädt den Linux-Kern und dann fli4l
als ein einziges Image --- danach kann fli4l ohne weiteren Zugriff auf
andere Medien booten. Beispiele dafür sind die Boottypen \emph{integrated}, 
\emph{attached}, \emph{netboot} und \emph{cd}.
\item [Split Image] Der Bootloader lädt den Linux-Kern und dann ein
rudimentäres fli4l-Image, dass die Bootmedien einbindet und die
Konfiguration und restlichen Dateien aus einem dort liegenden Archiv
holt. Beispiele dafür sind diese Boottypen: \emph{hd (Typ A)}, \emph{ls120},
\emph{attached} und \emph{cd-emul}.
\item [Installation auf einem Medium] Der Bootloader lädt den
Linux-Kern und dann ein rudimentäres fli4l-Image, das eine bereits
vorhandene fli4l-Installation in sein Dateisystem einbindet und damit
keine weiteren Archive auspacken muss. Eine HD-Installation vom Typ B
ist ein Beispiel dafür.
\end{description}

Man sollte jedoch zunächst erst einmal fli4l in einer minimalen Version 
installieren und damit Erfahrungen sammeln. Möchte man später fli4l 
zusätzlich als Anrufbeantworter und als HTTP-Proxy einsetzen, so hat 
man vorher schon mal Erfahrungen mit einem grundsätzlich laufenden Router.

Für die Installation ergeben sich daraus die folgenden fünf Varianten:

\begin{description}
\item[USB-Stick] Router auf einem USB-Stick
\item[CD-router] Router auf einer CD
\item[Netzwerk] Netzwerkboot
\item[HD-Installation Typ A] Router auf Festplatte, CF, DoC~-- nur eine FAT-Partition
\item[HD-Installation Typ B] Router auf Festplatte, CF, DoC~-- je eine FAT- und ext3-Partition
\end{description}

\marklabel{INSTALLTYP0}{\subsection{Router auf einem USB-Stick}}

USB-Sticks werden von Linux als Festplatten angesprochen, daher gelten hier
die Ausführungen zur Festplatteninstallation entsprechend. Bitte beachten Sie,
dass mittels des \var{OPT\_\-USB} die entspechenden Treiber geladen werden
müssen, damit der Stick mittels \var{OPT\_\-HDINSTALL} eingebunden werden kann.

\marklabel{INSTALLTYP0}{\subsection{Router auf einer CD oder Netzwerkboot}}

Alle benötigten Dateien liegen auf dem Bootmedium und werden beim
Booten in eine dynamische RAM-Disk entpackt.  In einer
Minimalkonfiguration ist damit ein Betrieb des Routers mit nur 64 MiB
RAM möglich. Die maximale Konfiguration wird nur durch die Kapazität
des Bootmediums und des Hauptspeichers limitiert.

\marklabel{INSTALLTYPA}{\subsection{Typ A: Router auf Festplatte~-- nur eine FAT-Partition}}

Dies entspricht der CD-version, nur dass die Dateien hierbei auf
einer Festplatte liegen, wobei der Begriff \glqq{}Festplatte\grqq{} hier auch
Compact-Flash-Medien ab 8 MiB und andere Geräte, welche Linux als
Festplatte ansprechen kann, mit einschließt. Seit fli4l 2.1.4 können
auch DiskOnChip Flash-Speicher von M-Sys oder SCSI-Festplatten benutzt
werden.

Die Beschränkung des Archivs opt.img durch die
Diskettenkapazität wird aufgehoben, aber alle diese Dateien müssen in
einer RAM-Disk mit der entsprechenden Größe beim Boot installiert werden.
Dies erhöht den RAM-Bedarf beim Einsatz vieler Pakete.

Für ein Update der Softwarepakete (d.h. des Archivs opt.img und der
rc.cfg über das Netzwerk) muss die FAT-Partition genügend Platz für den
Kernel, das RootFS und die DOPPELTE Größe des opt.img haben!
Falls auch die Notfall-Option genutzt werden soll, erhöht sich der
Platzbedarf noch einmal um die Größe des opt.img.

\marklabel{INSTALLTYPB}{\subsection{Typ B: Router auf Festplatte~-- je eine FAT- und ext3-Partition}}

Im Gegensatz zum Typ A werden hier nicht alle Dateien in die Ramdisk
gepackt, sondern bei dem erstmaligen Start nach der Installation oder
nach einem Update aus dem Archiv opt.img direkt auf eine
ext3-Partition kopiert und im späteren Betrieb von dort geladen. Bei
dieser Version ist der Speicherbedarf für die RAM-Disk am geringsten
und damit meist auch ein Betrieb mit sehr wenig RAM möglich.

Weitere Informationen zur Installation auf Festplatten finden Sie in
der Dokumentation des Pakets HD (separat herunterzuladen)-
beginnend bei der Beschreibung der Variablen \var{OPT\_\-HDINSTALL}.
