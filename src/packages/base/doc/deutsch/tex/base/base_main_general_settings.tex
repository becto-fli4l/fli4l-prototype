% Last Update: $Id$
\section{Allgemeine Einstellungen}

\begin{description}

  \config{HOSTNAME}{HOSTNAME}{HOSTNAME}

  Standardwert: \var{HOSTNAME='fli4l'}

  Als erstes sollte man seinem fli4l-Router einen Namen geben.


  \config{PASSWORD}{PASSWORD}{PASSWORD}
  
  Standardwert: \var{PASSWORD='fli4l'}
  
  Das hier angegebene Passwort wird für das Einloggen in den fli4l-Rechner
  benötigt~-- sei es per Tastatur direkt am Router oder per SSH von einem
  anderen Rechner aus (hierzu wird das sshd-Paket benötigt). Es muss aus
  mindestens einem und darf aus höchstens 126 Zeichen bestehen.


  \config{BOOT\_TYPE}{BOOT\_TYPE}{BOOTTYPE}

  Standardwert: \var{BOOT\_TYPE='hd'}

  \var{BOOT\_TYPE} legt im weitesten Sinne das Bootmedium fest. Diese Variable
  steuert, welche zusätzlichen Treiber (Kernel-Module) und Start-Skripte mit in
  das RootFS aufgenommen werden. Zum Verständnis eine kurze Skizze des
  Bootvorgangs:

 \begin{itemize}
    \item Das BIOS des Rechners lädt/startet den Bootloader auf dem Bootmedium.
    \item Der Bootloader (i.d.R syslinux) entpackt, lädt und startet den
          Kernel.
    \item Der Kernel entpackt das RootFS (= das grundlegende Dateisystem mit
          darin enthalten Programmen und Skripte), mountet das RootFS und
          beginnt die Start-Skripte abzuarbeiten.
    \item Je nach \var{BOOT\_TYPE} werden nun die Kernel-Module für das jeweilige
          Bootmedium geladen, die Boot-Partition gemountet und das OPT-Archiv
          (\texttt{opt.img}) mit den zusätzlichen Programmen entpackt.
    \item Im Anschluss beginnt die Konfiguration der einzelnen Dienste des fli4l.
  \end{itemize}

 Zur Zeit sind folgende Werte für \var{BOOT\_TYPE} gültig:

  \begin{description}
  \item[ls120] Boot von LS120/240 sowie ZIP Disks.
  \item[hd] Boot von Festplatte. IDE und SATA Geräte werden direkt erkannt, für SCSI, USB oder besondere Controller
            wird das Paket HD und/oder USB benötigt.
            Näheres ist der \jump{sec:hdinstall}{Dokumentation} zum Paket HD zu entnehmen.
  \item[cd] Boot von CD-ROM.  Es wird lediglich das ISO-Image fli4l.iso der CD erzeugt, welches
            anschließend mit dem jeweiligen Lieblingsbrennprogramm selbst auf
            CD gebrannt werden muss. Bezüglich SCSI, USB und spezielle Controller ist das Paket HD bzw. USB nötig.
  \item[integrated] Bei diesem Typ wird kein Bootmedium zu Grunde gelegt,
                    sondern das OPT-Archiv vollständig ins RootFS
                    integriert. Somit entfällt das Mounten des Bootmediums
                    und der Kernel kann gleich alles entpacken. Dieser
                    \var{BOOT\_TYPE} wird z.B. fürs Booten vom Netzwerk
                    benötigt.\\
                    \textbf{Hinweis: } Ein remote Update ist natürlich in diesem
                    Fall nicht möglich.
  \item[attached] Ähnlich wie \textbf{integrated}, jedoch wird das
                  OPT-Archiv als Datei \texttt{opt.img} ans RootFS
                  angehängt, mit der Folge, dass es wieder im Verzeichnis
                  \texttt{/boot} zu finden ist und gesondert beim Bootvorgang
                  entpackt wird.\\
                  Ansonsten gilt das unter \texttt{integrated} Gesagte.
  \item[netboot] Entspricht \textbf{integrated}. Es wird jedoch zusätzlich das
                 Skript \texttt{mknetboot.sh} gestarted, welches ein Image zum
                 Booten via LAN erzeugt. Weiteres ist bitte dem Wiki 
                 \altlink{https://ssl.nettworks.org/wiki/display/f/fli4l+und+Netzboot} 
                 zu entnehmen.
  \item[pxeboot] Es werden zwei Images generiert, kernel und rootfs.img. Das sind die beiden vom
		 PXE-Bootloader nachzuladenden Dateien. Beim Aufruf kann
		 die Lokation des tftp-Verzeichnisses angegeben werden und zusätzlich noch ein
		 Unterverzeichnis innerhalb des tftp-Verzeichnisses (--pxesubdir).
		 Weiteres auch hier im Wiki \altlink{https://ssl.nettworks.org/wiki/display/f/fli4l+und+Netzboot}           
  \end{description}

  \textbf{Hinweis:} wie ein fli4l als passender boot-server (pxe/tftp) zu 
  konfigurieren ist, können sie in der Dokumentation des Pakets dns\_dhcp nachlesen.\\
  
  \config{LIBATA\_DMA}{LIBATA\_DMA}{LIBATADMA}

  Mit dieser Variable kann eingestellt werden, ob DMA für libata basierte Geräte
  aktiviert werden soll. Dies ist z.B. bei einigen unvollständig verdrahteten
  IDE zu Compact-Flash Adaptern nötig. Um DMA zu aktivieren: 'enabled'
  Default: 'disabled'

  \config{MOUNT\_BOOT}{MOUNT\_BOOT}{MOUNTBOOT}
  
  Standardwert: \var{MOUNT\_\-BOOT='rw'}

  {Hier wird eingestellt, wie das Boot-Medium gemountet werden
    soll. Es gibt drei Möglichkeiten:

    \begin{description}
      \item[rw]~-- Read/Write~-- Schreiben und Lesen ist möglich
      \item[ro]~-- Read-Only~-- Nur Lesen ist möglich
      \item[no]~-- None~-- Medium wird nach dem Boot wieder
        abgemeldet und kann dann bei Bedarf entnommen werden.
    \end{description}

    Bei bestimmten Konfigurationen ist es unbedingt erforderlich, das
    Medium Read/\-Write anzumelden, z.B. wenn man den DHCP-Server
    einsetzen oder die imond-Log-Datei auf dem Medium anlegen
    möchte.}


  \config{BOOTMENU\_TIME}{BOOTMENU\_TIME}{BOOTMENUTIME}
    
  Standardwert: \var{BOOTMENU\_TIME='20'}

  {Hier wird eingestellt, wie lange der syslinux Bootloader warten soll,
    bis automatisch mit der Standard-Installation gebootet wird.

    Im Paket HD besteht die Möglichkeit, über \var{OPT\_RECOVER} eine Funktion
    zu aktivieren, mit der eine Notfallinstallation aus einer laufenden
    Installation erstellt werden kann. Diese kann im Bootmenü über die Wahl
    der Recover-Version aktiviert werden.

    Sollte hier der Wert '0' eingestellt sein, wartet der syslinux Bootloader
    bis der Anwender die Standard- oder die Recover-Version auswählt und
    aktiviert!}


  \config{TIME\_INFO}{TIME\_INFO}{TIMEINFO}

  Standardwert: \var{TIME\_INFO='MEZ-1MESZ,M3.5.0,M10.5.0/3'}
  
  Uhren ticken in der Unix-Welt und damit auch unter fli4l
  normalerweise nach der UTC (Universal Time Coordinated), einer
  weltweit einheitlichen Uhrzeit, die vor der Verwendung in die lokale
  Zeit umgerechnet wird. \var{TIME\_\-INFO} liefert fli4l die dafür
  notwendigen Informationen über die Namen der Zeitzonen, die
  Differenz zu UTC und Regeln, wann auf Sommerzeit und wieder zurück
  gewechselt wird. Damit das korrekt funktioniert, muss die Hardware
  Uhr auf UTC gestellt werden (entspricht der Londoner Winterzeit)
  oder über das Paket chrony mit einem Timeserver synchronisiert werden
  (diese liefern UTC aus).

  Die Einträge in \var{TIME\_\-INFO} bedeuten dabei folgendes:
  \begin{example}
    \begin{verbatim}
        TIME_INFO='MEZ-1MESZ,M3.5.0,M10.5.0/3'
    \end{verbatim}
  \end{example}
  \begin{itemize}
    \item \emph{MEZ-1:} Wir befinden uns in der mitteleuropäischen
      Zeitzone (\emph{MEZ}), die der UTC eine Stunde voraus ist \emph{MEZ-1=UTC}.
    \item \emph{MESZ:} In dieser Zeitzone gibt es Sommerzeit
      (Mitteleuropäische Sommerzeit). Da nichts weiter angegeben wird,
      kommt man zur Sommerzeit, indem man die Zeit eine Stunde
      vorstellt.
    \item \emph{M3.5.0,M10.5.0/3:} Am letzten Sonntag im März (um 2 Uhr) wird
      zur Sommerzeit gewechselt, am letzten Sonntag im Oktober (um 3 Uhr)
      wieder zurück.
  \end{itemize}

  Normalerweise braucht man diesen Wert nie anzufassen, es sei denn
  man sitzt in einer anderen Zeitzone. Will man die Werte anpassen,
  sollte man einen Blick auf die Spezifikation der Umgebungsvariable
  TZ werfen, die unter folgender URL zu finden ist (englisch):
  \altlink{http://pubs.opengroup.org/onlinepubs/009695399/basedefs/xbd_chap08.html}

  \config{RTC\_SYNC}{RTC\_SYNC}{RTCSYNC}{
  Standardwert: \var{RTC\_SYNC='hwclock'}

  In vielen Rechnern steckt eine batteriegepufferte Hardware-Uhr, die auch über
  die Dauer der Abschaltung mit Strom versorgt wird und die Uhrzeit weiterzählt,
  so dass sie beim nächsten Starten wieder als Systemzeit zur Verfügung steht.
  Es ist an dieser Stelle wichtig, zwischen der \emph{Systemzeit} und der
  \emph{Hardwarezeit} zu unterscheiden:

  \begin{itemize}
  \item Die \emph{Hardwarezeit} ist die Zeit, die in der Hardware-Uhr
  gespeichert und von dieser aktuell gehalten wird. Sie wird in der Regel beim
  Starten des Systems aus der Hardware-Uhr ausgelesen und als Systemzeit
  übernommen.

  \item Die \emph{Systemzeit} ist die eigentliche Zeit, die das Linux-System
  verwendet und z.~B.\ beim Aufruf des Befehls \texttt{date -u} angezeigt wird.
  Sie wird vom Linux-Kernel aktuell gehalten, etwa auf Basis regelmäßiger
  Hardware-Unterbrechungen (Timer-Interrupt), bezeichnet immer einen Zeitpunkt
  in koordinierter Weltzeit (UTC), und wird nicht von der Zeitzonen-Einstellung
  beeinflusst.

  \item Die \emph{lokalisierte Systemzeit} ist lediglich die Umrechnung der
  Systemzeit in eine andere Zeitzone, die auf dem fli4l-Router über die
  Umgebungsvariable \texttt{TZ} konfiguriert wird (siehe die Variable
  \jump{TIMEINFO}{\var{TIME\_INFO}}), und spielt im weiteren Verlauf dieses
  Abschnitts keine Rolle.
  \end{itemize}
   
  Mit Hilfe dieser Variable wird dem fli4l mitgeteilt, wie der Abgleich der
  Hardwarezeit mit der Systemzeit vorgenommen werden soll, d.~h.\ ob und wie
  oft die Hardwarezeit auf die Systemzeit gesetzt werden soll. Ein solcher
  Abgleich ist nötig, weil auch die beste Hardware-Uhr nicht zu 100~\% genau
  geht und zum systematischen Abdriften neigt, d.~h.\ sie geht auf Dauer
  gesehen etwas zu langsam oder etwas zu schnell.
  
  Es gibt prinzipiell zwei Möglichkeiten der Synchronisation:
  
  \begin{itemize}
  \item Modus ``kernel'': Ein NTP-Client wird verwendet, um von außen
  (in der Regel über das Internet oder eine externe (Funk-)Uhr) die tatsächliche
  Uhrzeit zu ermitteln und die Systemzeit des fli4l-Routers aktuell zu halten.
  Dabei wird der Linux-Kernel angewiesen, sich um die Aktualisierung der
  Hardwarezeit zu kümmern, so dass keine weitere Synchronisierung mehr nötig
  ist. Die Aktualisierung durch den Linux-Kernel ist etwas weniger genau als
  die Aktualisierung mittels \texttt{hwclock} (siehe Modus ``hwclock'' weiter
  unten), allerdings ist die Güte der Aktualisierung weit weniger wichtig, weil
  der zwangsläufige Fehler durch den NTP-Client ausgeglichen wird.
  
  Dieser Modus muss auch verwendet werden, wenn gar keine Hardware-Uhr
  existiert. Der Linux-Kernel wird in diesem Falle natürlich keine Hardwarezeit
  aktuell halten, weil es keine gibt. Es sollte dann allerdings unbedingt ein
  NTP-Client verwendet werden, damit der fli4l-Router überhaupt eine sinnvolle
  Systemzeit erhält.

  \item Modus ``hwclock'': Es findet beim Herunterfahren des Systems
  (bei der Ausführung des Stopp-Skripts \texttt{/etc/rc0.d/rc950.hwclock})
  sowie in regelmäßigen Abständen (alle 24 Stunden) eine Synchronisation mit
  Hilfe des \texttt{hwclock}-Programms statt. Dabei wird nicht nur die
  Hardwarezeit gesetzt, sondern \texttt{hwclock} misst auch, inwieweit die
  Systemzeit von der Hardwarezeit abweicht. Beim Starten des Systems wird dann
  die Systemzeit nicht direkt aus der Hardwarezeit übernommen, sondern es wird
  auch die Abweichung berücksichtigt, um das Abdriften der Systemzeit möglichst
  zu reduzieren. Die Abweichung wird in der Datei \texttt{/etc/adjtime}
  vermerkt. Ist ein beschreibbares persistentes Medium verfügbar, wird die
  Abweichung unter \texttt{/var/lib/persistent/base/adjtime} gespeichert; in
  diesem Falle ist \texttt{/etc/adjtime} eine symbolische Verknüpfung dorthin.
  
  Dieser Modus ist inkompatibel zu einer Aktualisierung der Systemzeit mit
  Hilfe eines NTP-Clients. Das liegt daran, dass ein NTP-Client automatisch
  das Aktualisieren der Hardwareuhr durch den Linux-Kernel aktiviert. Es ist
  jedoch wenig sinnvoll bzw.\ problematisch, dass sowohl \texttt{hwclock} als
  auch der Linux-Kernel gleichzeitig versuchen, die Hardwarezeit aktuell zu
  halten.
  \end{itemize}
  
  Es ist zu beachten, dass wenn eine Hardware-Uhr zur Verfügung steht, die
  darin gespeicherte Uhrzeit \emph{immer} als koordinierte Weltzeit (UTC)
  interpretiert wird. Die Zeitzone, die über die Variable \var{TIME\_INFO}
  gesetzt wird, wirkt sich nicht auf die in der Hardware-Uhr gespeicherte
  Zeit aus. Das Speichern einer lokalisierten Nicht-UTC-Zeit in der
  Hardware-Uhr wird von fli4l \emph{nicht} unterstützt.
  
  Das Ermitteln der Systemzeit aus der Hardwarezeit wird einmalig beim Starten
  des Systems vorgenommen. Dabei wird bereits durch den Linux-Kernel das
  Auslesen der Hardware-Uhr und das Setzen der Systemzeit unmittelbar am Beginn
  des Bootvorgangs vorgenommen. Im Modus ``hwclock'' wird dann später bei der
  Ausführung des Boot-Skripts \texttt{/etc/rc.d/rc100.hwclock} die Systemzeit
  erneut gesetzt, diesmal unter Berücksichtigung der systematischen Abweichung.
  }

  \config{KERNEL\_VERSION}{KERNEL\_VERSION}{KERNELVERSION}

  Legt die Version des zu verwendenden Kerns fest. Entsprechend dieser
  Variable werden der Kern aus \emph{img/kernel-$<$kernel
  version$>$.$<$compression extension$>$} und die Kernel-Module aus 
  \emph{opt/lib/modules/$<$kernel version$>$} selektiert.

  \config{KERNEL\_BOOT\_OPTION}{KERNEL\_BOOT\_OPTION}{KERNELBOOTOPTION}

  Standardwert: \var{KERNEL\_BOOT\_OPTION=''}
  
  Der Inhalt dieser Variable wird an die Kommandozeile des Kerns in
  der syslinux.cfg angehängt.
  Manche Systeme benötigen für korrekten Reboot 'reboot=bios'.
  Bei WRAP-Systemen also 'reboot=bios'.

  \config{COMP\_TYPE\_ROOTFS}{COMP\_TYPE\_ROOTFS}{COMPTYPEROOTFS}

  Standardwert: \var{COMP\_TYPE\_ROOTFS='xz'}
  
  Der Inhalt dieser Variable legt die Kompressionsmethode für das RootFS-Archiv
  fest. Mögliche Werte sind `xz', `lzma' und `bzip2'.

  \config{COMP\_TYPE\_OPT}{COMP\_TYPE\_OPT}{COMPTYPEOPT}

  Standardwert: \var{COMP\_TYPE\_OPT='xz'}
  
  Der Inhalt dieser Variable legt die Kompressionsmethode für das OPT-Archiv
  fest. Mögliche Werte sind `xz', `lzma' und `bzip2'.

  \config{POWERMANAGEMENT}{POWERMANAGEMENT}{POWERMANAGEMENT} 
  
    Standardwert: \var{POWERMANAGEMENT='acpi'}
     
    {Der Kern unterstützt verschiedene Formen des Powermanagements, das etwas betagte APM 
    und das aktuellere ACPI. Hier kann man einstellen, welche Form er verwenden soll. 
    Mögliche Werte sind 'none' (kein Powermanagement), 'acpi' und die beiden APM-Varianten 
    'apm' und 'apm\_rm'. Letzteres schaltet in einen speziellen Prozessormodus, bevor der 
    Router ausgeschaltet wird. }

  \config{FLI4L\_UUID}{FLI4L\_UUID}{FLI4LUUID}
  
    Standardwert: \var{FLI4L\_UUID=''}
    
    {Hier wird eine eindeutige UUID eingetragen, mit der der fli4l seine persistenten Daten 
    auf z.B. einem USB-Stick finden kann. Eine UUID kann auf einem beliebigen Linux-System
    (wie auch dem fli4l) mit dem Befehl \verb*?'cat /proc/sys/kernel/random/uuid'? erstellt werden.
    Dies gibt bei jedem Aufruf eine neue UUID aus. Diese muss nun in die Variable eintragen
    werden. Auf einem persistenten Medium (z.B. auf einer Festplatte (OPT\_HD) oder einem
    USB-Stick (OPT\_USB und OPT\_HD)) muss dann noch ein Verzeichnis mit demselben Namen 
    angelegt werden. Dort wird dann künftig alles gespeichert, das sich gegenüber der
    Konfiguration geändert hat, ebenso wie persistente Laufzeitdaten wie z.B. DHCP-Leases.
    Hierzu muss das entsprechende Paket dies natürlich unterstützen (siehe Dokumentation).
    Der entsprechende Eintrag für den Speicherpfad ist dort dann in der Regel 'auto'.

    Sollte der fli4l bereits vor dem Erstellen der UUID und dem Anlegen des Verzeichnisses
    einige Daten gespeichert haben, so sind diese unter /boot/persistent zu finden und müssen
    dann manuell an den neuen Speicherort verschoben werden. Deshalb empfiehlt es sich, die
    UUID gleich anfangs zu erstellen und nicht erst später zu migrieren.

    Zudem ist zu beachten, dass \var{MOUNT\_BOOT}='ro' nicht gewählt werden darf,
    solange das Verzeichnis sich irgendwo auf der /boot Partition befindet.

    Ein empfohlener Ort für das persistente Verzeichnis befindet sich auf der /data
    Partition (ganz oben) oder einem USB-Stick. Das Dateisystem des USB-Sticks sollte
    VFAT sein oder bei aktivem OPT\_HD alle dort unterstützen schreib-lese-fähigen
    Dateisysteme.}

  \config{IP\_CONNTRACK\_MAX}{IP\_CONNTRACK\_MAX}{IPCONNTRACKMAX}
  
  Standardwert: \var{IP\_CONNTRACK\_MAX=''}
  
  Mit Hilfe dieser Variable kann man die maximale mögliche Anzahl gleichzeitiger
  Verbindungen manuell einstellen. Normalerweise wird anhand des eingebauten
  Arbeitsspeichers automatisch ein sinnvoller Wert ermittelt. In Tabelle
  \ref{tab:connectiontracking} sind die verwendeten Voreinstellungen
  zusammengefasst dargestellt.

    \begin{table}[ht!]
        \centering
        \caption{Automatische Einstellung der maximalen Verbindungsanzahl}\marklabel{tab:connectiontracking}{}
        \begin{tabular}{p{6cm}p{6cm}}
            Arbeitsspeicher in MiB   &    gleichzeitige Verbindungen \\\hline
            16                       &    1024 \\
            24                       &    1280 \\
            32                       &    2048 \\
            64                       &    4096 \\
            128                      &    8192 \\
        \end{tabular}
    \end{table}

   Bei Einsatz von FileSharing-Programmen hinter oder auf dem Router und wenig
   Arbeitsspeicher ist die maximale Anzahl gleichzeitiger Verbindungen aber
   sehr schnell erreicht und zusätzliche Verbindungen können nicht mehr
   aufgebaut werden.\\ 
   Das äußert sich in Fehlermeldungen wie
   
    \begin{example}
      \begin{verbatim}
        ip_conntrack: table full, dropping packet
      \end{verbatim}
    \end{example}
   
   oder
   
   \begin{example}
      \begin{verbatim}
        ip_conntrack: Maximum limit of XXX entries exceeded
      \end{verbatim}
    \end{example}

    Mittels \var{IP\_\-CONNTRACK\_\-MAX} lässt sich nun die maximale
    Anzahl gleichzeitiger Verbindungen fest auf einen bestimmten Wert
    einstellen. Jede einzelne mögliche Verbindung kostet 350 Bytes Arbeitsspeicher, 
    der nicht mehr für andere Dinge genutzt werden kann.
    Setzt man also 10000, so sind gerundet 3,34 MB Arbeitsspeicher für den
    normalen Gebrauch verloren (Kernel, Ramdisks, Programme).

    Bei 32 MiB RAM sollte es kein Problem sein, mal eben 2 oder 3 MiB für die
    ip\_conntrack-Tabelle zu reservieren, bei 16 MiB wird es knapp und bei 12
    oder sogar 8 MiB ist absolute Sparwut angesagt.

    Die momentan benutze Einstellung lässt sich auf der Konsole mit

    \begin{example}
      \begin{verbatim}
        cat /proc/sys/net/ipv4/ip_conntrack_max
      \end{verbatim}
    \end{example}

    anzeigen und mit

    \begin{example}
      \begin{verbatim}
        echo "XXX" > /proc/sys/net/ipv4/ip_conntrack_max
      \end{verbatim}
    \end{example}

    zur Laufzeit setzen, wobei XXX für die Anzahl der Einträge steht.
    Die Einträge in der \var{IP\_CONNTRACK}-Tabelle selbst können auf der Konsole
    mit

    \begin{example}
      \begin{verbatim}
        cat /proc/net/ip_conntrack
      \end{verbatim}
    \end{example}

    angesehen und mit

    \begin{example}
      \begin{verbatim}
        cat /proc/net/ip_conntrack | grep -c use
      \end{verbatim}
    \end{example}

    gezählt werden.


  \config{LOCALE}{LOCALE}{LOCALE}

  Standardwert: \var{LOCALE}='de'

  Einige Komponenten sind mittlerweile mehrsprachfähig. Dazu zählen
  beispielsweise das Konsolen-Menü und die Weboberfläche. Mit dieser Variablen
  können Sie die bevorzugte Sprache auswählen. Verschiedene Komponenten haben
  noch eine eigene Einstellung womit diese Grundeinstellung, wenn nötig,
  überlistet werden kann. Wenn die hier angegebene Sprache bei einer Komponente
  (noch) nicht verfügbar ist, wird auf Englisch zurückgefallen.

  Bei \var{KEYBOARD\_LOCALE}='auto' wird versucht ein zu der
  \var{LOCALE}-Einstellung passendes Tastatur-Layout ein zu stellen.

  Bisher sind folgende Einstellungen möglich: de, en, es, fr, hu, nl.  

\end{description}
