% Last Update: $Id$

\marklabel{chap:bootmedien}{
  \chapter{Erzeugen der fli4l Archive/Bootmedien}
  }

  Sind alle Konfigurationsarbeiten erledigt, können die fli4l Archive/Bootmedien, sei
  es eine bootfähige Compact-Flash, ein bootfähiges ISO-Image oder nur die zum Remote-Update
  benötigten Dateien, erstellt werden.

\marklabel{sec:bootmedien_linux}{
  \section{Erzeugen der fli4l Archive/Bootmedien unter Linux bzw. anderen Unix-Derivaten und Mac OS X}
  }

  Dies geschieht mit Hilfe von Scripts
  (\texttt{.sh}), die im fli4l Wurzelverzeichnis zu finden sind.

  \begin{description}
    \item \texttt{mkfli4l.sh}
  \end{description}

  Das Build-Script erkennt selbständig die unterschiedlichen \jump{BOOTTYPE}{Bootvarianten}.

  Der einfachste Aufruf sieht unter Linux so aus:
  \begin{verbatim}
    sh mkfli4l.sh
  \end{verbatim}

  Die Aktionen des Build-Scripts werden durch drei Mechanismen gesteuert:
  \begin{itemize}
    \item Konfigurationsvariable \var{BOOT\_TYPE} aus der
          \texttt{$<$config$>$/base.txt}
    \item Konfigurationsdatei \texttt{$<$config$>$/mkfli4l.txt}
    \item Parameter des Build-Scripts
  \end{itemize}

  An Hand der Konfigurationsvariable \jump{BOOTTYPE}{\var{BOOT\_TYPE}}
  entscheidet sich, welche Aktion des Build-Scripts ausgeführt wird:
  \begin{itemize}
    \item Erstellen eines bootfähigen fli4l CD-ISO-Images
    \item Bereitstellen der fli4l Dateien, zwecks Remote-Update
    \item Erzeugen der fli4l Dateien und direktes Remote-Update per SCP
    \item usw.
  \end{itemize}

  Die Beschreibung der Variablen der Konfigurationsdatei
  \texttt{$<$config$>$/mkfli4l.txt} finden Sie im Kapitel
  \jump{sec:mkfli4lconf}{Steuerungsdatei mkfli4l.txt}.

  \subsection{Kommandozeilenoptionen}
  Der letzte Steuerungsmechanismus ist das Anhängen von Optionsparametern an
  den Aufruf des Build-Script auf der Kommandozeile. Die
  Steuerungsmöglichkeiten entsprechen denen der Steuerungsdatei \texttt{mkfli4l.txt}. 
  Die Angabe von Optionsparametern überschreiben die Werte aus
  der Steuerungsdatei. Aus Komfortgründen unterscheiden sich die Namen der
  Optionsparameter von den Namen der Variablen aus der Steuerungsdatei. Es
  existiert teilweise eine Kurz- und eine Langform:

  \begin{verbatim}
Usage: mkfli4l.sh [options] [config-dir]

-c, --clean         cleanup the build-directory
-b, --build <dir>   set build-directory to <dir> for the fli4l-files
-h, --help          display this usage
--batch             don't ask for user input

config-dir          set other config-directory - default is "config"

--hdinstallpath <dir> install a pre-install environment directly to
                    usb/compact flash device mounted or mountable to
                    directory <dir> in order to start the real installation
                    process directly from that device
                    device either has to be mounted and to be writable
                    for the user or it has to be mountable by the user
                    Do not use this for regular updates!

*** Remote-Update options
--remoteupdate        remote-update via scp, implies "--filesonly"
--remoteremount       make /boot writable before copying files and
                      read only afterwards
--remoteuser <name>   user name for remote-update - default is "fli4l"
--remotehost <host>   hostname or IP of remote machine - default
                      is HOSTNAME set in [config-dir]/base.txt
--remotepath <path>   pathname on remote maschine - default is "/boot"
--remoteport <portnr> portnumber of the sshd on remote maschine

*** Netboot options (only on Unix/Linux)
--tftpbootpath <path>   pathname to tftpboot directory
--tftpbootimage <name>  name of the generated bootimage file
--pxesubdir <path>      subdirectory for pxe files relative to tftpbootpath

*** Developer options
-u, --update-ver    set version to <fli4l_version>-rev<svn revision>
-v, --verbose       verbose - some debug-output
-k, --kernel-pkg    create a package containing all available kernel
                    modules and terminate afterwards.
                    set COMPLETE_KERNEL='yes' in config-directory/_kernel.txt
                    and run mkfli4l.sh again without -k to finish
    --filesonly     create only fli4l-files - do not create a boot-media
    --no-squeeze    don't compress shell scripts
    --rebuild       rebuild mkfli4l and related tools; needs make, gcc

  \end{verbatim}

   Eine HD-Vorinstallation einer passend formatierten (FAT16/FAT32) CompactFlash im
   USB-Cardreader oder eines USB-Sticks ist über die Option \verb+--hdinstallpath <dir>+ möglich.
   Dieses können Sie \emph{auf eigenes Risiko} zur Installation auf eine CompactFlash oder
   einen USB-Stick benutzen.
   Hierbei werden auf die angegebene Partition die nötigen Dateien des fli4l kopiert.
   Sie rufen dazu zunächst im fli4l-Verzeichnis

  \begin{verbatim}
     sh mkfli4l.sh --hdinstallpath <dir>
  \end{verbatim}
  \vspace{-2ex}
  auf. Dabei werden die fli4l Dateien auf eine CF-Card oder USB-Stick kopiert.

  Um die nächsten Schritte ausführen zu können, sind folgende Voraussetzungen zu erfüllen:

   \begin{itemize}
        \item \verb+chmod 777 /dev/brain+
        \item superuser-Rechte
        \item installiertes \verb+syslinux+
        \item installiertes \verb+fdisk+
   \end{itemize}

  Durch das Script erfolgt eine Kontrolle, ob dieser Datenträger tatsächlich ein USB-Laufwerk
  ist und die erste Partition eine FAT-Partition ist.
  Anschliessend werden der Bootloader und die nötigen Dateien auf den angegebenen Datenträger kopiert.
  Sie erhalten eine Meldung über den Erfolg oder Misserfolg.

 Nach dem Build müssen Sie

 \begin{verbatim}
   syslinux --mbr /dev/brain

    # make partition bootable using fdisk
    #     p - print partitions
    #     a - toggle bootable flag, specify number of fli4l partition
    #         usually '1'
    #     w - write changes and quit
    fdisk /dev/brain

    # install boot loader
    syslinux -i /dev/brain
 \end{verbatim}
 \vspace{-2ex}
 ausführen.
 Dann sollte die CF bzw. der USB-Stick bootfähig sein.
 Vergessen Sie nicht, den Datenträger auszuhängen (via \texttt{umount}).

  \bigskip

  Als letzter Optionsparameter  kann ein alternatives Konfigurationverzeichnis
  übergeben werden. Das normale Konfigurationsverzeichnis heißt \texttt{config}
  und liegt direkt im fli4l Wurzelverzeichnis. An diesem Ort legen alle fli4l
  Pakete die Konfigurationsdateien ab. Möchte man mehr als eine Konfiguration
  verwalten, so erstellt man sich ein weiteres Verzeichnis, z.B. \texttt{hd.conf},
  legt dort eine Kopie der Konfigurationsdateien ab und verändert diese den
  Anforderungen entsprechend. Hier einige Beispiele:
  \begin{verbatim}
     sh mkfli4l.sh --filesonly hd.conf
     sh mkfli4l.sh --no-squeeze config.test
  \end{verbatim}

\marklabel{sec:bootmedien_windows}{
  \section{Erzeugen der fli4l Archive/Bootmedien unter Windows}
  }

  Es wird das Tool `AutoIt3' verwendet (\altlink{http://www.autoitscript.com/site/autoit/}).
  Dieses ermöglicht eine `grafische' Ausgabe, sowie Dialoge, mit denen die in
  den folgenden Abschnitten beschriebenen Variablen beinflusst werden können.

  \begin{description}
    \item \texttt{mkfli4l.bat}
  \end{description}

  Das Build-Programm erkennt selbständig die unterschiedlichen \jump{BOOTTYPE}{Bootvarianten}.


  Der Aufruf von `mkfli4l.bat' kann direkt aus dem Windows Explorer
  erfolgen, wenn man keine optionalen Parameter verwenden möchte.

  Die Aktionen des Build-Programms werden durch verschiedene Mechanismen gesteuert:
  \begin{itemize}
    \item Konfigurationsvariable \var{BOOT\_TYPE} aus der
          \texttt{$<$config$>$/base.txt}
    \item Konfigurationsdatei \texttt{$<$config$>$/mkfli4l.txt}
    \item Parameter des Build-Programmes
    \item Interaktive Einstellung in der GUI
  \end{itemize}

  An Hand der Konfigurationsvariable \jump{BOOTTYPE}{\var{BOOT\_TYPE}}
  entscheidet sich, welche Aktion das Build-Programm ausführt:
  \begin{itemize}
    \item Erstellen eines bootfähigen fli4l CD-ISO-Images
    \item Bereitstellen der fli4l Dateien, zwecks Remote-Update
    \item Erzeugen der fli4l Dateien und direktes Remote-Update per SCP
    \item HD-pre-install einer passend formatierten CF im Cardreader
    \item usw.
  \end{itemize}

  Die Beschreibung der Variablen der Konfigurationsdatei
  \texttt{$<$config$>$/mkfli4l.txt} finden Sie im Kapitel
  \jump{sec:mkfli4lconf}{Steuerungsdatei mkfli4l.txt}.

  \subsection{Kommandozeilenoptionen}
  Ein weiterer Steuerungsmechanismus ist das Anhängen von Optionsparametern an
  den Aufruf des Build-Programms auf der Kommandozeile. Die
  Steuerungsmöglichkeiten entsprechen denen der Steuerungsdatei \texttt{mkfli4l.txt}.
  Die Angabe von Optionsparametern überschreiben die Werte aus
  der Steuerungsdatei. Aus Komfortgründen unterscheiden sich die Namen der
  Optionsparameter von den Namen der Variablen aus der Steuerungsdatei. Es
  existiert teilweise eine Kurz- und eine Langform:

  \begin{verbatim}
Usage: mkfli4l.bat [options] [config-dir]

-c, --clean             cleanup the build-directory
-b, --build <dir>       sets build-directory to <dir> for the fli4l-files
-v, --verbose           verbose - some debug-output
    --filesonly         creates only fli4l-files - does not create a disk
    --no-squeeze        don't compress shell scripts
-h, --help              display this usage

config-dir              sets other config-directory - default is "config"

*** Remote-Update options
--remoteupdate          remote-update via scp, implies "--filesonly"
--remoteuser <name>     user name for remote-update - default is "fli4l"
--remotehost <host>     hostname or IP of remote machine - default
                        is HOSTNAME set in [config-dir]/base.txt
--remotepath <path>     pathname on remote maschine - default is "/boot"
--remoteport <portnr>   portnumber of the sshd on remote maschine

*** GUI-Options
--nogui                 disable the config-GUI
--lang                  change language
                        [deutsch|english|espanol|french|magyar|nederlands]

  \end{verbatim}

  Als letzter Optionsparameter  kann ein alternatives Konfigurationverzeichnis
  übergeben werden. Das normale Konfigurationsverzeichnis heißt \texttt{config}
  und liegt direkt im fli4l Wurzelverzeichnis. An diesem Ort legen alle fli4l
  Pakete die Konfirgurationsdateien ab. Möchte man mehr als eine Konfiguration
  verwalten, so erstellt man sich ein weiteres Verzeichnis, z.B. \texttt{hd.conf},
  legt dort eine Kopie der Konfigurationsdateien ab und verändert diese den
  Anforderungen entsprechend. Hier einige Beispiele:
  \begin{verbatim}
     mkfli4l.bat hd.conf
     mkfli4l.bat -v
     mkfli4l.bat --no-gui config.hd
  \end{verbatim}

  \subsection{Konfigurationsdialog~-- Einstellung des Konfigurationsverzeichnis}

  Im Hauptfenster wird die Einstellung des Konfigurationsverzeichnis angezeigt
  und es kann ein Fenster geöffnet werden zur Auswahl des
  Konfigurationsverzeichnis.\\

  Zu beachten ist, dass eine Änderung des `Config-Dir' alle Optionen auf
  die Werte setzt, die in der dortigen
  \jump{sec:mkfli4lconf}{Steuerungsdatei `mkfli4l.txt'} gesetzt bzw.
  als Kommandozeilenparameter übergeben wurden.\\

  Findet mkfli4l.bat kein Verzeichnis fli4l-x.y.z$\backslash$config oder in
  dem Verzeichnis keine Datei mit dem Namen `base.txt' öffnet sich sofort das
  Fenster zur Auswahl des Konfigurationsverzeichnis. Dieses ermöglicht es auf
  einfache Weise im fli4l-Verzeichnis mehrere Konfigurationen zu verwalten.\\

  Beispiel:

\begin{example}
\begin{verbatim}
          fli4l-x.y.z\config
          fli4l-x.y.z\config.fd
          fli4l-x.y.z\config.cd
          fli4l-x.y.z\config.hd
          fli4l-x.y.z\config.hd-erstellen
\end{verbatim}
\end{example}

  \subsection{Konfigurationsdialog~-- allgemeine Einstellungen}
  \begin{figure}[ht!]
  \centering
  \includegraphics[width=\columnwidth]{win_build_build}
  \caption{Einstellungen}
  \label{fig:win_build_build}
  \end{figure}

  In diesem Dialog werden die Einstellungen für die Archiv/Bootmedienerstellung
  festgelegt:
  \begin{itemize}
    \item Build-Dir~-- Verzeichnis für die Archive/CD-Images/...
    \item \var{BOOT\_TYPE}~-- Anzeige des verwendeten/eingestellen \var{BOOT\_TYPE}~-- nicht änderbar
    \item Verbose~-- Aktivierung von zusätzlichen Ausgaben während der Erstellung
    \item Filesonly~-- es werden nur die Archive erstellt~-- kein bootmedium/kein Image
    \item Remoteupdate~-- Aktivierung des Remoteupdates per SCP
  \end{itemize}

  Mit der Schaltfläche \textbf{Aktuelle Einstellungen in mkfli4l.txt speichern}
  können die aktuell eingestellten Werte in der mkfli4l.txt gespeichert werden.

  \subsection{Konfigurationsdialog~-- Einstellungen für Remoteupdate}
  \begin{figure}[ht!]
  \centering
  \includegraphics[width=\columnwidth]{win_build_remoteupdate}
  \caption{Einstellungen für Remoteupdate}
  \label{fig:win_build_remoteupdate}
  \end{figure}

  In diesem Dialog werden die Einstellungen für den Remoteupdate festgelegt:
  \begin{itemize}
    \item IP-Adresse oder Hostname
    \item Benutzername auf dem Remote-Host
    \item Remote-Pfad (default: /boot)
    \item Remote-Port (default: 22)
    \item zu verwendendes SSH-Keyfile (ppk-Format von Putty)
  \end{itemize}

  \subsection{Konfigurationsdialog~-- Einstellungen für HD-pre-install}
  \begin{figure}[ht!]
  \centering
  \includegraphics[width=\columnwidth]{win_build_hd_install}
  \caption{Einstellungen für HD-pre-install}
  \label{fig:win_build_hd_install}
  \end{figure}

   In diesem Dialog können die Optionen für den HD-pre-install auf einer
   entsprechend partitionierten und formatierten CompactFlash-Karte
   in einem USB-Reader eingestellt werden.

   Mögliche Optionen:
   \begin{itemize}
     \item HD-pre-install aktivieren
     \item Laufwerksbuchstabe der CF-Karte
  \end{itemize}

  Hinweis zur Partionierung und Formatierung der CF:
  Für eine HD-Installation nach TYP A (siehe dazu Paket HD) muss auf der CF eine
  primäre aktive und formatierte FAT-Partition vorhanden sein. Möchte man
  weiterhin auch eine Datenpartiton benutzen, wird zusätzlich eine Linux-Partition,
  die mit dem Dateisystem ext3 formatiert ist, sowie die Datei \texttt{hd.cfg} auf der
  FAT-Partiton benötigt (hierzu sollten unbedingt die Hinweise im Paket HD beachtet
  werden).

\marklabel{sec:mkfli4lconf}{
  \section{Steuerungsdatei mkfli4l.txt}}
  Seit fli4l-Version 2.1.9 existiert die Steuerungsdatei
  \texttt{$<$config$>$/mkfli4l.txt}. Durch sie werden z.B. vom Standard
  abweichende Verzeichnisse übergeben. Die Steuerungsdatei hat einen
  ähnlichen Aufbau wie die normalen fli4l Konfigurationsdateien.
  Alle Konfigurationsvariablen sind hier optional, d.h. sie müssen nicht
  in der Konfigurationsdatei vorkommen oder können als Kommentar gekennzeichnet
  werden.
  \begin{description}

  \config {BUILDDIR}{BUILDDIR}{BUILDDIR}

  Standardwert: `build'

  Legt fest, in welchem Verzeichnis die fli4l Dateien erzeugt werden sollen.
  Ist die Variable undefiniert, setzt mkfli4l unter Windows `build' relativ zum fli4l
  Wurzelverzeichnis ein und meint damit also das Verzeichnis
  \texttt{build} im fli4l Wurzelverzeichnis:
  \begin{verbatim}
    Pfad/fli4l-x.y.z/build
  \end{verbatim}
  \vspace{-2ex}
  Unter *nix setzt mkfli4l \texttt{$<$config$>$/build} ein und legt damit die
  generierten Dateien zusammen mit der Konfiguration ab.

  Die konfigurierten Pfade in \var{BUILDDIR} müssen der jeweiligen Logik von
  Windows oder *unix entsprechen. Werden relative Pfade gesetzt, wird der Pfad
  durch den Buildprozess passend zu Windows oder *unix konvertiert.

  \config {VERBOSE}{VERBOSE}{VERBOSE}

  Standardwert: \var{VERBOSE='no'}

  Mögliche Werte sind \var{'yes'} oder \var{'no'}. Steuert die \emph{Geschwätzigkeit}
  des Build Prozesses.

  \config {FILESONLY}{FILESONLY}{FILESONLY}

  Standardwert: \var{FILESONLY='no'}

  Mögliche Werte \var{'yes'} oder \var{'no'}. Hiermit kann das Erstellen eines
  Boot-Mediums abgeschaltet werden, es werden also nur die Dateien erzeugt~--

  \config {REMOTEUPDATE}{REMOTEUPDATE}{REMOTEUPDATE}

  Standardwert: \var{REMOTEUPDATE='no'}

  Mögliche Werte \var{'yes'} oder \var{'no'}. Aktiviert das automatische
  Übertragen der erstellten Dateien mittels SCP auf den Router. Dieses setzt
  ein installiertes Paket \jump{OPTSSHD}{SSHD} mit aktiviertem \texttt{scp}
  voraus.  Siehe dazu auch die folgenden Variablen.

  \config {REMOTEHOSTNAME}{REMOTEHOSTNAME}{REMOTEHOSTNAME}

  Standardwert: \var{REMOTEHOSTNAME=''}

  Gibt den Ziel-Hostnamen für den SCP Datentransfer an.
  Sollte kein Name angegeben sein, wird dieser der Variable
  \jump{HOSTNAME}{\var{HOSTNAME}} entnommen.

  \config {REMOTEUSERNAME}{REMOTEUSERNAME}{REMOTEUSERNAME}

  Standardwert: \var{REMOTEUSERNAME='fli4l'}

  Username für den SCP Datentransfer.

  \config {REMOTEPATHNAME}{REMOTEPATHNAME}{REMOTEPATHNAME}

  Standardwert: \var{REMOTEPATHNAME='/boot'}

  Ziel-Pfad für den SCP Datentransfer.

  \config {REMOTEPORT}{REMOTEPORT}{REMOTEPORT}

  Standardwert: \var{REMOTEPORT='22'}

  Zielport für den SCP Datentransfer.

  \config {SSHKEYFILE}{SSHKEYFILE}{SSHKEYFILE}

  Standardwert: \var{SSHKEYFILE=''}

  Hier kann man eine SSH-Keydatei für den SCP-Remoteupdate angeben.
  Es kann somit ein Update ohne Angabe eines Passwortes erfolgen.
  
  \config {REMOTEREMOUNT}{REMOTEREMOUNT}{REMOTEREMOUNT}
  
  Standardwert: \var{REMOTEREMOUNT='no'}
  
  Mögliche Werte \var{'yes'} oder \var{'no'}. Wird hier \var{'yes'}
  gesetzt, wird ein eventuell Readonly eingehängtes Bootdevice "/boot"
  für das Remoteupdate Readwrite gemountet um das Remoteupdate möglich
  zu machen. 

  \config {TFTPBOOTPATH}{TFTPBOOTPATH}{TFTPBOOTPATH}

  Pfad an dem das Netboot-Image abgelegt wird.

  \config {TFTPBOOTIMAGE}{TFTPBOOTIMAGE}{TFTPBOOTIMAGE}

  Name des Netboot-Images.

  \config {PXESUBDIR}{PXESUBDIR}{PXESUBDIR}

  Unterverzeichnis für die PXE-Dateien relativ zu TFTPBOOTPATH.


  \config {SQUEEZE\_SCRIPTS}{SQUEEZE\_SCRIPTS}{SQUEEZESCRIPTS}

   Aktiviert bzw. deaktiviert das Squeezen (Kommprimieren) von
   Skripten.
   Das Komprimieren eines Skripts mit Squeeze entfernt alle Kommentare und
   Zeileneinrückungen.
   Im Normalfall sollte hier immer der Standardwert \var{'yes'} benutzt werden.

  \config {MKFLI4L\_DEBUG\_OPTION}{MKFLI4L\_DEBUG\_OPTION}{MKFLI4LDEBUGOPTION}

   Es können zum Debuggen zusätzliche Optionen an das \jump{mkfli4l}{mkfli4l-Programm} übergeben
   werden.

  \end{description}

  \chapter{Anbindung von PCs im LAN}

  Für jeden Rechner im LAN ist einzustellen:

  \begin{enumerate}
  \item IP-Adresse (siehe \smalljump{sec:pc-lan-ip}{IP-Adresse})
  \item Name des Rechners plus Wunsch-Domain-Name
    (siehe \smalljump{sec:pc-lan-name}{Rechnername und Domain})
  \item Standard-Gateway (siehe \smalljump{sec:pc-lan-gateway}{Gateway})
  \item IP-Adresse des DNS-Servers (siehe \smalljump{sec:pc-lan-dns}{DNS-Server})
  \end{enumerate}

  \marklabel{sec:pc-lan-ip}{\section{IP-Adresse}}
  Die IP-Adresse muss im gleichen Netz wie die IP-Adresse des
  fli4l-Routers (auf Ethernet-Seite) liegen, also z.B. 192.168.6.2,
  wenn der fli4l die Adresse 192.168.6.1 hat.
  Kein Rechner darf die gleiche IP-Adresse haben, weshalb man am
  besten (nur) die letzte Zahl ändert. Auch ist darauf zu achten, dass
  man hier die gleiche IP-Adresse angibt, wie man es für diesen
  Rechner in der Datei config/base.txt angegeben hat.

  \marklabel{sec:pc-lan-name}{\section{Rechnername und Domain}}
  Der Name des Rechners ist dann z.B. ``mein-pc'', die Domain ``lan.fli4l''.

  \wichtig{Die im PC eingestellte Domain muss identisch mit der
  gewählten Domain im fli4l-Rechner sein, wenn man den fli4l-Router
  als DNS-Server verwenden will. Sonst kann es im Netz erhebliche
  Probleme geben.}

  Grund: Windows-Rechner suchen regelmäßig nach Rechnern mit dem Namen
  ihrer Arbeitsgruppte: WORKGROUP.meine-domain.fli4l. Ist dies nicht die in fli4l
  eingestellte Domain (hier: meine-domain.fli4l), wird fli4l
  versuchen, diese Anfrage durch Weiterleiten ins Internet zu
  beantworten \ldots

  Einzutragen ist die Domain in den TCP/IP Einstellungen des Rechners.

  \subsection{Windows 2000}

  Für Windows 2000 findet man das unter:

  \noindent Start \pfeil\\
  \hspace*{2ex}Einstellungen \pfeil\\
  \hspace*{4ex}Systemsteuerung \pfeil\\
  \hspace*{6ex}Netzwerk- und DFÜ-Verbindungen \pfeil\\
  \hspace*{8ex}LAN-Verbindung \pfeil\\
  \hspace*{10ex}Eigenschaften \pfeil\\
  \hspace*{12ex}Internetprotokoll (TCP/IP) \pfeil\\
  \hspace*{14ex}Eigenschaften \pfeil\\
  \hspace*{16ex}Erweitert\ldots \pfeil\\
  \hspace*{18ex}DNS \pfeil\\
  \hspace*{20ex}DNS-Suffix hinzufügen \pfeil\\

  ``lan.fli4l'' (bzw. die eingestellte domain) eingeben (ohne ``''!)
  \pfeil OK drücken.

\subsection{NT 4.0}

  Start \pfeil\\
  \hspace*{2ex}Einstellungen \pfeil\\
  \hspace*{4ex}Systemsteuerung  \pfeil\\
  \hspace*{6ex}Netzwerk \pfeil\\
  \hspace*{8ex}Protokolle \pfeil\\
  \hspace*{10ex}TCP/IP \pfeil\\
  \hspace*{12ex}Eigenschaften \pfeil\\
  \hspace*{14ex}DNS \pfeil\\
  \hspace{16ex}\begin{itemize}
  \item Hostname eintragen (eigener Rechnername)
  \item Domäne eintragen (wie in config/base.txt)
  \item IP-Adresse vom fli4l-Router hinzufügen
  \item DNS-Suffix hinzufügen (Domäne hinzufügen~-- siehe
    2 Zeilen höher)
  \end{itemize}

\subsection{Win95/98}

  Start \pfeil\\
  \hspace*{2ex}Einstellungen \pfeil\\
  \hspace*{4ex}Systemsteuerung \pfeil\\
  \hspace*{6ex}Netzwerk \pfeil\\
  \hspace*{8ex}Konfiguration \pfeil\\
  \hspace*{10ex}TCP/IP (jenes, das an die Netzwerkkarte zum Router\\
  \hspace*{10ex}angebunden ist) \pfeil\\
  \hspace*{12ex}Eigenschaften \pfeil\\
  \hspace*{14ex}DNS-Konfiguration:

  DNS aktivieren und bei ``Domäne:'' dann ``lan.fli4l'' eingeben (ohne ``''!).

\subsection{Windows XP}

  Für Windows XP findet man das unter:

  \noindent Start \pfeil\\
  \hspace*{2ex}Einstellungen \pfeil\\
  \hspace*{4ex}Systemsteuerung \pfeil\\
  \hspace*{6ex}Netzwerkverbindungen \pfeil\\
  \hspace*{8ex}LAN-Verbindung \pfeil\\
  \hspace*{10ex}Eigenschaften \pfeil\\
  \hspace*{12ex}Internetprotokoll (TCP/IP) \pfeil\\
  \hspace*{14ex}Eigenschaften \pfeil\\
  \hspace*{16ex}Erweitert\ldots \pfeil\\
  \hspace*{18ex}DNS \pfeil\\
  \hspace*{20ex}DNS-Suffix für diese Verbindung \pfeil\\

  ``lan.fli4l'' (bzw. die eingestellte domain) eingeben (ohne ``''!)
  \pfeil OK drücken.

\subsection{Windows 7}

  Für Windows 7 findet man das unter:

  \noindent Windows Button (ex. Start) \pfeil\\
  \hspace*{2ex}Systemsteuerung \pfeil\\
  \hspace*{4ex}Netzwerk und Internet \pfeil\\
  \hspace*{6ex}Netzwerk- und Freigabecenter \pfeil\\
  \hspace*{8ex}LAN-Verbindung \pfeil\\
  \hspace*{10ex}Eigenschaften \pfeil\\
  \hspace*{12ex}Internetprotokoll Version 4 (TCP/IPv4) \pfeil\\
  \hspace*{14ex}Eigenschaften \pfeil\\
  \hspace*{16ex}Erweitert\ldots \pfeil\\
  \hspace*{18ex}DNS \pfeil\\
  \hspace*{20ex}DNS-Suffix für diese Verbindung \pfeil\\

  ``lan.fli4l'' (bzw. die eingestellte domain) eingeben (ohne ``''!)
  \pfeil OK drücken.

\subsection{Windows 8}

  Für Windows 8 findet man das unter:

  \noindent Gleichzeitig Windows- und X-Taste drücken \pfeil\\
  \hspace*{2ex}Systemsteuerung \pfeil\\
  \hspace*{4ex}Netzwerk und Internet \pfeil\\
  \hspace*{6ex}Netzwerk- und Freigabecenter \pfeil\\
  \hspace*{8ex}Ihr Netzwerk wählen (Ehternet oder WLAN) \pfeil\\
  \hspace*{10ex}Eigenschaften \pfeil\\
  \hspace*{12ex}Internetprotokoll Version 4 (TCP/IPv4) \pfeil\\
  \hspace*{14ex}Eigenschaften \pfeil\\
  \hspace*{16ex}Erweitert\ldots \pfeil\\
  \hspace*{18ex}DNS \pfeil\\
  \hspace*{20ex}DNS-Suffix für diese Verbindung \pfeil\\

  ``lan.fli4l'' (bzw. die eingestellte domain) eingeben (ohne ``''!)
  \pfeil OK drücken.

  \marklabel{sec:pc-lan-gateway}{\section{Gateway}}
  Die Angabe des Standard-Gateways ist unbedingt erforderlich, denn
  ohne die Angabe der richtigen IP-Adresse an dieser Stelle
  funktioniert nichts.  Es muß hier die IP-Adresse des fli4l-Routers
  (auf Ethernet-Seite) angegeben werden, also z.B. 192.168.6.4
  entsprechend der IP-Adresse, die hier in der Datei config/base.txt
  für den fli4l-Router angegeben wurde.

  Es ist falsch, den fli4l-Router als Proxy in der Windows- oder
  Browser- Konfiguration einzutragen~-- außer man setzt ein Proxy auf
  dem fli4l-Router ein. Im Normalfall ist fli4l kein Proxy, daher
  bitte \emph{nicht} fli4l als Proxy angeben!

\marklabel{sec:pc-lan-dns}{\section{DNS-Server}}

  Als IP-Adresse des DNS-Servers gibt man nicht die Adresse des
  Provider-DNS-Servers an, sondern die des fli4l-Routers (Ethernet),
  da dieser nun selbst Anfragen beantworten kann bzw. diese bei
  Unkenntnis ins Internet weiterleitet.

  Mit der Konstruktion von fli4l als DNS-Server werden viele von den
  Windows-PCs ausgeführten Anfragen nicht ins Internet weitergeroutet,
  sondern werden direkt vom fli4l-Router beantwortet.

\marklabel{sec:pc-lan-misc}{\section{Verschiedenes}}

  Die Punkte 1 bis 4 brauchen bei konfiguriertem DHCP-Server nicht
  eingetragen zu werden, da dann der fli4l-Router die notwendigen Daten
  automatisch übermittelt.

  \textbf{Internetoptionen:} Bei Verbindungen muß ``keine Verbindung wählen'' ausgewählt sein.
  Bei Einstellungen für lokales Netzwerk(LAN): es darf hier NICHTS
  angegeben werden (es sei dann es wird \var{OPT\_\-P}roxy verwendet).
  Beides sind Standardeinstellungen, die im Normalfall nicht geändert
  werden müssen.

\marklabel{IMONDSCHNITTSTELLE}{
    \chapter{Client-/Server-Schnittstelle imon}
  }

  \marklabel{sec:imond}{
    \section{imon-Server imond}}

  imond ist ein netzwerkfähiges Server-Programm, welches bestimmte
  Anfragen beantwortet oder auch Kommandos zur Steuerung des Routers
  entgegennehmen kann.

  Ausserdem steuert imond das Least-Cost-Routing. Dazu verwendet er
  eine Konfigurationsdatei /etc/imond.conf, welche beim Booten
  automatisch aus den \var{ISDN\_\-CIRC\_\-x\_\-XXX}-Variablen der Datei
  config/isdn.txt und anderen über ein Shell-Script erzeugt wird.

  imond läuft permanent als Daemon und horcht gleichzeitig auf
  TCP/IP-Port 5000 und Device /dev/isdninfo.

  Folgende Kommandos sind über den TCP/IP-Port 5000 möglich:
  \begin{table}
    \textbf{Admin-Befehle}

    \vspace{1ex}
    \begin{tabular}{lp{9cm}}

      addlink ci-index              & Channel zum Circuit hinzufügen
                                      (Channel-Bundling) \\
      adjust-time seconds           & Ändert die Uhrzeit des Routers um die
                                      angegebenen Sekunden \\
      delete filename pw            & Löscht die Datei auf dem Router \\
      hup-timeout \#ci-index [value]& Anzeigen bzw. Setzen des HUP-Timeout für
                                      ISDN-Circuits \\
      removelink ci-index           & Zusätzlichen Channel wieder entfernen \\
      reset-telmond-log-file        & Löschen der telmond-Protokolldatei \\
      reset-imond-log-file          & Löschen der imond-Protokolldatei \\
      receive filename \#bytes pw   & Eine Datei auf den Router übertragen.
                                      Dazu quittiert imond den Befehl mit
                                      einem ACK (0x06). Danach wird die Datei
                                      in 1024er-Blöcken übertragen, die imond
                                      auch jeweils mit einem ACK bestätigt.
                                      Als letztes übermittelt imond noch ein
                                      OK. \\
      send filename pw              & Wenn das Passwort stimmt und die Datei
                                      existiert, liefert imond ein OK \#bytes.
                                      Anschliessend überträgt imond die Datei
                                      in 1024er Blöcken, die jeweils mit
                                      einem ACK (0x06) bestätigt werden
                                      müssen. Als letztes liefert imond noch
                                      ein OK. \\
      support pw                    & Liefert den Status/Konfiguration vom
                                      Router \\
      sync                          & Synchronisiert den Cache von gemounteten
                                      Laufwerken \\
    \end{tabular}
  \end{table}


  \begin{table}
    \textbf{Admin- oder User-Befehle}

    \vspace{1ex}
    \begin{tabular}{lp{9cm}}

      dial                      &    Wählt den Provider an
                                     (Default-Route-Circuit) \\
      dialmode [auto|manual|off]&    Liefert bzw. setzt den Dialmode \\
      disable                   &    Hängt ein und setzt dialmode auf ``off''
                                      \\
      enable                    &    Setzt dialmode auf ``auto'' \\
      halt                      &    Fährt den Router sauber herunter \\
      hangup [\#channel-id]     &    Hängt ein \\
      poweroff                  &    Fährt den Router herunter und schaltet ab \\
      reboot                    &    Reboot vom i4l-Router! \\
      route [ci-index]          &    Setzen Default-Route auf Circuit X
                                     (0=automatisch) \\
    \end{tabular}
  \end{table}


  \begin{table}
    \textbf{User-Befehle}

    \vspace{1ex}
    \begin{tabular}{lp{9cm}}
      channels                  & Ausgabe Anzahl der verfügbaren ISDN-Kanäle\\
      charge \#channel-id       & Ausgabe der Online-Kosten für einen
                                  Channel\\
      chargetime \#channel-id   & Online-Zeit unter Berücksichtigung des
                                  Taktes\\
      circuit [ci-index]        & Ausgabe eines Circuit-Namens\\
      circuits                  & Ausgabe Anzahl der Default-Route-Circuits\\
      cpu                       & Liefert die Auslastung der CPU in Prozent\\
      date                      & Ausgabe Datum/Uhrzeit\\
      device ci-index           & Liefert das Device des Circuits\\
      driverid \#channel-id     & Ausgabe Driver-Id für Channel X\\
      help                      & Ausgabe Hilfe\\
      inout \#channel-id        & Ausgabe der Richtung (incoming/outgoing)\\
      imond-log-file            & Ausgabe imond-Protokolldatei\\
      ip \#channel-id           & Ausgabe der IP\\
      is-allowed command        & Ausgabe, ob Befehl konfiguriert/gültig
                                  ist\newline
                                  Mögliche Befehle:
                                    dial|dialmode|route|reboot|
                                    imond-log|telmond-log|mgetty-log \\
      is-enabled                & Ausgabe, ob dialmode auf off (0) oder auto
                                  (1)\\
      links ci-index            & Ausgabe Anzahl momentaner Channel 0, 1 oder
                                  2, 0 heisst: Kein Channel-Bundling möglich\\
      log-dir imond|telmond|mgetty& Liefert das Logverzeichnis\\
      mgetty-log-file           & Ausgabe mgetty-Protokolldatei\\
      online-time \#channel-id  & Ausgabe Online-Zeit der akt. Verbindung in
                                  hh:mm:ss\\
      pass [password]           & Abfrage, ob Password nötig bzw. Password-
                                  Eingabe\newline
                                  1 Userpassword gesetzt\newline
                                  2 Adminpassword gesetzt\newline
                                  4 imond befindet sich im Admin-Modus\\
      phone \#channel-id        & Ausgabe Telefonnummer/Name des ``Gegners''\\
      pppoe                     & Liefert die Anzahl der pppoe-Devices (also 0
                                  oder 1)\\
      quantity \#channel-id     & Liefert die übertragenen Datenmengen (in
                                  Byte)\\
      quit                      & Beenden der Verbindung zu imond\\
      rate \#channel-id         & Ausgabe Übertragungsraten (incoming/outgoing
                                  in B/sec)\\
      status \#channel-id       & Ausgabe Status für Channel X\\
      telmond-log-file          & Ausgabe telmond-Protokolldatei\\
      time \#channel-id         & Ausgabe Summe Online-Zeiten, Format
                                  hh:mm:ss\\
      timetable [ci-index]      & Ausgabe der Zeittabelle für LC-Routing\\
      uptime                    & Ausgabe der Uptime des Routers in Sekunden\\
      usage \#channel-id        & Ausgabe Art der Verbindung, mögliche
                                  Antworten: Fax, Voice, Net, Modem, Raw\\
      version                   & Ausgabe der Protokoll- und
                                  Programm-Version\\
    \end{tabular}
  \end{table}


  Der TCP/IP-Port 5000 ist nur vom maskierten LAN aus erreichbar.
  Standardmäßig wird ein Zugriff von aussen über die
  Firewall-Konfiguration abgeblockt.

  Imond unterstützt zwei Benutzerebenen: den User- und den
  Admin-Modus.  Für beide Ebenen kann ein Passwort gesetzt werden
  mittels \var{IMOND\_\-PASS} bzw.  \var{IMOND\_ADMIN\_\-PASS}. Dadurch
  werden die imon-Clients von imond gezwungen, eine Password-Abfrage
  durchzuführen und anschließend das Password an imond zu übertragen.
  Solange dieses Password nicht übermittelt wurde, nimmt imond nur die
  beiden Kommandos ``pass'' und ``quit'' entgegen. Alle anderen werden
  mit einem Fehler zurückgewiesen.

  Möchte man das weiter einschränken, z.B. den Zugriff nur von nur
  einem PC erlauben, muss die Firewall-Konfiguration angepasst werden.

  Die Befehle

\begin{example}
\begin{verbatim}
         enable/disable/dialmode   dial/hangup   route   reboot/halt
\end{verbatim}
\end{example}

  können durch die Konfigurationsvariablen \var{IMOND\_\-XXX} global ein- oder
  abgeschaltet werden (s. Kapitel ``Konfiguration'').

  Mit einem Unix/Linux-Rechner (oder einem Windows-Rechner in der DOS-Box)
  kann man das Ganze leicht ausprobieren:

  Nach Eingabe von

\begin{example}
\begin{verbatim}
        telnet fli4l 5000        \# oder entsprechender Name des fli4l-Routers
\end{verbatim}
\end{example}

  kann man direkt die oben aufgeführten Kommandos eingeben und sich
  die Ausgabe anschauen.

  Zum Beispiel bekommt man mit ``help'' die Hilfe angezeigt, mit
  ``quit'' wird die Verbindung zum imond abgebaut.

\marklabel{sec:leastcostrouting}{
  \subsection{Least-Cost-Routing~-- Funktionsweise}
  }

  imond konstruiert aus der Konfigurationsdatei /etc/imond.conf
  (welche wiederum beim Booten aus den Konfigurationsvariablen
  \var{ISDN\_\-CIRC\_\-x\_\-TIMES} usw.  erstellt wird), eine zeitabhängige
  Tabelle (Time-Table). Diese umfasst eine komplette Kalenderwoche im
  1-Stunden-Raster (168 Stunden = 168 Bytes). Die Tabelle setzt sich
  jedoch lediglich aus Circuits zusammen, für die eine Default-Route
  definert ist.

  Mit dem imond-Kommando ``timetable'' kann man sich diese Tabelle
  anschauen.

  Hier ein Beispiel:

  Nehmen wir an, dass 3 Circuits definiert wurden, nämlich:

\begin{example}
\begin{verbatim}
        CIRCUIT_1_NAME='Addcom'
        CIRCUIT_2_NAME='AOL'
        CIRCUIT_3_NAME='Firma'
\end{verbatim}
\end{example}

  wobei lediglich die ersten beiden Circuits mit Default-Routen belegt
  sind, also die enstprechenden Variablen ISDN\_CIRC\_x\_ROUTE den
  Wert `0.0.0.0' haben.

  Wenn die dazugehörigen Variablen \var{ISDN\_\-CIRC\_\-x\_\-TIMES} folgendermaßen
  aussehen:

\begin{example}
\begin{verbatim}
        ISDN_CIRC_1_TIMES='Mo-Fr:09-18:0.0388:N Mo-Fr:18-09:0.0248:Y
                      Sa-Su:00-24:0.0248:Y'

        ISDN_CIRC_2_TIMES='Mo-Fr:09-18:0.019:Y Mo-Fr:18-09:0.049:N
                      Sa-Su:09-18:0.019:N Sa-Su:18-09:0.049:N'

        ISDN_CIRC_3_TIMES='Mo-Fr:09-18:0.08:N Mo-Fr:18-09:0.03:N
                      Sa-Su:00-24:0.03:N'
\end{verbatim}
\end{example}

  dann wird daraus folgende Datei /etc/imond.conf:

\begin{example}
\begin{verbatim}
        #day  hour  device  defroute  phone        name        charge  ch-int
        Mo-Fr 09-18 ippp0   no        010280192306 Addcom      0.0388   60
        Mo-Fr 18-09 ippp0   yes       010280192306 Addcom      0.0248   60
        Sa-Su 00-24 ippp0   yes       010280192306 Addcom      0.0248   60
        Mo-Fr 09-18 ippp1   yes       019160       AOL  0.019   180
        Mo-Fr 18-09 ippp1   no        019160       AOL  0.049   180
        Sa-Su 09-18 ippp1   no        019160       AOL  0.019   180
        Sa-Su 18-09 ippp1   no        019160       AOL  0.049   180
        Mo-Fr 09-18 isdn2   no        0221xxxxxxx  Firma       0.08     90
        Mo-Fr 18-09 isdn2   no        0221xxxxxxx  Firma       0.03     90
        Sa-Su 00-24 isdn2   no        0221xxxxxxx  Firma       0.03     90
\end{verbatim}
\end{example}

  imond erstellt dann im Speicher folgende Time-Table~-- hier die Ausgabe
  über das imond-Kommando ``timetable'':

\begin{example}
\begin{verbatim}
         0  1  2  3  4  5  6  7  8  9 10 11 12 13 14 15 16 17 18 19 20 21 22 23
     --------------------------------------------------------------------------
     Su  3  3  3  3  3  3  3  3  3  3  3  3  3  3  3  3  3  3  3  3  3  3  3  3
     Mo  2  2  2  2  2  2  2  2  2  4  4  4  4  4  4  4  4  4  2  2  2  2  2  2
     Tu  2  2  2  2  2  2  2  2  2  4  4  4  4  4  4  4  4  4  2  2  2  2  2  2
     We  2  2  2  2  2  2  2  2  2  4  4  4  4  4  4  4  4  4  2  2  2  2  2  2
     Th  2  2  2  2  2  2  2  2  2  4  4  4  4  4  4  4  4  4  2  2  2  2  2  2
     Fr  2  2  2  2  2  2  2  2  2  4  4  4  4  4  4  4  4  4  2  2  2  2  2  2
     Sa  3  3  3  3  3  3  3  3  3  3  3  3  3  3  3  3  3  3  3  3  3  3  3  3

     No.  Name                   DefRoute  Device  Ch/Min   ChInt
      1   Addcom                   no      ippp0   0.0388     60
      2   Addcom                   yes     ippp0   0.0248     60
      3   Addcom                   yes     ippp0   0.0248     60
      4   AOL               yes     ippp1   0.0190    180
      5   AOL               no      ippp1   0.0490    180
      6   AOL               no      ippp1   0.0190    180
      7   AOL               no      ippp1   0.0490    180
      8   Firma                    no      isdn2   0.0800     90
      9   Firma                    no      isdn2   0.0300     90
     10   Firma                    no      isdn2   0.0300     90
\end{verbatim}
\end{example}

  Für den Circuit 1 (Addcom) sind also drei Zeitbereiche (1-3)
  eingetragen, für Circuit 2 (AOL) vier Zeitbereiche (4-7) und
  für den letzen drei Zeitbereiche (8-10).

  In der Time-Table werden jeweils die Indices ausgegeben, welche für
  die jeweilige Stunde gültig sind. Hier tauchen lediglich die Indices
  2-4 auf, da alle anderen keine LC-Default-Routen sind.

  Sieht man in der Tabelle irgendwo Nullen, gibt es Lücken in den
  \var{ISDN\_\-CIRC\_\-X\_\-TIMES}-Werten. Dann existiert zu diesen Zeiten keine
  Default-Route, Internet-Zugang abgeknipst!

  Beim Programmstart ermittelt imond zunächst den Wochentag und die
  aktuelle Stunde. Anschließend wird dann über die Time-Table der
  Index ermittelt und damit dann auch der entsprechende Circuit. Auf
  diesen wird dann die Default-Route gesetzt.

  Bei Zustandsänderungen der Channels, z.B. Wechsel von online
  nach offline~-- jedoch spätestens nach 1 Minute~-- geht das Spiel von
  neuem los: Zeit ermitteln, Lookup in Tabelle, Default-Route-Circuit
  ermitteln.

  Ändert sich der aktuell verwendete Circuit, z.B. montags um 18:00
  Uhr, wird die alte Default-Route gelöscht, eine vielleicht
  bestehende Verbindung beendet (sorry\ldots) und anschließend die
  Default-Route auf den neuen Circuit gesetzt. Dies kann von imond bis
  zu 60 Sekunden später bemerkt werden, also wird spätestens um
  18:00:59 umgeschaltet.

  Bei Circuits, die keine Default-Route belegen, ändert sich überhaupt
  nichts. Hier wird der Inhalt von \var{ISDN\_\-CIRC\_\-x\_\-TIMES} lediglich zur
  Berechnung der Telefonkosten verwendet. Diese können dann relevant
  sein, wenn man über den Client imonc das LC-Routing temporär
  ausschaltet und einen Circuit manuell wählt.

  Man kann sich jedoch auch die Tabellen für andere
  Zeitbereich-Indices (im Beispiel von 1 bis 10) anschauen, auch die
  der ``Non-LC-Default-Route-Circuits''.

  Kommando:

\begin{example}
\begin{verbatim}
                    timetable index
\end{verbatim}
\end{example}

  Beispiel:

\begin{example}
\begin{verbatim}
                    telnet fli4l 5000
                    timetable 5
                    quit
\end{verbatim}
\end{example}

  Die Ausgabe sieht dann so aus:

\begin{example}
\begin{verbatim}
         0  1  2  3  4  5  6  7  8  9 10 11 12 13 14 15 16 17 18 19 20 21 22 23
     --------------------------------------------------------------------------
     Su  0  0  0  0  0  0  0  0  0  0  0  0  0  0  0  0  0  0  0  0  0  0  0  0
     Mo  5  5  5  5  5  5  5  5  5  0  0  0  0  0  0  0  0  0  5  5  5  5  5  5
     Tu  5  5  5  5  5  5  5  5  5  0  0  0  0  0  0  0  0  0  5  5  5  5  5  5
     We  5  5  5  5  5  5  5  5  5  0  0  0  0  0  0  0  0  0  5  5  5  5  5  5
     Th  5  5  5  5  5  5  5  5  5  0  0  0  0  0  0  0  0  0  5  5  5  5  5  5
     Fr  5  5  5  5  5  5  5  5  5  0  0  0  0  0  0  0  0  0  5  5  5  5  5  5
     Sa  0  0  0  0  0  0  0  0  0  0  0  0  0  0  0  0  0  0  0  0  0  0  0  0

     No.  Name                   DefRoute  Device  Ch/Min   ChInt
      5   AOL               no      ippp1   0.0490    180
\end{verbatim}
\end{example}

  Alles klar?

  Mit dem imond-Kommando ``route'' kann das LC-Routing ein- und
  ausgeschaltet werden. Bei Angabe eines positiven Circuit-Indices
  (1\ldots N) wird die Default-Route auf den angegebenen Circuit
  gelegt. Ist der Index 0, wird das LC-Routing wieder aktiviert und
  der Circuit automatisch ausgewählt.


  \subsection{Zur Berechnung der Onlinekosten}

  Das ganze Modell zur Berechnung der Onlinekosten funktioniert nur
  korrekt, wenn der Zeittakt für einen Circuit (Variable
  \var{ISDN\_\-CIRC\_\-x\_\-CHARGEINT}) über die ganze Woche konstant ist. Dies
  ist im Normalfall bei Internet-Providern die Regel. Wählt man sich
  jedoch über die Telekom (ich meine nicht T-Online!) z.B. in sein
  Firmennetz ein, gilt das als ganz normales Telefongespräch. Und da
  wechselt ab 18:00 der Takt von 90 Sekunden auf 4 Minuten (Stand Juni
  00). Deshalb ist die Definition von

\begin{example}
\begin{verbatim}
        ISDN_CIRC_3_CHARGEINT='90'
        ISDN_CIRC_3_TIMES='Mo-Fr:09-18:0.08:N Mo-Fr:18-09:0.03:N Sa-Su:00-24:0.03:N'
\end{verbatim}
\end{example}

  eigentlich nicht ganz korrekt. Es sind zwar abends umgerechnet auf
  die Minute 3 Pfennig (4 Minuten kosten 12 Telekom-Pfennige), jedoch
  ist der Takt falsch. Deshalb können bei der Kostenanzeige
  Differenzen zu den tatsächlichen Zahlen auftreten.

  Hier ist ein Tipp, wie verschieden lange Taktzeiten dennoch richtig
  berücksichtigt werden können (auch wichtig für
  \var{ISDN\_\-CIRC\_\-x\_\-CHARGEINT}): Man definiert einfach 2 Circuits,
  einen für tagsüber mit \var{ISDN\_\-CIRC\_\-1\_\-CHARGEINT}=`90' und den
  anderen mit \var{ISDN\_\-CIRC\_\-2\_\-CHARGEINT}=`240'.
  Natürlich muss man dann auch noch \var{ISDN\_\-CIRC\_\-x\_\-TIMES}
  entsprechend wählen, damit tagsüber Circuit 1 und abends Circuit 2
  verwendet wird.

  Wie gesagt: Bei Nutzung von Verbindungen zu Internet-Providern gibt
  es das Problem nicht, weil dort der Zeittakt immer konstant ist und
  lediglich die Kosten pro Minute wechseln (oder gibt es sowas doch???
  Ich traue T-* alles zu :-).

  % Do not remove the next line
% Synchronized to r30003

  \marklabel{sec:winimonc}{
    \section{Client Windows imonc.exe}}

  \subsection{Introduction}

  Le démon Imond sur le routeur fli4l gère deux modes d'utilisations différents~:
  le mode Administrateur (Admin) et le mode Utilisateur. Dans le mode Admin
  toutes les commandes sont activées automatiquement. Dans le mode Utilisateur
  vous devez activer les variables \jump{IMONDENABLE}{\var{IMOND\_ENABLE}},
  \jump{IMONDDIAL}{\var{IMOND\_DIAL}}, \jump{IMONDROUTE}{\var{IMOND\_ROUTE}} et
  \jump{IMONDREBOOT}{\var{IMOND\_REBOOT}}, dans le fichier /config/base.txt pour avoir
  les commandes. Si les variables sont sur `no' les commandes ne seront pas activées,
  même les commandes Exit et mode Admin ne seront pas activées dans le client imonc.
  Le choix de l'utilisation entre le mode Utilisateur et le mode Admin se fait
  par l'intermédiaire d'un Mot de Passe qui sera transféré au routeur. Vous pouvez
  activer le mode Admin ou Utilisateur, en cliquant sur l'icone située dans
  la barre de taches et entrer le Mot de Passe n'oubliez pas de redémarrer
  le client imonc.

  Lorsque imonc a démarré, une icône supplémentaire apparait dans la barre de taches,
  il indique le statut des canaux de la connexion Internet pour le (numéris).

  Les couleurs de l'icône ignifient~:

  \begin{description}
    \item[Rouge]~: offline (déconnecté)
    \item[Jaune]~: en cours de connexion 
    \item[Vert clair]~: online (en ligne il y a du trafic sur le canal)
    \item[Vert foncé]~: online (en ligne il n'y a pas de trafic sur le canal)
  \end{description}

  \noindent Suivant le Windows que vous utilisez le comportement d'imonc diverge,
  il peut être réduit à une icone dans la barre des taches près de l'heure. pour
  ouvrir la fenêtre il suffit de faire un double clic avec le bouton gauche de
  la souris sur l'icone. Pour ouvrir le menu contextuel vous utilisez le bouton
  droit, delà vous pouvez choisir directement les commandes imonc.

  Un (grand nombre de paramètres) peuvent être adaptés selon vos propres besoins,
  ils seront enregistrés et sauvegardés dans la base de registre de Windows à cet
  endroit HKCU{\textbackslash}Software{\textbackslash}fli4l.

  Il y a toujours quelques erreurs dans la documentation d'imonc et du routeur
  fli4l, malgré des relectures. Si vous rencontrez des problèmes, allez dans la
  page "A propos" cliquer sur le bouton systeminfo puis sur le bouton support info,
  ensuite le mot de passe du routeur vous sera demandé (pas le mot de passe d'imond~!).
  Imonc produira un fichier fli4lsup.txt, qui inclura toutes les informations
  importantes sur le routeur fli4l et sur imonc. Ce fichier peut être ajouté
  dans le Newsgroup pour demander de l'aide. Cela maximisera les chances
  d'avoir de l'aide plus rapidement.

  Vous pouvez trouve des détails concernant le développement du client imonc pour
  Windows sur le site \altlink{http://www.imonc.de/}, vous trouverez des informations
  sur les nouveaux dispositifs les futures versions d'imonc, les résolutions de bug
  et aussi la dernière version à télécharger (si elle n'est pas déjà inclue dans
  la distribution fli4l).

  \subsection{Paramètre de démarrage}

  Le client imonc à besoin du Nom ou de Adresse IP du routeur fli4l pour pouvoir
  établir une connexion avec celui-ci "l'ordinateur fli4l". Si l'ordinateur du client
  imonc est enregistré correctement dans le DNS, il devrait fonctionner sans problème.
  Voici les paramètres que l'on peut transmettre~: 

  \begin{itemize}
    \item /Server:IP ou Nom d'Hôte du routeur (Forme abrégée~: /S:IP ou Nom d'Hôte)
    \item /Password:Mot de Passe (Forme abrégée: /P:Mot de Passe)
    \item /log Active le protocole de communication entre imonc et imond, lorsque
      cette option est activée un fichier imonc.log est créé. Ce fichier enregistre
      toutes les communications, il peut être très volumineux. C'est pour cette
      raison que l'on active ce paramètre uniquement si il y a des problèmes de
      configurations.
    \item /iport:N$^\circ$ port Par défaut imond écoute sur le Port~: 5000
    \item /tport:N$^\circ$ port Par défaut telmond écoute sur le Port~: 5001
    \item /rc:"Commande" Les commandes écrites ici sont transmis au routeur sans
      aucun contrôle supplémentaire. Si plusieurs commandes sont exportées
      simultanément elles doivent être séparées par un point virgule. Pour être
      sur du fonctionnement de imonc vous devez retaper le Mot de Passe
      (si configuré?) car il n'y aura aucune redemande de Mot de Passe.
      les commandes possibles sont documentées dans le Chapitre 8.1. La commande
      dialtimesync n'ai plus utilisée elle est remplacée par \flqq{}dial; timesync\frqq{},
      qui force le routeur à synchroniser l'heure avec le client.
    \item /d:"Répertoire-fli4l" Cette option permet d'écrire le répertoire du
      dossier fli4l avec des paramètres de démarrage, c'est intéressant pour ceux
      qui utilisent plusieurs versions de fli4l.
    \item /wait Si le Nom d'Hôte ne peut pas être résolu, imonc se bloque,
      il faut redémarrer imonc par un double clic sur l'icône de celui-ci.
    \item /nostartcheck Cela coupe le contrôle d'imonc, s'il est en fonction.
      c'est uniquement nécessaire si vous avez plusieurs routeurs fli4l différents
      à surveiller dans votre Réseau. Si des fonctions supplémentaires étaient
      connectées comme syslog ou e-mail ils resteront désactivées.
  \end{itemize}

  Utilisation (enregistrement de lien)~:

\begin{example}
\begin{verbatim}
X:\...imonc.exe [/Server:Nom d'Hôte] [/Password:Mot de passe] [/iport:Numéro port]
            [/log] [/tport:Numéro port] [/rc:"Commande"]
\end{verbatim}
\end{example}

  Exemple d'enregistrement avec une adresse-IP~:

\begin{example}
\begin{verbatim}
        C:\wintools\imonc /Server:192.168.6.4
\end{verbatim}
\end{example}

  Ou avec le nom et le Mot de Passe~:

\begin{example}
\begin{verbatim}
        C:\wintools\imonc /S:fli4l /P:secret
\end{verbatim}
\end{example}

  Ou avec le nom, le Mot de Passe et une commande au routeur~:

\begin{example}
\begin{verbatim}
        C:\wintools\imonc /S:fli4l /P:secret /rc:"dialmode manual"
\end{verbatim}
\end{example}

  \subsection{Concernant l'aperçu de imonc}

  Imonc client Windows interroge imond pour avoir les informations sur les
  connexions Internet existantes, il les affichent dans un tableau. Sur cette
  page il y a aussi le statut général du routeur, l'heure, la date, le bouton
  synchronisation, etc. Voici les descriptions de ces fenêtres~:

  \begin{tabular}{lp{9cm}}
    Statut             &Calling/Online/Offline (appel/en ligne/raccrocher)\\
    Nom                &Le numéro de Tél ou le Nom du circuit\\
    Direction          &On voie si c'est une connexion entrante ou sortante\\
    IP                 &Adresse IP qui à été assignée\\
    I/Octets           &Octets Entrants\\
    O/Octets           &Octets Sortants\\
    T/enligne          &Temps en ligne\\
    T/Total            &Temps total en ligne\\
    Prix/Unit          &Prix de l'unité par connexion\\
    Prix               &Prix total de la connexion\\
  \end{tabular}

  \medskip

  Les données seront actualisées toutes les deux secondes. (Maintenant) cette
  intervalle peut être changé. Dans le menu on est en mesure de voir le
  canal sur lequel le routeur est en ligne en temps réel. Copiez l'Adresse IP
  réelle dans le presse-papier et installez le canal indiqué explicitement.
  Ceci peut être intéressant s'il y a plusieurs connexions différents par ex.
  une pour naviguer sur Internet et l'autre connectée à votre entreprise,
  de cette façon vous pouvez débrancher l'une ou l'autre connexion.

  En plus si vous avez activé telmond sur votre routeur fli4l, imonc sera en
  mesure d'afficher les informations sur les appels téléphoniques entrants
  (le nom et le numéro de Tél du correspondant). Le dernier appel téléphonique
  reçu sera vu au-dessus des boutons de commande. Un protocole des appels
  téléphoniques entrants peut être vu en utilisant les pages d'appels.

  Les six boutons mentionnés ci-dessous vous permettront de choisir les commandes
  suivantes~:

  \begin{tabular}{clp{9cm}}
    Bouton & Description      & Fonction\\
    1& Connecter/Raccrocher   & Connecter ou raccrocher la ligne\\
    2& Ajout Canal/Supp Canal & Ajoute ou supprime un canal, cette caractéristique
                                n'est disponible que dans le Mode Admin\\
    3& Redémarrer             & Redémarre fli4l!\\
    4& Éteindre               & Arrête fli4l proprement et met le routeur hors tension\\
    5& Arrêter                & Arrête fli4l proprement, pour éteindre le routeur 
                                en toute sécurité\\
    6& Sortir                 & Sort du programme client imonc\\
  \end{tabular}

  \medskip

  \noindent Les cinq premières commandes en mode Utilisateur peuvent être activées
  ou désactivées dans le fichier de configuration /config/base.txt pour le
  routeur fli4l. En mode administrateur toutes les commandes sont toujours
  activées.
  Le choix de la commande Dialmode modifie le comportement du routeur~:

  \begin{tabular}{lp{9cm}}
    Auto    & Le routeur établira automatiquement une connexion
              Internet s'il y a une demande dans réseau local.\\
    Manuel  & L'utilisateur doit établir la connexion manuellement.\\
    Couper  & Il n'y a aucune connexion possible, ni manuellement ni
              automatiquement. La selection du bouton de "connexion" est
              désactivée.\\
  \end{tabular}

  \noindent La volonté de fli4l par défaut c'est d'établir automatiquement une
  connexion Internet sur une demande de requéte Internet par n'importe quel
  Hôte du réseau local. En principe on ne doit jamais modifier la commande pour
  se connecter \ldots

  Il y a également la possibilité de changer manuellement le Circuit-Défaut-Route,
  c.à d. commuter marche/arrêt ou automatique, c'est pourquoi la liste de sélection
  de "Default route" (choix du FAI) est prévu dans la version de Windows d'imonc.
  En outre, on peut maintenant configurer directement dans imonc l'heure de déconnexion.
  Utiliser le Bouton "config" en dessous de Défaut-Route ici la configuration de
  tous les circuits pour le routeur sont indiqués. La valeur de la variable
  Hup-timeout peut être éditée directement dans le fichier isdn.txt du Circuit ISDN
  (ne fonctionne pas pour le moment avec la DSL).

  Un aperçu de LCR-Routing se trouve sur la page Admin/Plage Horaire. Là,
  vous pouvez voir, le Circuit qui sera démarré automatiquement.

  \subsection{Paramètres de configuration}

  On peut accéder à la configuration par le bouton "config" dans la barre d'état.
  La fenêtre qui s'ouvre est divisée en deux, dans le tableau de gauche vous
  avez les répertoires et sous-répertoires, dans celui de droite la configuration
  de imonc. Voici les répertoires en détail~:

  \begin{itemize}
  \item Répertoire général~:
    \begin{itemize}
    \item Synchroniser tous les~: on ajuste ici le nombre de rafraîchissement
      en seconde de la page d'accueil.
    \item Synchroniser au démarrage~: synchronise l'heure et la date du routeur
      avec le client au démarrage. on peut activer cette fonction manuellement
      avec le bouton "Synchroniser" sur la page d'accueil. 
    \item Réduire au démarrage~: au démarrage le programme sera réduit en icône.
      Vous verrez seulement l'icône à côté de l'heure.
    \item Lancer imonc au démarrage Windows~: ici le client imonc démarre
      automatiquement aprés le démarrage de Windows. on peut entrer dans la
      fenêtre "paramètre" des commandes supplémentaires.
    \item Voir l'actualité de fli4l.de~: ici on peut recevoir les (News) du site
      fli4l.de chargées automatiquement par imonc, les titres sont alors montrés
      dans la fenêtre "Nouvelles" et qui pourront être lus.
    \item Appel du fichier log~: on indique ici le nom du fichier pour
      enregistrer la liste des appels locaux.
    \item Attendre la réponse du routeur~: temps d'attente d'une réponse du
      routeur en seconde, avant que la connexion soit perdue.
    \item Langue~: on choisit ici la langue pour imonc.
    \item Confirmer les commandes du routeur~: si la case est cochée, toutes
      les commandes envoyées au routeur demande une confirmation,
      ex. redémarrage, déconnexion etc \ldots
    \item Arrêt même avec trafic~: si aucune réponse n'aboutit, la connexion
      s'arrête même si il y a toujours du trafic sur cette connexion.
    \item Reconnexion automatique au routeur~: une reconnexion du routeur est
      faite automatiquement, si une coupure de la connexion a eu lieu,
      (p. ex. un redémarrage du routeur).
    \item Reduire la fenêtre système~: si activée, en cliquant sur le bouton
      "Sortir" imonc se réduit en icône vers la barre des taches à côté de
      l'heure, au lieu de s'arrêter.
    \end{itemize}

  \item Sous-répertoire proxy~: ici on enregistre le proxy pour les demandes http.
    Celui-ci est utilisé à présent pour l'actualisation des fenêtres, time-table
    et news.
    \begin{itemize}
    \item Active le proxy pour le protocole http~: ici on active Proxy
          \begin{itemize}
            \item Adresse~: ici l'adresse du serveur proxy
            \item Port~: ici le numéro de port du serveur proxy (defaut: 8080)
          \end{itemize}
    \end{itemize}

  \item Sous-répertoire icône~: ici on peut personnaliser les couleurs des icônes.
    Dans l'avenir on pourra choisir les couleurs de fond de l'icône pour dialmode
    (mode de connexion) qui sera placé dans la barre de tache.

  \item Répertoire d'appel, le réglage de la position de la fenêtre avis d'appel
    sur l'écran, sera stockée et sauvegardée dans la base de Registre. Vous pouvez
    déplacer la fenêtre à l'endroit de votre choix. Après ce réglage, la fenêtre
    apparaitra exactement cet endroit à chaque fois.
    \begin{itemize}
      \item Mise à jour~: on peut choisir ici, comment imonc reçois les
        informations des nouveaux appels tél, Il y a trois possibilités différentes.
        La premier consiste à interroger régulièrement de service telmond sur le
        routeur. Une autre possibilité consiste à interroger les annonces de
        Syslog, cette variante est la préférer~-- On doit Naturellement activer
        Syslog dans le client imonc. Imonc doit être connecté à une direction approprié,
        la troisième possibilité proposé est d'utiliser le paquetage Capi2Text pour
        la signalisation d'appel.
      \item Effacer premier zéro~: parfois devant le numéro de téléphonique est
        placé un zéro supplémentaire. Celui-ci peut être supprimé avec cette option.
      \item Indicatif régional~: la présélection personnelle du numéro de tél
        peut être écrite ici. Quand un appel arrive avec la même présélection.
        La présélection ne sera pas visible.
      \item Annuaire~: ici on indique le fichier dans lequel l'annuaire
        téléphonique local sera sauvegardé pour les numéros de téléphones.
        Si le fichier n'existe pas, il est automatiquement installé.
      \item Fichier log~: on indique ici le nom du fichier, utile pour enregistrer
        la liste des appels sur l'ordinateur local. Ce paramètre est visible
        uniquement si la variable \var{TELMOND\_\-LOG} être sur `yes', c'est
        également valable pour la liste des appels réelle.
      \item Recherche externe~: un programme peut être indiqué dans cette fenêtre,
        que l'on appelle si un numéro de téléphone ne peut pas être résolu au moyen
        de l'annuaire téléphonique local. Des infos plus précises devraient être
        jointes aux programmes correspondants. Il y a jusqu'à présent un CD
        d'annuaire téléphonique de Marcel Wappler KlickTel ainsi qu'un lien vers
        une base de données.
    \end{itemize}

  \item Sous-Répertoire des appels tél~:
    ces options sont destinées, à détailler des instructions des appels téléphoniques
    et de les afficher, voir les illustrations ci-dessous.
    \begin{itemize}
      \item Notification d'appel actif~: détermine si des appels doivent être signalés.
      \item Indication des notifications d'appels~: lors d'un appels tél une fenêtre d'
        apparait, elle détaille les Infos suivantes~: l'appel MSN, le numéro de tél du
        correspondant et la date/heure de l'appel. Pour cela il est nécessaire que la
        variable \var{OPT\_\-TELMOND} soit placée sur `yes' dans le fichier
        config/isdn.txt
        \begin{itemize}
          \item Ne pas enregister les numéros non transmis~: les appels ne doivent pas
            être écrit dans la fenêtre d'appel, si aucun numéro de Tél n'a été transféré.
          \item Temps d'affichage~: cette indication influe sur la durée de fermeture de
            la fenêtre avis d'appel, la fenêtre doit rester ouverte un certain temps.
            Si on indique "0" la fenêtre ne se fermera pas automatiquement.
          \item Fontsize (ou police)~: ici on choisit la taille des caractères pour la
            fenêtre. Celle-ci affecte la taille de la fenêtre, puisque la taille de la
            fenêtre sera calculée par rapport à la taille du message.
          \item Couleur~: ici on choisit la couleur des textes dans la fenêtre d'appel.
            J'emploie le rouge pour l'identification des messages.
      \end{itemize}
    \end{itemize}

  \item Sous-répertoire annuaire~: la fenêtre contient l'annuaire téléphonique
    qui est utilisé pour la définition des numéros de téléphones des appels
    entrants et aussi si vous possédez un MSN. Cette fenêtre apparait même si la
    variable \var{TELMOND\_\-LOG} est sur `no' parce que cette fenêtre est utilisée
    aussi pour le dernier appel entrant vu dans la fenêtre principale. On peut choisir
    un fichier qui sera placé sur le routeur.

    Construction d'un appel entrant~:

\begin{example}
\begin{verbatim}
  # Format:
  # Telefonnummer=anzuzeigender Name[, Wavefilename]
  # 0241123456789=Testuser
  00=unbekannt
  508402=Fax
  0241606*=Elsa AG Aachen
\end{verbatim}
\end{example}

    Les trois premières lignes sont des commantaires. La quatrième
    ligne est créée si aucun numéro n'est transmis, "unbekannt" (ou inconnu)
    sera affiché. La cinquième ligne indique le numéro de tél "508402" et le
    Nom "Fax" , dans tous les cas le format sera toujours le même, Numéro de
    Tél=Nom. La sixième ligne détermine l'ensemble des numéros de Tél, pour
    toutes appel ex. 0241606 le Nom sera affiché. Souvenez-vous que le dernier
    numéro d'appel du correspondant est indiqué sur la première fenêtre
    principale. Optionnel, un fichier son peut être défini et sera joué lors
    d'un appel Tél.

    Dés la Version 1.5.2, il est possible d'installer un annuaire Téléphonique
    sur le routeur sous la forme d'un fichier il sera enregistré et synchronisé
    dans (/etc/phonebook). Si un même numéro de téléphone avec un Nom différent
    sont enregistrés dans l'annuaire Du routeur et dans l'annuaire de imonc,
    il sera demandé à l'utilisateur qu'elle est l'entrée valide. les appels ne
    sont pas juste recopiés mais sont enregistrés sur les deux annuaires. La
    synchronisation du fichier d'annuaire est faite dans la mémoire RAM, cela
    veut dire, lorsque l'on reboot (redémarre) le routeur, le fichier sera perdu.

  \item Répertoire son, les fichiers son qui seront installés ici seront joués,
    si l'événement indiqué se produit.
    \begin{itemize}
      \item Courriel~: le fichier son sera joué, si un nouveau courriel se trouve
        sur votre Serveur POP3. 
      \item Erreur courriel~: le fichier son sera joué, si une erreur se produit
        lors de la réception du Courriel.
      \item Connexion perdu~: le fichier son sera joué, si la connexion avec
        le routeur est perdue (ex. redémarrage du routeur). Si l'option
        "reconnexion automatique au routeur" n'est pas activée, un messagebox
        s'ouvrira pour demander une nouvelle connexion au routeur.
      \item Connexion~: le fichier son sera joué, si le routeur établit
        une connexion Internet.
      \item Déconnexion~: le fichier son sera joué, lorsque le routeur déactive
        la connexion Internet.
      \item Avis appel~: le fichier son sera joué, si l'annonce des appels est
        activée et si un nouvel appel est reçu.
      \item Annonce de fax~: le fichier son sera joué, après la réception de
        nouveaux fax.
    \end{itemize}

  \item Répertoire courriel
    \begin{itemize}
      \item Comptes~: cette fenêtre sert à configurer les comptes POP3.
      \item Activer le contrôle courriel~: si vous avez un compte courriel il
        recherchera automatiquement les nouveaux courriels.
        \begin{itemize}
          \item Vérifier x/Min~: cette option définit un intervalle temps entre
            chaque contrôle Courriel sur le compte. Attention~: en définissant un 
            intervalle trop court, le routeur peut rester constamment en ligne!
            Ceci se produit lorsque l'intervalle est plus court que "Delai attente"
            du circuit utilisé.
          \item Temps d'attente x/Sec~: temps d'attente d'une réponse du Serveur POP3
            avant l'arrêt de celui-ci, si la valeur est à "0" cela signifie qu'aucun
            TimeOut (temps d'attente) n'est installé.
          \item routeur déconnecté~: cette option permet au routeur de se connecter
            automatiquement pour rechercher les nouveaux courriels sur le Serveur POP3.
            Aprés le téléchargement des courriels le routeur se déconnecte. pour pouvoir
            utiliser ce dispositif on doit mettre Dialmode sur 'auto'. Attention~:
            cela occasionne des frais supplémentaires de connexion si aucun tarif
            unitaire est utilisé~!
          \item Circuit à utiliser~: cette option définit le circuit qui sera utilisé
            pour la connexion aux courriels.
          \item Rester en ligne aprés contrôle~: la déconnexion doit être faire
            manuellement ou l'arrêt doit être réalisé automatiquement par l'option
            Délai attente.
          \item Charger en-têtes des courriels~: télécharger les en-têtes des courriels ou
            uniquement le nombre de courriel disponible? Cette option doit être activée
            pour supprimer les courriels directement sur le serveur POP3.
         \item M'avertir de nouveaux courriels~: faut-il un message sonore et une icône
            dans la barre de tâche pour m'annoncer de nouveaux courriels.
         \item Exécuter le programme de messagerie~: démarrer automatiquement le
            programme de messagerie pour lire les nouveaux courriels disponibles.
         \item Programme~: indiquer ici le programme de messagerie.
         \item Paramètre~: entrer les paramètres additionnels qui seront transférés
            au démarrage du programme de messagerie. Si Outlook est utilisé comme
            programme courriel (pas Outlook Express!) vous pouvez entrer comme paramètre
            "/recycle" empêche de lancer Outlook dans une nouvelle fenêtre s'il est
            déjà ouvert.
      \end{itemize}
    \end{itemize}

  \item Repertoire Admin
    \begin{itemize}
      \item Mot de passe Root~: ici on entre le mot de passe du routeur qui est
        dans le fichier (/config/base.txt dans la variable \verb+PASSWORD+) pour pouvoir
        par exemple configurer Portforwarding sur votre ordinateur et l'envoyer
        sur le routeur.
      \item Voir les fichiers sur le routeur~: tous les fichiers log (ou journal)
        qui se trouvent sur le routeur sont à indiquer ici, ils peuvent être lus,
        avec un simple clic de la souris dans la page Admin/fichier, ainsi on peut
        afficher les fichiers log du routeur directement dans imonc.
      \item Fichier de configuration~: ici on peut choisir, si tous les fichiers
        seront ouverts avec le programme éditeur de texte ou uniquement les fichiers
        *.txt pour étudier et travailler dessus. On peut égalment ouvrir un ensemble
        de fichiers.
      \item DynEisfairLog~: si vous avez créé un compte sur DynEisfair, vous pouvez
        enregistrer ici les données d'accés et de voir avec le fichier Log les mises
        à jours des fonctions sur la page Admin/DynEisfairLog.
    \end{itemize}

  \item Répertoire démarrage auto, sert à configurer une liste de programmes qui
    sera lancée automatiquement. Celle-ci est exportée après une connexion réussie
    si l'option "Activer la liste des programmes" est cochée.
    \begin{itemize}
      \item Programme~: tous les programmes installés ici seront lancés
        automatiquement, si le routeur est connecté et que La Liste des
        programmes est cochée.
      \item Activer la liste des programmes~: la liste doit-elle être activé pour
        exécution des programmes aprés une connexion réussie~?
    \end{itemize}

  \item Répertoire trafic du réseau, est utilisé pour la configuration
    (personnalisée) de la fenêtre de Info trafic. Un utilisateur m'a averti qu'il
    y avait quelques erreurs sur la définition des données avec des versions
    anciennes de DirectX.
    \begin{itemize}
      \item Voir les informations sur le trafic~: voulez-vous afficher une utilisation
        graphique des canaux dans une fenêtre à part? Dans le menu contextuel vous
        pouvez choisir l'attribut StayOnTop, cette option provoque l'affichage de la
        fenêtre sur toutes les autres fenêtres. Cette option sera enregistrée dans
        la base de registre et sera en service aprés un redémarrage du programme.
      \item Voir les titres~: doit-on monter la barre de titre dans la fenêtre
        Traffic-Info? Cette fenêtre montrera les informations des circuits utilisés
        par le routeur.
        \begin{itemize}
          \item Voir l'utilisation CPU~: montrer l'utilisation du CPU dans la barre
            de titre?
          \item Voir le temps de communication~: le temps en ligne du canal doit-il
            aussi être indiqué dans la barre de titre~?
        \end{itemize}
      \item Fenêtre semi-transparente~: la fenêtre doit-elle être représentée en
        transparence? Cette fonction n'est disponible que sous
        Windows 2000 et Windows XP.
      \item Couleur~: les couleurs sont définies ici pour la fenêtre Traffic-Info.
        Maintenant le canal DSL et le premier canal ISND utiliseront les mêmes
        couleurs. 
      \item Limite~: entrer les valeurs maximales des taux de transmission xDSL~--
        pour T-Online~: Débit Montant (upload) 128 Ko/s et Débit Descendant
       (download) 1024 Ko/s.
    \end{itemize}

  \item Répertoire Syslog, est utilisé pour la configuration de l'affichage
    des messages Syslog.
    \begin{itemize}
      \item Activer le client Syslog~: montrez les messages Syslog dans imonc~?
        Cette option doit être arrêtée, si vous utilisez un autre client Syslog
        externe, par exemple le client Kiwi's Syslog.
      \item Indiquer les messages Syslog~: monter les messages Syslog avec un
        niveau de prioritaire~? Vous pouvez indiquer ici les niveaux prioritaires
        des messages Syslog, par defaut le message débug est coché, vous pouvez
        cocher le niveau selon vos besoins.
      \item Enregistrer les messages Syslog~: les messages lus doivent-ils être
        sauvegardés? Dans la fenêtre on peut choisir les messages que l'on veut
        sauvegarder. On peut insérer des caractères supplémentaires avec nom de
        fichier à sauvegarder, les voici~:
        \begin{description}
          \item[\%y]~-- On l'ajoute pour avoir l'année actuel
          \item[\%m]~-- On l'ajoute pour avoir le mois actuel
          \item[\%d]~-- On l'ajoute pour avoir le jour actuel
        \end{description}
      \item Voir le nom des ports~: doit on afficher la description du port au
        lieu du numéro de port~?
      \item Voir les messages pare-feu~: ici, on indique les messages du firewall
        (ou pare-feu), il seront aussi indiqués en mode utilisateur.
    \end{itemize}

  \item Répertoire fax, sert à configurer les fax (ou télécopie) dans imonc. Pour
    que ce dossier soit visible vous devez installer sur le routeur le paquetage
    mgetty et/ou faxrcv, (vous pouvez les trouver sur le site de fli4l).
    \begin{itemize}
      \item Fichier Log pour fax~: ici on peut enregister les fax reçus sous forme
        de fichier dans un dossier de l'ordinateur.
      \item Répertoire local des fax~: configurer le répertoire pour stocker les
        fax reçus, avant de les consulter.
      \item Actualisation~: il y deux possibilités, lorsque imonc reçoit un
        nouveau fax. Soit c'est imonc Syslog qui reçoit les fax (naturellement
        le client imonc-Syslog doit être activé), soit imonc regarde régulièrement
        le fichier log. La première variante est la meilleure. Si vous utilisé la
        deuxième variante, vous pouvez indiquer combien de fois la page d'aperçu de
        fax doit être actualisée. Il faut faire attention cette valeur n'est pas
        une indication en seconde, mais c'est une indication en multiple,
        en général c'est une intervalle d'actualisation.
    \end{itemize}

  \item Répertoire tableau, sert à ajuster les colonnes des (tableaux) dans imonc
    par rapport à vos besoins. D'une part, pour chaque tableau on peut régler les
    en-têtes les colonnes qui doivent être affichées, d'autre part pour chaque
    service de communication il y a un tableau différent, appel Tél, fax, on peut
    régler le moment ou les Infos doivent être affichées.
  \end{itemize}

  \subsection{Concernant les appels tél}

  L'annuaire Téléphonique sera uniquement vu, que si la variable \var{TELMOND\_\-LOG}
  est placée sur 'yes', sinon aucun journal d'appels ne sera conservée. Dans cet
  annuaire sera enregistré tous les appels téléphoniques qui seront entrées sur
  le routeur. Vous pouvez commuter entre les appels enregistrés sur le PC local
  et les appels enregistrés sur le routeur, vous pouvez effacer le fichier sur
  le routeur avec le bouton-Réinitialisé.

  Dans l'annuaire téléphonique, vous pouvez cliquer avec le bouton droit
  de la souris sur le Numéro de Tél pour attribuer un Nom au numéro, comme
  cela le Nom apparaitra à la place du Numéro de Tél.

  \subsection{Concernant les connexions}

  L'affichage des connexions internet par le routeur dans une page est utilisé
  de puis la Version 1.4, elle donnera une bonne vue d'ensemble du comportement
  du routeur connecté à Internet. Pour voir cette page la variable \var{IMOND\_\-LOG}
  doit être placée sur `yes' dans le fichier /config/base.txt.

  De la même façon que l'annuaire-Tél, vous pouvez commuter les connexions
  enregistrées localement et celles enregistrées sur le routeur. Vous pouvez
  aussi effacer le fichier des données sur le routeur en cliquant sur
  le bouton-rafraîchir.

  Affichage du tableau de connexions.

  \begin{itemize}
  \item Nom du FAI
  \item Date et heure de départ
  \item Date et heure de fin
  \item Temps en ligne
  \item Prix de l'unité
  \item Prix Total
  \item Réception du signal
  \item Émission du signal
  \end{itemize}

  \subsection{Concernant les FAX}

  Pour que soit affichée la page FAX il faut installer le paquetage
  \var{OPT\_\-MGETTY} par M. Michael Heimbach sur le routeur ou
  \var{OPT\_\-MGETTY} par M. Felix Eckhofer. Sur le site Internet de
  fli4l, à la page d'accueil vous avez un raccourci pour les
  paquetages-OPT. Dans cette fenêtre tous les FAX reçus seront enregistrés,
  le menu contextuel offre plusieurs options de configuration qui seront
  uniquement disponibles en mode Administrateur~:

  \begin{itemize}
  \item Concernant les Fax reçus, il faut correctement configurer le chemin
  d'accès pour fli4l dans répertoire Admin/Remoteupdate, pour que les FAX
  reçus sur le routeur soient enregistrés et compressés avec le programme
  gzip, qui se trouve dans le paquetage fli4l, le programme gzip.exe et le
  fichier win32gnu.dll peuvent aussi être copié dans le répertoire imonc.
  Si gzip.exe n'est pas trouvé dans l'un des deux emplacements, et si le
  routeur est connecté à Internet, il recherchera le programme sur internet
  (directement sur le site CGIs).
  \item Supprimer un FAX reçu. Cela signifie que le FAX sera supprimé sur
    votre PC local et sur le routeur (le fichier FAX réel, et aussi dans
    le fichier log).
  \item Supprimer tous les FAX présents sur routeur. Ici tous les FAX
    sur le routeur dans le fichier log seront effacés. Les FAX ne seront
    pas effacés du fichier log de votre PC local.
  \end{itemize}

  Comme dans la page des appels Tél, vous pouvez commuter entre les Fax
  enregister localement et les Fax enregistrés sur le routeur.

  \subsection{Concernant les courriels}

  Cette page apparait, si dans le répertoire Config courriel il y a au moins
  un compte \mbox{courriel} avec serveur POP3 qui a été configuré et activé.

  Description de la page \mbox{courriel}. Maintenant on a intégré dans cette section
  le contrôleur de \mbox{courriel}. Si l'option "le routeur n'est pas en ligne" dans
  config \mbox{courriel} n'est pas activée, le contrôleur de Mail vérifiera tous les
  comptes \mbox{courriel}, ensuite il utilisera l'intervalle Temps pour vérifier le
  Serveur (le routeur doit être connecté, il utilise le circuit
  présélectionné). Si le routeur n'est pas connecté, activer l'option "le
  routeur n'est pas en ligne" et indiquer le circuit à utiliser, il établira
  une connexion en utilisant le circuit choisi et téléchargera les \mbox{courriels} de
  tous les comptes \mbox{courriels} configurés ensuite il fermera la connexion. Pour
  utiliser cette option vous devrez placer Dialmode sur "auto".

  Si des \mbox{courriels} sont disponible sur le serveur POP3, le programme
  \mbox{courriel} client sera démarré automatiquement ou une icône apparaitra
  prés de l'heure dans la barre de tache, il indiquera le nombre de \mbox{courriel}
  sur le serveur. En double cliquant dessus l'ensemble du \mbox{courriel} client
  sera lancés. Si une erreur se produit sur un compte \mbox{courriel}, d'une part,
  une note sur l'erreur sera écrit dans le dossier Histoire du \mbox{courriel},
  d'autre part, l'icone du \mbox{courriel} affichera dans le coin supérieur droit
  une couleur rouge.

  Dans la fenêtre \mbox{courriel}, on peut effacer directement les \mbox{courriels} sur le
  Serveur sans les avoir préalablement téléchargés. Il faut avoir téléchargé
  les en-têtes des courriels, vous devez marquer les cellules à supprimer, puis
  en cliquant sur le bouton droit de la souris le menu contextuel s'ouvre,
  et cliquer sur Delete MailMessage.

  \subsection{Admin}

  Cette partie est uniquement disponible si imonc est démarré en mode Admin.

  Premier point, cette page offre une vue d'ensemble des circuits utilisés,
  ~--les fournisseurs d'accès Internet~-- qui ont été choisis automatiquement
  par le routeur (par l'intermédiaire du LC Routing). En double cliquant sur un
  fournisseur d'accès dans l'aperçu fournisseur d'accès vous obtiendrez
  l'affichage des définitions des plages horaires pour ce fournisseur qui à été
  défini dans /config/base.txt.

  Deuxième point, cette page donne l'occasion d'installer les mises à jour à
  distance sur le routeur. Vous pouvez choisir l'un ou les cinq programmes
  (Kernel, fichier système, fichier OPT, rc.cfg et syslinux.cfg) qui seront
  copiés sur le routeur. Pour pouvoir faire la mise à jour à distance, vous
  devrez indiquer le répertoire de fli4l dans imonc et les fichiers nécessaires
  à copier. En plus, vous devez écrire le sous-répertoire des fichiers de
  configuration (par défaut: /config/*.txt) pour devez construire tous les
  fichiers systèmes de fli4l. Il est conseillé de Rebooter (ou redémarrer) après
  avoir envoyé les fichiers système sur le routeur pour que les modifications
  soient prises en compte. Si un mot de passe est demandé par le routeur, il
  est inscrit dans la variable PASSWORD dans /config/base.txt.

  Troisième point, cette page traite des contraintes du Port Forwarding,
  un port est connecté exactement et uniquement à un ordinateur client.
  Maintenant il est possible d'éditer et de configurer Port-Forwarding du
  routeur. aprés les modifications des ports ils seront activées, la
  connexion doit être active. Puisque les fichiers sont enregistés dans la
  mémoire virtuelle (Ramdisk), tous les changements seront uniquement
  sauvegardés jusqu'au prochain redémarrage du routeur. Pour sauvegarder des
  changements de manière permanente vous devez changer des Port Forward dans
  le fichier /config/base.txt et installer le nouveau fichier-OPT sur le routeur.

  Quatrième point, dans la fenêtre Admin, puis~-- fichier~-- vous pouvez utilisée
  et voir la configuration des fichiers Log du routeur, en cliquant
  simplement sur la souris. La liste de choix peut être configurée dans le
  dossier config-Admin d'imonc "voir les fichiers sur le routeur". Ensuite,
  vous pouvez simplement choisir les fichiers qui sont indiqués dans le menu
  déroulant.

  Cinquième point, cette fenêtre montre DynEisfair log, elle apparaît
  uniquement si dans le répertoire de configuration Config-Admin les
  enregistrements les données pour un accès à un compte DynEisfair a
  été configuré (pour simuler une IP fixe, lorsque l'on a une IP dynamique).
  Si cela est fait, le fichier log des services sera indiqué dans cette fenêtre.

  Dernier point, fenêtre hôtes, tous les ordinateurs enregistrés dans le fichier
  /etc/hosts sont indiqués ici, à l'avenir on essaiera de configurer chacun des
  ordinateurs enregistrés pour pouvoir les "pinger" (ou interroger)
  individuellement, ainsi on pourra rapidement vérifier l'ordinateur qui est 
  connecté au réseau local.

  \subsection{Concernant les erreurs syslog et firewall}

  Les pages erreur, syslog et Firewall (pare-feu), s'affiche uniquement
  s'il y a des événements enregistrés dans ce fichier, en plus il faut
  être en mode Admin pour que les pages soit affichées.

  Toutes les erreurs spécifiques à imonc/imond seront enregistrées dans la 
  fenêtre erreur. Si vous avez des problèmes vous pouvez aller vérifier dans
  cette liste pour voir les causes des erreurs que vous avez rencontrées.

  Dans la fenêtre Syslog les messages de syslog seront affichés, excepté des
  messages du pare-feu. Ceux-ci sont affichés dans une page indépendante
  (voir ci-dessous). Pour que la page syslog  fonctionne vous devrez placer
  la variable \var{OPT\_\-SYSLOGD} sur "yes" dans le fichier de configuration
  /config/base.txt En plus dans la variable \var{SYSLOGD\_\-DEST} on doit
  placer l'adresse IP du client qui bien entendu utilise imonc (par exemple~:
  \var{SYSLOGD\_\-DEST}='@ 100.100.100.100~-- adresse IP de votre client!).
  Il n'y aura pas que les messages syslog qui seront affichés, mais aussi
  la date, l'heure, l'IP et le niveau de priorité.

  Des messages du Firewall (pare-feu) seront affichés dans une page
  indépendante. Pour que la page fonctionne, vous devez placer la variable
  \var{OPT\_\-KLOGD} sur 'yes' dans le fichier de configuration /config/base.txt.

  \subsection{Concernant les News}

  Cette page News (ou d'actualité), doit être activée dans le répertoire
  config-Imonc. Les News mentionnés sur la page accueil du site fli4l, seront
  visibles directement dans Imonc à la page accueil. On peut directement aller
  sur le site http://www.fli4l.de/german/news.xml avec le bouton-plus. Vous
  avez une fenêtre à côté des titres des News, qui indique les 10 derniers
  paquetages-OPT enregistrés sur le site
  http://www.fli4l.de/german/imonc\_opt\_show.php, en double cliquant sur le
  paquetage choisi, vous allez directement sur le site. En plus, il est
  indiqué dans la barre de statut en bas de Imonc, les titres des News.


  \marklabel{sec:imonc}{
    \section{Unix/Linux-Client imonc}}

  Für Linux gibt es mittlerweile 2 Versionen: eine textbasierte
  (imonc) und eine mit graphischer Oberfläche (ximonc). Den Source zu
  ximonc findet man im Verzeichnis src. Die Dokumentation für ximonc
  wird erst in der 1.5-Final-Version zur Verfügung stehen. Ein
  erfahrener Linux-User sollte aber mit den Sources keine Probleme
  haben.

  Beschränken wir uns daher hier zunächst auf die textbasierte Version
  von imonc: Dieses ist ein curses-basiertes Programm, hat also keine
  graphische Oberfläche. Der Source liegt im Verzeichnis unix.

  Installation:

\begin{example}
\begin{verbatim}
        cd unix
        make install
\end{verbatim}
\end{example}

  imonc wird dabei in /usr/local/bin installiert.

  Aufruf:

\begin{example}
\begin{verbatim}
        imonc hostname
\end{verbatim}
\end{example}

  Dabei ist als hostname der Name oder die IP-Adresse des
  fli4l-Routers anzugeben, also z.B.

\begin{example}
\begin{verbatim}
        imonc fli4l
\end{verbatim}
\end{example}


  imonc zeigt folgende Informationen:

  \begin{itemize}
  \item Datum/Uhrzeit des fli4l-Routers

  \item Momentan eingestellte Route

  \item Default-Route-Circuits

  \item ISDN-Kanäle
    \begin{description}
    \item[Status]:         Calling/Online/Offline
    \item[Name]:           Telefonnummer des Gegners oder Circuit-Name
    \item[Time]:           Online-Zeit
    \item[Charge-Time]:    Online-Zeit unter Berücksichtigung des Zeittaktes
    \item[Charge]:         Berechnete Kosten
    \end{description}
  \end{itemize}

  Mögliche Kommandos sind:

  \begin{tabular}{lll}
    Nr   &Befehl             &Bedeutung\\
    0   &quit                &Programm beenden\\
    1   &enable              &Aktivieren\\
    2   &disable             &Deaktivieren\\
    3   &dial                &Wählen\\
    4   &hangup              &Einhängen\\
    5   &reboot              &Neu booten\\
    6   &timetable           &Zeittabelle ausgeben\\
    7   &dflt route          &Neuen Default-Route-Circuit bestimmen\\
    8   &add channel         &2. Kanal hinzuschalten\\
    9   &rem channel         &2. Kanal deaktivieren\\
  \end{tabular}

  \medskip

  \noindent Zu den Kommandos im Einzelnen:

  \begin{description}
  \item[0~-- quit] Die Verbindung zum imond-Server wird abgebaut und
    das Programm beendet.


  \item[1~-- enable] Alle Cirucits werden auf dialmode ``auto''
    gestellt. Das ist auch der Default-Zustand von fli4l nach dem
    Booten. Das heisst, dass fli4l bei einem Verbindungsaufbauwunsch
    eines Rechners im Netz automatisch rauswählt.


  \item[2~-- disable] Alle Circuits werden auf dialmode ``off''
    gestellt. Damit ist fli4l so gut wie tot, bis er mit dem
    Enable-Kommando wieder geweckt wird.


  \item[3~-- dial] Manuelle Wahl auf dem Default-Route-Circuit. Ist
    eher für Testzwecke gedacht, da fli4l normalerweise automatisch
    wählt.


  \item[4~-- hangup] Manuelles Einhängen: damit kann man dem
    automatischen Einhängen von fli4l zuvorkommen.


  \item[5~-- reboot] fli4l wird neu gebootet. Ziemlich überflüssiges
    Kommando \ldots


  \item[6~-- timetable] Es wird die Zeittabelle für die
    Default-Route-Circuits ausgegeben.  Beispiel: s.o.


  \item[7~-- default route circuit] Manuelles Wechseln des
    Default-Route-Circuits. Kann z.B. dann sinnvoll sein, wenn man das
    automatische LC-Routing von fli4l für eine Weile ausser Kraft
    setzen will, da einige Provider einen Zugriff auf das eigene
    Postfach nur über den eigenen Internet-Zugang erlauben.


  \item[8~-- add channel] Hier kann der 2. ISDN-Kanal manuell
    hinzugeschaltet werden.  Voraussetzung:
      \var{ISDN\_\-CIRC\_\-x\_\-BUNDLING}
    ist `yes'.


  \item[9~-- remove channel] Abschalten des 2. ISDN-Kanals. Siehe auch
    ``add channel''.

  \end{description}

  \noindent Sonst gelten bei diesen Kommandos dieselben Bemerkungen wie für den
  Windows-imond-Client \verb+imonc.exe+.

  Noch zu bemerken ist: Ab fli4l-1.4 ist es nun auch möglich, auf dem
  fli4l-Router selbst einen ``minimalisierten'' imon-Client zu
  installieren, nämlich durch Setzen von
  \smalljump{OPTIMONC}{\var{OPT\_\-IMONC}}=`yes' im Paket
  \smalljump{sec:tools}{\var{TOOLS}}.

  Damit kann man nun auch an der fli4l-Konsole bestimmte
  Einstellungen, z.B. Routing etc. mit imonc vornehmen. Achtung:
  Dieser Mini-imonc funktioniert nur auf dem fli4l-Router selbst! Auf
  einem Linux-/Unix- Client ist immer der ``große Bruder'' unix/imonc
  zu verwenden.
