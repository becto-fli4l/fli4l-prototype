% Last Update: $Id$
\chapter{Entwickler-Dokumentation}

Dieses Kapitel führt in die Aspekte von fli4l ein, die hauptsächlich für jene
interessant sind, die mit dem Gedanken spielen, fli4l durch die Entwicklung
eigener Pakete und OPTs zu erweitern. Es wird erläutert, wie fli4l "`unter der
Haube"' arbeitet und wie man in gewisse Prozesse eingreifen kann, um die
Funktionalität von fli4l zu verändern oder zu verbessern.

\section{Allgemeine Regeln}

Damit ein neues Paket in die OPT-Datenbank auf der fli4l-Homepage
aufgenommen wird, müssen einige Regeln beachtet werden. Pakete, die
sich nicht an diese Regeln halten, können ohne Vorwarnung aus der
Datenbank entfernt werden.

  \begin{enumerate}
    \item \emph{Keine} Kopieraktionen von Seiten des Benutzers! fli4l bietet ein
    ausgefeiltes System, um die Daten der fli4l-Pakete in die
    Installationsarchive einzupacken. Alle Dateien, die auf den Router sollen,
    liegen im Verzeichnis \texttt{opt/}.

    \item Pakete richtig packen und komprimieren: Die Pakete müssen so
    aufgebaut sein, dass sie sich mühelos ins fli4l-Verzeichnis entpacken
    lassen.

    \item Die Pakete sollen sich \emph{vollständig} über die Konfigurationsdatei
    konfigurieren lassen. Ein weiteres Bearbeiten der Konfigurationsdateien
    darf nicht vom Benutzer verlangt werden. Schwierige Entscheidungen dem
    Benutzer abnehmen oder in einen erweiterten Bereich verlagern (ans Ende
    der Konfigurationsdatei mit einem dicken Hinweis, etwa: ONLY MODIFY IF YOU
    KNOW WHAT YOU DO).

    \item Noch ein Hinweis zur Konfigurationsdatei: Anhand des Namens einer
    Variablen muss sich eindeutig erkennen lassen, zu welchem OPT sie gehört.
    So gehören z.\,B. zum \var{OPT\_\-HTTPD} die Variablen \var{OPT\_\-HTTPD},
    \var{HTTPD\_\-USER\_\-N}, usw.

    \item Bitte, bitte, macht möglichst kleine Binaries (Programme)! Das
    passiert automatisch, wenn ihr das FBR aus dem \emph{src}-Paket verwendet.
    Denkt auch daran, unnötige Features zu deaktivieren. Bei
    vorkompilierten Programmen hilft u.\,U. ein
\begin{example}
\begin{verbatim}
      strip -R .comment -R .note <Dateiname>
\end{verbatim}
\end{example}
    weiter. Nichts ist frustrierender als ein Download von 2~MiB, wenn man
    nachher feststellt, dass 500~kiB gereicht hätten\ldots

    \wichtig{Bitte auf diese Art und Weise keine Kernel-Module behandeln! Sie
    werden hinterher nicht mehr funktionieren!}

    \item Prüft euer Copyright! Wenn ihr Dateien als Vorlagen benutzt, achtet
    bitte darauf, das Copyright entsprechend zu ändern. Dies gilt besonders
    für die Config-, Check- und Opt-Textdateien. Ersetzt hier das Copyright
    durch euren eigenen Namen. Bei wortwörtlich kopierter Dokumentation muss
    natürlich das Copyright des Orginal-Autors erhalten bleiben!

    \item Bitte als Archivtypen nur verbreitete, freie Formate benutzen. Dazu
    gehören:
    \begin{itemize}
      \item ZIP  (\texttt{.zip})
      \item GZIP (\texttt{.tgz} oder \texttt{.tar.gz})
    \end{itemize}
    Andere Formate wie RAR, ACE, Blackhole, LHA etc.\ bitte nicht verwenden.
    Auch Windows-Installer-Dateien (\texttt{.msi}) oder selbstextrahierende Archive und
    Installer (\texttt{.exe}) sind nicht zu benutzen.
  \end{enumerate}

\marklabel{sec:libc}{\section{Übersetzen von Programmen}}

Die für das Übersetzen von Programmen zum Einsatz auf dem fli4l-Router
erforderliche Umgebung wird im separat erhältlichen \emph{src}-Paket angeboten.
Dort wird auch dokumentiert, wie sich eigene Programme für den fli4l übersetzen
lassen.

% Last Update: $Id: dev_main_modular.tex 52055 2018-03-14 05:43:07Z kristov $

\section{Modulkonzept}

fli4l wird in Module (Pakete) aufgeteilt, u.\,a. in:
\begin{itemize}
    \item \emph{fli4l-\version}~~\(\longleftarrow\) Das Basis-Paket
    \item \emph{dns\_dhcp}
    \item \emph{dsl}
    \item \emph{isdn}
    \item \emph{sshd}
    \item und viele weitere\ldots
\end{itemize}

Mit dem Basis-Paket ist fli4l ein reiner Ethernet-Router. Für ISDN
und/oder DSL ist das Paket \emph{isdn} und/oder \emph{dsl} in dem
fli4l-Verzeichnis auszupacken. Entsprechendes gilt für die anderen Pakete.

\marklabel{mkfli4l}{\subsection{mkfli4l}}

Aus den Paketen werden in Abhängigkeit von der konkreten Konfiguration
eine Konfigurationsdatei namens \texttt{rc.cfg} und zwei Archive namens
\texttt{rootfs.img} und \texttt{opt.img} erstellt, die alle
Konfigurationsinformationen und alle benötigten Dateien enthalten. Diese Dateien
werden mit Hilfe von \texttt{mkfli4l} erzeugt, welches die einzelnen Pakete
einliest und auf Fehler in der Konfiguration prüft.

\texttt{mkfli4l} akzeptiert die in Tabelle \ref{tab:mkfli4l} angegebenen
Parameter. Fehlen sie, werden die in Klammern angegebenen Werte angenommen.
Eine vollständige Liste der Optionen (Tabelle \ref{tab:mkfli4l}) erhält man,
wenn man
\begin{verbatim}
    mkfli4l -h
\end{verbatim}
aufruft.

\begin{table}[htbp]
  \centering
  \caption{Parameter für \texttt{mkfli4l}}
  \begin{tabular}{|lp{2cm}|p{8cm}|}
    \hline
    \multicolumn{1}{|c|}{\textbf{Option}} & \multicolumn{2}{c|} \textbf{{Bedeutung}} \\
    \hline
    -c, -\,-config    & \multicolumn{2}{|p{11cm}|} { Setzen des
      Verzeichnisses, in dem \texttt{mkfli4l} die config-Dateien der Pakete
      sucht (Standard: config)} \\
    -x, -\,-check     & \multicolumn{2}{|p{11cm}|} { Setzen des
      Verzeichnisses, in dem \texttt{mkfli4l} die zum Prüfen der Pakete
      benötigten Dateien sucht (\texttt{<package>.txt}, \texttt{<package>.exp} und
      \texttt{<package>.ext}; Standard: check)} \\
    -l, -\,-log       & \multicolumn{2}{|p{11cm}|} { Setzen der Logdatei;
      \texttt{mkfli4l} protokolliert Fehlermeldungen und Warnungen in
      dieser Datei (Standard: \texttt{img/mkfli4l.log})} \\
    -p, -\,-package   & \multicolumn{2}{|p{11cm}|} { Angabe der Pakete,
      die geprüft werden sollen. Diese Option kann mehrmals angegeben
      werden, wenn man mehrere Pakete im Zusammenhang prüfen will. Bei
      Verwendung von -p wird allerdings grundsätzlich zuerst die Datei
      \texttt{<check\_dir>/base.exp} eingelesen, um die allgemeinen
      regulären Ausdrücke, die vom Basis-Paket bereitgestellt werden,
      zur Verfügung zu stellen. Diese Datei muss also existieren.} \\
    -i, -\,-info       & \multicolumn{2}{|p{11cm}|} { Gibt Auskunft über
      den Verlauf der Prüfung (welche Dateien werden gelesen, welche
      Prüfungen werden durchgeführt, welche besonderen Dinge traten
      während des Prüfprozesses auf)} \\
    -v, -\,-verbose    & \multicolumn{2}{|p{11cm}|} { Ausführlichere
      Variante von -i} \\
    -h, -\,-help       & \multicolumn{2}{|p{11cm}|} { Zeigt die Hilfe an} \\
    \html{\multirow{8}{*}{top}} \latex{\multirow{8}{*}{}}{-d, -\,-debug} &
      \multicolumn{2}{|p{11cm}|} { Erleichtert die Fehlersuche im Prüfprozess.
      Dies ist als Hilfe für Paketentwickler gedacht, die etwas genauer wissen
      möchten, wie die Prüfung des Paketes abläuft.} \\
    \cline{2-3}
    \latex{&} \multicolumn{1}{|c|}{\textbf{Debugoption}} & \multicolumn{1}{c|}{\textbf{Bedeutung}} \\
    \cline{2-3}
    \latex{&} \multicolumn{1}{|l|}{check} & show check process \\
    \latex{&} \multicolumn{1}{|l|}{zip-list} & show generation of zip list \\
    \latex{&} \multicolumn{1}{|l|}{zip-list-skipped} & show skipped files \\
    \latex{&} \multicolumn{1}{|l|}{zip-list-regexp} & show regular expressions for ziplist \\
    \latex{&} \multicolumn{1}{|l|}{opt-files} & check all files in \texttt{opt/<package>.txt} \\
    \latex{&} \multicolumn{1}{|l|}{ext-trace} & show trace of extended checks \\
    \hline
  \end{tabular}
  \label{tab:mkfli4l}
\end{table}

\subsection{Aufbau}

Ein Paket kann mehrere OPTs enthalten, wenn es aber nur eins enthält, ist
es allerdings zweckmäßig, das Paket genauso wie das OPT zu nennen. Im
Folgenden ist \texttt{<PAKET>} durch den jeweiligen Paket-Namen zu ersetzen.
Ein Paket besteht aus folgenden Teilen:

\begin{itemize}
\item Verwaltungsdateien
\item Dokumentation
\item Entwickler-Dokumentation
\item Client-Programme
\item Quellcode
\item Weitere Dateien
\end{itemize}

Die einzelnen Teile sind im Folgenden näher beschrieben.

\subsection{Die Konfiguration der Pakete}

In der Datei \texttt{config/<PAKET>.txt} werden vom Benutzer Änderungen an der
Konfiguration des Pakets vorgenommen. Alle Variablen eines OPTs
sollten einheitlich mit dem Namen des OPTs beginnen, also zum
Beispiel:

\begin{example}
\begin{verbatim}
    #-------------------------------------------------------------------
    # Optional package: TELNETD
    #-------------------------------------------------------------------
    OPT_TELNETD='no'        # install telnetd: yes or no
    TELNETD_PORT='23'       # telnet port
\end{verbatim}
\end{example}

Ein OPT sollte in der Konfigurationsdatei durch einen Header (siehe
oben) entsprechend abgegrenzt werden. Dies erhöht die Übersichtlichkeit,
zumal ein Paket ja auch mehrere OPTs enthalten kann. Die dem OPT
zugehörigen Variablen sollten~--- ebenfalls im Interesse der
Übersichtlichkeit~--- nicht weiter eingerückt werden. Kommentare und
Leerzeilen sind erlaubt, wobei Kommentare einheitlich in Spalte 33
beginnen sollen. Ist eine Variable inklusive ihrer Belegung länger als
32 Zeichen, ist der Kommentar eine Zeile versetzt ab Spalte 33 einzufügen.
Längere Kommentare werden jeweils ab Spalte 33 beginnend auf mehrere Zeilen
aufgeteilt. Diese Maßnahmen sollen die Lesbarkeit der Konfigurationsdatei
erhöhen.

Alle Werte hinter
dem Gleichheitszeichen müssen in Hochkommata\footnote{Es können sowohl
einfache Hochkommata als auch doppelte Hochkommata verwendet werden.
Man kann also \texttt{FOO='bar'} oder auch \texttt{FOO="bar"}
schreiben.  Die Verwendung von doppelten Hochkommata sollte
allerdings die Ausnahme sein und man sollte sich vorher
unbedingt darüber informieren, wie eine Unix-Shell mit einfachen und
doppelten Hochkommata umgeht.} eingefasst werden, da es sonst beim
Booten zu Problemen kommen kann.

    Variablen, die aktiv sind (s.\,u.), werden in die \texttt{rc.cfg} übernommen,
    alles andere wird ignoriert. Einzige Ausnahme sind Variablen mit
    dem Namen \var{<PAKET>\_DO\_DEBUG}. Diese dienen zur Fehlersuche in
    Paketen und werden pauschal übernommen.

\marklabel{sec:opt_txt}{
  \subsection{Die Liste der zu kopierenden Dateien}
}

    Die Datei \texttt{opt/<PAKET>.txt} enthält Anweisungen, die beschreiben
\begin{itemize}
\item welche Dateien zu welchem OPT gehören,
\item wann sie in das zu generierende \texttt{opt}- bzw. \texttt{rootfs}-Archiv
    übernommen werden sollen,
\item welche UID\footnote{User ID: Eigentümer der Datei}, GID\footnote{Group ID:
Gruppe der Datei} und Rechte\footnote{Darf die Datei gelesen,
beschrieben oder ausgeführt werden?} jede Datei bekommen soll,
\item welche Konvertierungen vor Aufnahme ins Archiv erfolgen sollen.
\end{itemize}

\texttt{mkfli4l} generiert darauf basierend die erforderlichen Archive.

    Leere Zeilen und Zeilen, die mit \texttt{\#} anfangen, werden
ignoriert. In einer der ersten Zeilen sollte die Version des
Paket-Dateiformats wie folgt stehen:

\begin{example}
\begin{verbatim}
    <erste Spalte>      <zweite Spalte> <dritte Spalte>
    opt_format_version  1                    -
\end{verbatim}
\end{example}

    Die restlichen Zeilen haben folgende Syntax:

\begin{example}
\begin{verbatim}
    <erste Spalte>  <zweite Spalte> <dritte Spalte> <folgende spalten>
    Variable        Wert            Datei           Optionen
\end{verbatim}
\end{example}

    \begin{enumerate}
    \item
        In der ersten Spalte steht der Name einer Variable, von
        deren Wert das Übernehmen der in der dritten Spalte stehenden
        Datei abhängt. Der Name einer Variable kann beliebig oft in
        der ersten Spalte auftauchen, falls mehrere Dateien von ihr
        abhängen. Jede Variable, die in der Datei \texttt{opt/<PAKET>.txt}
        auftaucht, wird von \texttt{mkfli4l} markiert.

        Falls mehrere Variablen auf denselben Wert geprüft werden sollen, kann
        auch eine Liste von Variablen (durch Kommata getrennt) verwendet werden.
        In diesem Falle reicht es aus, wenn mindestens \emph{eine} Variable
        den in der zweiten Spalte geforderten Wert enthält. Wichtig ist dabei,
        dass zwischen den einzelnen Variablen \emph{keine} Leerzeichen stehen!

        Bei OPT-Variablen (also Variablen, die mit \var{OPT\_} beginnen und
        typischerweise den Typ \var{YESNO} haben) kann das Präfix
        "`\var{OPT\_}"' weggelassen werden. Des Weiteren ist es unwichtig, ob
        Variablen in Groß- oder in Kleinbuchstaben (oder beliebig gemischt)
        notiert werden.

      \item In der zweiten Spalte steht ein Wert. Stimmt die in der
        ersten Spalte stehende Variable mit diesem Wert überein und
        ist die Variable aktiv (s.\,u.), wird die
        Datei in der dritten Spalte übernommen. Steht eine \%-Variable
        in der ersten Spalte, wird über alle Indizes iteriert und
        geprüft, ob irgendein Element des Arrays mit dem Wert übereinstimmt.
        Ist das der Fall, wird kopiert. Zusätzlich wird vermerkt, dass
        aufgrund des aktuellen Wertes der Variable eine Datei kopiert
        wurde.

        Es ist möglich, vor den Wert ein "`!"' zu schreiben. In diesem Falle
        wird die Prüfung negiert, d.\,h.\ die Datei wird genau dann kopiert,
        wenn die Variable diesen Wert \emph{nicht} enthält.

      \item  In der dritten Spalte steht der Name einer Datei. Die
        Pfadangabe erfolgt relativ zum \texttt{opt}-Verzeichnis. Die Datei muss
        existieren und lesbar sein, sonst gibt es beim Generieren der
        Archive einen Fehler und die Generierung wird abgebrochen.

        Beginnt der Dateiname mit \texttt{rootfs:}, wird die Datei in die Liste
        der ins \texttt{rootfs}-Archiv aufzunehmenden Dateien übernommen. Der
        Präfix wird vorher entfernt. Taucht dieselbe Datei sowohl mit als auch
        ohne \texttt{rootfs:}-Präfix auf, wird sie nur ins
        \texttt{rootfs}-Archiv übernommen.

        Liegt die Datei unterhalb des verwendeten Konfigurationsverzeichnisses,
        wird sie in die Liste der aus dem Konfigurationsverzeichnis zu
        übernehmenden Dateien aufgenommen, andernfalls wird die unter
        \texttt{opt/} liegende Datei genommen.

        Ist die zu kopierende Datei ein Kernel-Modul, kann man die
        konkrete Kernel-Version durch \texttt{\$\{KERNEL\_VERSION\}} ersetzen.
        \texttt{mkfli4l} nimmt dann die Version aus der Konfiguration und setzt
        sie hier ein. Dadurch kann man einem Paket Module für verschiedene
        Kern-Versionen mitgeben und es wird immer die für den Kern richtige
        Version mit auf den Router kopiert.
        Für Kernel-Module kann der Pfad auch vollkommen entfallen.
        \texttt{mkfli4l} findet das Modul anhand der Dateien
        \texttt{modules.dep} und \texttt{modules.alias}, siehe Abschnitt
        \jump{subsec:automatic-dependencies}{"`Automatische Auflösung von
        Abhängigkeiten für Kernel-Module"'}.

        \begin{table}[ht!]
          \centering
          \small
          \caption{Optionen für Dateien}
          \label{table:options}
          \begin{tabular}{|p{2.5cm}|p{7.5cm}|p{3.5cm}|}
            \hline
            Option & Bedeutung & Standardwert \\
            \hline
            type= & Der Typ des Eintrags:\newline\newline
            \begin{tabular}{ll}
            \emph{local} & Dateisystem-Objekt\\
            \emph{file} & Datei\\
            \emph{dir} & Verzeichnis\\
            \emph{node} & Gerät\\
            \emph{symlink} & (symbolische) Verknüpfung
            \end{tabular}\newline\newline
            Wenn vorhanden, muss diese
            Option an erster Stelle stehen. Der Typ "`local"' steht hierbei
            für den Typ eines im Dateisystem existierenden Objekts und
            entspricht somit "`file"', "`dir"', "`node"' oder "`symlink"' (je
            nachdem). Die anderen Typen mit Ausnahme von "`file"' können
            Einträge im Archiv erzeugen, die nicht im lokalen Dateisystem
            vorliegen müssen. Das wird z.\,B. benutzt, um Gerätedateien im
            \texttt{rootfs}-Archiv anzulegen. & local \\
            uid= & Der Eigentümer der Datei, entweder numerisch oder
            als Name aus passwd & root \\
            gid= & Die Gruppe der Datei, entweder numerisch oder als
            Name aus group & root \\
            mode= & Die Zugriffsrechte &
            Dateien und Geräte:\newline\verb?rw-r--r--? (644)\newline
            Verzeichnisse:\newline\verb?rwxr-xr-x? (755)\newline
            Verknüpfungen:\newline
            \verb?rwxrwxrwx? (777)\newline\\
            flags=\newline
            (type=file) & Konvertierungen vor der Aufnahme ins Archiv:\newline\newline
            \begin{tabular}{lp{6cm}}
            \emph{utxt} & Konvertierung ins Unix-Format\\
            \emph{dtxt} & Konvertierung ins DOS-Format\\
            \emph{sh}   & Shell-Skript: Konvertierung ins Unix-Format, Entfernen überflüssiger Zeichen\\
            \emph{luac} & Lua-Skript: Übersetzung in Bytecode der Lua-VM
            \end{tabular}
            & \\
            name= & Alternativer Name, unter dem der Eintrag ins Archiv
aufgenommen wird  &  \\
            devtype=\newline
            (type=node) & Beschreibt den Typ des
            Geräts ("`c"' für zeichenorientierte und "`b"' für blockorientierte
            Geräte). Muss an zweiter Stelle stehen. & \\
            major=\newline
            (type=node) & Beschreibt die so genannte
            "`Major"'-Nummer der Gerätedatei. Muss an dritter Stelle stehen. & \\
            minor=\newline
            (type=node) & Beschreibt die so genannte
            "`Minor"'-Nummer der Gerätedatei. Muss an vierter Stelle stehen. & \\
            linktarget=\newline
            (type=symlink) & Beschreibt das Ziel der
            symbolischen Verknüpfung. Muss an zweiter Stelle stehen. & \\
            \hline

          \end{tabular}
        \end{table}

      \item In den anderen Spalten können die in Tabelle~\ref{table:options}
        aufgeführten Optionen für den Eigentümer, die Gruppe, die Rechte der Dateien
        und Konvertierungen stehen.

    \end{enumerate}

    Einige Beispiele:
    \begin{itemize}
    \item kopiere Datei, wenn \verb+OPT_TELNETD='yes'+, setze
UID/GID auf root und die Rechte auf 755 (\verb?rwxr-xr-x?):

\begin{example}
\begin{verbatim}
    telnetd     yes    usr/sbin/in.telnetd mode=755
\end{verbatim}
\end{example}

    \item kopiere Datei (\texttt{OPT\_BASE='yes'} gilt implizit immer); setze
UID/GID auf root, die Rechte auf 555 (\verb?r-xr-xr-x?) und konvertiere die
Datei ins Unix-Format bei gleichzeitigem Entfernen aller überflüssigen
Zeichen:

\begin{example}
\begin{verbatim}
    base    yes     etc/rc0.d/rc500.killall mode=555 flags=sh
\end{verbatim}
\end{example}

    \item kopiere Kernel-Modul, wenn \verb+PCMCIA_PCIC='i82365'+; setze
UID/GID auf root und die Rechte auf 644 (\verb?rw-r--r--?):

\begin{example}
\begin{verbatim}
    pcmcia_pcic i82365 lib/modules/${KERNEL_VERSION}/pcmcia/i82365.ko
\end{verbatim}
\end{example}

    \item kopiere Kernel-Modul, wenn \texttt{PCMCIA\_\-PCIC='i82365'}; setze
UID/GID auf root und die Rechte auf 644 (\verb?rw-r--r--?) (alternative Form):

\begin{example}
\begin{verbatim}
    pcmcia_pcic i82365 i82365.ko
\end{verbatim}
\end{example}

    \item kopiere Kernel-Modul, wenn mindestens einer der Einträge in
\var{NET\_DRV\_\%} den Wert \texttt{3c503} besitzt; setze
UID/GID auf root und die Rechte auf 644 (\verb?rw-r--r--?):
\begin{example}
\begin{verbatim}
    net_drv_%   3c503  3c503.ko
\end{verbatim}
\end{example}

    \item kopiere Datei, wenn die Variable \var{POWERMANAGEMENT} \emph{nicht}
    den Wert "`none"' enthält:

\begin{example}
\begin{verbatim}
    powermanagement !none etc/rc.d/rc100.pm mode=555 flags=sh
\end{verbatim}
\end{example}

    \item kopiere Datei, wenn irgendeine der OPT-Variablen \var{OPT\_MYOPTA}
    oder \var{OPT\_MYOPTB} den Wert "`yes"' enthält:

\begin{example}
\begin{verbatim}
    myopta,myoptb yes usr/local/bin/myopt-common.sh mode=555 flags=sh
\end{verbatim}
\end{example}

    Dieses Beispiel ist letztlich nur eine Kurzschreibweise für:

\begin{example}
\begin{verbatim}
    myopta yes usr/local/bin/myopt-common.sh mode=555 flags=sh
    myoptb yes usr/local/bin/myopt-common.sh mode=555 flags=sh
\end{verbatim}
\end{example}

    Und letzteres ist eine Kurzschreibweise für:

\begin{example}
\begin{verbatim}
    opt_myopta yes usr/local/bin/myopt-common.sh mode=555 flags=sh
    opt_myoptb yes usr/local/bin/myopt-common.sh mode=555 flags=sh
\end{verbatim}
\end{example}

    \item kopiere Datei \texttt{opt/usr/bin/beep.sh} ins \texttt{rootfs}-Archiv,
    aber benenne sie vorher in \texttt{bin/beep} um:

\begin{example}
\begin{verbatim}
    base yes rootfs:usr/bin/beep.sh mode=555 flags=sh name=bin/beep
\end{verbatim}
\end{example}

    \end{itemize}

    Wenn im Paket eine Variable referenziert wird, die nicht vom Paket
    selbst definiert wird, kann es passieren, dass das entsprechende
    Paket nicht installiert ist. Das führt für gewöhnlich zu einer Fehlermeldung
    in \texttt{mkfli4l}, da \texttt{mkfli4l} erwartet, dass alle von \texttt{opt/<PAKET>.txt}
    referenzierten Variablen definiert sind.

    Um diese Situation korrekt handhaben zu können, wurde die
    "`weak"'-Deklaration eingeführt. Sie hat das folgende Format:

\begin{example}
\begin{verbatim}
    weak        <Variable>    -
\end{verbatim}
\end{example}

    Dadurch wird die Variable definiert und ihr Wert intern auf
    "`undefiniert"' gesetzt, wenn sie nicht bereits definiert worden ist. Dabei ist
    jedoch zu beachten, dass hier das "`\var{OPT\_}"'-Präfix \emph{nicht} weggelassen
    werden darf (falls es existiert), weil sonst eine Variable \emph{ohne}
    dieses Präfix definiert wird.
    
    Ein Beispiel aus der \texttt{opt/rrdtool.txt}:
\begin{example}
\begin{verbatim}
    weak opt_openvpn -
    [...]
    openvpn    yes    usr/lib/collectd/openvpn.so
\end{verbatim}
\end{example}

    Ohne die \texttt{weak}-Definition würde \var{mkfli4l} bei der Nutzung des
    Pakets "`rrdtool"' eine Fehlermeldung anzeigen, wenn das "`openvpn"'-Paket
    nicht ebenfalls vorliegt. Mit Hilfe der \texttt{weak}-Definition kommt auch
    in dem Fall, dass das "`openvpn"'-Paket nicht vorliegt, keine Fehlermeldung.

\marklabel{subsec:konfigspezdatei}{
\subsubsection{Konfigurations-spezifische Dateien}
}

In manchen Situationen möchte man originale Dateien im \texttt{opt}- oder
\texttt{rootfs}-Archiv durch Konfigurations-spezifische Dateien 
wie z.\,B. Host-Keys, eigene Firewall-Scripte, \ldots{} ersetzen.
\texttt{mkfli4l} unterstützt dieses Szenario, indem es prüft, ob eine zu
kopierende Datei im Konfigurationsverzeichnis zu finden ist, und übernimmt in
diesem Falle diese Datei in die Liste der ins Archiv aufzunehmenden
Dateien.

Eine weitere Möglichkeit, Konfigurations-spezifische Dateien ins Archiv
aufzunehmen, wird im Abschnitt
\jump{subsec:addtoopt}{"`Erweiterte Prüfungen der Konfiguration"'} beschrieben.


\marklabel{subsec:automatic-dependencies}{
\subsubsection{Automatische Auflösung von Abhängigkeiten für Kernel-Module}}
Kernel-Module bauen unter Umständen auf anderen Kernel-Modulen
auf. Diese müssen vor ihnen geladen werden und daher gleichfalls in
das Archiv aufgenommen werden. \texttt{mkfli4l} bestimmt diese
Abhängigkeiten anhand von \texttt{modules.dep} und \texttt{modules.alias},
zweier beim Kernel-Bauen generierter Dateien, und nimmt automatisch
alle benötigten Module in die Archive auf. So führt z.\,B. folgender
Eintrag

\begin{example}
\begin{verbatim}
    net_drv_%   ne2k-pci    ne2k-pci.ko
\end{verbatim}
\end{example}

dazu, dass auch \texttt{8390.ko} ins Archiv aufgenommen wird, da
\texttt{ne2k-pci.ko} davon abhängt.

Die notwendigen Einträge in \texttt{modules.dep} und \texttt{modules.alias}
werden in das \texttt{rootfs}-Archiv mit aufgenommen und können von
\texttt{modprobe} zum Laden der Treiber genutzt werden.

\marklabel{subsec:dev:var-check}{
\subsection{Die Prüfung der Konfiguration-Variablen}
}

Mit Hilfe der Datei \texttt{check/<PAKET>.txt} können die Inhalte der Variablen auf Gültigkeit
überprüft werden. Diese Überprüfung war in früheren Versionen fest im
Programm \var{mkfli4l} eingebaut, wurde aber im Zuge der Modularisierung von
fli4l in die Check-Dateien ausgelagert. In dieser Datei ist für jede
Variable aus den Konfigurationsdateien eine Zeile vorhanden. Diese Zeilen
bestehen aus vier bis fünf Spalten, welche folgende Funktionen haben:

\begin{enumerate}

\item Variable: Diese Spalte gibt den Namen der zu überprüfenden
  Variable aus der Konfigurationsdatei an. Wenn es sich dabei um eine so
  genannte \emph{Array-Variable} handelt, die mehrmals mit verschiedenen Indizes auftauchen
  kann, wird an Stelle der Nummer ein Prozentzeichen (\%) in den
  Variablenname eingefügt. Dieses wird immer als "`\var{\_\%\_}"' in der Mitte
  eines Namens bzw.\ "`\var{\_\%}"' am Ende eines Namens verwendet. Der Name
  kann dabei mehrere Prozentzeichen
  enthalten, so dass man auch mehrdimensionale Arrays realisieren
  kann. Dann sollte zwischen den Prozentzeichen allerdings etwas
  stehen, muss aber nicht, was dann allerdings zu so seltsamen
  Namen wie "`\var{FOO\_\%\_\_\%}"' führt.

  Oftmals hat man das Problem, dass bestimmte Variablen Optionen
  beschreiben, die man nur in bestimmten Situationen benötigt. Deshalb
  können Variablen als optional markiert werden. Optionale Variablen
  werden mit einem vorangestellten "`+"' gekennzeichnet. Sie können dann
  da sein, müssen aber nicht. Arrays können auch mit einem "`++"' Präfix
  versehen werden. Steht ein "`+"' davor, kann das Array da sein oder
  ganz fehlen. Steht "`++"' davor, können zusätzlich auch noch einzelne
  Elemente des Arrays fehlen.


\item \var{OPT\_\-VARIABLE}: Diese Spalte teilt die Variable einem bestimmten
  OPT zu. Die Variable wird nur auf Gültigkeit überprüft, wenn die
  hier angegebene Variable auf "`yes"' steht.  Gibt es keine
  OPT-Variable, ist hier ein "`-"' anzugeben. In diesem Fall muss die Variable
  in der Konfigurationsdatei definiert werden, es sei denn, es wird eine
  Standard-Belegung definiert (s.\,u.). Der Name der OPT-Variable kann beliebig
  sein, er sollte jedoch mit dem Präfix "`\var{OPT\_}"' beginnen.

  Falls eine Variable von keiner OPT-Variablen abhängt, gilt sie als
  \emph{aktiv}. Falls sie von einer OPT-Variablen abhängig ist, ist sie genau
  dann aktiv, wenn

  \begin{itemize}
  \item ihre OPT-Variable aktiv ist und
  \item ihre OPT-Variable den Wert "`yes"' enthält.
  \end{itemize}

  Andernfalls ist die Variable inaktiv.

  \textbf{Hinweis:} Inaktive OPT-Variablen werden, wenn sie in der Konfiguration
  mit "`yes"' belegt werden, auf den Wert "`no"' zurückgesetzt; dies wird von
  \var{mkfli4l} auch mit einer entsprechenden Warnmeldung
  (bspw.\ "`\verb+OPT_Y='yes' ignored, because OPT_X='no'+"') kommentiert. Bei
  transitiven Abhängigkeitsketten (\var{OPT\_Z} hängt von \var{OPT\_Y} ab, das
  wiederum von \var{OPT\_X} abhängt) funktioniert dies aber nur dann
  zuverlässig, wenn die Namen aller OPT-Variablen mit "`\var{OPT\_}"' beginnen.

\item \var{VARIABLE\_\-N}: Steht in der ersten Spalte eine Variable mit einem
  \% im Namen, wird hier die Variable angegeben, die die Häufigkeit des
  Auftretens der Variable beschreibt (die so genannte \emph{N-Variable}). Ist
  die Variable mehrdimensional, wird die Häufigkeit des letzten Index
  beschrieben. Hängt die Variable von einem OPT ab, muss die N-Variable vom
  selben OPT oder von keinem OPT abhängig sein. Ist die Variable von keinem
  OPT abhängig, darf auch die N-Variable von keinem OPT abhängig sein. Gibt es
  keine N-Variable, ist hier ein "`-"' anzugeben.

  Aus Kompatibilitätsgründen mit zukünftigen fli4l-Versionen \emph{muss} die
  hier angegebene Variable identisch sein mit der Variable in
  \var{OPT\_VARIABLE}, wobei das letzte "`\%"' durch ein "`N"' ersetzt und alles
  dahinter entfernt wurde. Ein Array \var{HOST\_\%\_IP4} bekommt also zwingend
  die N-Variable \var{HOST\_N} zugewiesen und ein Array
  \var{PF\_USR\_CHAIN\_\%\_RULE\_\%} also die N-Variable
  \var{PF\_USR\_CHAIN\_\%\_RULE\_N}, und diese N-Variable ist selbst wieder eine
  Array-Variable mit der zugehörigen N-Variable \var{PF\_USR\_CHAIN\_N}.
  \emph{Alle anderen Benennungen der N-Variable werden mit zukünftigen
  fli4l-Versionen inkompatibel sein!}

\item \var{VALUE}: Diese Spalte gibt an, welche Werte für diese Variable
  eingegeben werden können. Es sind dabei z.\,B. folgende Angaben möglich:

  \begin{tabular}[ht!]{|l|l|}
    \hline
    Name & Bedeutung \\
    \hline
    \hline
    \var{NONE}     &  Es wird keine Überprüfung vorgenommen\\
    \var{YESNO}    &  Die Variable muss "`yes"' oder "`no"' sein\\
    \var{NOTEMPTY} &  Die Variable darf nicht leer sein\\
    \var{NOBLANK}  &  Die Variable darf kein Leerzeichen enthalten\\
    \var{NUMERIC}  &  Die Variable muss numerisch sein\\
    \var{IPADDR}   &  Die Variable muss eine IP-Adresse sein\\
    \var{DIALMODE} &  Die Variable muss "`on"', "`off"' oder "`auto"' sein\\
    \hline
  \end{tabular}
  \\

  Werden die Werte mit einem "`\var{WARN\_}"'-Präfix versehen, so führt ein
  illegaler Wert nicht zu einer Fehlermeldung und damit zu einem Abbruch von
  \var{mkfli4l}, sondern nur zur Ausgabe einer Warnung.

  Die möglichen Prüfungen werden durch reguläre Ausdrücke in
  \texttt{check/base.exp} definiert. Diese Datei kann erweitert werden und
  enhält neuerdings z.\,B. zusätzlich folgende Prüfungen: \var{HEX}, \var{NUMHEX},
  \var{IP\_ROUTE}, \var{DISK} und \var{PARTITION}.

  Die Anzahl der Ausdrücke kann jederzeit erweitert werden, hier ist
  Rückmeldung von den Paket-Entwicklern erforderlich.

  Zusätzlich können reguläre Ausdrücke auch direkt in den Check-Dateien
  angegeben werden, wobei man auch Bezug auf existierende Ausdrücke
  nehmen kann. Statt \var{YESNO} könnte man z.\,B. auch
\begin{example}
\begin{verbatim}
    RE:yes|no
\end{verbatim}
\end{example}
schreiben. Sinnvoll ist es dann, wenn ein
Test nur ein einziges Mal ausgeführt wird und relativ einfach ist. Für
genauere Informationen siehe nächstes Kapitel.

\item Standard-Belegung: In dieser Spalte kann optional ein Standard-Wert für
die Variable stehen, falls die Variable nicht in der Konfiguration steht.

\textbf{Hinweis:} Dies funktioniert zur Zeit jedoch nicht für Array-Variablen.
Auch darf die Variable nicht optional sein, es darf also kein "`+"' vor dem
Variablennamen stehen.

Beispiel:
\begin{example}
\begin{verbatim}
    OPT_TELNETD     -      -      YESNO    "no"
\end{verbatim}
\end{example}

Fehlt \var{OPT\_TELNETD} nun in der Konfigurationsdatei, wird "`no"' angenommen
und dieser Wert auch in die \texttt{rc.cfg} geschrieben.

\end{enumerate}

    Die Sache mit dem Prozentzeichen lässt sich am Besten mit einem
    Beispiel erklären. Nehmen wir an, in der \texttt{check/base.txt} steht:
\begin{example}
\begin{verbatim}
    NET_DRV_N          -                  -                  NUMERIC
    NET_DRV_%          -                  NET_DRV_N          NONE
    NET_DRV_%_OPTION   -                  NET_DRV_N          NONE
\end{verbatim}
\end{example}

      Das heißt, dass je nach Wert von \var{NET\_\-DRV\_\-N} die Variablen \var{NET\_\-DRV\_\-N},
      \var{NET\_\-DRV\_\-1\_\-OPTION}, \var{NET\_\-DRV\_\-2\_\-OPTION}, \var{NET\_\-DRV\_\-3\_\-OPTION}, usw. überprüft werden.

\subsection{Eigene Definitionen zum Prüfen der Konfigurationsvariablen}

\subsubsection{Einführung regulärer Ausdrücke}

  In der Version 2.0 gab es nur die oben angeführten sieben Werte-Bereiche,
  auf die Variablen geprüft werden können: \var{NONE}, \var{NOTEMPTY}, \var{NUMERIC},
  \var{IPADDR}, \var{YESNO}, \var{NOBLANK}, \var{DIALMODE}. Die Überprüfung war in \var{mkfli4l}
  fest eingebaut, nicht erweiterbar und beschränkte sich auf
  wesentliche "`Datentypen"', die mit vertretbarem Aufwand geprüft
  werden können.

  Mit der Version 2.1 wurde diese Prüfung neu implementiert.  Ziel der
  neuen Implementierung ist eine flexiblere Prüfung der Variablen, die
  auch in der Lage ist, komplexere Ausdrücke zu prüfen. Deshalb werden
  reguläre Ausdrücke verwendet, die in einem oder mehreren separaten
  Dateien abgespeichert werden.  Dadurch wird es zum einen möglich,
  Variablen zu prüfen, die im Augenblick noch nicht geprüft werden, und
  zum anderen können Entwickler optionaler Pakete eigene Ausdrücke
  definieren, um die Konfiguration ihrer Pakete prüfen zu lassen.

  Eine Beschreibung regulärer Ausdrücke findet man via "`man 7 regex"'
  oder z.\,B. hier: \altlink{http://unixhelp.ed.ac.uk/CGI/man-cgi?regex+7}.


\subsubsection{Spezifikation regulärer Ausdrücke}

  Spezifizieren kann man die Ausdrücke auf zwei Wegen:

  \begin{enumerate}
  \item Paketspezifische exp-Datei \texttt{check/<PAKET>.exp}

    Diese Datei liegt im \texttt{check}-Verzeichnis und trägt den gleichen Namen
    wie das dazugehörige Paket, also z.\,B. \texttt{check/base.exp}. Sie enthält
    Definitionen für Ausdrücke, die in der Datei \texttt{check/<PAKET>.txt} referenziert werden
    können. So enthält \texttt{check/base.exp} im Augenblick Definitionen für die
    bekannten Prüfungen und \texttt{check/isdn.exp} eine Definition für die Variable
    \var{ISDN\_\-CIRC\_\-x\_ROUTE} (das Fehlen dieser Überprüfung war der Auslöser
    dieser Änderungen).

Die Syntax lautet wie folgt, wobei man auch hier bei Bedarf doppelte
Hochkommata verwenden kann:
\begin{example}
\begin{verbatim}
    <Name> = '<Regulärer Ausdruck>' : '<Fehlermeldung>'
\end{verbatim}
\end{example}
oder am Beispiel aus \texttt{check/base.exp}:
\begin{example}
\begin{verbatim}
    NOTEMPTY = '.*[^ ]+.*'          : 'should not be empty'
    YESNO    = 'yes|no'             : 'only yes or no are allowed'
    NUMERIC  = '0|[1-9][0-9]*'      : 'should be numeric (decimal)'
    OCTET    = '1?[0-9]?[0-9]|2[0-4][0-9]|25[0-5]'
             : 'should be a value between 0 and 255'
    IPADDR   = '((RE:OCTET)\.){3}(RE:OCTET)' : 'invalid ipv4 address'
    EIPADDR  = '()|(RE:IPADDR)'
             : 'should be empty or contain a valid ipv4 address'
    NOBLANK  = '[^ ]+'              : 'should not contain spaces'
    DIALMODE = 'auto|manual|off'    : 'only auto, manual or off are allowed'
    NETWORKS = '(RE:NETWORK)([[:space:]]+(RE:NETWORK))*'
             : 'no valid network specification, should be one or more
                network address(es) followed by a netmask,
                for instance 192.168.6.0/24'
\end{verbatim}
\end{example}

In den regulären Ausdrücken können auch Referenzen auf bereits
existierende Definitionen enthalten sein. Diese werden dann einfach an
der Stelle eingefügt. Dadurch ist es einfacher, reguläre Ausdrücke zu
konstruieren. Eingefügt werden die Referenzen einfach durch
'(RE:Referenz)'. (Siehe die Definition des Ausdrucks \var{NETWORKS} oben für
ein entsprechendes Beispiel.)

Die Fehlermeldungen tendieren dazu, zu lang zu werden. Daher besteht
die Möglichkeit, sie über mehrere Zeilen zu verteilen. Die folgenden
Zeilen müssen dann immer mit einem Leerzeichen oder Tabulator
beginnen. Beim Einlesen der \texttt{check/<PAKET>.exp}-Datei werden überflüssige
Leerzeichen auf eins reduziert und Tabulatoren durch Leerzeichen
ersetzt. Ein Eintrag in der \texttt{check/<PAKET>.exp} könnte dann so aussehen:

\begin{example}
\begin{verbatim}
    NUM_HEX         = '0x[[:xdigit:]]+'
                    : 'should be a hexadecimal number
                       (a number starting with "0x")'
\end{verbatim}
\end{example}

\item  Reguläre Ausdrücke direkt in der Check-Datei \texttt{check/<PAKET>.txt}

Manche Ausdrücke kommen nur einmal vor, dann lohnt es sich nicht,
dafür einen regulären Ausdruck in einer \texttt{check/<PAKET>.exp}-Datei zu definieren. Dann kann man
diesen Ausdruck einfach in die Check-Datei schreiben, z.\,B.:

\begin{example}
\begin{verbatim}
    # Variable      OPT_VARIABLE    VARIABLE_N     VALUE
    MOUNT_BOOT      -               -              RE:ro|rw|no
\end{verbatim}
\end{example}

\var{MOUNT\_\-BOOT} kann lediglich die Werte "`ro"', "`rw"' oder "`no"' annehmen,
alles andere wird abgelehnt.

Will man Bezug auf existierende reguläre Ausdrücke nehmen, fügt man
einfach eine Referenz via "`(RE:...)"' ein. Beispiel:

\begin{example}
\begin{verbatim}
    # Variable      OPT_VARIABLE    VARIABLE_N     VALUE
    LOGIP_LOGDIR    OPT_LOGIP       -              RE:(RE:ABS_PATH)|auto
\end{verbatim}
\end{example}

\end{enumerate}


\subsubsection{Erweiterung existierender regulärer Ausdrücke}

Fügt ein optionales Paket einen zusätzlichen Wert für eine Variable
hinzu, die von einem regulären Ausdruck geprüft wird, muss der reguläre
Ausdruck erweitert werden. Dies geschieht einfach durch Definition der
neuen möglichen Werte durch einen regulären Ausdruck (wie oben
beschrieben) und Ergänzung des bestehenden regulären Ausdrucks in
einer eigenen \texttt{check/<PAKET>.exp}-Datei. Dass ein bestehender
Ausdruck modifiziert werden soll, kennzeichnet ein führendes "`+"'.
Der neue Ausdruck ergänzt den bestehenden Ausdruck, indem der neue Wert
als Alternative an den bestehenden Wert angehängt wird. Verwendet ein
anderer Ausdruck den ergänzten Ausdruck, gilt auch dort die Ergänzung.
Die angegebene Fehlermeldung wird einfach an die vorhandene hinten angehängt.

Am Beispiel der Ethernet-Treiber könnte das wie folgt aussehen:

\begin{itemize}
\item Das Basis-Paket stellt eine Menge von Ethernet-Treibern bereit
  und prüft die Variable \var{NET\_DRV\_x} mit dem regulären Ausdruck \var{NET\_DRV},
  der wie folgt spezifiert ist:

\begin{example}
\begin{verbatim}
    NET_DRV         = '3c503|3c505|3c507|...'
                    : 'invalid ethernet driver, please choose one'
                      ' of the drivers in config/base.txt'
\end{verbatim}
\end{example}
\item Das Paket "`pcmcia"' stellt jetzt zusätzliche Gerätetreiber bereit,
  muss also \var{NET\_DRV} ergänzen. Das sieht dann wie folgt aus:

\begin{example}
\begin{verbatim}
    PCMCIA_NET_DRV = 'pcnet_cs|xirc2ps_cs|3c574_cs|...' : ''
    +NET_DRV       = '(RE:PCMCIA_NET_DRV)' : ''
\end{verbatim}
\end{example}
\end{itemize}

Nun kann man zusätzlich auch noch PCMCIA-Treiber auswählen.


\subsubsection{Regulären Ausdruck in Abhängigkeit von \var{YESNO}-Variablen erweitern}

Wenn man \var{NET\_DRV} wie oben um die PCMCIA-Treiber erweitert hat, aber
das Paket "`pcmcia"' deaktiviert hat, könnte man dennoch einen PCMCIA-Treiber
in der \texttt{config/base.txt} auswählen, ohne dass eine Fehlermeldung beim
Erstellen der Archive auftritt. Um das zu verhindern, kann man den regulären
Ausdruck auch abhängig von einer \var{YESNO}-Variablen in der Konfiguration
erweitern. Dazu wird der Name der Variablen, die bestimmt ob der Ausdruck
erweitert wird, mit runden Klammern direkt hinter den Namen des Ausdrucks
gehängt. Ist die Variable aktiv und hat den Wert "`yes"', wird der Ausdruck erweitert,
sonst nicht.

\begin{example}
\begin{verbatim}
    PCMCIA_NET_DRV       = 'pcnet_cs|xirc2ps_cs|3c574_cs|...' : ''
    +NET_DRV(OPT_PCMCIA) = '(RE:PCMCIA_NET_DRV)' : ''
\end{verbatim}
\end{example}

Wird jetzt \verb+OPT_PCMCIA='no'+ gesetzt, und in der \texttt{config/base.txt} wird z.\,B. der
PCMCIA-Treiber \texttt{xirc2ps\_cs} benutzt, gibt es beim Erstellen der Archive eine
Fehlermeldung.

\textbf{Hinweis:} Dies funktioniert \emph{nicht}, wenn die Variable nicht
explizit in der Konfigurationsdatei gesetzt wird, sondern ihren Wert
über eine Standard-Belegung in der \texttt{check/<PAKET>.txt} erhält. In
diesem Fall muss man also in der Konfigurationsdatei die Variable explizit
setzen und ggf.\ auf die Standard-Belegung verzichten.

\marklabel{sec:regexp-dependencies}{
  \subsubsection{Regulären Ausdruck in Abhängigkeit von anderen Variablen erweitern}
}

Alternativ kann man auch beliebige Werte von Variablen als Bedingung
verwenden, die Syntax sieht dann wie folgt aus:

\begin{example}
\begin{verbatim}
    +NET_DRV(KERNEL_VERSION=~'^3\.18\..*$') = ...
\end{verbatim}
\end{example}

Wenn \var{KERNEL\_VERSION} zu dem angegebenen regulären Ausdruck passt, also
irgendein Kernel aus der 3.18er Versionsreihe genutzt wird, dann wird die Liste der
erlaubten Netzwerktreiber um die angegebenen Treiber ergänzt.

\textbf{Hinweis:} Dies funktioniert \emph{nicht}, wenn die Variable nicht
explizit in der Konfigurationsdatei gesetzt wird, sondern ihren Wert
über eine Standard-Belegung in der \texttt{check/<PAKET>.txt} erhält. In
diesem Fall muss man also in der Konfigurationsdatei die Variable explizit
setzen und ggf.\ auf die Standard-Belegung verzichten.

\subsubsection{Fehlermeldungen}

Findet die Prüfung einen Fehler, erscheint eine Fehlermeldung der
folgenden Art:

\begin{example}
\begin{verbatim}
    Error: wrong value of variable HOSTNAME: '' (may not be empty)
    Error: wrong value of variable MOUNT_OPT: 'rx' (user supplied regular expression)
\end{verbatim}
\end{example}

Beim ersten Fehler wurde der Ausdruck in einer \texttt{check/<PAKET>.exp}-Datei definiert und
ein Hinweis auf den Fehler wird mit ausgegeben. Im zweiten Falle wurde
der Ausdruck direkt in einer \texttt{check/<PAKET>.txt}-Datei spezifiziert, deshalb gibt es keinen
zusätzlichen Hinweis auf die Fehlerursache.


\subsubsection{Definition regulärer Ausdrücke}

Reguläre Ausdrücke sind wie folgt definiert:

Regulärer Ausdruck: Eine oder mehrere Alternativen, getrennt durch
'$|$', z.\,B. "`ro$|$rw$|$no"'. Trifft eine der Alternativen zu, trifft der
ganze Ausdruck zu (hier wären "`ro"', "`rw"' und "`no"' gültige Ausdrücke).

Eine Alternative ist eine Verkettung mehrerer Teilstücke, die einfach
aneinandergereiht werden.

Ein Teilstück ist ein "`Atom"', gefolgt von einem einzelnen "`*"', "`+"',
"`?"' oder "`\{min, max\}"'. Die Bedeutung ist wie folgt:
\begin{itemize}
\item "`a*"'~--- beliebig viele "`a"'s (erlaubt auch den Fall, das gar kein "`a"' da ist)
\item   "`a+"'~--- mindestens ein "`a"'
\item   "`a?"'~--- kein oder ein "`a"'
\item   "`a\{2,5\}"'~--- zwei bis fünf "`a"'s
\item   "`a\{5\}"'~--- genau fünf "`a"'s
\item   "`a\{2,\}"'~--- mindestens zwei "`a"'s
\item   "`a\{,5\}"'~--- höchstens fünf "`a"'s
\end{itemize}

Ein "`Atom"' ist ein
\begin{itemize}
\item  regulärer Ausdruck eingeschlossen in Klammern, z.\,B. trifft "`(a$|$b)+"'
          auf eine beliebige Zeichenkette zu, die mindestens
          ein "`a"' oder "`b"' enthält, sonst aber beliebig viele und in
          beliebiger Reihenfolge
        \item   ein leeres Paar Klammern steht für einen "`leeren"'
          Ausdruck
        \item   ein Ausdruck mit eckigen Klammern "`[\,]"' (siehe weiter unten)
        \item ein Punkt "`."', der auf irgendein einzelnes Zeichen zutrifft,
          z.\,B. trifft "`.+"' auf eine beliebige Zeichenkette zu, die
          mindestens ein Zeichen enthält
        \item ein "`\^\,"' steht für den Zeilenanfang, z.\,B. trifft "`\^\,a.*"' auf
          eine Zeichenkette zu, die mit einem "`a"' anfängt und in der
          beliebige Zeichen folgen, etwa "`a"' oder "`adkadhashdkash"'
        \item ein "`\$"' steht für das Zeilenende
        \item ein "`$\backslash$"' gefolgt von einem der Sonderzeichen
          \texttt{\^\,.\,[\,\$\,(\,)\,$|$\,*\,+\,?\,\{\,$\backslash$} steht für genau das zweite Zeichen
          ohne seine spezielle Bedeutung
        \item  ein normales Zeichen trifft auf genau das Zeichen zu,
          z.\,B. trifft "`a"' genau auf "`a"' zu.
\end{itemize}

Ein Ausdruck in rechteckigen Klammern bedeutet Folgendes:
\begin{itemize}
\item "`x-y"'~--- trifft auf irgendein Zeichen zu, das zwischen
                  "`x"' und "`y"' liegt, z.\,B. steht "`[0-9]"' für alle Zeichen
                  zwischen "`0"' und "`9"'; "`[a-zA-Z]"' steht für alle Buchstaben,
                  egal ob groß oder klein

                \item "`\^\,x-y"'~--- trifft auf irgendein Zeichen zu, das \emph{nicht} im
                  angegebenen Intervall liegt; so steht z.\,B. "`[\^\,0-9]"' für alle
                  Zeichen, die \emph{keine} Ziffern sind

                \item "`[:\emph{character-class}:]"'~--- trifft auf ein Zeichen der Zeichenklasse \emph{character-class}
                  zu. Relevante Standardzeichenklassen sind: \texttt{alnum}, \texttt{alpha},
                  \texttt{blank}, \texttt{digit}, \texttt{lower}, \texttt{print}, \texttt{punct}, \texttt{space}, \texttt{upper} und \texttt{xdigit}.
                So steht "`[\,[:alpha:]\,]"' für alle Groß- und Kleinbuchstaben und ist somit identisch zu "`[\,[:lower:]\,[:upper:]\,]"'.
\end{itemize}


\subsubsection{Beispiele für reguläre Ausdrücke}

Sehen wir uns das mal an einigen Beispielen an!

\var{NUMERIC}: Ein numerischer Wert besteht aus mindestens einer, aber ansonsten
beliebig vielen Ziffern. "`Mindestens ein"' drückt man
mit "`+"' aus, eine Ziffer hatten wir schon als Beispiel. Zusammengesetzt
ergibt das:

\begin{example}
\begin{verbatim}
    NUMERIC = '[0-9]+'
\end{verbatim}
\end{example}
oder alternativ
\begin{example}
\begin{verbatim}
    NUMERIC = '[[:digit:]]+'
\end{verbatim}
\end{example}

\var{NOBLANK}: Ein Wert, der keine Leerzeichen enthält, ist ein beliebiges
Zeichen (außer dem Leerzeichen) und davon beliebig viele:

\begin{example}
\begin{verbatim}
    NOBLANK = '[^ ]*'
\end{verbatim}
\end{example}

bzw.\ wenn der Wert zusätzlich auch nicht leer sein darf:

\begin{example}
\begin{verbatim}
    NOBLANK = '[^ ]+'
\end{verbatim}
\end{example}

\var{IPADDR}: Sehen wir uns das Ganze nochmal am Beispiel der IPv4-Addresse an. Eine
IPv4-Adresse besteht aus vier "`Octets"', die durch einen Punkt ("`."') voneinander getrennt sind. Ein
Octet kann eine Zahl zwischen 0 und 255 sein. Definieren wir als
erstes ein Octet. Es kann\\

\begin{tabular}[ht!]{lr}
  eine Zahl zwischen 0 und 9 sein: &       [0-9]\\
  eine Zahl zwischen 10 und 99: &     [1-9][0-9]\\
  eine Zahl zwischen 100 und 199:&   1[0-9][0-9]\\
  eine Zahl zwischen 200 und 249: &  2[0-4][0-9]\\
  eine Zahl zwischen 250 und 255 sein: & 25[0-5]\\
\end{tabular}\\

Das Ganze sind Alternativen, also fassen wir sie einfach mittels "`$|$"' zu
einem Ausdruck zusammen: "`[0-9]$|$[1-9][0-9]$|$1[0-9][0-9]$|$2[0-4][0-9]$|$25[0-5]"' und haben
damit ein Octet. Daraus können wir nun eine IPv4-Adresse machen, vier
Octets mit Punkten voneinander getrennt (der Punkt muss mittels eines \emph{Backslashs}
maskiert werden, da er sonst für ein beliebiges Zeichen
steht). Basierend auf der Syntax der exp-Dateien sieht das Ganze dann
wie folgt aus:

\begin{example}
\begin{verbatim}
    OCTET  = '[0-9]|[1-9][0-9]|1[0-9][0-9]|2[0-4][0-9]|25[0-5]'
    IPADDR = '((RE:OCTET)\.){3}(RE:OCTET)'
\end{verbatim}
\end{example}


\subsubsection{Unterstützung beim Entwurf regulärer Ausdrücke}

Will man reguläre Ausdrücke entwerfen und testen, kann man dazu das
"`regexp"'-Programm verwenden, das sich in dem Verzeichnis \texttt{unix}
bzw.\ \texttt{windows} des Pakets "`base"' befindet. Es akzeptiert die folgende
Syntax:

\begin{example}
\begin{verbatim}
    usage: regexp [-c <check dir>] <regexp> <string>
\end{verbatim}
\end{example}

Dabei bedeuten die Parameter Folgendes:
\begin{itemize}
\item \texttt{<check dir>} ist das Verzeichnis, das die Check-Dateien und damit
auch die exp-Dateien enthält. Diese werden von "`regexp"' eingelesen,
damit man auf bereits definierte Ausdrücke zurückgreifen kann.



\item \texttt{<regexp>} ist der reguläre Ausdruck (im Zweifelsfall immer in \verb+'...'+
oder \verb+"..."+ angeben, wobei doppelte Anführungsstriche nur nötig sind, wenn
einfache Hochkommata in dem Ausdruck vorkommen sollen)


\item \texttt{<string>} ist die zu prüfende Zeichenkette
\end{itemize}

Das könnte z.\,B. wie folgt aussehen:
\begin{example}
\begin{verbatim}
./i586-linux-regexp -c ../check '[0-9]' 0
adding user defined regular expression='[0-9]' ('^([0-9])$')
checking '0' against regexp '[0-9]' ('^([0-9])$')
'[0-9]' matches '0'

./i586-linux-regexp -c ../check '[0-9]' a
adding user defined regular expression='[0-9]' ('^([0-9])$')
checking 'a' against regexp '[0-9]' ('^([0-9])$')
regex error 1 (No match) for value 'a' and regexp '[0-9]' ('^([0-9])$')

./i586-linux-regexp -c ../check IPADDR 192.168.0.1
using predefined regular expression from base.exp
adding IPADDR='((RE:OCTET)\.){3}(RE:OCTET)'
 ('^(((1?[0-9]?[0-9]|2[0-4][0-9]|25[0-5])\.){3}(1?[0-9]?[0-9]|2[0-4][0-9]|25[0-5]))$')
'IPADDR' matches '192.168.0.1'

./i586-linux-regexp -c ../check IPADDR 192.168.0.256
using predefined regular expression from base.exp
adding IPADDR='((RE:OCTET)\.){3}(RE:OCTET)'
 ('^(((1?[0-9]?[0-9]|2[0-4][0-9]|25[0-5])\.){3}(1?[0-9]?[0-9]|2[0-4][0-9]|25[0-5]))$')
regex error 1 (No match) for value '192.168.0.256' and regexp
 '((RE:OCTET)\.){3}(RE:OCTET)'
(unknown:-1) wrong value of variable cmd_var: '192.168.0.256' (invalid ipv4 address)
\end{verbatim}
\end{example}


\subsection{Erweiterte Prüfungen der Konfiguration}

    Manchmal ist es notwendig, komplexere Überprüfungen durchzuführen.
    Beispiele für solche komplexeren Dinge wären z.\,B. Abhängigkeiten
    zwischen Paketen oder Bedingungen, die nur erfüllt sein müssen,
    wenn Variablen bestimmte Werte annehmen. So muss z.\,B. bei Auswahl
    eines PCMCIA-ISDN-Adapters auch das Paket "`pcmcia"' installiert
    werden.

    Um diese Überprüfungen durchführen zu können, kann man in
    \texttt{check/<PAKET>.ext} (auch ext-Skript genannt) kleinere Tests
    schreiben. Die Sprache besteht aus folgenden Elementen:

    \begin{enumerate}
    \item Schlüsselwörter:

      \begin{itemize}
      \item Kontrollfluss:

        \begin{itemize}
        \item \texttt{if (\textit{expr}) then \textit{statement} else \textit{statement} fi}
        \item \texttt{foreach \textit{var} in \textit{set\_var} do \textit{statement} done}
        \item \texttt{foreach \textit{var} in \textit{set\_var\_1 ... set\_var\_n} do \textit{statement} done}
        \item \texttt{foreach \textit{var} in \textit{var\_n} do \textit{statement} done}
        \end{itemize}

      \item
        Abhängigkeiten:
        \begin{itemize}
        \item \texttt{provides \textit{package} version \textit{x.y.z}}
        \item \texttt{depends on \textit{package} version \textit{x1.y1 x2.y2.z2 x3.y3 \ldots}}
        \end{itemize}

      \item Aktionen:
        \begin{itemize}
        \item \texttt{warning "\textit{warning}"}
        \item \texttt{error   "\textit{error}"}
        \item \texttt{fatal\_error "\textit{fatal error}"}
        \item \texttt{set \textit{var} = \textit{value}}
        \item \texttt{crypt (\textit{variable})}
        \item \texttt{stat (\textit{filename}, \textit{res})}
        \item \texttt{fgrep (\textit{filename}, \textit{regex})}
        \item \texttt{split (\textit{string}, \textit{set\_variable}, \textit{character})}
        \end{itemize}
      \end{itemize}
    \item Datentypen:      Zeichenketten, positive ganze Zahlen, Versionsnummern
    \item Logische Operationen:    \texttt{<}, \texttt{==}, \texttt{>}, \texttt{!=}, \texttt{!}, \texttt{\&\&}, \texttt{||},
      \texttt{=}\verb+~+, \texttt{copy\_pending}, \texttt{samenet}, \texttt{subnet}
    \end{enumerate}

\marklabel{subsec:dev:data-types}{
\subsubsection{Datentypen}
}

    Zu den Datentypen ist zu sagen, dass Variablen auf Grund des zugehörigen
    regulären Ausdrucks fest einem Datentyp zugeordnet werden:

\begin{itemize}
\item Variablen, deren Typ mit "`\var{NUM}"' beginnt, sind numerisch und
      enthalten positive ganze Zahlen
\item Variablen, die eine N-Variable für irgendein Array sind, sind ebenfalls
      numerisch
\item alle anderen Variablen werden wie Zeichenketten verarbeitet
\end{itemize}

    Das bedeutet unter anderem, dass eine Variable vom Typ \var{ENUMERIC}
    \emph{nicht} als Index beim Zugriff auf eine Array-Variable benutzt werden
    kann, auch wenn man sich vorher vergewissert hat, dass sie nicht leer ist.
    Der folgende Code funktioniert somit nicht:
\begin{example}
\begin{verbatim}
    # sei TEST eine Variable vom Typ ENUMERIC
    if (test != "")
    then
        # Fehler: You can't use a non-numeric ID in a numeric 
        #         context. Check type of operand.
        set i=my_array[test]
        # Fehler: You can't use a non-numeric ID in a numeric 
        #         context. Check type of operand.
        set j=test+2
    fi
\end{verbatim}
\end{example}

    Eine Lösung für dieses Problem bietet \jump{subsec:split}{\texttt{split}}:
\begin{example}
\begin{verbatim}
    if (test != "")
    then
        # alle Elemente von test_% sind numerisch
        split(test, test_%, ' ', numeric)
        # OK
        set i=my_array[test_%[1]]
        # OK
        set j=test_%[1]+2
    fi
\end{verbatim}
\end{example}

\marklabel{subsec:dev:string-rewrite}{
\subsubsection{Zeichenketten und Variablenersetzung}
}

    An verschiedenen Stellen werden Zeichenketten benötigt, etwa wenn eine
    \jump{subsec:dev:print}{Warnung} ausgegeben werden soll. In einigen Fällen,
    die in dieser Dokumentation beschrieben werden, wird eine solche
    Zeichenkette dabei nach Variablen durchsucht; werden welche gefunden,
    werden diese durch ihren Inhalt oder andere Attribute \emph{ersetzt}.
    Diese Ersetzung wird \emph{Variablenersetzung} genannt.

    Dies soll an einem Beispiel verdeutlicht werden. Es gelte die Konfiguration:

\begin{example}
\begin{verbatim}
    # config/base.txt
    HOSTNAME='fli4l'
    # config/dns_dhcp.txt
    HOST_N='1' # Anzahl der Hosts
    HOST_1_NAME='client'
    HOST_1_IP4='192.168.1.1'
\end{verbatim}
\end{example}

    Dann werden die Zeichenketten wie folgt umgeschrieben, wenn die
    Variablenersetzung in dem jeweiligen Kontext aktiv ist:

\begin{example}
\begin{verbatim}
    "Mein Router heißt $HOSTNAME"
    # --> "Mein Router heißt fli4l"
    "HOSTNAME ist Teil des Pakets %{HOSTNAME}"
    # --> "HOSTNAME ist Teil des Pakets base"
    "@HOST_N ist $HOST_N"
    # --> " # Anzahl der Hosts ist 1"
\end{verbatim}
\end{example}

    Wie man sehen kann, gibt es prinzipiell drei Möglichkeiten der Ersetzung:
    \begin{itemize}
    \item \texttt{\$<Name>} bzw.\ \texttt{\$\{<Name>\}}: Ersetzt den
          Variablennamen durch den Inhalt der Variable. Dies ist die häufigste
          Form der Ersetzung. Der Name muss in \texttt{\{...\}} stehen, wenn
          direkt danach in der Zeichenkette ein Zeichen kommt, das gültiger
          Bestandteil eines Variablennamens sein kann, also ein Buchstabe,
          eine Ziffer oder ein Unterstrich. In allen anderen Fällen ist die
          Verwendung von geschweiften Klammern möglich, aber nicht zwingend.

    \item \texttt{\%<Name>} bzw.\ \texttt{\%\{<Name>\}}: Ersetzt den
          Variablennamen durch den Namen des Pakets, in dem die Variable
          definiert ist. Dies funktioniert \emph{nicht} bei im Skript via
          \jump{subsec:dev:assignment}{\texttt{set}} zugewiesenen Variablen oder
          bei Laufvariablen einer \jump{subsec:dev:control}{\texttt{foreach}-Schleife},
          da solche Variablen kein Paket besitzen und für Laufvariablen diese
          Syntax eine andere Bedeutung erhält.

    \item \texttt{@<Name>} bzw.\ \texttt{@\{<Name>\}}: Ersetzt den
          Variablennamen durch den Kommentar, der in der Konfiguration hinter
          der Variablen steht. Auch dies ergibt keinen Sinn für im Skript
          definierte Variablen.
    \end{itemize}

    Will man ein "`\$"', "`@"' oder "`\%"' im Text haben, schreibt man "`\$\$"',
    "`@@"' bzw.\ "`\%\%"'.

    \textbf{Hinweis:} Elemente von Array-Variablen können auf diese Weise
    \emph{nicht} in Zeichenketten integriert werden, weil es keine Möglichkeit
    gibt, einen Index anzugeben.

    Generell unterliegen nur \emph{Konstanten} der Variablenersetzung;
    Zeichenketten, die über eine Variable hereinkommen, bleiben unverändert.
    Ein Beispiel soll dies verdeutlichen - es sei die folgende Konfiguration
    gegeben:

\begin{example}
\begin{verbatim}
    HOSTNAME='fli4l'
    TEST='${HOSTNAME}'
\end{verbatim}
\end{example}

    Dann führt der Code:

\begin{example}
\begin{verbatim}
    warning "${TEST}"
\end{verbatim}
\end{example}

    zur Ausgabe von:
    
\begin{example}
\begin{verbatim}
    Warning: ${HOSTNAME}
\end{verbatim}
\end{example}

    und \emph{nicht} zur Ausgabe von:

\begin{example}
\begin{verbatim}
    Warning: fli4l
\end{verbatim}
\end{example}

    In den folgenden Abschnitten wird explizit darauf hingewiesen, unter
    welchen Umständen Zeichenketten der Variablenersetzung unterliegen.

\subsubsection{Definition eines Dienstes mit einer dazugehörenden
    Versionsnummer: \texttt{provides}}

    Damit kann z.\,B. ein OPT deklarieren, dass es einen Drucker-Dienst
    oder einen Webserver-Dienst
    bereitstellt. Es kann jeweils nur ein einziges Paket geben, dass
    einen Dienst bereitstellt. Damit kann man verhindern, dass
    z.\,B. zwei Webserver parallel installiert werden, was
    naheliegenderweise nicht gehen würde, da sich die beiden Server um
    den Port 80 streiten würden. Zusätzlich wird die aktuelle Version
    des Dienstes angegeben, so dass Weiterentwicklungen Rechnung getragen
    werden kann. Die Versionsnummer besteht aus zwei- oder drei Zahlen, die
    durch Punkte voneinander getrennt sind, etwa "`4.0"' oder "`2.1.23"'.

    Typischerweise werden Dienste auf OPTs, nicht auf ganze Pakete abgebildet.
    So besitzt etwa das Paket "`tools"' eine ganze Reihe von Programmen, die
    jeweils ihre eigene \texttt{provides}-Anweisung definieren, so sie denn
    via \verb+OPT_...='yes'+ aktiviert sind.

    Die Syntax lautet:

\begin{example}
\begin{verbatim}
    provides <Name> version <Version>
\end{verbatim}
\end{example}

    Beispiel aus dem Paket "`easycron"':

\begin{example}
\begin{verbatim}
    provides cron version 3.10.0
\end{verbatim}
\end{example}

    Die Versionsnummer sollte vom OPT-Entwickler in der dritten Komponente
    angehoben werden, wenn lediglich Funktionserweiterungen vorgenommen wurden
    und die Schnittstelle zum OPT kompatibel geblieben ist. Die Versionsnummer
    sollte in der ersten oder zweiten Komponente angehoben werden, wenn sich
    die Schnittstelle in irgendeiner Weise inkompatibel verändert hat (z.\,B.
    auf Grund von Variablenumbenennungen, Pfad-Änderungen, verschwundenen oder
    umbenannten Dienstprogrammen etc.).

\subsubsection{Definition einer Abhängigkeit zu einem Dienst mit einer
    bestimmten Version: \texttt{depends}}

    Benötigt man zur Erbringung der eigenen Funktionalität einen anderen
    Dienst (z.\,B. einen Webserver), kann man hiermit diese Abhängigkeit
    zu einem Dienst mit einer bestimmten Version spezifizieren. Die
    Version kann zweistellig (z.\,B. "`2.1"') oder dreistellig (z.\,B. "`2.1.11"')
    angegeben werden, wobei die zweistellige Variante alle Versionen akzeptiert,
    die ebenfalls so beginnen, während die dreistellige Version nur genau
    diese angegebene Version akzeptiert. Des Weiteren kann eine Liste von
    solchen Versionsnummern angegeben werden, falls mehrere Versionen des
    Dienstes kompatibel mit dem Paket sind.

    Die Syntax lautet:

\begin{example}
\begin{verbatim}
    depends on <Name> version <Version>+
\end{verbatim}
\end{example}

    Ein Beispiel: Paket "`server"' enthalte:
\begin{example}
\begin{verbatim}
    provides server version 1.0.1
\end{verbatim}
\end{example}

    Sei ein Paket "`client"' gegeben. Darin seien folgende
    \texttt{depends}-Anweisungen beispielhaft enthalten:\footnote{Natürlich nur
    jeweils eine zur selben Zeit!}

\begin{example}
\begin{verbatim}
    depends on server version 1.0       # OK, '1.0' passt zu '1.0.1'
    depends on server version 1.0.1     # OK, '1.0.1' passt zu '1.0.1'
    depends on server version 1.0.2     # Fehler, '1.0.2' passt nicht zu '1.0.1'
    depends on server version 1.1       # Fehler, '1.1' passt nicht zu '1.0.1'
    depends on server version 1.0 1.1   # OK, '1.0' passt zu '1.0.1'
    depends on server version 1.0.2 1.1 # Fehler, weder '1.0.2' noch '1.1' passen
                                        # zu '1.0.1'
\end{verbatim}
\end{example}

\marklabel{subsec:dev:print}{
\subsubsection{Kommunikation mit dem Nutzer: \texttt{warning}, \texttt{error}, \texttt{fatal\_error}}
}

    Mit Hilfe dieser drei Funktionen kann man Nutzer warnen, einen
    Fehler signalisieren oder die Prüfung sofort abbrechen. Die Syntax
    sieht wie folgt aus:

    \begin{itemize}
    \item \verb+warning "text"+
    \item \verb+error "text"+
    \item \verb+fatal_error "text"+
    \end{itemize}

    Alle an diese Funktionen übergebenen Zeichenketten-Konstanten unterliegen
    der \jump{subsec:dev:string-rewrite}{Variablenersetzung}.

\marklabel{subsec:dev:assignment}{
\subsubsection{Zuweisungen}
}

    Benötigt man aus irgendeinem Grund eine temporäre Variable, kann
    man diese einfach mit "`\texttt{set var [= value]}"' anlegen. \emph{Die Variable
    darf kein Konfigurationsvariable sein!}\footnote{Dies ist eine bewusste
    Entscheidung: Durch check-Skripte lässt sich die Benutzerkonfiguration
    \emph{nicht} verändern.} Lässt man den "`= value"' Teil weg,
    wird die Variable einfach auf "`yes"' gesetzt, so dass man sie
    hinterher einfach in einer \texttt{if}-Anweisung testen kann. Wird ein
    Zuweisungsteil angegeben, kann hinter dem Gleichheitszeichen alles
    stehen: normale Variablen, indizierte Variablen, Zahlen,
    Zeichenketten, Versionsnummern.

    Zu beachten ist, dass durch diese Zuweisung gleichzeitig der \emph{Typ} der
    temporären Variablen festgelegt wird. Wird eine Zahl zugewiesen, "`merkt"'
    \var{mkfli4l} sich, dass diese Variable eine Zahl enthält, und erlaubt später
    das Rechnen damit. Versucht man, mit einer anders getypten Variable zu
    rechnen, wird dies fehlschlagen. Beispiel:

\begin{example}
\begin{verbatim}
    set i=1   # OK, i ist eine numerische Variable
    set j=i+1 # OK, j ist eine numerische Variable und enthält den Wert 2
    set i="1" # OK, i ist nun eine Zeichenketten-Variable
    set j=i+1 # Fehler "You can't use a non-numeric ID in a numeric 
              #         context. Check type of operand."
              # --> mit Zeichenketten kann man nicht rechnen!
\end{verbatim}
\end{example}

    Man kann auch temporäre Arrays (siehe unten) anlegen. Beispiel:

\begin{example}
\begin{verbatim}
    set prim_%[1]=2
    set prim_%[2]=3
    set prim_%[3]=5
    warning "${prim_n}"
\end{verbatim}
\end{example}

    Dabei wird die Anzahl der Elemente in dem Array in der Variable
    \var{prim\_n} von \var{mkfli4l} verwaltet. Der obige Code führt somit zu
    folgender Ausgabe:

\begin{example}
\begin{verbatim}
    Warning: 3
\end{verbatim}
\end{example}

    Wenn auf der rechten Seite einer Zuweisung eine Zeichenketten-Konstante
    steht, unterliegt sie zum Zeitpunkt der Zuweisung der
    \jump{subsec:dev:string-rewrite}{Variablenersetzung}. Dies wird im
    folgenden Beispiel demonstriert. Der Code:

\begin{example}
\begin{verbatim}
    set s="a"
    set v1="$s" # v1="a"
    set s="b"
    set v2="$s" # v2="b"
    if (v1 == v2)
    then
      warning "gleich"
    else
      warning "ungleich"
    fi
\end{verbatim}
\end{example}

    produziert die Ausgabe "`ungleich"', weil die Variablen \var{v1} und
    \var{v2} bereits während der Zuweisung den aktuellen Inhalt der Variablen
    \var{s} ersetzen.

    \textbf{Hinweis:} Eine in einem Skript gesetzte Variable ist bei der
    Abarbeitung weiterer Skripte sichtbar~-- es existiert zur Zeit kein
    Lokalitätsprinzip für derart eingeführte Variablen. Da die Reihenfolge, in
    der die Skripte verschiedener Pakete abgearbeitet wird, nicht definiert ist,
    sollte man sich nie darauf verlassen, dass Variablen irgendwelche Werte
    besitzen bzw.\ von einem anderen Paket übernommen haben.

\subsubsection{Arrays}

    Will man auf einzelne Elemente einer \%-Variablen (eines Arrays)
    zugreifen, muss man den Original-Namen der Variable, wie er in der
    \texttt{check/<PAKET>.txt}-Datei steht, verwenden, und dabei für
    jedes "`\%"'-Zeichen einen Index mit Hilfe von "`[\emph{Index}]"' anhängen.

    Beispiel: Will man auf die Elemente der Variable
    \var{PF\_USR\_CHAIN\_\%\_RULE\_\%} zugreifen, benötigt man zwei Indizes,
    weil die Variable zwei "`\%"'-Zeichen besitzt. Alle Elemente ausgeben
    kann man z.\,B. mit Hilfe des folgenden Codes (die \texttt{foreach}-Schleife
    wird \jump{subsec:dev:control}{weiter unten} erläutert):

\begin{example}
\begin{verbatim}
    foreach i in pf_usr_chain_n
    do
        # nur ein Index nötig, da nur ein '%' im Variablennamen
        set j_n=pf_usr_chain_%_rule_n[i]
        # Achtung: ein
        # foreach j in pf_usr_chain_%_rule_n[i]
        # ist leider nicht möglich, deshalb der Umweg über j_n!
        foreach j in j_n
        do
            # zwei Indizes nötig, da zwei '%' im Variablennamen
            set rule=pf_usr_chain_%_rule_%[i][j]
            warning "Rule $i/$j: ${rule}"
        done
    done
\end{verbatim}
\end{example}

    Mit der folgenden Beispiel-Konfiguration

\begin{example}
\begin{verbatim}
    PF_USR_CHAIN_N='2'
    PF_USR_CHAIN_1_NAME='usr-chain_a'
    PF_USR_CHAIN_1_RULE_N='2'
    PF_USR_CHAIN_1_RULE_1='ACCEPT'
    PF_USR_CHAIN_1_RULE_2='REJECT'
    PF_USR_CHAIN_2_NAME='usr-chain_b'
    PF_USR_CHAIN_2_RULE_N='1'
    PF_USR_CHAIN_2_RULE_1='DROP'
\end{verbatim}
\end{example}

    gibt es dann die folgenden Ausgaben:

\begin{example}
\begin{verbatim}
    Warning: Rule 1/1: ACCEPT
    Warning: Rule 1/2: REJECT
    Warning: Rule 2/1: DROP
\end{verbatim}
\end{example}

    Alternativ kann man direkt über alle Werte des Arrays iterieren, kennt dann
    allerdings nicht die exakten Indizes der Einträge (was auch nicht immer
    erforderlich ist):

\begin{example}
\begin{verbatim}
    foreach rule in pf_usr_chain_%_rule_%
    do
        warning "Rule %{rule}='${rule}'"
    done
\end{verbatim}
\end{example}

    Das produziert mit der Beispiel-Konfiguration von oben die folgenden
    Ausgaben:

\begin{example}
\begin{verbatim}
    Warning: Rule PF_USR_CHAIN_1_RULE_1='ACCEPT'
    Warning: Rule PF_USR_CHAIN_1_RULE_2='REJECT'
    Warning: Rule PF_USR_CHAIN_2_RULE_1='DROP'
\end{verbatim}
\end{example}

    An dem zweiten Beispiel sieht man auch schön die Bedeutung der
    \texttt{\%{<Name>}}-Syntax: Innerhalb der Zeichenkette wird
    \texttt{\%{rule}} durch den \emph{Namen} der betrachteten Variable ersetzt
    (also z.\,B. \var{PF\_USR\_CHAIN\_1\_RULE\_1}), während \texttt{\${rule}}
    durch dessen \emph{Inhalt} (also z.\,B. \var{ACCEPT}) ersetzt wird.

\subsubsection{Verschlüsseln eines Passwortes: \texttt{crypt}}

Einige Variablen enthalten Passwörter, die nicht im Klartext in der
\texttt{rc.cfg} stehen sollen. Diese Variablen können mittels \texttt{crypt}
verschlüsselt werden und werden damit in das Format überführt, dass
auch auf dem Router benötigt wird. Verwendet wird das wie folgt:

\begin{example}
\begin{verbatim}
    crypt (<Variable>)
\end{verbatim}
\end{example}

Die \texttt{crypt}-Funktion ist die \emph{einzige} Stelle, an der eine
Konfigurationsvariable verändert werden kann.

\marklabel{subsec:statdatei}{
\subsubsection{Abfragen von Eigenschaften einer Datei: \texttt{stat}}
}

    \texttt{stat} ermöglicht es, Eigenschaften einer Datei abzufragen. Zur
    Verfügung gestellt wird im Augenblick lediglich die Größe einer
    Datei. Wenn man
    auf Dateien unterhalb des aktuellen Konfigurationsverzeichnisses testen
    will, kann man die interne Variable \var{config\_dir} benutzen. Die Syntax
    lautet:

\begin{example}
\begin{verbatim}
    stat (<Dateiname>, <Schlüssel>)
\end{verbatim}
\end{example}

    Der Aufruf sieht wie folgt aus (wobei die
    verwendeten Parameter nur Beispiele sind):

\begin{example}
\begin{verbatim}
    foreach i in openvpn_%_secret
    do
       stat("${config_dir}/etc/openvpn/$i.secret", keyfile)
       if (keyfile_res != "OK")
       then
          error "OpenVPN: missing secretfile <config>/etc/openvpn/$i.secret"
       fi
    done
\end{verbatim}
\end{example}

    In dem Beispiel wird geprüft, ob eine Datei im aktuellen
    Konfigurationsverzeichnis vorhanden ist. Wenn also \verb+OPENVPN_1_SECRET='test'+
    in der Konfigurationsdatei gesetzt wird, prüft die Schleife im ersten
    Durchlauf, ob im aktuellen  Konfigurationsverzeichnis die Datei
    \texttt{etc/openvpn/test.secret} vorhanden ist.

    Nach dem Aufruf sind zwei Variablen definiert:

    \begin{itemize}
    \item \texttt{<Schlüssel>\_res}: Resultat des Systemaufrufs stat() ("`OK"', wenn
      Systemruf erfolgreich, sonst Fehlermeldung des Systemaufrufs)
    \item \texttt{<Schlüssel>\_size}: Größe der Datei
    \end{itemize}

    Das könnte dann z.\,B. so aussehen:

\begin{example}
\begin{verbatim}
    stat ("unix/Makefile", test)
    if ("$test_res" == "OK")
    then
            warning "test_size = $test_size"
    else
            error "Error '$test_res' while trying to get size of 'unix/Makefile'"
    fi
\end{verbatim}
\end{example}

    Ein als Zeichenketten-Konstante übergebener Dateiname unterliegt der
    \jump{subsec:dev:string-rewrite}{Variablenersetzung}.

\marklabel{subsec:fgrepdatei}{
\subsubsection{Durchsuchen von Dateien: \texttt{fgrep}}
}

    Wenn Sie in einer Datei per "`grep"'\footnote{"`grep"' ist ein auf
    Unix-Betriebsystemen verbreitetes Kommando zum Filtern von Textströmen.}
    suchen wollen, steht Ihnen das
    \texttt{fgrep}-Kommando zur Verfügung. Die Syntax lautet:

\begin{example}
\begin{verbatim}
    fgrep (<Dateiname>, <RegEx>)
\end{verbatim}
\end{example}

    Wenn die Datei \texttt{<Dateiname>} nicht existiert wird \var{mkfli4l}
    mit einem fatalen Fehler beendet! Wenn Sie also nicht sicher sind,
    ob die Datei immer vorhanden ist, testen Sie die Existenz von
    \texttt{<Dateiname>} vorher mit \texttt{stat} ab. Nach dem Aufruf von
    \texttt{fgrep} steht Ihnen das Suchresultat in dem Array
    \var{FGREP\_MATCH\_\%} zur Verfügung, wobei der Index \emph{x} wie üblich
    von eins bis \var{FGREP\_MATCH\_N} reicht. \var{FGREP\_MATCH\_1} verweist
    dabei auf den gesamten Bereich der Zeile, auf den der reguläre Ausdruck
    gepasst hat, während \var{FGREP\_MATCH\_2} bis \var{FGREP\_MATCH\_N} den
    jeweils \emph{n-1}-ten geklammerten Teil beinhalten.

    Ein erstes einfaches Beispiel soll die Verwendung demonstrieren. Die Datei
    \texttt{opt/etc/shells} enthält die Zeile:

\begin{example}
\begin{verbatim}
/bin/sh
\end{verbatim}
\end{example}

    Der folgende Code

\begin{example}
\begin{verbatim}
    fgrep("opt/etc/shells", "^/(.)(.*)/")
    foreach v in FGREP_MATCH_%
    do
      warning "%v='$v'"
    done
\end{verbatim}
\end{example}

    produziert die folgende Ausgabe:

\begin{example}
\begin{verbatim}
    Warning: FGREP_MATCH_1='/bin/'
    Warning: FGREP_MATCH_2='b'
    Warning: FGREP_MATCH_3='in'
\end{verbatim}
\end{example}

    Der reguläre Ausdruck hat (nur) auf "`/bin/"' gepasst, deshalb steht auch
    (nur) dieser Teil der Zeile in der Variable \var{FGREP\_MATCH\_1}. Der erste
    geklammerte Teil im Ausdruck passt auf das erste Zeichen hinter dem ersten
    "`/"', deshalb steht auch nur "`b"' in \var{FGREP\_MATCH\_2}. Der zweite
    geklammerte Teil umfasst den Rest hinter den "`b"' bis zum letzten "`/"',
    somit steht "`in"' in der Variable \var{FGREP\_MATCH\_3}.

    Das folgende zweite Beispiel demonstriert eine praxisnahe Verwendung von
    \texttt{fgrep} an einem Beispiel aus der \texttt{check/base.ext}. Hier
    wird getestet, ob alle in der \var{PF\_FORWARD\_x} angegebenen
    \texttt{tmpl:}-Referenzen vorhanden sind:

\begin{example}
\begin{verbatim}
    foreach n in pf_forward_n
    do
      set rule=pf_forward_%[n]
      if (rule =~ "tmpl:([^[:space:]]+)")
      then
        foreach m in match_%
        do
          stat("$config_dir/etc/fwrules.tmpl/$m", tmplfile)
          if(tmplfile_res == "OK")
          then
            add_to_opt "etc/fwrules.tmpl/$m"
          else
            stat("opt/etc/fwrules.tmpl/$m", tmplfile)
            if(tmplfile_res == "OK")
            then
              add_to_opt "etc/fwrules.tmpl/$m"
            else
              fgrep("opt/etc/fwrules.tmpl/templates", "^$m[[:space:]]+")
              if (fgrep_match_n == 0)
              then
                error "Can't find tmpl:$m for PF_FORWARD_${n}='$rule'!"
              fi
            fi
          fi
        done
      fi
    done
\end{verbatim}
\end{example}

    Sowohl ein als Zeichenketten-Konstante übergebener Dateiname als auch als
    Zeichenketten-Konstante übergebener regulärer Ausdruck unterliegen der
    \jump{subsec:dev:string-rewrite}{Variablenersetzung}.

\marklabel{subsec:split}{
\subsubsection{Auseinandernehmen von Parametern: \texttt{split}}
}

    Oftmals werden Variablen mit mehreren Parametern belegt, die dann
    in Startup-Skripten erst wieder auseinandergenommen werden. Will
    man diese bereits vorher auseinandernehmen und Tests auf ihnen
    durchführen, nimmt man \texttt{split}. Die Syntax lautet:

\begin{example}
\begin{verbatim}
    split (<Zeichenkette>, <Array>, <Trennzeichen>)
\end{verbatim}
\end{example}

    Die Zeichenkette kann durch eine Variable oder direkt als
    Konstante angegeben werden. \var{mkfli4l} zerlegt ihn an den Stellen, an
    denen das Trennzeichen auftaucht, und erzeugt pro Teil ein Element des
    Arrays. Über diese Elemente kann man dann hinterher iterieren und Prüfungen
    vornehmen. Steht zwischen zwei Trennzeichen nichts, wird ein Array-Element
    mit einer leeren Zeichenkette als Wert erzeugt. Ausnahme ist "` "':
    Hier werden alle Leerzeichen konsumiert und keine leeren Variablen
    erzeugt.

    Sollen die bei der Zerlegung entstandenen Elemente in einem
    numerischen Kontext verwendet werden (z.\,B. als Indizes), muss das
    beim Aufruf von \texttt{split} spezifiert werden. Das geschieht durch das
    zusätzliche Attribut "`numeric"'. Der Aufruf sieht dann wie folgt
    aus:

\begin{example}
\begin{verbatim}
    split (<Zeichenkette>, <Array>, <Trennzeichen>, numeric)
\end{verbatim}
\end{example}

   Ein Beispiel:

\begin{example}
\begin{verbatim}
    set bar="1.2.3.4"
    split (bar, tmp_%, '.', numeric)
    foreach i in tmp_%
    do
            warning "%i = $i"
    done
\end{verbatim}
\end{example}

    Die produzierte Ausgabe ist:

\begin{example}
\begin{verbatim}
    Warning: TMP_1 = 1
    Warning: TMP_2 = 2
    Warning: TMP_3 = 3
    Warning: TMP_4 = 4
\end{verbatim}
\end{example}

    \textbf{Hinweis:} Wenn die "`numeric"'-Variante verwendet wird, dann prüft
    \var{mkfli4l} zum Zeitpunkt der Zerlegung \emph{nicht}, ob die
    Teil-Zeichenketten auch wirklich numerisch sind! Bei einer späteren
    Verwendung in einem numerischen Kontext (etwa beim Addieren) löst
    \var{mkfli4l} jedoch einen fatalen Fehler aus, wenn eine solche Variable
    doch nicht numerisch ist. Beispiel:

\begin{example}
\begin{verbatim}
    set bar="a.b.c.d"
    split (bar, tmp_%, '.', numeric)
    # Fehler: invalid number 'a'
    set i=tmp_%[1]+1
\end{verbatim}
\end{example}

    Eine an \texttt{split} im ersten Parameter übergebene
    Zeichenketten-Konstante unterliegt der
    \jump{subsec:dev:string-rewrite}{Variablenersetzung}.

\marklabel{subsec:addtoopt}{
\subsubsection{Hinzufügen von Dateien zum Archiv: \texttt{add\_to\_opt}}
}

    Mit der Funktion \texttt{add\_to\_opt} können zusätzliche Dateien
    ans \texttt{opt}- oder \texttt{rootfs}-Archiv angehängt werden. Es können dabei \emph{alle}
    Dateien unterhalb von \texttt{opt/} oder aus dem Konfigurationsverzeichnis
    ausgewählt werden. Eine Beschränkung nur auf die Dateien, die mit einem
    bestimmten Paket geliefert werden, gibt es nicht. Liegt eine Datei
    sowohl unter \texttt{opt/} als auch im Konfigurationsverzeichnis im
    gleichen Pfad, bevorzugt \texttt{add\_to\_opt} die Dateien aus dem
    Konfigurationsverzeichnis. Die Funktion \texttt{add\_to\_opt} wird in der
    Regel dann eingesetzt, wenn komplexe logische Regeln darüber entscheiden,
    ob und welche Dateien in das Archiv aufgenommen werden müssen.

    Die Syntax sieht wie folgt aus:
\begin{example}
\begin{verbatim}
    add_to_opt <Datei> [<Flags>]
\end{verbatim}
\end{example}

    Die Flags sind optional. Es gelten die in Tabelle~\ref{table:options}
    aufgeführten Standard-Werte, falls keine Flags angegeben sind.

    Es folgt ein Beispiel aus dem Paket "`sshd"':

\begin{example}
\begin{verbatim}
    if (opt_sshd)
    then
       foreach pkf in sshd_public_keyfile_%
       do
         stat("$config_dir/etc/ssh/$pkf", publickeyfile)
         if(publickeyfile_res == "OK")
         then
             add_to_opt "etc/ssh/$pkf" "mode=400 flags=utxt"
         else
             error "sshd: missing public keyfile %pkf=$pkf"
         fi
       done
    fi
\end{verbatim}
\end{example}

    Mit \jump{subsec:statdatei}{\texttt{stat}} wird zunächst geprüft, ob die
    Datei im Konfigurationsverzeichnis existiert. Ist die Datei vorhanden, wird
    sie ans Archiv angehängt, andernfalls bricht \var{mkfli4l} mit einer
    entsprechenden Fehlermeldung ab.

    \textbf{Hinweis:} Auch bei \texttt{add\_to\_opt}
    \jump{subsec:konfigspezdatei}{prüft} \var{mkfli4l} zuerst, ob die zu
    kopierende Datei im Konfigurationsverzeichnis zu finden ist.

    Sowohl ein als Zeichenketten-Konstante übergebener Dateiname als auch als
    Zeichenketten-Konstante übergebene Flags unterliegen der
    \jump{subsec:dev:string-rewrite}{Variablenersetzung}.

\marklabel{subsec:dev:control}{
\subsubsection{Kontrollfluss}
}

\begin{example}
\begin{verbatim}
    if (expr)
    then
            statement
    else
            statement
    fi
\end{verbatim}
\end{example}

    Eine klassische Fallunterscheidung, wie man sie kennt. Ist die
    Bedingung wahr, wird der \texttt{then}-Teil ausgeführt, ist die Bedingung
    falsch, wird der \texttt{else}-Teil ausgeführt.

    Will man Tests über Array-Variablen durchführen, muss man jede einzelne
    Variable testen. Dazu gibt es die \texttt{foreach}-Schleife in zwei
    Varianten.

    \begin{enumerate}
    \item Iterieren über Array-Variablen:

\begin{example}
\begin{verbatim}
    foreach <Laufvariable> in <Array-Variable>
    do
            <Anweisung>
    done

    foreach <Laufvariable> in <Array-Variable-1> <Array-Variable-2> ...
    do
            <Anweisung>
    done
\end{verbatim}
\end{example}

    Diese Schleife iteriert über alle angegebenen Array-Variablen, jeweils
    angefangen beim ersten Element bis hin zum letzten; die Anzahl der Elemente
    im Array wird dabei der dem Array zugeordneten N-Variable
    entnommen. Die Lauf\-vari\-able nimmt dabei die jeweiligen Werte der
    Array-Variablen an. Zu beachten ist dabei, dass bei optionalen
    Array-Variablen, die in der Konfiguration nicht vorhanden sind,
    ein leeres Element generiert wird. Unter Umständen muss das im Skript
    berücksichtigt werden, was man z.\,B. wie folgt tun kann:

\begin{example}
\begin{verbatim}
    foreach i in template_var_opt_%
    do
        if (i != "")
        then
            warning "%i is present (%i='$i')"
        else
            warning "%i is undefined (empty)"
        fi
    done
\end{verbatim}
\end{example}

    Wie man auch am Beispiel erkennen kann, lässt sich der \emph{Name} der
    jeweiligen Array-Variablen durch die \texttt{\%<Laufvariable>}-Konstruktion
    ermitteln.

    Die Anweisung in der Schleife kann eine der oben beschriebenen
    Kontrollelemente oder Funktionen (\texttt{if}, \texttt{foreach},
    \texttt{provides}, \texttt{depends}, \ldots) sein.

    Will man auf genau ein Element eines Arrays zugreifen, kann man dieses
    mittels der Syntax \texttt{<Array>[<Index>]} ansprechen. Der Index kann
    dabei eine normale Variable, eine Zahlenkonstante oder wiederum ein
    indiziertes Array sein.

    \item Iterieren über N-Variablen:

\begin{example}
\begin{verbatim}
    foreach <Laufvariable> in <N-Variable>
    do
            <Anweisung>
    done
\end{verbatim}
\end{example}

    Diese Schleife läuft von 1 bis zum Wert, der in der N-Variable steht. Man
    kann die Laufvariable dazu benutzen, um Array-Variablen zu indizieren. Will
    man also nicht nur über eine Array-Variable iterieren, sondern über mehrere
    gleichzeitig, die alle durch \emph{dieselben} N-Variable kontrolliert
    werden, nimmt man diese Variante der Schleife und verwendet die
    Laufvariable zum Indizieren mehrerer Array-Variablen. Beispiel:

\begin{example}
\begin{verbatim}
    foreach i in host_n
    do
        set name=host_%_name[i]
        set ip4=host_%_ip4[i]
        warning "$i: name=$name ip4=$ip4"
    done
\end{verbatim}
\end{example}

    Das ergibt bei entsprechend gefüllten \var{HOST\_\%\_NAME}- und
    \var{HOST\_\%\_IP4}-Arrays beispielsweise:

\begin{example}
\begin{verbatim}
    Warning: 1: name=berry ip4=192.168.11.226
    Warning: 2: name=fence ip4=192.168.11.254
    Warning: 3: name=sandbox ip4=192.168.12.254
\end{verbatim}
\end{example}

    \end{enumerate}

\subsubsection{Ausdrücke}

    Ausdrücke verknüpfen Werte und Operatoren zu einem neuen Wert. Ein Wert
    kann dabei eine gewöhnliche Variable, ein Array-Element oder eine Konstante
    (Zahl, Zeichenkette oder Versionsnummer) sein. Alle
    Zeichenketten-Konstanten, die in Ausdrücken auftreten, unterliegen der
    \jump{subsec:dev:string-rewrite}{Variablenersetzung}.

    Operatoren erlauben so gut wie alles, was man von einer
    Programmiersprache gewöhnt ist. Ein Test auf die Gleichheit zweier Variablen
    könnte also so aussehen:

\begin{example}
\begin{verbatim}
    var1 == var2
    "$var1" == "$var2"
\end{verbatim}
\end{example}

    Zu beachten ist dabei, dass der Vergleich in Abhängigkeit vom Typ
    der Variable erfolgt, der in \texttt{check/<PAKET>.txt} festgelegt
    wurde. Ist eine der beiden Variablen \jump{subsec:dev:data-types}{numerisch},
    erfolgt der Vergleich auf numerischer Basis,
    d.\,h.\ die Zeichenketten werden in Zahlen umgewandelt und dann
    verglichen. Sonst erfolgt der Vergleich auf Zeichenketten-Basis; ein
    Vergleich \texttt{"05"\ == "5"} ergibt "`falsch"', ein Vergleich
    \texttt{"18"\ < "9"} ergibt "`wahr"' auf Grund der lexikographischen Ordnung
    auf Zeichenketten: die Ziffer "`1"' liegt vor der Ziffer "`9"' im zugrunde
    liegenden ASCII-Zeichensatz.

    Für den Vergleich von Versionen wird das Hilfskonstrukt
    \texttt{numeric(version)} eingeführt, welches den numerischen Wert für
    eine Versionsnummer für Vergleichszwecke bestimmt. Dabei gilt:

\begin{example}
\begin{verbatim}
    numeric(version) := major * 10000 + minor * 1000 + sub
\end{verbatim}
\end{example}

    wobei "`major"' die erste Komponente der Versionsnummer darstellt,
    "`minor"' die zweite und "`sub"' die dritte; fehlt "`sub"', entfällt der
    Term in der obigen Summe einfach (oder anders ausgedrückt, für "`sub"' wird
    null angenommen).

    Eine vollständige Auflistung aller Ausdrücke ist in Tabelle
    \ref{tab:expr} zu finden. Dabei steht "`val"' für einen beliebig getypten
    Wert, "`number"' für einen numerischen Wert und "`string"' für eine
    Zeichenkette.

    \begin{table}[htb]
      \centering
      \caption{Logische Ausdrücke}
      \label{tab:expr}
      \begin{tabular}{ll}
        \hline
        Ausdruck &                     wahr wenn\\
        \hline
        \hline
       id                    &    id == "`yes"'\\
       val  == val           &    identisch getypte Werte sind gleich\\
       val  != val           &    identisch getypte Werte sind ungleich\\
       val  == number        &    numerischer Wert von val == number\\
       val  != number        &    numerischer Wert von val != number\\
       val  $<$  number      &    numerischer Wert von val $<$ number\\
       val  $>$  number      &    numerischer Wert von val $>$ number\\
       val  == version       &    numeric(val) == numeric(version) \\
       val  $<$  version     &    numeric(val) $<$  numeric(version) \\
       val  $>$  version     &    numeric(val) $>$  numeric(version) \\
       val  =\verb?~? string &    regulärer Ausdruck in string auf val passt\\
       ( expr )              &    Ausdruck in Klammern ist wahr\\
       expr \&\& expr        &    beide Ausdrücke sind wahr\\
       expr || expr          &    mind. einer der beiden Audrücke ist wahr\\
       copy\_pending(id)     &    siehe Beschreibung\\
       samenet (string1, string2) & string1 das gleiche netz wie
       string2 beschreibt\\
       subnet (string1, string2)  & string1 ein Subnetz von string2 beschreibt\\
        \hline
      \end{tabular}
    \end{table}

\subsubsection{Match-Operator}

Mit dem Match-Operator \verb?=~? kann man prüfen, ob ein regulärer
Ausdruck auf den Wert einer Variable passt. Weiterhin kann man
den Operator auch nutzen, um Teilausdrücke aus einer Variablen zu
extrahieren. Nach erfolgreichem Anwenden eines regulären Ausdrucks auf
eine Variable enthält das Array \var{MATCH\_\%} die gefundenen Teile. Das
könnte z.\,B. wie folgt aussehen:

\begin{example}
\begin{verbatim}
    set foo="foobar12"
    if ( foo =~ "(foo)(bar)([0-9]*)" )
    then
            foreach i in match_%
            do
                    warning "match %i: $i"
            done
    fi
\end{verbatim}
\end{example}

Ein \var{mkfli4l}-Aufruf führt dann zu folgender Ausgabe:

\begin{example}
\begin{verbatim}
    Warning: match MATCH_1: foo
    Warning: match MATCH_2: bar
    Warning: match MATCH_3: 12
\end{verbatim}
\end{example}

Bei Verwendung von \verb?=~? kann Bezug auf alle existierenden
regulären Ausdrücke genommen werden. Will man z.\,B. prüfen, ob ein
PCMCIA-Ethernet-Treiber ausgewählt wurde, ohne dass \var{OPT\_PCMCIA} auf
"`yes"' gesetzt wurde, könnte das wie folgt aussehen:

\begin{example}
\begin{verbatim}
    if (!opt_pcmcia)
    then
        foreach i in net_drv_%
        do
           if (i =~ "^(RE:PCMCIA_NET_DRV)$")
           then
               error "If you want to use ..."
           fi
        done
    fi
\end{verbatim}
\end{example}

Wie in dem Beispiel demonstriert wird, ist es wichtig, den regulären Ausdruck
mit Hilfe von \texttt{\^} und \texttt{\$} zu \emph{verankern}, wenn man den
Ausdruck auf die \emph{gesamte} Variable anwenden will. Ansonsten liefert der
Match-Ausdruck schon "`wahr"', wenn nur ein \emph{Teil} der Variable vom
regulären Ausdruck abgedeckt wird, was in diesem Fall sicherlich nicht
erwünscht ist.

\subsubsection{Prüfen, ob in Abhängigkeit vom Wert einer Variable eine
  Datei kopiert wurde: \texttt{copy\_pending}}

        Mit den im Check-Prozess gewonnenen Informationen prüft die
        Funktion \texttt{copy\_pending}, ob in Abhängigkeit vom Wert einer
        Variable eine Datei kopiert wurde oder nicht. Das kann man
        verwenden, um z.\,B. zu testen, ob der vom Nutzer angegebene
        Treiber auch wirklich existiert und kopiert
        wurde. \texttt{copy\_pending} akzeptiert den zu prüfenden Namen in Form
        einer Variablen oder einer Zeichenkette.\footnote{Wie eingangs
        beschrieben unterliegt die Zeichenkette der Variablenersetzung, so dass
        man z.\,B. mittels einer
        \jump{subsec:dev:control}{\texttt{foreach}-Schleife} und der
        \jump{subsec:dev:string-rewrite}{\texttt{\%<Name>}-Ersetzung} alle
        Elemente eines Arrays prüfen kann.} \texttt{copy\_pending} prüft dazu,
        ob

        \begin{itemize}
        \item die Variable aktiv ist (wenn sie von einem OPT abhängt,
           muss dieses auf "`yes"' gesetzt sein),

         \item die Variable in einer \texttt{opt/<PAKET>.txt}-Datei referenziert
           wurde, und

         \item ob in Abhängigkeit vom aktuellen Wert eine Datei kopiert
           wurde.
        \end{itemize}

        Dabei liefert \texttt{copy\_pending} "`wahr"' zurück, wenn im letzten
        Schritt festgestellt wurde, dass \emph{keine} Datei kopiert wurde,
        das Kopieren also somit noch aussteht (also "`pending"' ist).

    Ein kleines Beispiel für die Anwendung all dieser Funktionen
    findet man in \texttt{check/base.ext}:

\begin{example}
\begin{verbatim}
    foreach i in net_drv_%
    do
        if (copy_pending("%i"))
        then
            error "No network driver found for %i='$i', check config/base.txt"
        fi
    done
\end{verbatim}
\end{example}

    Hier werden alle Elemente des Arrays \var{NET\_DRV\_\%} angemeckert, für
    die keine Kopieraktion vorgenommen wurde, für die also in der
    \texttt{opt/base.txt} kein entsprechender Eintrag existiert.

\subsubsection{Vergleich von Netzwerkadressen: \texttt{samenet} und \texttt{subnet}}

Zum Prüfen von Routen benötigt man ab und zu einen Test, ob zwei
Netzwerke identisch sind oder eines ein Subnetz eines anderen
ist. Dazu gibt es die beiden Funktionen \texttt{samenet} und
\texttt{subnet}. Dabei liefert

\begin{example}
\begin{verbatim}
    samenet (netz1, netz2)
\end{verbatim}
\end{example}

"`wahr"', wenn beide Netze identisch sind, und

\begin{example}
\begin{verbatim}
    subnet (netz1, netz2)
\end{verbatim}
\end{example}

gibt "`wahr"' zurück, wenn "`netz1"' ein Subnetz von "`netz2"' ist.

\subsubsection{Erweitern der Kernel-Kommandozeile}

Ist ein OPT gezwungen, dem Kernel andere Boot-Parameter zu übergeben, so musste
früher die Variable \var{KERNEL\_BOOT\_OPTION} geprüft werden, ob der nötige
Parameter enthalten war, und ggf.\ eine Warnung oder eine Fehlermeldung
ausgegeben werden. Mit der internen Variable \var{KERNEL\_BOOT\_OPTION\_EXT}
kann man nötige, aber fehlende Optionen direkt im ext-Skript ergänzen. Ein
Beispiel aus der \texttt{check/base.ext}:

\begin{example}
\begin{verbatim}
    if (powermanagement =~ "apm.*|none")
    then
        if ( ! kernel_boot_option =~ "acpi=off")
        then
            set kernel_boot_option_ext="${kernel_boot_option_ext} acpi=off"
        fi
    fi
\end{verbatim}
\end{example}

Damit wird "`acpi=off"' an den Kernel übergeben, falls keine Energieverwaltung
oder welche vom Typ "`APM"' gewünscht ist.

\subsection{Unterstützung verschiedener Kernelversionslinien}

Verschiedene Kernelversionslinien unterscheiden sich häufig in einigen Details:
\begin{itemize}
\item es stehen andere Treiber zur Verfügung, einige sind weggefallen,
  andere hinzugekommen
\item die Module heißen teilweise einfach anders
\item die Modul-Abhängigkeiten sehen anders aus
\item die Module liegen woanders
\end{itemize}

Diese Unterschiede werden zum großen Teil durch \var{mkfli4l}
automatisch behandelt. Um die zur Verfügung stehenden Module zu
beschreiben, kann man zum einen die zur Prüfung verwendeten Prüfungen in
Abhängigkit von der Version erweitern
(\jump{sec:regexp-dependencies}{bedingte reguläre Ausdrücke}), und zum anderen
erlaubt \var{mkfli4l} \emph{versionsabhängige}
\texttt{opt/<PAKET>.txt}-Dateien. Dies heißen dann
\texttt{opt/<PAKET>\_<Kernel-Version>.txt}, wobei die Komponenten der
Kernel-Version durch Unterstriche voneinander getrennt werden. Ein Beispiel:
Das Paket "`base"' enthält die folgenden Dateien im \texttt{opt}-Verzeichnis:

\begin{itemize}
\item \texttt{base.txt}
\item \texttt{base\_3\_18.txt}
\item \texttt{base\_3\_19.txt}
\end{itemize}

Die erste Datei (\texttt{base.txt}) wird \emph{immer} verarbeitet. Die anderen
beiden Dateien werden nur verarbeitet, wenn die Kernelversion "`3.18(.*)"'
bzw.\ "`3.19(.*)"' lautet. Wie man sieht, können Versionskomponenten im Dateinamen
weggelassen werden, wenn man eine ganze Gruppe von Kerneln "`erschlagen"'
möchte. Unter Annahme von \verb+KERNEL_VERSION='3.18.9'+ werden für ein Paket
\texttt{<PAKET>} die folgenden Dateien (sofern vorhanden) eingelesen und
verarbeitet:

\begin{itemize}
\item \texttt{<PAKET>.txt}
\item \texttt{<PAKET>\_3.txt}
\item \texttt{<PAKET>\_3\_18.txt}
\item \texttt{<PAKET>\_3\_18\_9.txt}
\end{itemize}

\subsection{Dokumentation}

    Die Dokumentation wird in den Dateien

    \begin{itemize}
    \item \texttt{doc/<SPRACHE>/opt/<PAKET>.txt}
    \item \texttt{doc/<SPRACHE>/opt/<PAKET>.html}
    \end{itemize}

    abgelegt. Die HTML-Dateien können auch aufgeteilt werden, d.\,h.\ für jedes
    enthaltene OPT eine. Dann muss trotzdem eine \texttt{<PAKET>.html} angelegt
    werden, die auf die anderen Dateien verweist.
    Änderungen sollten in folgenden Dateien dokumentiert werden:

    \begin{itemize}
    \item \texttt{changes/<PAKET>.txt}
    \end{itemize}

    Die gesamte Text-Dokumentation darf keine Tabulatoren enthalten und muss nach
    spätestens 79 Zeichen einen harten Zeilenumbruch haben. Die stellt sicher,
    dass die Dokumentation auch mit einem Editor ohne automatischen
    Zeilenumbruch richtig gelesen werden kann.

    Wer mag kann auch eine Dokumentation im \LaTeX-Format erstellen und daraus
    dann HTML- und PDF-Fassungen erzeugen. Als Beispiel kann
    die Dokumentation von fli4l dienen. Einen Rahmen für die
    Dokumentation und die minimal benötigten \LaTeX-Macros kann man im
    Paket "`template"' finden. Eine kurze Beschreibung ist in den
    folgenden Unterabschnitten zu finden.

    Die fli4l-Dokumentation steht zur Zeit in den folgenden Sprachen zur
    Verfügung: deutsch, englisch (\texttt{<SPRACHE>} = "`english"') und
    französisch (\texttt{<SPRACHE>} = "`french"'). Es steht einem
    Paket-Entwickler jedoch frei, sein Paket in beliebigen Sprachen zu
    dokumentieren. Im Sinne der Verständlichkeit wird jedoch empfohlen,
    eine Dokumentation in deutsch und/oder englisch (idealerweise in beiden
    Sprachen) anzufertigen.

\subsubsection{Voraussetzungen für die Erstellung einer \LaTeX-Dokumentation}

  Zum Erstellen der Dokumentation aus \LaTeX-Quellen gibt es folgende
  Anforderungen an die Umgebung:

  \begin{itemize}
  \item Linux/OS~X-Umgebung: Zur einfachen Erzeugung gibt es ein Makefile,
    mit dem alle weiteren Aufrufe automatisiert sind (Cygwin müsste auch
    funktionieren, wird aber nicht vom fli4l-Team getestet)
  \item LaTeX2HTML für die HTML-Version
  \item natürlich \LaTeX\ (empfohlen wird "`TeX Live"' für Linux/OS~X und
  "`MiKTeX"' für Microsoft Windows) mit dem "`pdftex"'-Programm und folgenden
    \TeX-Paketen:
    \begin{itemize}
    \item aktuelles KOMA-Skript (mindestens Version 2)
    \item alle notwendigen Pakete für pdftex
    \item ausgepacktes Dokumentationspaket für fli4l, welches die
      benötigten Makefiles und \TeX-Stile bereitstellt
  \end{itemize}
  \end{itemize}


\subsubsection{Dateinamen}

Die Dateien der Dokumentation werden nach folgendem Schema benannt:

\begin{description}
\item [\texttt{<PAKET>\_main.tex}:] Diese Datei enthält den Hauptteil der
  Dokumentation. \texttt{<PAKET>} steht hier für den Namen des
  Pakets, das beschrieben werden soll (in Kleinbuchstaben).
\item[\texttt{<PAKET>\_appendix.tex}:] Sollen zu diesem Paket noch
  weitere Anmerkungen im Anhang hinzugefügt werden, so werden
  diese hier abgelegt.
\end{description}

Diese Dateien werden im Verzeichnis
\texttt{fli4l/<PAKET>/doc/<SPRACHE>/tex/<PAKET>}
abgelegt.  Für das Paket "`sshd"' sieht das z.\,B. wie folgt aus:

\begin{verbatim}
    $ ls fli4l/doc/deutsch/tex/sshd/
    Makefile sshd_appendix.tex  sshd_main.tex  sshd.tex
\end{verbatim}

Das Makefile ist für die Generierung der Dokumentation verantwortlich,
die \texttt{sshd.tex}-Datei stellt einen Rahmen für die eigentliche
Dokumentation und den Anhang bereit, der sich in den anderen beiden Dateien
befindet. Ansehen kann man sich das am Beispiel der
Dokumentation des "`template"'-Pakets.


\subsubsection{\LaTeX-Grundlagen}

\LaTeX\ arbeitet ähnlich wie HTML "`Tag-orientiert"', nur dass die Tags
hier "`Kommandos"' heißen und folgendes Format aufweisen: \verb*?\kommando?
bzw.\ \verb*?\begin{umgebung}? \ldots \verb*?\end{umgebung}?

Nach Möglichkeit sollte man mit Hilfe von Kommandos eher die \emph{Bedeutung}
des jeweiligen Textes auszeichnen und weniger dessen \emph{Darstellung}. Es
ist also vorteilhaft, z.\,B.

\begin{example}
\verb*?\warning{Bitte nicht ... tun}?
\end{example}

\noindent statt

\begin{example}
\verb*?\emph{Bitte nicht ... tun}?
\end{example}

\noindent zu verwenden.

Jedes Kommando bzw.\ jede Umgebung kann noch weitere Parameter
aufnehmen, die mit \verb*?\kommando{parameter1}{paramter2}{paramterN}?
geschrieben werden.

Manche Kommandos haben optionale Parameter, die in eckigen (statt geschweiften)
Klammern stehen: \verb*?\kommando[optionalerParameter]{parameter1}?
\ldots\ Dabei kommt im Normalfall nur ein optionaler Parameter vor, in
seltenen Fällen aber auch mehrere.

Einzelne Absätze werden im Dokument durch Leerzeilen
getrennt.  Innerhalb dieser Absätze nimmt \LaTeX\ selbst den
Zeilenumbruch und die Worttrennung vor.

Folgende Buchstaben haben eine spezielle Bedeutung in \LaTeX\ und müssen,
sollten sie in normalem Text vorkommen, mit einem vorangestellten \verb*?\?
maskiert werden: \# \$ \& \_ \% \{ \}. "`\verb?~?"' und "`\verb?^?"' müssen
wie folgt geschrieben werden: \verb!\verb?~?! \verb!\verb?^?!

Die wichtigsten \LaTeX-Kommandos werden in der Dokumentation des
"`template"'-Pakets verwendet und erklärt.

\subsection{Dateiformate}

    Alle Textdateien (sowohl Dokumentation als auch Skripte, die später auf
    dem Router liegen) müssen im DOS-Dateiformat, also mit CR/LF statt nur LF
    am Zeilenende in das Paket gelegt werden. Dadurch wird erreicht, dass
    Windows-Nutzer die Dokumentation auch mit "`notepad"' lesen können und durch eine Änderung
    eines Skripts unter Windows das Ganze später auf dem Router trotzdem
    lauffähig bleibt.
    Die Skripte werden beim Bauen der Archive in das auf dem Router
    benötigte Format konvertiert (siehe die Beschreibung der Flags in Tabelle~\ref{table:options}).

\subsection{Entwickler-Dokumentation}

    Sollte ein Programm aus dem Paket eine neue Schnittstelle definieren,
    die andere Programme nutzen können, so ist die Dokumentation dieser
    Schnittstelle  in einer separaten Dokumentation unter \texttt{doc/dev/<PAKET>.txt} abzulegen.

\subsection{Client-Programme}

    Sollte ein Paket zusätzliche Client-Programme mitliefern, so sind diese
    im Verzeichnis \texttt{windows/} für Windows-Clients und im Verzeichnis \texttt{unix/} für
    Unix- und Linux-Clients abzulegen.

\subsection{Quellcode}

    Angepasste Programme und Quellcodes können im Verzeichnis \texttt{src/<PAKET>/} beigelegt
    werden. Sollen die
Programme genauso wie die restlichen fli4l-Programme gebaut werden,
bitte einen Blick in die Dokumentation des \jump{buildroot}{"`src"'-Pakets} werfen.

\marklabel{sec:script_names}{
  \subsection{Weitere Dateien}
}

    Alle Dateien, die nachher auf dem Router liegen, werden unter \texttt{opt/}
    abgelegt. Dabei liegen unter:
    \begin{itemize}
    \item \texttt{opt/etc/boot.d/} und \texttt{opt/etc/rc.d/} Skripte, die beim Starten des
      Systems ausgeführt werden sollen
    \item \texttt{opt/etc/rc0.d/} Skripte, die beim Herunterfahren des Systems
      ausgeführt werden
    \item \texttt{opt/etc/ppp/} Skripte, die beim Einwählen und Auflegen
      ausgeführt werden
    \item \texttt{opt/} die ausführbaren Programme und sonstige Dateien
      entsprechend ihrer Positionen im Dateisystem (d.\,h.\ die Datei
      \texttt{opt/bin/busybox} wird später auf dem Router im Verzeichnis
      \texttt{/bin} liegen)
    \end{itemize}

    Die Skripte in \texttt{opt/etc/boot.d/}, \texttt{opt/etc/rc.d/} und
    \texttt{opt/etc/rc0.d/}
    werden wie folgt benannt:

    \begin{example}
    \begin{verbatim}
    rc<nummer>.<name>
    \end{verbatim}
    \end{example}

    Die Nummer entscheidet über die
    Reihenfolge der Ausführung, der Name gibt einen Hinweis darauf,
    welches Programm/Paket von diesem Skript behandelt wird.


% Last Update: $Id: dev_main_scripting.tex 35981 2014-12-28 19:32:31Z kristov $

\section{Allgemeine Skript-Erstellung auf fli4l}


Hier folgt jetzt \emph{keine} allgemeine Einführung in Shell-Skripte, das
kann jeder im Internet selber nachlesen, es wird nur auf die spezielle
Gegebenheiten bei fli4l eingegangen. Informationen dazu gibt es in den diversen
Unix-/Linux-Hilfeseiten. Folgende Links können als Einstiegspunkte zu
diesem Thema dienen:
\begin{itemize}
\item Einführung in Shell-Skripte:
  \begin{itemize}
  \item \altlink{http://cip.physik.uni-freiburg.de/main/howtos/sh.php}
  \end{itemize}
 \item
   Hilfeseiten online:
   \begin{itemize}
   \item \altlink{http://linux.die.net/}
   \item \altlink{http://heapsort.de/man2web}
   \item \altlink{http://man.he.net/}
   \item \altlink{http://www.linuxcommand.org/superman_pages.php}
   \end{itemize}
\end{itemize}

\subsection{Aufbau}

    In der Unix-Welt ist es nötig, ein Skript mit dem Namen des Interpreters
    zu beginnen, daher steht in der ersten Zeile:
\begin{example}
\begin{verbatim}
      #!/bin/sh
\end{verbatim}
\end{example}

    Damit man später leichter erkennen kann, was ein Skript macht und wer es
    geschrieben hat, sollte jetzt ein kurzer Header folgen, in etwa so:

\begin{example}
\begin{verbatim}
      #--------------------------------------------------------------------
      # /etc/rc.d/rc500.dummy - start my cool dummy server
      #
      # Creation:     19.07.2001  Toller Hecht <toller-hecht@example.net>
      # Last Update:  11.11.2001  Süße Maus <suesse-maus@example.net>
      #--------------------------------------------------------------------
\end{verbatim}
\end{example}

    Nun kann das eigentliche Skript folgen...


\marklabel{dev:sec:config-variables}{
\subsection{Umgang mit Konfigurationsvariablen}
}


    Pakete werden über die Datei \texttt{config/<PACKAGE>.txt}
    konfiguriert. Die darin enthaltenen und
    \jump{subsec:dev:var-check}{aktiven Variablen} werden beim Erzeugen
    des Boot-Mediums in die Datei \texttt{rc.cfg} übernommen. Beim Booten des
    Routers wird diese Datei eingelesen, bevor irgend ein rc-Skript
    (Skripte unter \texttt{/etc/rc.d/}) gestartet wird. Diese Skripte können
    dadurch auf alle Konfigurationsvariablen einfach durch
    \var{\$<Variablenname>} zugreifen.

    Benötigt man Werte von Konfigurationsvariablen auch noch nach dem
    Booten, dann kann man sie aus der \texttt{/etc/rc.cfg} extrahieren, in
    welche während des Bootens die Konfiguration des Boot-Mediums geschrieben
    wurde. Möchte man beispielsweise den Wert der Variable \texttt{OPT\_DNS}
    in einem Skript auslesen, so kann man dies folgendermaßen tun:

\begin{example}
\begin{verbatim}
    eval $(grep "^OPT_DNS=" /etc/rc.cfg)
\end{verbatim}
\end{example}

    Das funktioniert auch mit mehreren Variablen effizient (d.\,h.\ mit nur
    einem Aufruf des \texttt{grep}-Programms):

\begin{example}
\begin{verbatim}
    eval $(grep "^\(HOSTNAME\|DOMAIN_NAME\|OPT_DNS\|DNS_LISTEN_N\)=" /etc/rc.cfg)
\end{verbatim}
\end{example}

\marklabel{dev:sec:persistent-data}{
\subsection{Persistente Speicherung von Daten}
}

Gelegentlich benötigt ein Paket die Möglichkeit, Daten persistent abzulegen,
die also einen Neustart des Routers überleben. Dazu existiert die Funktion
\texttt{map2persistent}, die von einem Skript in \texttt{/etc/rc.d/}
aufgerufen werden kann. Sie erwartet eine Variable, die einen Pfad enthält,
und ein Unterverzeichnis. Die Idee ist, dass die Variable entweder einen
tatsächlichen Pfad beschreibt~-- dann wird dieser Pfad auch genommen, denn der
Nutzer hat ihn so gewünscht, oder die Zeichenkette "`auto"'~-- dann wird
unterhalb eines Verzeichnisses auf einem persistenten Medium ein entsprechendes
Unterverzeichnis gemäß dem zweiten Parameter erzeugt. Die Funktion liefert
das Resultat in eben der Variable zurück, deren Name im ersten Parameter
übergeben wurde.

Ein Beispiel soll dies verdeutlichen. Sei \var{VBOX\_SPOOLPATH} eine Variable,
die einen Pfad oder die Zeichenkette "`auto"' enthält. Dann führt der Aufruf

\begin{example}
\begin{verbatim}
    begin_script VBOX "Configuring vbox ..."
    [...]
    map2persistent VBOX_SPOOLPATH /spool
    [...]
    end_script
\end{verbatim}
\end{example}

dazu, dass die Variable \var{VBOX\_SPOOLPATH} entweder gar nicht verändert
wird (falls sie einen Pfad enthält), oder dass sie durch den Pfad
\texttt{/var/lib/persistent/vbox/spool} ersetzt wird (falls sie die Zeichenkette
"`auto"' enthält). Dabei verweist\footnote{mit Hilfe eines so genannten
"`bind"'-Mounts} \texttt{/var/lib/persistent} auf ein
Verzeichnis auf einem beschreibbaren und nicht flüchtigen Speichermedium, und
\var{<SCRIPT>} stellt das aufrufende Skript in Kleinbuchstaben dar (dieser Name
wird vom ersten Argument des
\jump{subsec:dev:bug-searching}{\texttt{begin\_script}-Aufrufs} abgeleitet).
Falls kein geeignetes Medium existieren sollte (was durchaus sein kann),
ist \texttt{/var/lib/persistent} ein Verzeichnis in der RAM-Disk.

Zu beachten ist, dass der Pfad, der von \texttt{map2persistent} zurückgegeben
wird, \emph{nicht} automatisch erzeugt wird~-- das muss der Aufrufer selbst
erledigen (etwa durch einen Aufruf von \texttt{mkdir -p <Pfad>}).

In der Datei \texttt{/var/run/persistent.conf} kann nachgeschaut werden, ob
die persistente Speicherung von Daten möglich ist. Beispiel:

\begin{example}
\begin{verbatim}
    . /var/run/persistent.conf
    case $SAVETYPE in
    persistent)
        echo "Persistente Speicherung möglich!"
        ;;
    transient)
        echo "Persistente Speicherung NICHT möglich!"
        ;;
    esac
\end{verbatim}
\end{example}

\marklabel{subsec:dev:bug-searching}{
\subsection{Fehlersuche}
}

    Bei Start-Skripten ist es oft sinnvoll, diese bei Bedarf im Debug-Modus
    der Shell laufen zu lassen, um festzustellen, wo "`der Wurm drin ist"'.
    Dazu wird am Anfang und am Ende folgendes eingefügt:

\begin{example}
\begin{verbatim}
      begin_script <OPT-Name> "start message"
      <script code>
      end_script
\end{verbatim}
\end{example}

Im normalen Betrieb erscheint jetzt beim Start des Skriptes der
angegebene Text und am Ende der gleiche Text mit einem vorangestellten
"`finished"'.

Will man die Skripte debuggen, muss man zwei Dinge tun:

\begin{enumerate}

\item Man muss \jump{DEBUGSTARTUP}{\var{DEBUG\_\-STARTUP}} auf "`yes"'
  setzen.
\item Man muss das Debuggen für dieses OPT aktivieren. Das tut man in
  der Regel durch den Eintrag
\begin{example}
\begin{verbatim}
      <OPT-Name>_DO_DEBUG='yes'
\end{verbatim}
\end{example}
in der Konfigurationsdatei.\footnote{Manchmal werden mehrere
  Start-Skripte verwendet, die dann auch verschiedene Namen für ihre
  Debug-Variablen haben. Hier hilft ein kurzer Blick in die Skripte.}
    Jetzt wird während der Ausführung am Bildschirm genau dargestellt, was
    passiert.
\end{enumerate}


\subsubsection{Weitere beim Debuggen hilfreiche Variablen}

\begin{description}

  \config{DEBUG\_ENABLE\_CORE}{DEBUG\_ENABLE\_CORE}{DEBUGENABLECORE}

  Diese Variable gestattet das Erzeugen von "`Core-Dumps"' (Speicherauszügen).
  Stürzt ein Programm aufgrund eines Fehlers ab, wird ein Abbild des aktuellen
  Zustandes im Dateisystem abgelegt, der hinterher zur Analyse des
  Problems verwendet werden kann. Die Core-Dumps werden unter
  \texttt{/var/log/dumps/} abgelegt.

  \config{DEBUG\_IP}{DEBUG\_IP}{DEBUGIP}

  Wird diese Variable gesetzt, werden alle Aufrufe des Programms \texttt{ip}
  protokolliert.

  \config{DEBUG\_IPUP}{DEBUG\_IPUP}{DEBUGIPUP}

  Wird diese Variable auf "`yes"' gesetzt, werden während der
  Ausführung der \texttt{ip-up}/\texttt{ip-down}-Skripte die ausgeführten
  Anweisungen mitgezeichnet und im System-Protokoll gespeichert.

  \config{LOG\_BOOT\_SEQ}{LOG\_BOOT\_SEQ}{LOGBOOTSEQ}

  Wird diese Variable auf "`yes"' gesetzt, protokolliert der \texttt{bootlogd}
  während des Bootens alle auf der Konsole getätigten Ausgaben in der Datei
  \texttt{/var/tmp/boot.log}. Diese Variable hat standardmäßig den Wert "`yes"'.

  \config{DEBUG\_KEEP\_BOOTLOGD}{DEBUG\_KEEP\_BOOTLOGD}{DEBUGKEEPBOOTLOGD}

  Normalerweise wird der \texttt{bootlogd} am Ende des Bootvorganges
  beendet. Ein Setzen dieser Variable unterbindet das und erlaubt ein
  Protokollieren der Konsolenausgaben über den Bootvorgang hinaus.

  \config{DEBUG\_MDEV}{DEBUG\_MDEV}{DEBUGMDEV}

  Ein Setzen dieser Variable generiert ein Protokoll des \texttt{mdev}-Daemons,
  der für das Anlegen der Geräte-Dateien unter \texttt{/dev} zuständig ist.

\end{description}
\subsection{Hinweise}
\begin{itemize}
\item  Es ist \emph{immer} besser, geschweifte Klammern "`\{\ldots\}"' an Stelle von runden
      Klammern "`(\ldots)"' zu benutzen. Allerdings muss dabei darauf geachtet werden,
      dass nach der öffnenden Klammer ein Leerzeichen oder eine neue Zeile vor
      dem nächsten Befehl kommt und vor der schließenden Klammer ein
      Semikolon oder auch eine neue Zeile kommt. Beispielsweise ist

\begin{example}
\begin{verbatim}
        { echo "cpu"; echo "quit"; } | ...
\end{verbatim}
\end{example}

      \noindent gleichbedeutend mit:

\begin{example}
\begin{verbatim}
        {
                echo "cpu"
                echo "quit"
        } | ...
\end{verbatim}
\end{example}


      \item Ein Skript kann mit "`exit"' vorzeitig beendet
        werden. Dies ist aber bei den Start-Skripten (\texttt{opt/etc/boot.d/...},
        \texttt{opt/etc/rc.d/...}), den Stopp-Skripten (\texttt{opt/etc/rc0.d/...}) und den
        \texttt{ip-up}/\texttt{ip-down}-Skripten (\texttt{opt/etc/ppp/...}) geradezu tödlich, da
        auch nachfolgende Skripte nicht mehr ausgeführt werden. Im
        Zweifelsfall immer weglassen.


      \item KISS~-- Keep it small and simple. Du willst Perl als
        Skript-Sprache verwenden? Dir reichen die
        Skripting-Möglichkeiten von fli4l nicht?  Überdenke deine
        Einstellung! Ist dein OPT wirklich nötig? fli4l ist immer noch
        "`nur"' ein Router, ein Router sollte eigentlich keine
        Serverdienste anbieten.


      \item Die Fehlermeldung "`: not found"' heißt meistens, dass das
        Skript noch im DOS-Format vorliegt. Weitere Fehlerquelle: Das
        Skript ist nicht ausführbar. In beiden Fällen sollte die
\texttt{opt/<PACKAGE>.txt}-Datei daraufhin geprüft werden, ob sie die
korrekten Optionen (in Bezug auf "`mode"', "`gid"', "`uid"' und Flags) enthält. Wenn das Skript erst
bem Booten erzeugt wird, sollte ein "`chmod +x <Skriptname>"' ausgeführt
werden.

      \item Für temporäre Dateien sollte der Pfad \texttt{/tmp} genutzt werden.
        Es ist aber unbedingt darauf zu achten, dass hier nur wenig
        Platz ist, weil dies in der RootFS-RAM-Disk liegt! Wenn mehr
        Platz benötigt wird, muss man sich eine eigene RAM-Disk
        erstellen und mounten. Details dazu verrät der Abschnitt "`RAM-Disks"'
        in dieser Dokumentation.

    \item Damit temporäre Dateien eindeutige Namen erhalten, sollte man
      grundsätzlich die aktuelle Prozess-ID, die in der Shell-Variable "`\$"'
      gespeichert ist, an den Dateiname anhängen.
      \texttt{/tmp/<OPT-Name>.\$\$} stellt somit einen guten Dateinamen dar,
      \texttt{/tmp/<OPT-Name>} eher weniger, wobei \texttt{<OPT-Name>} natürlich
      nicht so stehen bleiben soll, sondern geeignet ersetzt werden muss.

\end{itemize}


% Synchronized to r49912

\providecommand{\fwaction}[1]{{\small\textsf{#1}}}
\providecommand{\fwchain}[1]{\texttt{#1}}
\providecommand{\fwtable}[1]{\textsc{#1}}
\providecommand{\fwmatch}[1]{\texttt{#1}}
\providecommand{\fwpktstate}[1]{\texttt{#1}}
\providecommand{\fwloglevel}[1]{\texttt{#1}}

\section{Using The Packet Filter}
\subsection{Adding Own Chains And Rules}

A set of routines is provided to manipulate the packet filter to add or delete
so-called ``chains'' and ``rules''. A chain is a named list of ordered rules.
There is a set of predefined chains (\fwchain{PREROUTING}, \fwchain{INPUT},
\fwchain{FORWARD}, \fwchain{OUTPUT}, \fwchain{POSTROUTING}), using this set of
routines more chains can be created as needed.

\begin{description}
\item [\texttt{add\_chain/add\_nat\_chain <chain>}:]
  Adds a chain to the ``filter-'' or ``nat-'' table.
\item [\texttt{flush\_chain/flush\_nat\_chain <chain>}:]
  Deletes all rules from a chain of the ``filter-'' or ``nat-'' table.
\item [\texttt{del\_chain/del\_nat\_chain <chain>}:]
  Deletes a chain from the ``filter-'' or ``nat-'' table. Chains must be empty
  prior to deleting and all references to them have to be deleted as well before.
  Such a reference i.e. can be a \fwaction{JUMP}-action with the chain defined
  as its target.
\item[\texttt{add\_rule/ins\_rule/del\_rule}:] Adds rules to the end
  (\texttt{add\_rule}) resp. at any place of a chain (\texttt{ins\_rule})
  or deletes rules from a chain (\texttt{del\_rule}). Use the syntax like here:

\begin{example}
\begin{verbatim}
    add_rule <table> <chain> <rule> <comment>
    ins_rule <table> <chain> <rule> <position> <comment>
    del_rule <table> <chain> <rule> <comment>
\end{verbatim}
\end{example}

  \noindent where the parameters have the following meaning:
  \begin{description}
  \item[table] The table in which the chain is
  \item[chain] The chain, in which the rule is to be inserted
  \item[rule] The rule which is to be inserted, the format
     corresponds to that used in the configuration file
  \item[position] The position at which the rule will be added (only
     in \texttt{ins\_rule})
  \item[comment] A comment that appears with the rule
     when somebody looks at the packet filter.
  \end{description}
\end{description}


\subsection{Integrating Into Existing Rules}

fli4l configures the packet filter with a certain default rule set.
If you want to add your own rules, you will usually want to insert
them after the default rule set. You will also need to know what the
action is desired by the user when dropping a packet. This information
can be obtained for
\fwchain{FORWARD}- and \fwchain{INPUT} chains by calling two functions,
\texttt{get\_defaults} and \texttt{get\_count}. After calling

\begin{example}
\begin{verbatim}
    get_defaults <chain>
\end{verbatim}
\end{example}

the following results are obtained:

\begin{description}
\item[\var{drop}:] This variable contains the chain to which is branched
   when a packet is discarded.
\item[\var{reject}:] This variable contains the chain to which is branched
   when a packet is rejected.
\end{description}

After calling

\begin{example}
\begin{verbatim}
    get_count <chain>
\end{verbatim}
\end{example}

the variable \var{res} contains the number of rules in the chain
\texttt{<chain>}. This position is of importance because you can \emph{not}
simply use \texttt{add\_rule} to add a rule at the end of the predefined
``filter''-chains \fwchain{INPUT}, \fwchain{FORWARD} and \fwchain{OUTPUT}.
This is because these chains are completed with a default rule valid for
all remaining packets depending on the content of the \var{PF\_<chain>\_POLICY}-variable.
Adding a rule \emph{after} this last rule hence has no effect. The function
\texttt{get\_count} instead allows to detect the position right \emph{in front of}
this last rule and to pass this position to the \texttt{ins\_rule}-function as a
parameter \texttt{<position>} in order to add the rule in the place at the end of the
appropriate chain, but right in front of this last default rule targeting
all remaining packets.

An example from the script \texttt{opt/etc/rc.d/rc390.dns\_dhcp} from the
package ``dns\_dhcp'' shall make this clear:

\begin{example}
\begin{verbatim}
    case $OPT_DHCPRELAY in
        yes)
            begin_script DHCRELAY "starting dhcprelay ..."

            idx=1
            interfaces=""
            while [ $idx -le $DHCPRELAY_IF_N ]
            do
                eval iface='$DHCPRELAY_IF_'$idx

                get_count INPUT
                ins_rule filter INPUT "prot:udp  if:$iface:any 68 67 ACCEPT" \
                    $res "dhcprelay access"

                interfaces=$interfaces' -i '$iface
                idx=`expr $idx + 1`
            done
            dhcrelay $interfaces $DHCPRELAY_SERVER

            end_script
        ;;
  esac
\end{verbatim}
\end{example}

Here you can see in the middle of the loop a call to \texttt{get\_count}
followed by a call to the \texttt{ins\_rule} function and, among other things,
the \var{res} variable is passed as \texttt{position} parameter.


\subsection{Extending The Packet Filter Tests}

fli4l uses the syntax \fwmatch{match:params} in packet filter rules
to add additional conditions for packet matching (see \fwmatch{mac:},
\fwmatch{limit:}, \fwmatch{length:}, \fwmatch{prot:}, \ldots). If you
want to add tests you have to do this as follows:

\begin{enumerate}
\item Define a suitable name. The first character of this name has
to be lower case a-z. The rest of the name can consist of any character
or digit.

\achtung{If the packet filter test is used within IPv6 rules it is
to make sure that the name is not a valid IPv6 address component!}

\item Creating a file \texttt{opt/etc/rc.d/fwrules-<name>.ext}.
The content of this file is something like this:

\begin{example}
\begin{verbatim}
    # IPv4 extension is available
    foo_p=yes

    # the actual IPv4 extension, adding matches to match_opt
    do_foo()
    {
        param=$1
        get_negation $param
        match_opt="$match_opt -m foo $neg_opt --fooval $param"
    }

    # IPv6 extension is available
    foo6_p=yes

    # the actual IPv6 extension, adding matches to match_opt
    do6_foo()
    {
        param=$1
        get_negation6 $param
        match_opt="$match_opt -m foo $neg_opt --fooval $param"
    }
\end{verbatim}
\end{example}

\mtr{The packet filter test does not have to be implemented for both IPv4 and
IPv6 (though this would be preferred if reasonable for both layer 3 protocols).}

\item Testing the extension:

\begin{example}
\begin{verbatim}
    $ cd opt/etc/rc.d
    $ sh test-rules.sh 'foo:bar ACCEPT'
    add_rule filter FORWARD 'foo:bar ACCEPT'
    iptables -t filter -A FORWARD -m foo --fooval bar -s 0.0.0.0/0 \
        -d 0.0.0.0/0 -m comment --comment foo:bar ACCEPT -j ACCEPT
\end{verbatim}
\end{example}

\item Adding the extension and all other needed files
(\texttt{iptables} components) to the archive using
the known mechanisms.
\item Allowing the extension in the configuration by extending
of \var{FW\_GENERIC\_MATCH} and/or \var{FW\_GENERIC\_MATCH6} in an exp-file,
for example:

\begin{example}
\begin{verbatim}
    +FW_GENERIC_MATCH(OPT_FOO) = 'foo:bar' : ''
    +FW_GENERIC_MATCH6(OPT_FOO) = 'foo:bar' : ''
\end{verbatim}
\end{example}
\end{enumerate}


% Last Update: $Id$

\section{CGI-Erstellung für das \emph{httpd}-Paket}

\subsection{Allgemeines zum Webserver}
Der Webserver, der bei fli4l verwendet wird, ist der \texttt{mini\_httpd} von ACME
Labs. Die Quellen können unter
\altlink{http://www.acme.com/software/mini_httpd/} heruntergeladen werden.
Allerdings wurden für fli4l ein paar Änderungen vorgenommen. 
Die Anpassungen befinden sich im \emph{src}-Paket im Verzeichnis
\texttt{src/""fbr/""buildroot/""package/""mini\_httpd}.

\subsection{Skriptnamen}

Der Skriptname sollte möglichst vielsagend sein, damit er von
anderen Skripten leichter zu unterscheiden ist und es keine
Namensüberschneidungen bei verschiedenen OPTs gibt.

Damit die Skripte ausführbar gemacht werden und DOS-Zeilenumbrüche in
Unix-Zeilen\-um\-brüche umgewandelt werden, muss in der \texttt{opt/<PAKET>.txt}
ein entsprechender Eintrag gemacht werden, siehe
Tabelle~\jump{table:options}{\ref{table:options}}.

\subsection{Menü-Einträge}

Um einen Eintrag im Menü vorzunehmen, muss eine Eintragung in der Datei
\texttt{/etc/httpd/menu} vorgenommen werden. Dieser Mechanismus erlaubt es OPTs,
auch im laufendem Betrieb Änderungen am Menü vorzunehmen. Dies sollte nur mit
dem Skript \texttt{httpd-menu.sh} gemacht werden, da dieses darauf achtet, dass
das Dateiformat dieser Datei immer konsistent ist. Um neue Menüpunkte
einzufügen, wird es folgendermaßen aufgerufen:

\begin{example}
\begin{verbatim}
    httpd-menu.sh add [-p <priority>] <link> <name> [section] [realm]
\end{verbatim}
\end{example}

So wird ein Eintrag mit dem Namen \texttt{<name>} in den Abschnitt
\texttt{[section]} eingetragen. Wenn \texttt{[section]} weggelassen wird, wird
es standardmäßig in den Abschnitt "`OPT-Pakete"' eingetragen. \texttt{<link>}
gibt das Ziel des neuen Links an. \texttt{<priority>} spezifiziert die Priorität
eines Menüeintrags in seinem Abschnitt. Wird sie nicht angegeben, wird die
Standardpriorität 500 benutzt. Die Priorität sollte eine dreistellige Nummer
sein. Je kleiner die Priorität, desto weiter oben steht der Link in dem
Abschnitt. Soll ein Eintrag möglichst weit nach unten, so ist z.\,B. die Priorität
900 zu wählen. Bei gleicher Priorität werden die Einträge nach dem Ziel des
Links sortiert. Bei \texttt{[realm]} wird der Bereich angegeben, für den ein
angemeldeter Benutzer mindestens die Berechtigung \emph{view} haben muss, damit
der Menüpunkt angezeigt wird. Wird \texttt{[realm]} nicht angegeben, wird der
Menüpunkt immer angezeigt. Siehe hierzu auch den Abschnitt
\jump{sec:rights}{"`Benutzerrechte"'}.

Beispiel:

\begin{example}
\begin{verbatim}
    httpd-menu.sh add "neuedatei.cgi" "Hier klicken" "Tools" "tools"
\end{verbatim}
\end{example}

Dieses Beispiel erzeugt im Abschnitt "`Tools"' einen Link mit dem Titel 
"`Hier klicken"' und dem Link-Ziel "`neuedatei.cgi"' und legt den Abschnitt,
falls dieser nicht vorhanden ist, an.

Das Skript kann auch Einträge aus dem Menü wieder entfernen:

\begin{example}
\begin{verbatim}
    httpd-menu.sh rem <link>
\end{verbatim}
\end{example}

Mit diesem Aufruf wird der Eintrag mit dem Link \texttt{<link>} wieder
entfernt.

\wichtig{Wenn mehrere Menüeinträge auf die gleiche Datei verweisen,
werden alle diese Einträge aus dem Menü entfernt.}

Da Abschnitte auch Prioritäten haben können, können diese auch manuell angelegt
werden. Wird ein Abschnitt automatisch beim Hinzufügen eines Eintrages
angelegt, erhält er automatisch die Priorität 500. Der Syntax zum Anlegen von
Abschnitten lautet:

\begin{example}
\begin{verbatim}
    httpd-menu.sh addsec <priority> <name>
\end{verbatim}
\end{example}

Auch hier sollte \texttt{<priority>} nur dreistellige numerische Werte annehmen.

Um sinnvolle Prioritäten in Erfahrung zu bringen, lohnt es sich, auf
einem laufenden fli4l in die Datei \texttt{/etc/httpd/menu} zu schauen, die
Prioritäten stehen in der zweiten Spalte.

Der Vollständigkeit halber wird hier kurz auf das Dateiformat der Menüdatei
eingegangen. Wem die Funktion von \texttt{httpd-menu.sh} reicht, der kann diesen
Abschnitt überspringen. Die Datei \texttt{/etc/httpd/menu} hat den folgenden
Aufbau: Sie ist in vier Spalten eingeteilt. In der ersten Spalte steht ein
Kennbuchstabe, der Überschriften und Einträge unterscheidet. In der zweiten
Spalte steht die Sortierungspriorität. Die dritte Spalte enthält bei Einträgen
das Ziel des Links und bei Überschriften einen Bindestrich, da dieses Feld bei
Überschriften keine Bedeutung hat. Im Rest der Zeile steht der Text, der
nachher im Menü erscheint.

Überschriften benutzen den Kennbuchstaben "`t"', dann wird ein neuer
Menüabschnitt begonnen. Normale Menüeinträge benutzen den Kennbuchstaben "`e"'.
Ein Beispiel:

\begin{example}
\begin{verbatim}
    t 300 - Mein tolles OPT
    e 200 meinopt.cgi Mach etwas Tolles
    e 500 meinopt.cgi?mehr=ja Mach mehr Tolles
\end{verbatim}
\end{example}

Beim Bearbeiten dieser Datei muss man darauf achten, dass das
\texttt{httpd-menu.sh}-Skript die Datei immer sortiert abspeichert. Die
einzelnen Abschnitte sind sortiert und die Einträge pro Abschnitt sind in diesem
Abschnitt sortiert. Der genaue Sortieralgorithmus kann aus
\texttt{httpd-menu.sh} übernommen werden~-- besser wäre allerdings, dieses
Skript um mögliche neue Funktionen zu erweitern, damit alle Menü-Bearbeitungen
an zentraler Stelle passieren.

\subsection{Aufbau eines CGI-Skriptes}

\subsubsection{Die Kopfzeilen}
Alle Skripte des Webservers sind einfache Shell-Skripte (Interpreter wie z.\,B.
Perl, PHP, etc.\ sind viel zu groß für fli4l). Sie sollten mit dem
obligatorischen Skript-Header anfangen (Verweis auf den Interpreter,
Name, Sinn des Skriptes, Autor, Lizenz).

\subsubsection{Das Hilfsskript "'cgi-helper"'}
Gleich nach den Kopfzeilen sollte dann schon das Hilfsskript \texttt{cgi-helper}
mit folgendem Aufruf eingebunden werden:

\begin{example}
\begin{verbatim}
    . /srv/www/include/cgi-helper
\end{verbatim}
\end{example}

Wichtig ist ein Leerzeichen zwischen Punkt und Schrägstrich!

Dieses Skript stellt diverse Hilfsfunktionen bereit, die das Erstellen von CGIs
für fli4l wesentlich vereinfachen sollen. Außerdem werden mit dem Einbinden
des \texttt{cgi-helper} auch noch Standardaufgaben ausgeführt, wie
beispielsweise das Parsen von Variablen, die mit Formularen oder über die URL
übergebenen wurden, oder das Laden von Sprach- und CSS-Dateien.

Tabelle~\ref{tab:dev:cgi-helper} gibt einen Überblick über die Funktionen des
\texttt{cgi-helper}-Skriptes.

\begin{table}[htbp]
  \centering
  \caption{Funktionen des \texttt{cgi-helper}-Skriptes}
  \label{tab:dev:cgi-helper}
  \begin{small}
    \begin{tabular}{|l|p{0.8\textwidth}|}
      \hline
      Name                         & Funktion         \\
      \hline
      \texttt{check\_rights}      & Überprüfung der Benutzerrechte \\
      \texttt{http\_header}       & Ausgabe eines Standard-HTTP-Headers oder eines speziellen HTTP-Headers, beispielsweise zum Download von Dateien\\
      \texttt{show\_html\_header} & Ausgabe des kompletten Seitenheaders (inkl. HTTP-Header, Kopfzeile und Menü)\\
      \texttt{show\_html\_footer} & Ausgabe des Abschlusses der HTML-Seite \\
      \texttt{show\_tab\_header}  & Ausgabe eines Inhalt-Fensters mit Reitern\\
      \texttt{show\_tab\_footer}  & Ausgabe des Abschlusses des Inhaltsfensters\\
      \texttt{show\_error}        & Ausgabe einer Box für Fehlermeldungen (Hintergrundfarbe: rot)\\
      \texttt{show\_warn}         & Ausgabe einer Box für Warnmeldungen (Hintergrundfarbe: gelb)\\
      \texttt{show\_info}         & Ausgabe einer Box für Informationen oder Erfolgsmeldungen (Hintergrundfarbe: grün)\\
      \hline
    \end{tabular}
  \end{small}
\end{table}

\subsubsection{Der Inhalt eines CGI-Skriptes}

Um ein einheitliches Design und vor allem die Kompatibilität
mit zukünftigen fli4l-Versionen zu gewährleisten, ist es sehr zu empfehlen, die
Funktionen des Hilfsskriptes \texttt{cgi-helper} zu benutzen, auch wenn man in
einem CGI theoretisch alle Ausgaben selbst generieren kann.

Eine einfaches CGI-Skript könnte wie folgt aussehen:

\begin{example}
\begin{verbatim}
    #!/bin/sh
    # --------------------
    # Header (c) Autor Datum
    # --------------------
    # get main helper functions
    . /srv/www/include/cgi-helper

    show_html_header "Mein erstes CGI"
    echo '   <h2>Willkommen</h2>'
    echo '   <h3>Dies ist ein Beispiel-CGI-Skript</h3>'
    show_html_footer
\end{verbatim}
\end{example}

\subsubsection{Die Funktion \texttt{show\_html\_header}}

Die Funktion \texttt{show\_html\_header} erwartet eine Zeichenkette als
Parameter. Diese Zeichenkette stellt den Titel der zu generierenden Seite dar.
Die Funktion generiert automatisch das Menü und bindet ebenso automatisch zum
Skript gehörende CSS- und Sprachdateien ein. Voraussetzung dafür ist, dass
diese sich in den Verzeichnissen \texttt{/srv/www/css}
bzw.\ \texttt{/srv/www/lang} befinden und denselben Namen (aber natürlich eine
andere Endung) wie das Skript haben. Ein Beispiel:

\begin{example}
\begin{verbatim}
    /srv/www/admin/OpenVPN.cgi
    /srv/www/css/OpenVPN.css
    /srv/www/lang/OpenVPN.de
\end{verbatim}
\end{example}

Sowohl das Benutzen von Sprachdateien als auch von CSS-Dateien ist optional.
Immer eingebunden werden \texttt{css/main.css} und \texttt{lang/main.<lang>},
wobei \texttt{<lang>} der gewählten Sprache entspricht.

Der Funktion \texttt{show\_html\_header} können aber neben dem Titel noch
weitere Parameter übergeben werden. Ein Aufruf mit allen möglichen Parametern
könnte so aussehen:

\begin{example}
\begin{verbatim}
    show_html_header "Titel" "refresh=$time;url=$url;cssfile=$cssfile;showmenu=no"
\end{verbatim}
\end{example}

Alle zusätzlichen Parameter müssen, wie im Beispiel gezeigt, mit
Anführungszeichen umschlossen und durch ein Semikolon getrennt werden. Eine
Überprüfung der Syntax erfolgt \emph{nicht}! Es ist also notwendig, sehr genau
auf die Parameterübergabe zu achten.

Hier eine kurze Übersicht über die Funktion der Parameter:

\begin{itemize}
 \item \texttt{refresh=}\emph{time}: Zeit in Sekunden in der die Seite vom
    Browser neu geladen werden soll
 \item \texttt{url=}\emph{url}: die URL, die bei einem Refresh neu geladen wird
 \item \texttt{cssfile=}\emph{cssfile}: Name einer CSS-Datei, wenn diese vom
    Namen des CGIs abweicht
 \item \texttt{showmenu=no}: unterdrückt die Anzeige des Menüs und des Headers
\end{itemize}

Sonstige Richtlinien zum Inhalt:

\begin{itemize}
 \item Fasst euch kurz :-)
 \item Schreibt sauberes HTML (SelfHTML\footnote{siehe
    \altlink{http://de.selfhtml.org/}} ist da ein guter Ansatzpunkt).
 \item Verzichtet auf hochmodernen Schnick-Schnack (JavaScript ist i.\,O., wenn
    es nicht stört, sondern den Benutzer unterstützt, das Ganze muss auch ohne
    JavaScript funktionieren).
\end{itemize}

\subsubsection{Die Funktion \texttt{show\_html\_footer}}

Die Funktion \texttt{show\_html\_footer} schließt den Block im CGI-Skript ab,
der durch die Funktion \texttt{show\_html\_header} eingeleitet wurde.

\subsubsection{Die Funktion \texttt{show\_tab\_header"'}}

Damit der Inhalt eurer mit Hilfe des CGIs erzeugten Webseite auch hübsch
geordnet aussieht, könnt ihr die \texttt{cgi-helper}-Funktion
\texttt{show\_tab\_header} nutzen. Sie erzeugt dann anklickbare Reiter ("`Tabs"'
genannt), so dass ihr eure Seite in mehrere logisch voneinander getrennte
Bereiche unterteilen könnt.

Der \texttt{show\_tab\_header}-Funktion werden Parameter immer in Paaren
übergeben. Der erste Wert entspricht dem Titel eines Reiters, der zweite dem
Link. Wird als Link die Zeichenkette "`no"' übergeben, wird lediglich der Titel
ausgegeben und der Reiter ist nicht anklickbar (und blau).

Im folgenden Beispiel wird ein "`Fenster"' mit dem Titel "`Ein tolles Fenster"'
erzeugt. Im Fenster steht "`foo bar"':

\begin{example}
\begin{verbatim}
    show_tab_header "Ein tolles Fenster" "no"
    echo "foo"
    echo "bar"
    show_tab_footer
\end{verbatim}
\end{example}

Im nächsten Beispiel werden zwei anklickbare Reiter generiert, die dem
Skript die Variable \var{action} mit verschiedenen Werten übergibt.

\begin{example}
\begin{verbatim}
    show_tab_header "1. Auswahltab" "$myname?action=machdies" \
                    "2. Auswahltab" "$myname?action=machjenes"
    echo "foo"
    echo "bar"
    show_tab_footer
\end{verbatim}
\end{example}

Nun kann das Skript den Inhalt der Variablen \var{FORM\_action} (siehe weiter
unten zur Variablenauswertung) auswerten und je nachdem unterschiedliche Inhalte
bereitstellen. Damit der angeklickte Reiter auch ausgewählt erscheint und nicht
mehr angeklickt werden kann, müsste der Funktion statt dem Link wie schon
erwähnt ein "`no"' übergeben werden. Das geht aber auch einfacher, wenn man sich
an die Konvention in folgendem Beispiel hält:

\begin{example}
\begin{verbatim}
    _opt_machdies="1. Auswahltab"
    _opt_machjenes="2. Auswahltab"
    show_tab_header "$_opt_machdies" "$myname?action=opt_machdies" \
                    "$_opt_machjenes" "$myname?action=opt_machjenes"
    case $FORM_action in
        opt_machdies) echo "foo" ;;
        opt_machjenes) echo "bar" ;;
    esac
    show_tab_footer
\end{verbatim}
\end{example}

Wird also für den Titel eine Variable verwendet, deren Namen dem Inhalt der
Variablen \var{action} mit führendem Unterstrich (\texttt{\_}) entspricht, wird
der entsprechende Reiter ausgewählt dargestellt.

\subsubsection{Die Funktion \texttt{show\_tab\_footer}}

Die Funktion \texttt{show\_tab\_footer} schließt den Block im CGI-Skript ab,
der durch die Funktion \texttt{show\_tab\_header} eingeleitet wurde.

\subsubsection{Mehrsprachfähigkeit}

Das Hilfsskript \texttt{cgi-helper} enthält weiterhin Funktionen, um CGI-Skripte
mehrsprachfähig zu machen. Dazu müssen "`nur"' für alle Textausgaben Variablen
mit führenden Unterstrichen (\texttt{\_}) verwendet und diese Variablen in den
entsprechenden Sprachdateien definiert werden.

Beispiel:

\texttt{lang/opt.de} enthalte:

\begin{example}
\begin{verbatim}
    _opt_machdies="Eine Ausgabe"
\end{verbatim}
\end{example}

\texttt{lang/opt.en} enthalte:

\begin{example}
\begin{verbatim}
    _opt_machdies="An Output"
\end{verbatim}
\end{example}

\texttt{admin/opt.cgi} enthalte:

\begin{example}
\begin{verbatim}
    ...
    echo $_opt_machdies
    ...
\end{verbatim}
\end{example}


\subsubsection{Formular-Auswertung}

Um Formulare zu verarbeiten, muss man noch einige Dinge wissen. Egal ob die
Formular-Methode \var{GET} oder \var{POST} verwendet wird, die Parameter finden
sich nach dem Einbinden des \texttt{cgi-helper}-Skripts (welches wiederum das
Hilfsprogramm proccgi aufruft) in den Variablen \var{FORM\_<Parameter>} wieder.
Wenn das Formularfeld also den Namen "`eingabe"' hatte, kann im CGI-Skript
mit \var{\$FORM\_eingabe} darauf zugegriffen werden.

Weitere Informationen zum Programm \texttt{proccgi} finden sich unter
\altlink{http://www.fpx.de/fp/Software/ProcCGI.html}.

\marklabel{sec:rights}{
    \subsubsection{Benutzerrechte: Die Funktion \texttt{check\_rights}}
}

Um zu prüfen, ob der Benutzer ausreichende Rechte zur Nutzung eines CGI-Skripts
besitzt, muss am Anfang des CGI-Skripts die Funktion \texttt{check\_rights}
wie folgt aufgerufen werden:

\begin{example}
\begin{verbatim}
    check_rights <Bereich> <Aktion>
\end{verbatim}
\end{example}

Das CGI-Skript wird dann nur ausgeführt, wenn der aktuell angemeldete Benutzer
\begin{itemize}
\item alle Rechte hat (\verb+HTTPD_RIGHTS_x='all'+), oder
\item alle Rechte für den angegebenen Bereich hat
    (\verb+HTTPD_RIGHTS_x='<Bereich>:all'+), oder
\item das Recht hat, die angegebene Aktion im angegebenen Bereich auszuführen\\
    (\verb+HTTPD_RIGHTS_x='<Bereich>:<Aktion>'+).
\end{itemize}

% More examples?

\subsubsection{Die Funktion \texttt{show\_error}}

Diese Funktion gibt eine Fehlermeldung in einer roten Box aus. Sie erwartet
zwei Parameter: einen Titel sowie eine Meldung. Beispiel:

\begin{example}
\begin{verbatim}
    show_error "Error: No key" "No key was specified!"
\end{verbatim}
\end{example}

\subsubsection{Die Funktion \texttt{show\_warn}}

Diese Funktion gibt eine Warnmeldung in einer gelben Box aus. Sie erwartet
zwei Parameter: einen Titel sowie eine Meldung. Beispiel:

\begin{example}
\begin{verbatim}
    show_info "Warnung" "Derzeit besteht keine Verbindung!"
\end{verbatim}
\end{example}

\subsubsection{Die Funktion \texttt{show\_info}}

Diese Funktion gibt eine Informations- oder Erfolgsmeldung in einer grünen Box
aus. Sie erwartet zwei Parameter: einen Titel sowie eine Meldung. Beispiel:

\begin{example}
\begin{verbatim}
    show_info "Info" "Aktion wurde erfolgreich ausgeführt!"
\end{verbatim}
\end{example}

\subsubsection{Das Hilfsskript "'cgi-helper-ip4"'}

Gleich nach den Hilfsskript "cgi-helper" kann dann das Hilfsskript cgi-helper-ip4
mit folgendem Aufruf eingebunden werden:

\begin{example}
\begin{verbatim}
. /srv/www/include/cgi-helper-ip4
\end{verbatim}
\end{example}

Wichtig ist ein Leerzeichen zwischen Punkt und Schrägstrich!

Dieses Skript stellt Hilfsfunktionen bereit, um Prüfungen von IPv4-Adressen
vornehmen zu können.

\subsubsection{Die Funktion \texttt{ip4\_isvalidaddr}}

Diese Funktion überprüft, ob eine gültige IPv4-Adresse übergeben wurde.
Beispiel:

\begin{example}
\begin{verbatim}
    if ip4_isvalidaddr ${FORM_inputip}
    then
        ...
    fi
\end{verbatim}
\end{example}

\subsubsection{Die Funktion \texttt{ipv4\_normalize}}

Diese Funktion entfernt aus der übergebenen IPv4-Adresse führende Nullen.
Beispiel:

\begin{example}
\begin{verbatim}
    ip4_normalize ${FORM_inputip}
    IP=$res
    if [ -n "$IP" ]
    then
        ...
    fi
\end{verbatim}
\end{example}

\subsubsection{Die Funktion \texttt{ipv4\_isindhcprange}}

Diese Funktion prüft, ob die übergebene IPv4-Adresse sich im Bereich der
übergebenen Start- und Zieladresse befindet. Beispiel:
 
\begin{example}
\begin{verbatim}
    if ip4_isindhcprange $FORM_inputip $ip_start $ip_end
    then
        ...
    fi
\end{verbatim}
\end{example}

\subsection{Sonstiges}

Dies und das (ja, das ist auch noch wichtig!):

\begin{itemize}
 \item Der \texttt{mini\_httpd} schützt Unterverzeichnisse nicht mit einem
    Passwort. Es muss für jedes Verzeichnis eine eigene \texttt{.htaccess}-Datei
    oder ein Link auf eine andere \texttt{.htaccess}-Datei angelegt werden.
 \item KISS - Keep it simple, stupid!
 \item Diese Angaben können sich jederzeit ohne Vorankündigung ändern!
\end{itemize}

\subsection{Fehlersuche}

Um die Fehlersuche innerhalb eines CGI-Skripts zu erleichtern, kann man vor
der Einbindung des \texttt{cgi-helper}-Skripts den Debugging-Modus aktivieren.
Dazu muss die Variable \var{set\_debug} auf den Wert "`yes"' gesetzt werden.
Dies führt zur Erstellung der Datei \texttt{debug.log}, die über die URL
\texttt{http://<fli4l-Host>/admin/debug.log} heruntergeladen werden kann.
Diese enthält alle Aufrufe des CGIs. Die \texttt{set\_debug}-Variable ist nicht
global, muss also in jedem Problem-CGI erneut gesetzt werden. Beispiel:

\begin{example}
\begin{verbatim}
    set_debug="yes"
    . /srv/www/include/cgi-helper
\end{verbatim}
\end{example}

Alternativ kann auch die jeweilige CGI-URL um den Parameter \texttt{debug=yes}
ergänzt werden, etwa \texttt{http://<fli4l-Host>/admin/meinopt.cgi?debug=yes}.

Des Weiteren eignet sich cURL\footnote{siehe
\altlink{http://de.wikipedia.org/wiki/CURL}} hervorragend zur Fehlersuche,
insbesondere wenn die HTTP-Kopfzeilen nicht korrekt zusammengesetzt werden oder
der Browser nur weiße Seiten anzeigt. Auch ist das Cachingverhalten moderner
Webbrowser bei der Fehlersuche hinderlich.

Beispiel: Mit dem folgenden Aufruf wird der HTTP-Header ("`\emph{d}ump"',
\texttt{-D}) sowie die normale Ausgabe des CGIs \texttt{admin/mein.cgi}
ausgegeben. Es soll der Benutzername ("`\emph{u}ser"', \texttt{-u}) "`admin"'
verwendet werden.

\begin{example}
\begin{verbatim}
    curl -D - http://fli4l/admin/mein.cgi -u admin
\end{verbatim}
\end{example}


% Last Update: $Id$

\section{Hochfahren, Herunterfahren, Einwählen und Auflegen unter fli4l}

\subsection{Bootkonzept}

fli4l 2.0 sollte eine saubere Installation auf eine Festplatte oder ein
CompactFlash(TM)-Medium bieten, aber auch eine Installation auf ein
Zip-Medium oder die Erstellung einer bootfähigen CD-ROM sollte möglich
sein. Zusätzlich sollte die Festplattenversion sich nicht grundlegend von
einer Installation auf Diskette\footnote{Ursprünglich konnte fli4l auch
von einer Diskette betrieben werden. Da fli4l inzwischen dafür zu groß geworden
ist, wird dies nicht mehr unterstützt.} unterscheiden.

Diese Anforderungen wurden realisiert, indem die Dateien des
\texttt{opt.img}-Archivs aus der
bisherigen RAM-Disk auf eine anderes Medium verlagert werden können. Dies
kann eine Partition auf einer Festplatte bzw.\ einem CF-Medium sein. Dieses
zweite Volume wird unter \texttt{/opt} eingehängt, und dort liegende Programme
werden nur noch über symbolische Links in das RootFS integriert. Das entstehende Layout im
RootFS-Dateisystem entspricht dann dem im \texttt{opt}-Verzeichnis der
ausgepackten fli4l-Distribution mit einer Ausnahme~-- das \texttt{files}-Präfix
entfällt. Die Datei \texttt{opt/etc/rc} ist dann also direkt unter \texttt{/etc/rc} zu
finden, \texttt{opt/bin/busybox} unter
\texttt{/bin/busybox}. Dass diese Dateien unter Umständen nur symbolische Verknüpfungen
auf ein im Nur-Lese-Modus eingehängtes Volume sind, kann man ignorieren,
solange man die Dateien nicht modifizieren möchte. Will man dies tun,
muss man die Dateien vorher mit \texttt{mk\_writable} (s.\,u.) schreibbar machen.

\subsection{Start- und Stopp-Skripte}

Skripte, die beim Hochfahren des Systems ausgeführt werden sollen, befinden
sich in den Verzeichnissen \texttt{opt/etc/boot.d/} und \texttt{opt/etc/rc.d/}
und werden auch in dieser Reihenfolge ausgeführt. Des Weiteren befinden sich in
\texttt{opt/etc/rc0.d/} Skripte, die beim Herunterfahren des Systems ausgefühlt
werden.

\wichtig{Da zum Ausführen dieser Skripte kein separater Prozess
erzeugt wird, dürfen sie nicht mit "`exit"' abgeschlossen
werden. Ein solcher Befehl führt zum vorzeitigen Abbruch des Bootvorgangs!}

\subsubsection{Start-Skripte in \texttt{opt/etc/boot.d/}}

Skripte, die in diesem Verzeichnis liegen, werden als erstes
ausgeführt. Ihre Aufgabe ist es, das Boot-Volume einzuhängen, die
auf dem Boot-Medium liegende Konfigurationsdatei \texttt{rc.cfg} einzulesen und
das Opt-Archiv zu entpacken. Je nach \jump{BOOTTYPE}{Boot-Typ}
sind diese Skripte mehr oder weniger kompliziert und tun die folgenden
Dinge:

\begin{itemize}
\item Laden von Hardware-Treibern (optional)
\item Boot-Volume einhängen (optional)
\item Konfigurationsdatei \texttt{rc.cfg} vom Boot-Volume einlesen und
      in die Datei \texttt{/etc/rc.cfg} schreiben
\item Einhängen des Opt-Volumes (optional)
\item Extrahieren des Opt-Archivs (optional)
\end{itemize}

Damit die Skripte überhaupt eine Chance haben, etwas über die
fli4l-Konfiguration zu erfahren, wird die Konfigurationsdatei auch ins
RootFS-Archiv unter \texttt{/etc/rc.cfg} eingebunden; die
Konfigurationsvariablen in dieser Datei stehen den Start-Skripten in
\texttt{opt/etc/boot.d/} direkt zur Verfügung. Nach dem Einhängen des
Boot-Volumes wird die \texttt{/etc/rc.cfg} durch die Konfigurationsdatei
vom Boot-Volume ersetzt, so dass den Start-Skripten in \texttt{opt/etc/rc.d/}
(s.\,u.) die aktuelle Konfiguration vom Boot-Volume zur Verfügung steht.
\footnote{Normalerweise sind diese beiden Dateien identisch. Eine Abweichung
entsteht nur, wenn die Konfigurationsdatei auf dem Boot-Volume händisch
editiert wird, etwa um die Konfiguration nachträglich abzuändern, ohne die
fli4l-Archive neu zu bauen.}

\subsubsection{Start-Skripte in \texttt{opt/etc/rc.d/}}

Befehle, die immer beim Starten des Routers ausgeführt werden müssen,
können in Skripten im Verzeichnis \texttt{opt/etc/rc.d/} abgelegt werden. Hierbei
gelten folgende Konventionen:

\begin{enumerate}
  \item Es gilt folgende Namenskonvention:

\begin{example}
\begin{verbatim}
    rc<dreistellige Zahl>.<Name des OPTs>
\end{verbatim}
\end{example}
      
    Die Skripte werden in aufsteigender Reihenfolge der Zahlen
    gestartet. Ist mehreren Skripten dieselbe Zahl zugeordnet,
    entscheidet die alphabetische Sortierung nach dem Punkt. Falls
    der Start eines Pakets erst nach einem anderen erfolgen darf,
    wird das durch die Zahl festgelegt.

    Hier eine ungefähre Richtlinie, welche Nummern für welche Aufgaben
    verwendet werden sollten:

    \begin{table}[htbp]
    \centering
    \begin{tabular}{lp{0.8\textwidth}}
            \hline
            Nummer        &       Aufgabe         \\
            \hline
            \hline
            000-099       &       Grundsystem (Hardware, Zeitzone, Dateisystem) \\
            100-199       &       Kernel-Module (Treiber) \\
            200-299       &       externe Verbindungen (PPPoE, ISDN4Linux, PPtP) \\
            300-399       &       Netzwerk (Routing, Interfaces, Paketfilter) \\
            400-497       &       Server (DHCP, HTTPD, Proxy, etc.) \\
            498-499       &       reserviert (Aktivierung des Circuit-Systems) \\
            500-997       &       nach Belieben (ab hier können Skripte von Circuits
                                  kontrollierte Verbindungen nutzen und laufen
                                  möglicherweise parallel zu einer Einwahl) \\
            998-999       &       reserviert (bitte nicht benutzen!) \\
            \hline
    \end{tabular}
    \end{table}

  \item In diesen Skripten \emph{müssen} alle Funktionen, die das RootFS
    verändern, hinterlegt werden, etwa das Anlegen eines
    Verzeichnisses \texttt{/var/log/lpd}.

  \item In diesen Skripten dürfen \emph{keine} Schreibzugriffe auf
    Dateien erfolgen, die Teil des Opt-Archivs sein können, da diese Dateien
    auf einem im Nur-Lese-Modus eingehängten Volume liegen können.
    Muss man eine solche Datei modifizieren, muss man sie
    vorher mit Hilfe der Funktion \texttt{mk\_writable} (s.\,u.) beschreibbar
    machen. Durch den Aufruf der Funktion wird die Datei (wenn nötig) als
    beschreibbare Kopie im RootFS abgelegt. Ist die Datei bereits beschreibbar,
    bewirkt der \texttt{mk\_writable}-Aufruf nichts.
    
    \wichtig{\texttt{mk\_writable} muss direkt auf Dateien im RootFS angewandt
    werden, nicht über den Umweg des \texttt{opt}-Verzeichnisses. Will man
    also \texttt{/usr/local/bin/foo} modifizieren, ruft man
    \texttt{mk\_writable} mit dem Argument \texttt{/usr/local/bin/foo} auf.}

  \item Diese Skripte müssen vor der Ausführung der eigentlichen
    Befehle prüfen, ob das dazugehörige OPT auch aktiv ist. Das ist
    normalerweise durch eine einfache Fallunterscheidung erledigt:

\begin{example}
\begin{verbatim}
    if [ "$OPT_<OPT-Name>" = "yes" ]
    then
        ...
        # Hier OPT starten!
        ...
    fi
\end{verbatim}
\end{example}

  \item Um die Fehlersuche zu erleichtern, sollten die Skripte mit
    \texttt{begin\_script} und \texttt{end\_script} geklammert werden:

\begin{example}
\begin{verbatim}
    if [ "$OPT_<OPT-Name>" = "yes" ]
    then
        begin_script FOO "configuring foo ..."
        ...
        end_script
    fi
\end{verbatim}
\end{example}

    Die Fehlersuche einzelner Start-Skripte kann man dann einfach via
    \verb+FOO_DO_DEBUG='yes'+ aktivieren.

  \item Den Skripten stehen alle Konfigurationsvariablen
    direkt zur Verfügung. Im Abschnitt \jump{dev:sec:config-variables}{"`Umgang
    mit Konfigurationsvariablen"'} wird erklärt, wie man von anderen Skripten
    aus auf die Konfigurationsvariablen zugreifen kann.

  \item Der Pfad \texttt{/opt} darf auch nicht als Speicherplatz für
    Datenbestände der OPTs benutzt werden. Falls zusätzlicher
    Speicherplatz benötigt wird, sollte dem Benutzer über eine
    Konfigurationsvariable die Möglichkeit gegeben werden, einen geeigneten
    Pfad auszuwählen. Je nach Art der zu speichernden Daten (persistente oder
    transiente Daten) sind verschiedene Standard-Belegungen sinnvoll. So bieten
    sich für transiente Daten etwa Pfade unterhalb von \texttt{/var/run/} an;
    für persistente Daten sollte die Funktion
    \jump{dev:sec:persistent-data}{map2persistent} in Kombination mit einer
    geeigneten Konfigurationsvariable verwendet werden.

\end{enumerate}

\subsubsection{Stopp-Skripte in \texttt{opt/etc/rc0.d/}}

Jeder Rechner muss mal heruntergefahren oder neu gestartet werden. Nun kann
es vorkommen, dass man Vorgänge ausführen muss, bevor der Rechner
heruntergefahren oder neu gestartet wird. Zum Herunterfahren und Neustarten
sind die Befehle "`halt"' bzw.\ "`reboot"' zuständig. Diese Befehle werden auch
aufgerufen, wenn man im IMONC oder in der Web-GUI auf die entsprechenden
Schaltflächen klickt.

Alle Stopp-Skripte liegen im Verzeichnis \texttt{opt/etc/rc0.d/}. Ihre Dateinamen
werden analog zu den Start-Skripten gebildet. Sie werden ebenfalls in
\emph{auf}steigender Reihenfolge der Zahlen ausgeführt.

\subsection{Hilfsfunktionen}

In \texttt{/etc/boot.d/base-helper} werden verschiedene Funktionen
bereitgestellt, die von den Start- und anderen Skripten verwendet werden
können. Das betrifft Dinge wie Unterstützung zur Fehlersuche, Laden von
Kernel-Modulen oder Ausgabe von Meldungen.
Die einzelnen Funktionen werden im Folgenden aufgelistet und kurz beschrieben.

\subsubsection{Skript-Steuerung}

\begin{description}

\item[\texttt{begin\_script <Symbol> <Meldung>}:]
Gibt eine Meldung aus und aktiviert die Fehlersuche im Skript mittels
\texttt{set -x}, falls \var{<Symbol>\_DO\_DEBUG} auf "`yes"' steht.

\item[\texttt{end\_script}:]
Gibt eine Abschluss-Meldung aus und deaktiviert
die Fehlersuche, falls sie mit \texttt{begin\_script} aktiviert wurde. Für
jeden \texttt{begin\_script}-Aufruf muss es einen zugehörigen
\texttt{end\_script}-Aufruf geben (und umgekehrt).

\end{description}

\subsubsection{Laden von Kernel-Modulen}

\begin{description}

\item[\texttt{do\_modprobe [-q] <Modul> <Parameter>*}:]
Lädt ein Kernel-Modul inkl.
eventueller Parameter bei gleichzeitiger Auflösung der Modulabhängigkeiten.
Der Parameter "`-q"' verhindert, dass im Fehlerfall eine Meldung auf der Konsole
ausgegeben und ins Boot-Protokoll geschrieben wird.
Die Funktion liefert im Erfolgsfall den Rückgabewert null zurück und im
Fehlerfall einen Wert ungleich null. Damit lässt sich Code schreiben, der
ein Fehlschlagen des Ladens eines Kernel-Moduls geeignet behandelt:

\begin{example}
\begin{verbatim}
    if do_modprobe -q acpi-cpufreq
    then
        # kein CPU-Frequenzsteuerung via ACPI
        log_error "CPU-Frequenzsteuerung via ACPI nicht verfügbar!"
        # [...]
    else
        log_info "CPU-Frequenzsteuerung via ACPI aktiviert."
        # [...]
    fi
\end{verbatim}
\end{example}

\item[\texttt{do\_modrobe\_if\_exists [-q] <Modulpfad> <Modul> <Parameter>*}:]\mbox{}\\
Prüft, ob das Modul \texttt{/lib/modules/<Kernel-Version>/<Modulpfad>/<Modul>}
existiert und ruft bei Vorhandensein die Funktion \texttt{do\_modprobe} auf.

\wichtig{Das Modul muss tatsächlich unter dem angegebenen Modulnamen existieren,
der Modulname darf kein Alias sein. Bei einem Alias wird direkt
\texttt{do\_modprobe} aufgerufen.}

\end{description}

\subsubsection{Meldungen und Fehlerbehandlung}

\begin{description}

\item[\texttt{log\_info <Meldung>}:] Schreibt eine Meldung auf
die Konsole und nach \texttt{/bootmsg.txt}. Wird keine Meldung als
Parameter übergeben, liest \texttt{log\_info} von der Standard-Eingabe. Die
Funktion liefert als Rückgabewert immer null zurück.

\item[\texttt{log\_warn <Meldung>}:] Schreibt eine Warnmeldung auf
die Konsole und nach \texttt{/bootmsg.txt}, wobei vor die Meldung die
Zeichenkette \texttt{WARN:} gesetzt wird. Wird keine Meldung als
Parameter übergeben, liest \texttt{log\_warn} von der Standard-Eingabe.
Die Funktion liefert als Rückgabewert immer null zurück.

\item[\texttt{log\_error <Meldung>}:] Schreibt eine Fehlermeldung auf
die Konsole und nach \texttt{/bootmsg.txt}, wobei vor die Meldung die
Zeichenkette \texttt{ERR:} gesetzt wird. Wird keine Meldung als
Parameter übergeben, liest \texttt{log\_error} von der Standard-Eingabe.
Die Funktion liefert als Rückgabewert immer einen Wert ungleich null zurück.

\item[\texttt{set\_error <Meldung>}:] Gibt die Fehlermeldung aus und setzt
eine interne Fehlervariable, das später via \texttt{is\_error} geprüft werden
kann.

\item[\texttt{is\_error}:] Setzt die interne Fehlervariable zurück und liefert
wahr zurück, falls sie vorher durch \texttt{set\_error} gesetzt wurde.

\end{description}

\subsubsection{Netzwerk-Funktionen}

\begin{description}

\item[\texttt{translate\_ip\_net <Wert> <Variablenname> [<Ergebnisvariable>]}:]
Ersetzt symbolische Referenzen in
Parametern. Momentan finden folgende Übersetzungen statt:
\begin{description}
\item[Host- oder Netzwerk-Adressen] werden nicht übersetzt
\item[\texttt{any}] wird durch \texttt{0.0.0.0/0} ersetzt
\item[\texttt{dynamic}] wird durch die dynamische IPv4-Adresse des Routers
ersetzt; das ist die IPv4-Adresse, die der Router beim Einwählen über einen
Circuit mit einer Default-Route zugewiesen bekommt
\item[\var{IP\_NET\_x}] wird durch das in der Konfiguration stehende
Netzwerk ersetzt; referenziert die Variable einen Circuit, so wird das dem
Circuit zugewiesene IPv4-Netz zurückgegeben
\item[\var{IP\_NET\_x\_IPADDR}] wird durch die in der Konfiguration stehende
IP-Adresse ersetzt; referenziert die Variable einen Circuit, so wird die dem
Circuit zugewiesene IPv4-Adresse zurückgegeben
\item[\var{IP\_ROUTE\_x}] wird durch das in der Konfiguration stehende
geroutete Netzwerk ersetzt
\item[\texttt{@<Hostname>}] wird durch die in der Konfiguration für den
angegeben Host spezifizierte IP-Adresse ersetzt
\item[\texttt{\{<circuit>\}}] wird durch das dem Circuit zugewiesene IPv4-Netz
ersetzt
\end{description}

Das Ergebnis der Übersetzung wird in der Variable gespeichert, deren Name im
dritten Parameter übergeben wird; fehlt dieser Parameter, wird das Ergebnis in
der Variable \var{res} gespeichert. Der Variablenname, der im zweiten Parameter
übergeben wird, wird nur für Fehlermeldungen benutzt, falls die Übersetzung
fehlschlägt; hier kann also vom Aufrufer die Quelle des zu übersetzenden Wertes
angegeben werden. Im Fehlerfall wird dann eine Meldung wie

\begin{example}
\begin{verbatim}
    Unable to translate value '<Wert>' contained in <Variablenname>.
\end{verbatim}
\end{example}

ausgegeben.

Der Rückgabewert ist null, falls die Übersetzung erfolgreich war, und ungleich
null, falls ein Fehler aufgetreten ist.

\end{description}

\subsubsection{Diverses}

\begin{description}

\item[\texttt{mk\_writable <Datei>}:]
Stellt sicher, dass die übergebene Datei beschreibbar ist. Befindet sich die
Datei auf einem im Nur-Lese-Modus eingehängten Volume und ist lediglich über
eine symbolische Verknüpfung ins Dateisystem eingebunden, wird eine lokale Kopie
angelegt, die dann beschreibbar ist.

\item[\texttt{list\_unique <Liste>}:]
Entfernt Duplikate aus der übergebenen Liste. Das Resultat wird in die
Standard-Ausgabe geschrieben.

\end{description}

\subsection{mdev-Regeln}

Für OPTs ist es möglich, zusätzliche mdev-Regeln zu etablieren, die spezielle
Aktionen beim Erscheinen oder Verschwinden bestimmter Geräte vornehmen. Das
\var{OPT\_AUTOMOUNT} im hd-Paket verwendet beispielsweise eine solche Regel,
um auftauchende Speichermedien automatisch einzuhängen. Will man eine
zusätzliche mdev-Regel integrieren, muss man ein Skript der Form

\begin{verbatim}
    /etc/mdev.d/mdev<Nummer>.<Name>
\end{verbatim}

ins RootFS einbauen, wobei die Nummer ähnlich den Start- und Stopp-Skripten aus
drei Ziffern bestehen muss und der Name beliebig gewählt werden kann. Innerhalb
dieses Skriptes werden sämtliche Ausgaben an die Standardausgabe in die
resultierende \texttt{/etc/mdev.conf} integriert. Ein Beispiel aus dem oben
erwähnten \var{OPT\_AUTOMOUNT}:

\begin{small}
\begin{verbatim}
#!/bin/sh
#----------------------------------------------------------------------------
# /etc/mdev.d/mdev500.automount - mdev HD automounting rules     __FLI4LVER__
#
#
# Last Update:  $Id$
#----------------------------------------------------------------------------

cat <<"EOF"
#
# mdev500.automount
#

-SUBSYSTEM=block;DEVTYPE=partition;.+       0:0 660 */lib/mdev/automount

EOF
\end{verbatim}
\end{small}

Zu der Syntax der Regeln sei auf den Dateikopf der \texttt{/etc/mdev.conf}
sowie auf die mdev-Dokumentation unter
\altlink{http://git.busybox.net/busybox/plain/docs/mdev.txt} verwiesen. Falls
eine Regel ein Skript aufruft (wie \texttt{/lib/mdev/automount} im obigen
Beispiel), dann hat es Zugriff auf alle Variablen des auslösenden
Kernel-``uevent''s, insbesondere auf:

\begin{itemize}
\item \var{ACTION} (typischerweise \texttt{add} oder \texttt{remove}, seltener
\texttt{change})
\item \var{DEVPATH} (sysfs-Pfad zu der betroffenen Komponente)
\item \var{SUBSYSTEM} (das betroffene Kernel-Subsystem, siehe unten)
\item \var{DEVNAME} (die betroffene Gerätedatei unter \texttt{/dev}; fehlt, wenn
es nicht um zu erstellende oder löschende Geräte geht, sondern z.B. um Module)
\item \var{MDEV} (wird von mdev auf den Namen der letzlich erzeugten Gerätedatei
gesetzt)
\end{itemize}

Beispiele für Kernel-Subsysteme:

\begin{description}
\item[block] Blockgeräte (Speichermedien) wie \texttt{sda} (erste Festplatte),
             \texttt{sr0} (erstes CD-Laufwerk) oder \texttt{ram1} (zweite
             RAM-Disk)
\item[input] Geräte für Eingaben von Tastatur, Maus etc. wie
             \texttt{input/event0}; welche Gerätedateien welchen Geräten
             zugeordnet sind, ist nicht festgelegt und muss im sysfs ermittelt
             werden
\item[mem]   Geräte zum Zugriff auf den Speicher und Hardware-Ports wie
             \texttt{mem} und \texttt{ports}; hier fallen auch Pseudo-Geräte wie
             \texttt{zero} (liefert ununterbrochen das ASCII-Zeichen mit Wert
             null) und \texttt{null} (liefert nichts, verschluckt alles)
             darunter
\item[sound] diverse Geräte für die Tonausgabe, Benennung uneinheitlich
\item[tty]   Geräte zum Zugriff auf physische und virtuelle Konsolen wie
             \texttt{tty1} (erste virtuelle Konsole) oder \texttt{ttyS0} (erste
             serielle Konsole)
\end{description}

Ein Beispiel für die ersten beiden seriellen Schnittstellen:

\begin{scriptsize}
\begin{verbatim}
mdev[42]: 30.050644 add@/devices/pnp0/00:04/tty/ttyS0
mdev[42]:   ACTION=add
mdev[42]:   DEVPATH=/devices/pnp0/00:04/tty/ttyS0
mdev[42]:   SUBSYSTEM=tty
mdev[42]:   MAJOR=4
mdev[42]:   MINOR=64
mdev[42]:   DEVNAME=ttyS0
mdev[42]:   SEQNUM=613

mdev[42]: 30.051477 add@/devices/platform/serial8250/tty/ttyS1
mdev[42]:   ACTION=add
mdev[42]:   DEVPATH=/devices/platform/serial8250/tty/ttyS1
mdev[42]:   SUBSYSTEM=tty
mdev[42]:   MAJOR=4
mdev[42]:   MINOR=65
mdev[42]:   DEVNAME=ttyS1
mdev[42]:   SEQNUM=614
\end{verbatim}
\end{scriptsize}

Ein Beispiel für eine angeschlossene MF II-Tastatur:

\begin{scriptsize}
\begin{verbatim}
mdev[41]: 4.030653 add@/devices/platform/i8042/serio0/input/input0
mdev[41]:   ACTION=add
mdev[41]:   DEVPATH=/devices/platform/i8042/serio0/input/input0
mdev[41]:   SUBSYSTEM=input
mdev[41]:   PRODUCT=11/1/1/ab41
mdev[41]:   NAME="AT Translated Set 2 keyboard"
mdev[41]:   PHYS="isa0060/serio0/input0"
mdev[41]:   PROP=0
mdev[41]:   EV=120013
mdev[41]:   KEY=4 2000000 3803078 f800d001 feffffdf ffefffff ffffffff fffffffe
mdev[41]:   MSC=10
mdev[41]:   LED=7
mdev[41]:   MODALIAS=input:b0011v0001p0001eAB41-e0,1,4,11,14,k71,72,73,74,75,76,77,79,
7A,7B,7C,7D,7E,7F,80,8C,8E,8F,9B,9C,9D,9E,9F,A3,A4,A5,A6,AC,AD,B7,B8,B9,D9,E2,ram4,l0,
1,2,sfw
mdev[41]:   SEQNUM=604
\end{verbatim}
\end{scriptsize}

Ein Beispiel für ein geladenes USB-Kernelmodul (\texttt{uhci\_hcd}):

\begin{scriptsize}
\begin{verbatim}
mdev[41]: 6.537506 add@/module/uhci_hcd
mdev[41]:   ACTION=add
mdev[41]:   DEVPATH=/module/uhci_hcd
mdev[41]:   SUBSYSTEM=module
mdev[41]:   SEQNUM=633
\end{verbatim}
\end{scriptsize}

Ein Beispiel für eine angeschlossene Festplatte:

\begin{scriptsize}
\begin{verbatim}
mdev[41]: 7.267527 add@/devices/pci0000:00/0000:00:07.1/ata1/host0/target0:0:0/0:0:0:0/block/sda
mdev[41]:   ACTION=add
mdev[41]:   DEVPATH=/devices/pci0000:00/0000:00:07.1/ata1/host0/target0:0:0/0:0:0:0/block/sda
mdev[41]:   SUBSYSTEM=block
mdev[41]:   MAJOR=8
mdev[41]:   MINOR=0
mdev[41]:   DEVNAME=sda
mdev[41]:   DEVTYPE=disk
mdev[41]:   SEQNUM=688
\end{verbatim}
\end{scriptsize}

Dies ist eine ATA/IDE-Festplatte (ata1), die über den Gerätenamen \texttt{sda}
angesprochen werden sollte.

Ein Beispiel für ein entferntes Blockgerät (die Zuordnung einer Image-Datei zu
einer fli4l-VM wurde via \texttt{virsh detach-device} gelöst):

\begin{scriptsize}
\begin{verbatim}
mdev[42]: 52.600646 remove@/devices/pci0000:00/0000:00:0a.0/virtio5/block/vdb/vdb1
mdev[42]:   ACTION=remove
mdev[42]:   DEVPATH=/devices/pci0000:00/0000:00:0a.0/virtio5/block/vdb/vdb1
mdev[42]:   SUBSYSTEM=block
mdev[42]:   MAJOR=254
mdev[42]:   MINOR=17
mdev[42]:   DEVNAME=vdb1
mdev[42]:   DEVTYPE=partition
mdev[42]:   SEQNUM=776

mdev[42]: 52.644642 remove@/devices/virtual/bdi/254:16
mdev[42]:   ACTION=remove
mdev[42]:   DEVPATH=/devices/virtual/bdi/254:16
mdev[42]:   SUBSYSTEM=bdi
mdev[42]:   SEQNUM=777

mdev[42]: 52.644718 remove@/devices/pci0000:00/0000:00:0a.0/virtio5/block/vdb
mdev[42]:   ACTION=remove
mdev[42]:   DEVPATH=/devices/pci0000:00/0000:00:0a.0/virtio5/block/vdb
mdev[42]:   SUBSYSTEM=block
mdev[42]:   MAJOR=254
mdev[42]:   MINOR=16
mdev[42]:   DEVNAME=vdb
mdev[42]:   DEVTYPE=disk
mdev[42]:   SEQNUM=778

mdev[42]: 52.644777 remove@/devices/pci0000:00/0000:00:0a.0/virtio5
mdev[42]:   ACTION=remove
mdev[42]:   DEVPATH=/devices/pci0000:00/0000:00:0a.0/virtio5
mdev[42]:   SUBSYSTEM=virtio
mdev[42]:   MODALIAS=virtio:d00000002v00001AF4
mdev[42]:   SEQNUM=779

mdev[42]: 52.644973 remove@/devices/pci0000:00/0000:00:0a.0
mdev[42]:   ACTION=remove
mdev[42]:   DEVPATH=/devices/pci0000:00/0000:00:0a.0
mdev[42]:   SUBSYSTEM=pci
mdev[42]:   PCI_CLASS=10000
mdev[42]:   PCI_ID=1AF4:1001
mdev[42]:   PCI_SUBSYS_ID=1AF4:0002
mdev[42]:   PCI_SLOT_NAME=0000:00:0a.0
mdev[42]:   MODALIAS=pci:v00001AF4d00001001sv00001AF4sd00000002bc01sc00i00
mdev[42]:   SEQNUM=780
\end{verbatim}
\end{scriptsize}

Wie man sehen kann, sind bei einer solchen Entfernung diverse Kernel-Subsysteme
involviert (hier \texttt{block}, \texttt{bdi}, \texttt{virtio} und
\texttt{pci}).

\subsection{ttyI-Geräte}

Für die ttyI-Geräte (\texttt{/dev/ttyI0} \ldots \texttt{/dev/ttyI15}), über
welche die "`Modem-Emulation"' der ISDN-Karte genutzt werden kann, existiert
ein Zähler, um Konflikte zwischen verschiedenen Paketen, die diese Geräte
nutzen, zu vermeiden.
Hierzu wird beim Start des Routers die Datei \texttt{/var/run/next\_ttyI}
angelegt, die von den verschiedenen OPTs abgefragt und fortgezählt werden kann.
Mit dem folgenden Beispielskript kann dieser Wert abgefragt, um eins
erhöht und wieder für das nächste OPT exportiert werden.

\begin{example}
\begin{verbatim}
    ttydev_error=
    ttydev=$(cat /var/run/next_ttyI)
    if [ $ttydev -le 16 ]
    then                                    # ttyI device available? yes
        ttydev=$((ttydev + 1))              # ttyI device + 1
        echo $ttydev >/var/run/next_ttyI    # save it
    else                                    # ttyI device available? no
        log_error "No ttyI device for <Name deines OPTs> available!"
        ttydev_error=true                   # set error for later use
    fi

    if [ -z "$ttydev_error" ]               # start OPT only if next tty device
    then                                    # was available to minimize error
        ...                                 # messages and minimize the
                                            # risk of uncomplete boot
    fi
\end{verbatim}
\end{example}

\subsection{Skripte beim Einwählen und Auflegen}

\subsubsection{Allgemeines}

Nach dem Herstellen bzw.\ Trennen einer Circuit-Verbindung werden die Skripte
in \texttt{/etc/ppp/} abgearbeitet. Hier können OPTs Aktionen hinterlegen,
die nach dem Herstellen bzw.\ Auflegen der Verbindung nötig
sind. Benannt werden die Dateien wie folgt:

\begin{table}[htbp]
\centering
\begin{tabular}{l}
    \texttt{ip-up<dreistellige Zahl>.<OPT-Name>}\\
    \texttt{ip-down<dreistellige Zahl>.<OPT-Name>}\\
\end{tabular}
\end{table}

Dabei werden die \texttt{ip-up}-Skripte nach dem \emph{Aufbau} und die
\texttt{ip-down}-Skripte nach dem \emph{Abbau} der Verbindung ausgeführt.

\wichtig{In den \texttt{ip-down}-Skripten dürfen keine Aktionen ausgeführt
werden, die zu einer erneuten Einwahl führen, da dadurch nur ein
Dauer-Online-Zustand erreicht wird, was für Nicht-Flatrate-Benutzer
ein teures Unterfangen ist.}

\wichtig{Da für die einzelnen Skripte kein eigener Prozess erzeugt wird, dürfen
auch diese Skripte \emph{nicht} mit "`exit"' abgeschlossen werden!}

\subsubsection{Variablen}

Durch das spezielle Aufrufkonzept stehen die folgenden Variablen den
\texttt{ip-up}- und \texttt{ip-down}-Skripten zur Verfügung:

\begin{table}[htbp]
\centering
\begin{tabular}{lp{10cm}}

    \var{real\_interface}    & die aktuelle Schnittstelle, also z.\,B.
                               \texttt{ppp0}, \texttt{ippp0}, \ldots\\
    \var{interface}          & das IMOND-Interface, also \texttt{pppoe},
                               \texttt{ippp0}, \ldots\\
    \var{tty}                & verbundenes Terminal, möglicherweise leer!\\
    \var{speed}              & die Verbindunggeschwindigkeit, bei ISDN z.\,B.
                               64000\\
    \var{local}              & die eigene IP-Adresse\\
    \var{remote}             & die IP-Adresse des Point-To-Point-Partners\\
    \var{is\_default\_route} & gibt an, ob das aktuelle
                               \texttt{ip-up}/\texttt{ip-down} für die
                               Schnittstelle durchgeführt wird, über welche die
                               Default-Route geht (kann "`yes"' oder "`no"'
                               sein)\\
\end{tabular}
\end{table}

\subsubsection{Default-Route}

Seit Version 2.1.0 werden die \texttt{ip-up}/\texttt{ip-down}-Skripte nicht nur
für die Schnittstelle ausgeführt, über welche die Default-Route geht, sondern
für alle Verbindungen, welche die \texttt{ip-up}- und \texttt{ip-down}-Skripte
aufrufen. Um das alte Verhalten zu simulieren, muss in \texttt{ip-up}- und
\texttt{ip-down}-Skripten die folgende Abfrage eingefügt werden:

\begin{example}
\begin{verbatim}
    # is a default-route-interface going up?
    if [ "$is_default_route" = "yes" ]
    then
        # die eigentlichen Aktionen
    fi
\end{verbatim}
\end{example}

Natürlich darf das neue Verhalten auch für spezielle Aktionen
ausgenutzt werden.


\section{Paket "`template"'}

Um einiges von dem hier Beschriebenen etwas zu veranschaulichen, liegt der
fli4l-Distribution das Paket "`template"' bei. Dieses erklärt an kleinen
Beispielen, wie:

\begin{itemize}
\item eine Konfigurationsdatei auszusehen hat (\texttt{config/template.txt})
\item eine Check-Datei geschrieben wird (\texttt{check/template.txt})
\item die erweiterten Prüfmöglichekiten verwendet werden (\texttt{check/template.ext})
\item Konfigurationsvariablen für spätere Verwendung abgelegt werden
  können \\(\texttt{opt/etc/rc.d/rc999.template})
\item abgelegte Konfigurationsvariablen wieder ausgelesen werden \\(\texttt{opt/usr/bin/template\_show\_config})
\end{itemize}



\section{Aufbau des Boot-Datenträgers}

Seit Version 1.5 wird das Programm \texttt{syslinux} zum Booten verwendet.
Dieses hat den Vorteil, dass ein DOS-kompatibles Dateisystem auf dem
Datenträger zur Verfügung steht.

Der Boot-Datenträger enthält folgende Dateien:

\begin{table}[htbp]
\centering
\begin{tabular}[h!]{lp{10cm}}
\texttt{ldlinux.sys}             & der Urlader ("`Boot loader"') \texttt{syslinux} \\
\texttt{syslinux.cfg}            & Konfigurationsdatei für \texttt{syslinux} \\
\texttt{kernel}                  & Linux-Kernel\\
\texttt{rootfs.img}              & RootFS: enthält zum Booten nötige Programme\\
\texttt{opt.img}                 & Optionale Dateien: Treiber und Pakete\\
\texttt{rc.cfg}                  & Konfigurationsdatei mit den benutzten Variablen aus
den Dateien des Konfigurationsverzeichnisses\\
\texttt{boot.msg}                & Texte für das \texttt{syslinux}-Bootmenü \\
\texttt{boot\_s.msg}             & Texte für das \texttt{syslinux}-Bootmenü \\
\texttt{boot\_z.msg}             & Texte für das \texttt{syslinux}-Bootmenü \\
\texttt{hd.cfg}                  & Konfigurationsdatei zur Zuordnung der Partitionen \\
\end{tabular}
\end{table}

Durch das Skript \texttt{mkfli4l.sh} (bzw. \texttt{mkfli4l.bat}) werden zunächst
die Dateien \texttt{opt.img}, \texttt{syslinux.cfg} und \texttt{rc.cfg} sowie das \texttt{rootfs.img} erzeugt.
Die dafür nötigen Dateien ermittelt das Programm \var{mkfli4l} (im \texttt{unix}- bzw.
\texttt{windows}-Unterverzeichnis). In den beiden Archiven
sind die benötigten Kernel- und andere Pakete enthalten. Die Datei
\texttt{rc.cfg} befindet sich sowohl im Opt-Archiv als auch auf dem
Boot-Datenträger.\footnote{Die Fassung im Opt-Archiv ist während der frühen
Boot-Phase nötig, weil zu diesem Zeitpunkt das Boot-Volume noch nicht
eingehängt ist.}

Anschließend werden die Dateien \texttt{kernel}, \texttt{rootfs.img}, \texttt{opt.img}
und \texttt{rc.cfg} zusammen mit den \texttt{syslinux}-Dateien auf den Datenträger kopiert.

Beim Booten von fli4l wird über das Skript \texttt{/etc/rc} die \texttt{rc.cfg} ausgewertet und das
komprimierte \texttt{opt.img}-Archiv in die RootFS-RAM-Disk integriert (je nach
Installationstyp werden dabei die Dateien direkt in die RootFS-RAM-Disk entpackt
oder über symbolische Verknüpfungen eingebunden).
Danach werden die Skripte in \texttt{/etc/rc.d/} in alphanumerischer Reihenfolge ausgeführt
und somit die Treiber geladen und die Dienste gestartet.


\section{Konfigurationsdateien}

Hier werden die Dateien kurz aufgeführt, die vom fli4l-Router
on-the-fly beim Booten erzeugt werden.

\begin{enumerate}
\item Konfiguration Provider
  \begin{itemize}
  \item         \texttt{etc/ppp/pap-secrets}

  \item         \texttt{etc/ppp/chap-secrets}

  \end{itemize}
\item Konfiguration DNS
  \begin{itemize}
  \item         \texttt{etc/resolv.conf}

  \item         \texttt{etc/dnsmasq.conf}

  \item         \texttt{etc/dnsmasq\_dhcp.conf}

  \item         \texttt{etc/resolv.dnsmasq}

  \end{itemize}


\item Hosts-Datei
  \begin{itemize}
  \item         \texttt{etc/hosts}
  \end{itemize}



\item imond-Konfiguration
  \begin{itemize}
  \item \texttt{etc/imond.conf}
  \end{itemize}

\end{enumerate}


\subsection{Konfiguration Provider}

Für den ausgesuchten Provider wird in \texttt{etc/ppp/pap-secrets}
die User-ID und das Passwort angepasst.

Beispiel für Provider Planet-Interkom:

\begin{example}
\begin{verbatim}
# Secrets for authentication using PAP
# client        server  secret                  IP addresses
"anonymer"      *       "surfer"                *
\end{verbatim}
\end{example}

Dabei ist im Beispiel "`anonymer"' die USER-ID. Als Remote-Server wird
prinzipiell jeder erlaubt (deshalb "`*"'). "`surfer"' ist das Passwort für
den Provider Planet-Interkom.


\subsection{Konfiguration DNS}


Man kann den fli4l-Router als DNS-Server einsetzen. Warum dies
sinnvoll und bei Windows-Rechnern im LAN sogar zwingend notwendig
ist, wird in der Dokumentation des "`base"'-Pakets erläutert.

Die Resolver-Datei \texttt{etc/resolv.conf} enthält den Domainnamen und den zu
verwendenten Nameserver. Sie hat folgenden Inhalt (wobei "`domain.de"' nur
ein Platzhalter für den Wert der Konfigurationsvariable
\var{DOMAIN\_\-NAME} ist):

\begin{example}
\begin{verbatim}
        search domain.de
        nameserver 127.0.0.1
\end{verbatim}
\end{example}

Der DNS-Server dnsmasq wird über die Datei \texttt{etc/dnsmasq.conf}
konfiguriert. Sie wird beim Booten vom Skript \texttt{rc040.dns-local} sowie
\texttt{rc370.dnsmasq} automatisch erzeugt und könnte wie folgt aussehen:

\begin{example}
\begin{verbatim}
user=dns
group=dns
resolv-file=/etc/resolv.dnsmasq
no-poll
no-negcache
bogus-priv
log-queries
domain-suffix=lan.fli4l
local=/lan.fli4l/
domain-needed
expand-hosts
filterwin2k
conf-file=/etc/dnsmasq_dhcp.conf
\end{verbatim}
\end{example}


\subsection{Hosts-Datei}

    Diese Datei enthält eine Zuordnung von Host-Namen zu IP-Adressen. Diese
    Zuordnung ist jedoch nur lokal auf dem fli4l verwendbar, für andere Rechner
    im LAN ist sie nicht sichtbar. Diese Datei ist eigentlich überflüssig, wenn
    zusätzlich ein lokaler DNS-Server gestartet wird.



\subsection{imond-Konfiguration}

Die Datei \texttt{etc/imond.conf} wird unter anderem aus den Konfigurationsvariablen
\var{CIRC\_x\_NAME}, \var{CIRC\_x\_ROUTE}, \var{CIRC\_x\_CHARGEINT}
und \var{CIRC\_x\_TIMES} konstruiert. Sie kann aus bis zu 32 Zeilen
(ohne die Kommentarzeilen) bestehen. Jede Zeile besteht aus acht Spalten:

\begin{enumerate}
\item  Bereich Wochentag bis Wochentag
\item  Bereich Stunde bis Stunde
\item  Device (\texttt{ippp}\emph{X} oder \texttt{isdn}\emph{X})
\item  Circuit mit Default-Route: "`yes"'/"`no"'
\item  Telefonnummer
\item  Name des Circuits
\item  Telefonkosten pro Minute in Euro
\item  Zeittakt ("`Charge interval"') in Sekunden
\end{enumerate}

    Hier ein Beispiel:

\begin{example}
\begin{verbatim}
#day  hour  device  defroute  phone        name        charge  ch-int
Mo-Fr 18-09 ippp0   yes       010280192306 Addcom      0.0248  60
Sa-Su 00-24 ippp0   yes       010280192306 Addcom      0.0248  60
Mo-Fr 09-18 ippp1   yes       019160       Compuserve  0.019   180
Mo-Fr 09-18 isdn2   no        0221xxxxxxx  Firma       0.08    90
Mo-Fr 18-09 isdn2   no        0221xxxxxxx  Firma       0.03    90
Sa-Su 00-24 isdn2   no        0221xxxxxxx  Firma       0.03    90
\end{verbatim}
\end{example}

Weitere Erklärungen zum Least-Cost-Routing findet man in der Dokumentation des "`base"'-Pakets.

\subsection{Die \texttt{/etc/.profile}-Datei}

Die Datei \texttt{/etc/.profile} enthält benutzerdefinierte Einstellungen für
die Shell. Um die Stan\-dard-Einstellungen zu überschreiben, ist es nötig,
unterhalb seines Konfigurationsverzeichnisses eine Datei \texttt{etc/.profile}
zu erstellen. Dort können dann Einstellungen zum Prompt oder Abkürzungen
(so genannte "`Aliase"') eingetragen werden.

\wichtig{Diese Datei darf kein \texttt{exit} enthalten!}

Beispiel:

\begin{example}
\begin{verbatim}
alias ll='ls -al'
\end{verbatim}
\end{example}

\subsection{Skripte in \texttt{/etc/profile.d/}}

In dem Verzeichnis \texttt{/etc/profile.d/} können Skripte abgelegt werden, die
beim Starten einer Shell ausgeführt und damit die Umgebung für die Shell
beeinflussen können. Typischerweise platzieren OPTs dort Skripte, welche
spezielle Umgebungsvariablen setzen, die für die Programme des OPTs notwendig
sind.

Falls sich sowohl Skripte in \texttt{/etc/profile.d/} befinden als auch die
Datei \texttt{/etc/.profile} existiert, werden die Skripte in
\texttt{/etc/profile.d/} \emph{nach} dem Skript \texttt{/etc/.profile}
ausgeführt.

\section{Inkompatibilitäten zwischen 3.x und 4.x}

Beim Umstellen von Paketen von Version 3.x auf 4.x sind die folgenden Dinge
zu beachten:

\begin{itemize}
\item Die zugrunde liegende $\mu$Clibc-Bibliothek ist aktualisiert worden. Deswegen
sollten alle Binärprogramme, die gegen eine ältere $\mu$Clibc gebunden wurden, neu
übersetzt werden. Das dafür nötige FBR (\emph{f}li4l \emph{B}uild\emph{r}oot)
ist im \emph{src}-Paket zu finden.

\item Das Circuit-Konzept ist stark ausgebaut worden, so dass die Skripte in
\texttt{/etc/ppp} häufiger aufgerufen werden als vorher (z.B. auch für alle
konfigurierten Routen). Des Weiteren können jetzt dieselben Skripte für
verschiedene Circuits parallel ausgeführt werden. Greift ein \texttt{ip-up}-
oder ein \texttt{ip-down}-Skript auf eine globale Ressource zu (z.B. eine Datei,
die von allen Circuits gleichermaßen genutzt wird), muss diese Ressource für die
Dauer der Bearbeitung gesperrt und hinterher wieder freigegeben werden. Die
dafür nötigen Funktionen \texttt{sync\_lock\_resource} und
\texttt{sync\_unlock\_resource} werden in Abschnitt \ref{TODO} beschrieben.

\item Der Circuit und das Gerät \texttt{pppoe} existiert nur noch aus
Kompatibilitätsgründen und sollte nicht mehr benutzt werden. Es repräsentiert
den ersten konfigurierten PPPoE-Circuit bzw. die ihm zugrunde liegende
Netzwerkschnittstelle. PPP-Circuits, die \emph{keine} PPPoE-Circuits sind (dies
trifft z.B. auf PPTP-, Fritz!DSL- oder UMTS-Circuits zu), werden über diesen
Alias \emph{nicht} mehr repräsentiert. Des Weiteren sollte in den
Circuit-Skripten unterhalb von \texttt{/etc/ppp} die Variable
\texttt{real\_interface} nicht mehr genutzt werden. Statt dessen sollte
die Variable \texttt{interface} verwendet werden. Eine Abbildung von der
tatsächlichen PPP-Schnittstelle auf \texttt{pppoe} findet nicht mehr statt, da
nicht mehr davon ausgegangen werden darf, dass es höchstens einen DSL-Circuit
gibt. Es ist somit auch keine Prüfung mehr vorzunehmen, ob die Variable
\texttt{interface} den Wert \texttt{pppoe} beinhaltet.
\end{itemize}

