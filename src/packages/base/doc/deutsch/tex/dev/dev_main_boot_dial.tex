% Last Update: $Id$

\section{Hochfahren, Herunterfahren, Einwählen und Auflegen unter fli4l}

\subsection{Bootkonzept}

fli4l 2.0 sollte eine saubere Installation auf eine Festplatte oder ein
CompactFlash(TM)-Medium bieten, aber auch eine Installation auf ein
Zip-Medium oder die Erstellung einer bootfähigen CD-ROM sollte möglich
sein. Zusätzlich sollte die Festplattenversion sich nicht grundlegend von
einer Installation auf Diskette\footnote{Ursprünglich konnte fli4l auch
von einer Diskette betrieben werden. Da fli4l inzwischen dafür zu groß geworden
ist, wird dies nicht mehr unterstützt.} unterscheiden.

Diese Anforderungen wurden realisiert, indem die Dateien des
\texttt{opt.img}-Archivs aus der
bisherigen RAM-Disk auf eine anderes Medium verlagert werden können. Dies
kann eine Partition auf einer Festplatte bzw.\ einem CF-Medium sein. Dieses
zweite Volume wird unter \texttt{/opt} eingehängt, und dort liegende Programme
werden nur noch über symbolische Links in das RootFS integriert. Das entstehende Layout im
RootFS-Dateisystem entspricht dann dem im \texttt{opt}-Verzeichnis der
ausgepackten fli4l-Distribution mit einer Ausnahme~-- das \texttt{files}-Präfix
entfällt. Die Datei \texttt{opt/etc/rc} ist dann also direkt unter \texttt{/etc/rc} zu
finden, \texttt{opt/bin/busybox} unter
\texttt{/bin/busybox}. Dass diese Dateien unter Umständen nur symbolische Verknüpfungen
auf ein im Nur-Lese-Modus eingehängtes Volume sind, kann man ignorieren,
solange man die Dateien nicht modifizieren möchte. Will man dies tun,
muss man die Dateien vorher mit \texttt{mk\_writable} (s.\,u.) schreibbar machen.

\subsection{Start- und Stopp-Skripte}

Skripte, die beim Hochfahren des Systems ausgeführt werden sollen, befinden
sich in den Verzeichnissen \texttt{opt/etc/boot.d/} und \texttt{opt/etc/rc.d/}
und werden auch in dieser Reihenfolge ausgeführt. Des Weiteren befinden sich in
\texttt{opt/etc/rc0.d/} Skripte, die beim Herunterfahren des Systems ausgefühlt
werden.

\wichtig{Da zum Ausführen dieser Skripte kein separater Prozess
erzeugt wird, dürfen sie nicht mit "`exit"' abgeschlossen
werden. Ein solcher Befehl führt zum vorzeitigen Abbruch des Bootvorgangs!}

\subsubsection{Start-Skripte in \texttt{opt/etc/boot.d/}}

Skripte, die in diesem Verzeichnis liegen, werden als erstes
ausgeführt. Ihre Aufgabe ist es, das Boot-Volume einzuhängen, die
auf dem Boot-Medium liegende Konfigurationsdatei \texttt{rc.cfg} einzulesen und
das Opt-Archiv zu entpacken. Je nach \jump{BOOTTYPE}{Boot-Typ}
sind diese Skripte mehr oder weniger kompliziert und tun die folgenden
Dinge:

\begin{itemize}
\item Laden von Hardware-Treibern (optional)
\item Boot-Volume einhängen (optional)
\item Konfigurationsdatei \texttt{rc.cfg} vom Boot-Volume einlesen und
      in die Datei \texttt{/etc/rc.cfg} schreiben
\item Einhängen des Opt-Volumes (optional)
\item Extrahieren des Opt-Archivs (optional)
\end{itemize}

Damit die Skripte überhaupt eine Chance haben, etwas über die
fli4l-Konfiguration zu erfahren, wird die Konfigurationsdatei auch ins
RootFS-Archiv unter \texttt{/etc/rc.cfg} eingebunden; die
Konfigurationsvariablen in dieser Datei stehen den Start-Skripten in
\texttt{opt/etc/boot.d/} direkt zur Verfügung. Nach dem Einhängen des
Boot-Volumes wird die \texttt{/etc/rc.cfg} durch die Konfigurationsdatei
vom Boot-Volume ersetzt, so dass den Start-Skripten in \texttt{opt/etc/rc.d/}
(s.\,u.) die aktuelle Konfiguration vom Boot-Volume zur Verfügung steht.
\footnote{Normalerweise sind diese beiden Dateien identisch. Eine Abweichung
entsteht nur, wenn die Konfigurationsdatei auf dem Boot-Volume händisch
editiert wird, etwa um die Konfiguration nachträglich abzuändern, ohne die
fli4l-Archive neu zu bauen.}

\subsubsection{Start-Skripte in \texttt{opt/etc/rc.d/}}

Befehle, die immer beim Starten des Routers ausgeführt werden müssen,
können in Skripten im Verzeichnis \texttt{opt/etc/rc.d/} abgelegt werden. Hierbei
gelten folgende Konventionen:

\begin{enumerate}
  \item Es gilt folgende Namenskonvention:

\begin{example}
\begin{verbatim}
    rc<dreistellige Zahl>.<Name des OPTs>
\end{verbatim}
\end{example}
      
    Die Skripte werden in aufsteigender Reihenfolge der Zahlen
    gestartet. Ist mehreren Skripten dieselbe Zahl zugeordnet,
    entscheidet die alphabetische Sortierung nach dem Punkt. Falls
    der Start eines Pakets erst nach einem anderen erfolgen darf,
    wird das durch die Zahl festgelegt.

    Hier eine ungefähre Richtlinie, welche Nummern für welche Aufgaben
    verwendet werden sollten:

    \begin{table}[htbp]
    \centering
    \begin{tabular}{lp{0.8\textwidth}}
            \hline
            Nummer        &       Aufgabe         \\
            \hline
            \hline
            000-099       &       Grundsystem (Hardware, Zeitzone, Dateisystem) \\
            100-199       &       Kernel-Module (Treiber) \\
            200-299       &       externe Verbindungen (PPPoE, ISDN4Linux, PPtP) \\
            300-399       &       Netzwerk (Routing, Interfaces, Paketfilter) \\
            400-497       &       Server (DHCP, HTTPD, Proxy, etc.) \\
            498-499       &       reserviert (Aktivierung des Circuit-Systems) \\
            500-997       &       nach Belieben (ab hier können Skripte von Circuits
                                  kontrollierte Verbindungen nutzen und laufen
                                  möglicherweise parallel zu einer Einwahl) \\
            998-999       &       reserviert (bitte nicht benutzen!) \\
            \hline
    \end{tabular}
    \end{table}

  \item In diesen Skripten \emph{müssen} alle Funktionen, die das RootFS
    verändern, hinterlegt werden, etwa das Anlegen eines
    Verzeichnisses \texttt{/var/log/lpd}.

  \item In diesen Skripten dürfen \emph{keine} Schreibzugriffe auf
    Dateien erfolgen, die Teil des Opt-Archivs sein können, da diese Dateien
    auf einem im Nur-Lese-Modus eingehängten Volume liegen können.
    Muss man eine solche Datei modifizieren, muss man sie
    vorher mit Hilfe der Funktion \texttt{mk\_writable} (s.\,u.) beschreibbar
    machen. Durch den Aufruf der Funktion wird die Datei (wenn nötig) als
    beschreibbare Kopie im RootFS abgelegt. Ist die Datei bereits beschreibbar,
    bewirkt der \texttt{mk\_writable}-Aufruf nichts.
    
    \wichtig{\texttt{mk\_writable} muss direkt auf Dateien im RootFS angewandt
    werden, nicht über den Umweg des \texttt{opt}-Verzeichnisses. Will man
    also \texttt{/usr/local/bin/foo} modifizieren, ruft man
    \texttt{mk\_writable} mit dem Argument \texttt{/usr/local/bin/foo} auf.}

  \item Diese Skripte müssen vor der Ausführung der eigentlichen
    Befehle prüfen, ob das dazugehörige OPT auch aktiv ist. Das ist
    normalerweise durch eine einfache Fallunterscheidung erledigt:

\begin{example}
\begin{verbatim}
    if [ "$OPT_<OPT-Name>" = "yes" ]
    then
        ...
        # Hier OPT starten!
        ...
    fi
\end{verbatim}
\end{example}

  \item Um die Fehlersuche zu erleichtern, sollten die Skripte mit
    \texttt{begin\_script} und \texttt{end\_script} geklammert werden:

\begin{example}
\begin{verbatim}
    if [ "$OPT_<OPT-Name>" = "yes" ]
    then
        begin_script FOO "configuring foo ..."
        ...
        end_script
    fi
\end{verbatim}
\end{example}

    Die Fehlersuche einzelner Start-Skripte kann man dann einfach via
    \verb+FOO_DO_DEBUG='yes'+ aktivieren.

  \item Den Skripten stehen alle Konfigurationsvariablen
    direkt zur Verfügung. Im Abschnitt \jump{dev:sec:config-variables}{"`Umgang
    mit Konfigurationsvariablen"'} wird erklärt, wie man von anderen Skripten
    aus auf die Konfigurationsvariablen zugreifen kann.

  \item Der Pfad \texttt{/opt} darf auch nicht als Speicherplatz für
    Datenbestände der OPTs benutzt werden. Falls zusätzlicher
    Speicherplatz benötigt wird, sollte dem Benutzer über eine
    Konfigurationsvariable die Möglichkeit gegeben werden, einen geeigneten
    Pfad auszuwählen. Je nach Art der zu speichernden Daten (persistente oder
    transiente Daten) sind verschiedene Standard-Belegungen sinnvoll. So bieten
    sich für transiente Daten etwa Pfade unterhalb von \texttt{/var/run/} an;
    für persistente Daten sollte die Funktion
    \jump{dev:sec:persistent-data}{map2persistent} in Kombination mit einer
    geeigneten Konfigurationsvariable verwendet werden.

\end{enumerate}

\subsubsection{Stopp-Skripte in \texttt{opt/etc/rc0.d/}}

Jeder Rechner muss mal heruntergefahren oder neu gestartet werden. Nun kann
es vorkommen, dass man Vorgänge ausführen muss, bevor der Rechner
heruntergefahren oder neu gestartet wird. Zum Herunterfahren und Neustarten
sind die Befehle "`halt"' bzw.\ "`reboot"' zuständig. Diese Befehle werden auch
aufgerufen, wenn man im IMONC oder in der Web-GUI auf die entsprechenden
Schaltflächen klickt.

Alle Stopp-Skripte liegen im Verzeichnis \texttt{opt/etc/rc0.d/}. Ihre Dateinamen
werden analog zu den Start-Skripten gebildet. Sie werden ebenfalls in
\emph{auf}steigender Reihenfolge der Zahlen ausgeführt.

\subsection{Hilfsfunktionen}

In \texttt{/etc/boot.d/base-helper} werden verschiedene Funktionen
bereitgestellt, die von den Start- und anderen Skripten verwendet werden
können. Das betrifft Dinge wie Unterstützung zur Fehlersuche, Laden von
Kernel-Modulen oder Ausgabe von Meldungen.
Die einzelnen Funktionen werden im Folgenden aufgelistet und kurz beschrieben.

\subsubsection{Skript-Steuerung}

\begin{description}

\item[\texttt{begin\_script <Symbol> <Meldung>}:]
Gibt eine Meldung aus und aktiviert die Fehlersuche im Skript mittels
\texttt{set -x}, falls \var{<Symbol>\_DO\_DEBUG} auf "`yes"' steht.

\item[\texttt{end\_script}:]
Gibt eine Abschluss-Meldung aus und deaktiviert
die Fehlersuche, falls sie mit \texttt{begin\_script} aktiviert wurde. Für
jeden \texttt{begin\_script}-Aufruf muss es einen zugehörigen
\texttt{end\_script}-Aufruf geben (und umgekehrt).

\end{description}

\subsubsection{Laden von Kernel-Modulen}

\begin{description}

\item[\texttt{do\_modprobe [-q] <Modul> <Parameter>*}:]
Lädt ein Kernel-Modul inkl.
eventueller Parameter bei gleichzeitiger Auflösung der Modulabhängigkeiten.
Der Parameter "`-q"' verhindert, dass im Fehlerfall eine Meldung auf der Konsole
ausgegeben und ins Boot-Protokoll geschrieben wird.
Die Funktion liefert im Erfolgsfall den Rückgabewert null zurück und im
Fehlerfall einen Wert ungleich null. Damit lässt sich Code schreiben, der
ein Fehlschlagen des Ladens eines Kernel-Moduls geeignet behandelt:

\begin{example}
\begin{verbatim}
    if do_modprobe -q acpi-cpufreq
    then
        # kein CPU-Frequenzsteuerung via ACPI
        log_error "CPU-Frequenzsteuerung via ACPI nicht verfügbar!"
        # [...]
    else
        log_info "CPU-Frequenzsteuerung via ACPI aktiviert."
        # [...]
    fi
\end{verbatim}
\end{example}

\item[\texttt{do\_modrobe\_if\_exists [-q] <Modulpfad> <Modul> <Parameter>*}:]\mbox{}\\
Prüft, ob das Modul \texttt{/lib/modules/<Kernel-Version>/<Modulpfad>/<Modul>}
existiert und ruft bei Vorhandensein die Funktion \texttt{do\_modprobe} auf.

\wichtig{Das Modul muss tatsächlich unter dem angegebenen Modulnamen existieren,
der Modulname darf kein Alias sein. Bei einem Alias wird direkt
\texttt{do\_modprobe} aufgerufen.}

\end{description}

\subsubsection{Meldungen und Fehlerbehandlung}

\begin{description}

\item[\texttt{log\_info <Meldung>}:] Schreibt eine Meldung auf
die Konsole und nach \texttt{/bootmsg.txt}. Wird keine Meldung als
Parameter übergeben, liest \texttt{log\_info} von der Standard-Eingabe. Die
Funktion liefert als Rückgabewert immer null zurück.

\item[\texttt{log\_warn <Meldung>}:] Schreibt eine Warnmeldung auf
die Konsole und nach \texttt{/bootmsg.txt}, wobei vor die Meldung die
Zeichenkette \texttt{WARN:} gesetzt wird. Wird keine Meldung als
Parameter übergeben, liest \texttt{log\_warn} von der Standard-Eingabe.
Die Funktion liefert als Rückgabewert immer null zurück.

\item[\texttt{log\_error <Meldung>}:] Schreibt eine Fehlermeldung auf
die Konsole und nach \texttt{/bootmsg.txt}, wobei vor die Meldung die
Zeichenkette \texttt{ERR:} gesetzt wird. Wird keine Meldung als
Parameter übergeben, liest \texttt{log\_error} von der Standard-Eingabe.
Die Funktion liefert als Rückgabewert immer einen Wert ungleich null zurück.

\item[\texttt{set\_error <Meldung>}:] Gibt die Fehlermeldung aus und setzt
eine interne Fehlervariable, das später via \texttt{is\_error} geprüft werden
kann.

\item[\texttt{is\_error}:] Setzt die interne Fehlervariable zurück und liefert
wahr zurück, falls sie vorher durch \texttt{set\_error} gesetzt wurde.

\end{description}

\subsubsection{Netzwerk-Funktionen}

\begin{description}

\item[\texttt{translate\_ip\_net <Wert> <Variablenname> [<Ergebnisvariable>]}:]
Ersetzt symbolische Referenzen in
Parametern. Momentan finden folgende Übersetzungen statt:
\begin{description}
\item[Host- oder Netzwerk-Adressen] werden nicht übersetzt
\item[\texttt{any}] wird durch \texttt{0.0.0.0/0} ersetzt
\item[\texttt{dynamic}] wird durch die dynamische IPv4-Adresse des Routers
ersetzt; das ist die IPv4-Adresse, die der Router beim Einwählen über einen
Circuit mit einer Default-Route zugewiesen bekommt
\item[\var{IP\_NET\_x}] wird durch das in der Konfiguration stehende
Netzwerk ersetzt; referenziert die Variable einen Circuit, so wird das dem
Circuit zugewiesene IPv4-Netz zurückgegeben
\item[\var{IP\_NET\_x\_IPADDR}] wird durch die in der Konfiguration stehende
IP-Adresse ersetzt; referenziert die Variable einen Circuit, so wird die dem
Circuit zugewiesene IPv4-Adresse zurückgegeben
\item[\var{IP\_ROUTE\_x}] wird durch das in der Konfiguration stehende
geroutete Netzwerk ersetzt
\item[\texttt{@<Hostname>}] wird durch die in der Konfiguration für den
angegeben Host spezifizierte IP-Adresse ersetzt
\item[\texttt{\{<circuit>\}}] wird durch das dem Circuit zugewiesene IPv4-Netz
ersetzt
\end{description}

Das Ergebnis der Übersetzung wird in der Variable gespeichert, deren Name im
dritten Parameter übergeben wird; fehlt dieser Parameter, wird das Ergebnis in
der Variable \var{res} gespeichert. Der Variablenname, der im zweiten Parameter
übergeben wird, wird nur für Fehlermeldungen benutzt, falls die Übersetzung
fehlschlägt; hier kann also vom Aufrufer die Quelle des zu übersetzenden Wertes
angegeben werden. Im Fehlerfall wird dann eine Meldung wie

\begin{example}
\begin{verbatim}
    Unable to translate value '<Wert>' contained in <Variablenname>.
\end{verbatim}
\end{example}

ausgegeben.

Der Rückgabewert ist null, falls die Übersetzung erfolgreich war, und ungleich
null, falls ein Fehler aufgetreten ist.

\end{description}

\subsubsection{Diverses}

\begin{description}

\item[\texttt{mk\_writable <Datei>}:]
Stellt sicher, dass die übergebene Datei beschreibbar ist. Befindet sich die
Datei auf einem im Nur-Lese-Modus eingehängten Volume und ist lediglich über
eine symbolische Verknüpfung ins Dateisystem eingebunden, wird eine lokale Kopie
angelegt, die dann beschreibbar ist.

\item[\texttt{list\_unique <Liste>}:]
Entfernt Duplikate aus der übergebenen Liste. Das Resultat wird in die
Standard-Ausgabe geschrieben.

\end{description}

\subsection{mdev-Regeln}

Für OPTs ist es möglich, zusätzliche mdev-Regeln zu etablieren, die spezielle
Aktionen beim Erscheinen oder Verschwinden bestimmter Geräte vornehmen. Das
\var{OPT\_AUTOMOUNT} im hd-Paket verwendet beispielsweise eine solche Regel,
um auftauchende Speichermedien automatisch einzuhängen. Will man eine
zusätzliche mdev-Regel integrieren, muss man ein Skript der Form

\begin{verbatim}
    /etc/mdev.d/mdev<Nummer>.<Name>
\end{verbatim}

ins RootFS einbauen, wobei die Nummer ähnlich den Start- und Stopp-Skripten aus
drei Ziffern bestehen muss und der Name beliebig gewählt werden kann. Innerhalb
dieses Skriptes werden sämtliche Ausgaben an die Standardausgabe in die
resultierende \texttt{/etc/mdev.conf} integriert. Ein Beispiel aus dem oben
erwähnten \var{OPT\_AUTOMOUNT}:

\begin{small}
\begin{verbatim}
#!/bin/sh
#----------------------------------------------------------------------------
# /etc/mdev.d/mdev500.automount - mdev HD automounting rules     __FLI4LVER__
#
#
# Last Update:  $Id$
#----------------------------------------------------------------------------

cat <<"EOF"
#
# mdev500.automount
#

-SUBSYSTEM=block;DEVTYPE=partition;.+       0:0 660 */lib/mdev/automount

EOF
\end{verbatim}
\end{small}

Zu der Syntax der Regeln sei auf den Dateikopf der \texttt{/etc/mdev.conf}
sowie auf die mdev-Dokumentation unter
\altlink{http://git.busybox.net/busybox/plain/docs/mdev.txt} verwiesen. Falls
eine Regel ein Skript aufruft (wie \texttt{/lib/mdev/automount} im obigen
Beispiel), dann hat es Zugriff auf alle Variablen des auslösenden
Kernel-``uevent''s, insbesondere auf:

\begin{itemize}
\item \var{ACTION} (typischerweise \texttt{add} oder \texttt{remove}, seltener
\texttt{change})
\item \var{DEVPATH} (sysfs-Pfad zu der betroffenen Komponente)
\item \var{SUBSYSTEM} (das betroffene Kernel-Subsystem, siehe unten)
\item \var{DEVNAME} (die betroffene Gerätedatei unter \texttt{/dev}; fehlt, wenn
es nicht um zu erstellende oder löschende Geräte geht, sondern z.B. um Module)
\item \var{MDEV} (wird von mdev auf den Namen der letzlich erzeugten Gerätedatei
gesetzt)
\end{itemize}

Beispiele für Kernel-Subsysteme:

\begin{description}
\item[block] Blockgeräte (Speichermedien) wie \texttt{sda} (erste Festplatte),
             \texttt{sr0} (erstes CD-Laufwerk) oder \texttt{ram1} (zweite
             RAM-Disk)
\item[input] Geräte für Eingaben von Tastatur, Maus etc. wie
             \texttt{input/event0}; welche Gerätedateien welchen Geräten
             zugeordnet sind, ist nicht festgelegt und muss im sysfs ermittelt
             werden
\item[mem]   Geräte zum Zugriff auf den Speicher und Hardware-Ports wie
             \texttt{mem} und \texttt{ports}; hier fallen auch Pseudo-Geräte wie
             \texttt{zero} (liefert ununterbrochen das ASCII-Zeichen mit Wert
             null) und \texttt{null} (liefert nichts, verschluckt alles)
             darunter
\item[sound] diverse Geräte für die Tonausgabe, Benennung uneinheitlich
\item[tty]   Geräte zum Zugriff auf physische und virtuelle Konsolen wie
             \texttt{tty1} (erste virtuelle Konsole) oder \texttt{ttyS0} (erste
             serielle Konsole)
\end{description}

Ein Beispiel für die ersten beiden seriellen Schnittstellen:

\begin{scriptsize}
\begin{verbatim}
mdev[42]: 30.050644 add@/devices/pnp0/00:04/tty/ttyS0
mdev[42]:   ACTION=add
mdev[42]:   DEVPATH=/devices/pnp0/00:04/tty/ttyS0
mdev[42]:   SUBSYSTEM=tty
mdev[42]:   MAJOR=4
mdev[42]:   MINOR=64
mdev[42]:   DEVNAME=ttyS0
mdev[42]:   SEQNUM=613

mdev[42]: 30.051477 add@/devices/platform/serial8250/tty/ttyS1
mdev[42]:   ACTION=add
mdev[42]:   DEVPATH=/devices/platform/serial8250/tty/ttyS1
mdev[42]:   SUBSYSTEM=tty
mdev[42]:   MAJOR=4
mdev[42]:   MINOR=65
mdev[42]:   DEVNAME=ttyS1
mdev[42]:   SEQNUM=614
\end{verbatim}
\end{scriptsize}

Ein Beispiel für eine angeschlossene MF II-Tastatur:

\begin{scriptsize}
\begin{verbatim}
mdev[41]: 4.030653 add@/devices/platform/i8042/serio0/input/input0
mdev[41]:   ACTION=add
mdev[41]:   DEVPATH=/devices/platform/i8042/serio0/input/input0
mdev[41]:   SUBSYSTEM=input
mdev[41]:   PRODUCT=11/1/1/ab41
mdev[41]:   NAME="AT Translated Set 2 keyboard"
mdev[41]:   PHYS="isa0060/serio0/input0"
mdev[41]:   PROP=0
mdev[41]:   EV=120013
mdev[41]:   KEY=4 2000000 3803078 f800d001 feffffdf ffefffff ffffffff fffffffe
mdev[41]:   MSC=10
mdev[41]:   LED=7
mdev[41]:   MODALIAS=input:b0011v0001p0001eAB41-e0,1,4,11,14,k71,72,73,74,75,76,77,79,
7A,7B,7C,7D,7E,7F,80,8C,8E,8F,9B,9C,9D,9E,9F,A3,A4,A5,A6,AC,AD,B7,B8,B9,D9,E2,ram4,l0,
1,2,sfw
mdev[41]:   SEQNUM=604
\end{verbatim}
\end{scriptsize}

Ein Beispiel für ein geladenes USB-Kernelmodul (\texttt{uhci\_hcd}):

\begin{scriptsize}
\begin{verbatim}
mdev[41]: 6.537506 add@/module/uhci_hcd
mdev[41]:   ACTION=add
mdev[41]:   DEVPATH=/module/uhci_hcd
mdev[41]:   SUBSYSTEM=module
mdev[41]:   SEQNUM=633
\end{verbatim}
\end{scriptsize}

Ein Beispiel für eine angeschlossene Festplatte:

\begin{scriptsize}
\begin{verbatim}
mdev[41]: 7.267527 add@/devices/pci0000:00/0000:00:07.1/ata1/host0/target0:0:0/0:0:0:0/block/sda
mdev[41]:   ACTION=add
mdev[41]:   DEVPATH=/devices/pci0000:00/0000:00:07.1/ata1/host0/target0:0:0/0:0:0:0/block/sda
mdev[41]:   SUBSYSTEM=block
mdev[41]:   MAJOR=8
mdev[41]:   MINOR=0
mdev[41]:   DEVNAME=sda
mdev[41]:   DEVTYPE=disk
mdev[41]:   SEQNUM=688
\end{verbatim}
\end{scriptsize}

Dies ist eine ATA/IDE-Festplatte (ata1), die über den Gerätenamen \texttt{sda}
angesprochen werden sollte.

Ein Beispiel für ein entferntes Blockgerät (die Zuordnung einer Image-Datei zu
einer fli4l-VM wurde via \texttt{virsh detach-device} gelöst):

\begin{scriptsize}
\begin{verbatim}
mdev[42]: 52.600646 remove@/devices/pci0000:00/0000:00:0a.0/virtio5/block/vdb/vdb1
mdev[42]:   ACTION=remove
mdev[42]:   DEVPATH=/devices/pci0000:00/0000:00:0a.0/virtio5/block/vdb/vdb1
mdev[42]:   SUBSYSTEM=block
mdev[42]:   MAJOR=254
mdev[42]:   MINOR=17
mdev[42]:   DEVNAME=vdb1
mdev[42]:   DEVTYPE=partition
mdev[42]:   SEQNUM=776

mdev[42]: 52.644642 remove@/devices/virtual/bdi/254:16
mdev[42]:   ACTION=remove
mdev[42]:   DEVPATH=/devices/virtual/bdi/254:16
mdev[42]:   SUBSYSTEM=bdi
mdev[42]:   SEQNUM=777

mdev[42]: 52.644718 remove@/devices/pci0000:00/0000:00:0a.0/virtio5/block/vdb
mdev[42]:   ACTION=remove
mdev[42]:   DEVPATH=/devices/pci0000:00/0000:00:0a.0/virtio5/block/vdb
mdev[42]:   SUBSYSTEM=block
mdev[42]:   MAJOR=254
mdev[42]:   MINOR=16
mdev[42]:   DEVNAME=vdb
mdev[42]:   DEVTYPE=disk
mdev[42]:   SEQNUM=778

mdev[42]: 52.644777 remove@/devices/pci0000:00/0000:00:0a.0/virtio5
mdev[42]:   ACTION=remove
mdev[42]:   DEVPATH=/devices/pci0000:00/0000:00:0a.0/virtio5
mdev[42]:   SUBSYSTEM=virtio
mdev[42]:   MODALIAS=virtio:d00000002v00001AF4
mdev[42]:   SEQNUM=779

mdev[42]: 52.644973 remove@/devices/pci0000:00/0000:00:0a.0
mdev[42]:   ACTION=remove
mdev[42]:   DEVPATH=/devices/pci0000:00/0000:00:0a.0
mdev[42]:   SUBSYSTEM=pci
mdev[42]:   PCI_CLASS=10000
mdev[42]:   PCI_ID=1AF4:1001
mdev[42]:   PCI_SUBSYS_ID=1AF4:0002
mdev[42]:   PCI_SLOT_NAME=0000:00:0a.0
mdev[42]:   MODALIAS=pci:v00001AF4d00001001sv00001AF4sd00000002bc01sc00i00
mdev[42]:   SEQNUM=780
\end{verbatim}
\end{scriptsize}

Wie man sehen kann, sind bei einer solchen Entfernung diverse Kernel-Subsysteme
involviert (hier \texttt{block}, \texttt{bdi}, \texttt{virtio} und
\texttt{pci}).

\subsection{ttyI-Geräte}

Für die ttyI-Geräte (\texttt{/dev/ttyI0} \ldots \texttt{/dev/ttyI15}), über
welche die "`Modem-Emulation"' der ISDN-Karte genutzt werden kann, existiert
ein Zähler, um Konflikte zwischen verschiedenen Paketen, die diese Geräte
nutzen, zu vermeiden.
Hierzu wird beim Start des Routers die Datei \texttt{/var/run/next\_ttyI}
angelegt, die von den verschiedenen OPTs abgefragt und fortgezählt werden kann.
Mit dem folgenden Beispielskript kann dieser Wert abgefragt, um eins
erhöht und wieder für das nächste OPT exportiert werden.

\begin{example}
\begin{verbatim}
    ttydev_error=
    ttydev=$(cat /var/run/next_ttyI)
    if [ $ttydev -le 16 ]
    then                                    # ttyI device available? yes
        ttydev=$((ttydev + 1))              # ttyI device + 1
        echo $ttydev >/var/run/next_ttyI    # save it
    else                                    # ttyI device available? no
        log_error "No ttyI device for <Name deines OPTs> available!"
        ttydev_error=true                   # set error for later use
    fi

    if [ -z "$ttydev_error" ]               # start OPT only if next tty device
    then                                    # was available to minimize error
        ...                                 # messages and minimize the
                                            # risk of uncomplete boot
    fi
\end{verbatim}
\end{example}

\subsection{Skripte beim Einwählen und Auflegen}

\subsubsection{Allgemeines}

Nach dem Herstellen bzw.\ Trennen einer Circuit-Verbindung werden die Skripte
in \texttt{/etc/ppp/} abgearbeitet. Hier können OPTs Aktionen hinterlegen,
die nach dem Herstellen bzw.\ Auflegen der Verbindung nötig
sind. Benannt werden die Dateien wie folgt:

\begin{table}[htbp]
\centering
\begin{tabular}{l}
    \texttt{ip-up<dreistellige Zahl>.<OPT-Name>}\\
    \texttt{ip-down<dreistellige Zahl>.<OPT-Name>}\\
\end{tabular}
\end{table}

Dabei werden die \texttt{ip-up}-Skripte nach dem \emph{Aufbau} und die
\texttt{ip-down}-Skripte nach dem \emph{Abbau} der Verbindung ausgeführt.

\wichtig{In den \texttt{ip-down}-Skripten dürfen keine Aktionen ausgeführt
werden, die zu einer erneuten Einwahl führen, da dadurch nur ein
Dauer-Online-Zustand erreicht wird, was für Nicht-Flatrate-Benutzer
ein teures Unterfangen ist.}

\wichtig{Da für die einzelnen Skripte kein eigener Prozess erzeugt wird, dürfen
auch diese Skripte \emph{nicht} mit "`exit"' abgeschlossen werden!}

\subsubsection{Variablen}

Durch das spezielle Aufrufkonzept stehen die folgenden Variablen den
\texttt{ip-up}- und \texttt{ip-down}-Skripten zur Verfügung:

\begin{table}[htbp]
\centering
\begin{tabular}{lp{10cm}}

    \var{real\_interface}    & die aktuelle Schnittstelle, also z.\,B.
                               \texttt{ppp0}, \texttt{ippp0}, \ldots\\
    \var{interface}          & das IMOND-Interface, also \texttt{pppoe},
                               \texttt{ippp0}, \ldots\\
    \var{tty}                & verbundenes Terminal, möglicherweise leer!\\
    \var{speed}              & die Verbindunggeschwindigkeit, bei ISDN z.\,B.
                               64000\\
    \var{local}              & die eigene IP-Adresse\\
    \var{remote}             & die IP-Adresse des Point-To-Point-Partners\\
    \var{is\_default\_route} & gibt an, ob das aktuelle
                               \texttt{ip-up}/\texttt{ip-down} für die
                               Schnittstelle durchgeführt wird, über welche die
                               Default-Route geht (kann "`yes"' oder "`no"'
                               sein)\\
\end{tabular}
\end{table}

\subsubsection{Default-Route}

Seit Version 2.1.0 werden die \texttt{ip-up}/\texttt{ip-down}-Skripte nicht nur
für die Schnittstelle ausgeführt, über welche die Default-Route geht, sondern
für alle Verbindungen, welche die \texttt{ip-up}- und \texttt{ip-down}-Skripte
aufrufen. Um das alte Verhalten zu simulieren, muss in \texttt{ip-up}- und
\texttt{ip-down}-Skripten die folgende Abfrage eingefügt werden:

\begin{example}
\begin{verbatim}
    # is a default-route-interface going up?
    if [ "$is_default_route" = "yes" ]
    then
        # die eigentlichen Aktionen
    fi
\end{verbatim}
\end{example}

Natürlich darf das neue Verhalten auch für spezielle Aktionen
ausgenutzt werden.
