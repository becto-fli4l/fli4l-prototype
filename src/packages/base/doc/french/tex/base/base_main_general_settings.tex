% Do not remove the next line
% Synchronized to r51346

\section{Configuration générale}

\begin{description}
  \config{HOSTNAME}{HOSTNAME}{HOSTNAME}

  Configuration par défaut~: \var{HOSTNAME='fli4l'}

  {Tout d'abord, vous devez donner un nom au routeur fli4l.}

  \config{PASSWORD}{PASSWORD}{PASSWORD}

  Configuration par défaut~: \var{PASSWORD='fli4l'}

    Ici le mot de passe est nécessaire pour accéder au routeur fli4l~--
    que ce soit par le clavier branché au routeur ou via le SSH depuis un autre
    ordinateur (pour cela, le paquetage sshd est nécessaire). Le mot de passe
	doit être composé d'au moins un à 126 caratères.

  \config{BOOT\_TYPE}{BOOT\_TYPE}{BOOTTYPE}

  Configuration par défaut~: \var{BOOT\_TYPE='hd'}

  \var{BOOT\_TYPE} dans cette variable on configure le média de boot. cette
  variable recherche le pilote supplémentaire (module kernel) et le script de
  démarrage dans RootFS. Voici une courte explication du processus de boot
  (ou démarrage)~:

  \begin{itemize}
  \item Le BIOS de l'ordinateur charge le média et démarre le Bootloader.
  \item Le Bootloader décompacte (en règle générale syslinux) et commence à
    exécuter le kernel.
  \item Le kernel décompacte RootFS (= il contient les fichiers systèmes, les
    programmes, les scripts), monte RootFS et commence à traiter les scripts.
  \item Maintenant selon le \var{BOOT\_TYPE} les modules du kernel sont chargés
    pour booter le média respectif, monte les partitions, décompacte l'archive OPT
    (\texttt{opt.img}) et charge les programmes supplémentaires.
  \item La configuration des différents services de fli4l commence.
  \end{itemize}

  Voici les options pour la variable \var{BOOT\_TYPE}~:

  \begin{description}
  \item[ls120] Démarre à partir d'un LS120/240 ou un disque ZIP.
  \item[hd] Démarre à partir d'un disque dur IDE ou SATA, il sont détectés
            directement. Pour un support SCSI, USB ou un contrôleur spéciale, le
            paquetage HD et/ou USB est requis pour l'installation. Pour plus informations
            voir la \jump{sec:hdinstall}{documentation} du paquetage hd.
  \item[cd] Démarre à partir d'un CD-ROM. Vous devez simplement créer une image ISO
            pour le CD, exemple fli4l.iso, qui sera ensuite graver sur le CD avec
            l'un de vos programmes préféré. Si vous avez besoins d'un pilote spécifique
			pour le CD-ROM, par exemple SCSI, USB ou un contrôleur spéciale, le paquetage
			HD et/ou USB est requis pour l'installation.
  \item[integrated] Choisissez cette option si vous ne prévoyez pas d'utiliser un support de
					boot classique, mais une installation par le réseau. L'archive OPT est
					intrégrée dans le RootFS, ainsi le kernel extrait tout à la fois et n'a
					pas besoin de monter un support de boot. La variable \var{BOOT\_TYPE} est
					nécessaire pour une installation depuis le réseau.\\
        \textbf{Notez~:} la mise à jour par le réseau n'est naturellement pas possible.
  \item[attached] Ce paramètre est similaire à \textbf{integrated} mais l'archive OPT
				  \texttt{opt.img} n'est pas intrégrée dans le RootFS. il sera copié dans
				  le répertoire \texttt{/boot} et sera extrait pendant le processus de démarrage.\\
                  La mise en garde décrite pour \texttt{integrated} est identique ici.
  \item[netboot] Ce paramètre est similaire à \textbf{integrated}. Toutefois, le script
                 \texttt{mknetboot.sh} sera exécuté pour créer l'image, celle-ci sera
                 exécuté sur le LAN (ou réseau local). S'il vous plaît lire la
                 documentation sur Wiki
                 \altlink{https://ssl.nettworks.org/wiki/display/f/fli4l+und+Netzboot}
                 pour plus d'informations.
  \item[pxeboot] Deux images seront générées, le kernel et le rootfs.img. Ces deux fichiers
				 seront utilisés par le chargeur de boot PXE. Pendant l'exécution vous pouvez
				 créer un répertoire TFTP, vous pouvez même créer un sous-répertoire TFTP
				 avec (--pxesubdir). Vous pouvez vous référer à la documentation sur Wiki,
				 à cette adresse~:
                 \altlink{https://ssl.nettworks.org/wiki/display/f/fli4l+und+Netzboot}.
  \end{description}

  \textbf{Notez~:} fli4l doit être configuré comme un serveur de boot avec
  les paramètres (pxe/tftp) appropriés, vous trouverez de la documentation
  dans le paquetage dns\_dhcp

  \config{LIBATA\_DMA}{LIBATA\_DMA}{LIBATADMA}

  Avec cette variable vous pouvez désactiver le DMA pour les périphériques basés
  sur libata. Il est parfois nécessaire d'utiliser cette fonction, lorsque
  plusieurs périphérique différent sont raccordés à l'IDE, exemple un adaptateurs
  Compact Flash. le paramètre par défaut est~: 'disabled'

  \config{MOUNT\_BOOT}{MOUNT\_BOOT}{MOUNTBOOT}

    Configuration par défaut~: \var{MOUNT\_\-BOOT='rw'}

  {Ici on règle, la manière de "monter" un média de boot. Il y a trois possibilités~:

    \begin{description}
    \item[rw]~-- Read/Write~-- Possibilité de lecture et d'écriture.
    \item[ro]~-- Read-Only~-- Possibilité de lecture uniquement.
    \item[no]~-- None~-- Le média sera démonté après le démarrage. Il pourra
    être enlevé si besoin.
    \end{description}

    Certaines configurations nécessitent le montage du média au démarrage
    avec le paramètre Read/Write, par exemple, si vous voulez si vous voulez
    installer un serveur DHCP ou installer un fichier log (ou journal)
    pour imond sur le média.}

  \config{BOOTMENU\_TIME}{BOOTMENU\_TIME}{BOOTMENUTIME}

    Configuration par défaut~: \var{BOOTMENU\_TIME='20'}

  {Ici on règle le temps d'attente du Bootloader de syslinux avant de lancer
   automatiquement l'installation standard.

    Dans le paquetage HD il y a la possibilité d'activer la fonction \var{OPT\_RECOVER}
    en cas d'instabilité de la version une installation secondaire peut être générée,
    au cas où l'installation courante aurait un problème. Celle-ci peut être
    activée dans le menu boot vous pouvez choisir la version Recover.

    Si vous mettez la valeur '0', le système attend que l'utilisateur active
    le chargement du Bootloader syslinux standard ou la version Recover
    que vous avez sélectionnée~!}

  \config{TIME\_INFO}{TIME\_INFO}{TIMEINFO}

  Configuration par défaut~: \var{TIME\_INFO='MEZ-1MESZ,M3.5.0,M10.5.0/3'}

  Les heures passent dans le monde d'Unix, elles passent aussi sous fli4l
  avec la norme UTC (Coordinated Universal Time) une heure unique dans le
  monde et qui sera convertit pour chaque localité. la variable \var{TIME\_\-INFO}
  donne les informations nécessaires à fli4l sur les noms du fuseau
  horaire et régle automatiquement les heures d'été et d'hiver. Pour que cela
  fonctionne correctement il faut régler l'heure UTC (correspond à l'heure
  d'hiver de Londres). on peut utiliser pour la synchronisation le paquetage
  chrony serveur de temps (il est livré avec UTC).

  On paramètre \var{TIME\_\-INFO} avec les valeurs suivantes~:
\begin{example}
\begin{verbatim}
        TIME_INFO='MEZ-1MESZ,M3.5.0,M10.5.0/3'
\end{verbatim}
\end{example}
  \begin{itemize}
  \item \emph{MEZ-1~:} fuseau horaire de l'Europe centrale (\emph{MEZ}),
    avec une heure d'avance \emph{MEZ-1=UTC}.
  \item \emph{MESZ~:} réglage de l'heure d'été (heure d'été en Europe
    centrale). S'il n'y a aucune indication une heure sera ajoutée automatiquement
    à l'heure été.
  \item \emph{M3.5.0,M10.5.0/3~:} date des changements d'horaires d'été et
  d'hiver, le dernier dimanche d'octobre c'est le passage en heure d'hiver.
  \end{itemize}

  Normalement on n'a pas besoin de toucher ces valeurs à moins que l'on
  soit dans un autre fuseau horaire. Si vous voulez adapter ces valeurs, vous devez
  d'abord jeter un coup d'oeil sur les spécifications des variables d'environnements
  elles se trouvent à cette adresse URL (en Anglais)~:
  \altlink{http://pubs.opengroup.org/onlinepubs/009695399/basedefs/xbd_chap08.html}

  \config{RTC\_SYNC}{RTC\_SYNC}{RTCSYNC}{
  Paramètre par défaut~: \var{RTC\_SYNC='hwclock'}

  De nombreux ordinateurs disposent d'une horloge matérielle qui est alimentée par
  une batterie, ainsi l'horloge continue de compter les l'heures pendant l'arrêt
  de l'ordinateur afin qu'elle soit à nouveau disponible en tant qu'horloge système au
  prochain démarrage. Il est important de faire la distinction entre \emph{l'horloge système}
  et \emph{l'horloge matériel}~:

  \begin{itemize}
  \item \emph{L'horloge matériel} est l'heure stockée et maintenue par l'horloge
  matérielle. L'heure est généralement lu à partir de l'horloge matérielle lorsque
  le système est démarré et adopté en tant qu'horloge système.

  \item \emph{L'horloge système} est l'heure réelle que le système Linux utilise,
  c'est elle qui est affichée quand vous utilisez la commande \texttt{date -u}.
  elle est maintenu à jour par le kernel (ou noyau) Linux, elle est basé sur l'interruption
  matérielle régulière (Timer Interrupt), elle désigne toujours l'heure en utilisant
  le temps universel coordonné (UTC) et n'est pas influencé par le fuseau horaire terrestre.

  \item \emph{L'horloge système localisée} est juste la conversion de l'heure système
  en appliquant l'heure du fuseau horaire terrestre, elle est configuré sur le routeur
  fli4l via la variable d'environnement \texttt{TZ} (voir la variable \jump{TIMEINFO}{\var{TIME\_INFO}}),
  elle ne joue aucun rôle dans la suite de cette section.
  \end{itemize}

  Cette variable informe comment fli4l calibre l'horloge matériel avec l'horloge du système,
  c'est-à-dire à quelle fréquence l'horloge matériel doit être réglée sur l'horloge système.
  Un tel ajustement est nécessaire, car même la meilleure horloge matérielle n'est pas
  précise à 100~\% et tend à dériver systématiquement, c'est-à-dire que l'horloge sera
  légèrement trop lente ou trop rapide à long terme.

  Il y a essentiellement deux façons de synchroniser l'heure~:

  \begin{itemize}
  \item Mode "Kernel"~: le client NTP est utilisé pour déterminer l'heure réelle extérieur
  (généralement via Internet ou par une horloge (radio) externe) et pour maintenir à
  jour l'horloge système du routeur fli4l. Le kernel Linux est chargé de mettre à jour l'horloge
  matériel, de sorte qu'aucune synchronisation supplémentaire ne soit nécessaire. La mise à
  jour par le kernel Linux est légèrement moins précise par rapport à l'utilisation de \texttt{hwclock}
  (voir le mode "hwclock" ci-dessous), cependant, la qualité de mise à jour est beaucoup moins
  importante, en raison des erreurs inévitables qui seront compensées par le client NTP.

  Ce mode doit également être utilisé s'il n'existe aucune horloge matérielle. Bien sûr,
  le kernel Linux ne mettra pas à jour l'horloge matérielle car il n'y en a pas. Cependant,
  le client NTP doit être utilisé pour s'assurer que le routeur fli4l dispose d'une horloge
  système raisonnable.

  \item Mode "Hwclock"~: en utilisant le programme \texttt{hwclock} l'horloge se synchronise
  à intervalle réguliére (toutes les 24 heures) et à l'arrêt du système (avec l'exécution
  du script stop \texttt{/etc/rc0.d/rc950.hwclock}). Non seulement l'horloge matérielle est
  définie, mais \texttt{hwclock} mesure également la diffèrence entre l'horloge système et
  l'horloge matérielle. Lors du démarrage du système, l'horloge système n'est pas prise en compte
  directement à partir de l'horloge matérielle, mais l'écart entre les deux sera pris en compte
  afin de réduire au maximum la dérive de l'horloge système. Cette écart est noté dans le fichier
  \texttt{/etc/adjtime}. Si un support réinscriptible est installé, l'écart sera stocké sous
  \texttt{/var/lib/persistent/base/adjtime}, dans ce cas, \texttt{/etc/adjtime} sera un lien
  symbolique.

  Ce mode est incompatible avec le client NTP pour la mise à jour de l'horloge système.
  En effet, le client NTP permet la mise à jour automatique de l'horloge matérielle
  par le kernel Linux. Il n'est pas logique que \texttt{hwclock} et le kernel
  Linux essayent de mettre à jour en même temps l'horloge matériel.
  \end{itemize}

  Veuillez noter que si une horloge matérielle est disponible (avec une batterie), l'heure
  stockée sera \emph{toujours} interprétée comme le temps universel coordonné (UTC). Le
  fuseau horaire défini dans la variable \var{TIME\_INFO} n'affecte pas l'heure stockée dans
  l'horloge matérielle. Le stockage d'une heure non UTC localisée par l'horloge matérielle
  n'est \emph{pas} supporté par fli4l.

  L'établissement de l'horloge système à partir de l'horloge matérielle est effectuée
  une fois que le système est démarragé. Le kernel Linux lit l'horloge matérielle et
  définit l'horloge système immédiatement au début du processus de démarrage. En mode
  "hwclock", l'horloge système sera définie plus tard lors de l'exécution du script
  de démarrage \texttt{/etc/rc.d/rc100.hwclock}, cette fois en tenant compte de l'écart
  systématique.
  }

  \config{KERNEL\_VERSION}{KERNEL\_VERSION}{KERNELVERSION}

  Ici on détermine la version du kernel (ou noyau) à utiliser, cette variable
  doit correspondre au kernel \emph{img/kernel-$<$kernel version$>$.$<$extension
  compression$>$}, on peut voir la version du Kernel dans le répertoire
  \emph{opt/lib/modules/$<$kernel version$>$}.

  \config{KERNEL\_BOOT\_OPTION}{KERNEL\_BOOT\_OPTION}{KERNELBOOTOPTION}

  Configuration par défaut~: \var{KERNEL\_BOOT\_OPTION=''}

  Ici vous pouvez ajouter les variables en ligne de commande pour le kernel,
  ils seront rajoutées dans le fichier syslinux.cfg. Par exemple certain système
  on besoin pour rebooter correctement d'ajouter 'reboot=bios', avec un système
  WRAP vous pouvez ajouter 'nokdb reboot=bios'.

  \config{COMP\_TYPE\_ROOTFS}{COMP\_TYPE\_ROOTFS}{COMPTYPEROOTFS}

  Configuration par défaut~: \var{COMP\_TYPE\_ROOTFS='xz'}

  Le contenu de cette variable détermine la méthode de compression pour
  l'archive RootFS. Les valeurs possibles sont 'xz', 'lzma' et 'bzip2'.

  \config{COMP\_TYPE\_OPT}{COMP\_TYPE\_OPT}{COMPTYPEOPT}

  Configuration par défaut~: \var{COMP\_TYPE\_OPT='xz'}

  Le contenu de cette variable détermine la méthode de compression pour
  l'archive OPT. Les valeurs possibles sont 'xz', 'lzma' et 'bzip2'.

  \config{POWERMANAGEMENT}{POWERMANAGEMENT}{POWERMANAGEMENT}
  
  Configuration par défaut~: \var{POWERMANAGEMENT='acpi'}

  {Le kernel supporte différents formats de gestion d'énergie, l'APM qui est un
  peu âgé et l'actuel ACPI. Vous pouvez placer ici le format que vous voulez
  utiliser. Les valeurs possibles sont~: 'none' (aucune gestion d'énergie),
  'ACPI' et les deux variantes de APM 'apm' et 'apm\_rm'. Ce dernier commute
  en mode processeur spécial, avant que le routeur s'arrête.}

  \config{FLI4L\_UUID}{FLI4L\_UUID}{FLI4LUUID}

  Configuration par défaut~: \var{FLI4L\_UUID=''}

  {Vous pouvez indiquer dans cette variable un UUID (ou IDentifiant Universellement
  Unique), dans lequel fli4l pourra enregistrer des données persistantes par
  exemple sur une clé USB. Cette UUID peut être créé avec n'importe quel
  Système Linux (et aussi avec fli4l) avec la commande \verb*?'cat /proc/sys/kernel/random/uuid'?
  Chaque exécution de la commande produit un nouvel UUID que vous devez entrer
  dans la variable. Sur un support persistant (par exemple, un disque dur
  (OPT\_HD) ou une clé USB (voir paquetage OPT\_USB et OPT\_HD) vous devez créer un
  répertoire avec le même nom que l'UUID. Ce répertoire sera utilisé pour stocker
  les changements de configuration ainsi que les données d'exécution persistante
  (par ex. pour le dhcp leases (ou baux DHCP). Naturellement les paquetages
  correspondants pour ce nouveau mécanisme doivent être supportés (voir leur
  documentation). Le paramètre pour sauvegarder le chemin sera généralement 'auto'.

  Si vous avez installé fli4l avant d'utiliser l'application UUID et que des
  données sont déjà stockées dans le répertoire fli4l, vous devez natuellement
  déplacer ces données dans le nouveau répertoire /boot/persistant. Il est 
  recommandé par conséquent de configurer l'UUID à l'installation de fli4l pour
  éviter de déplacer les données.

  En outre vous ne devez pas paramétrer la variable comme ceci \var{MOUNT\_BOOT}='ro',
  tant que l'emplacement de stockage fait partie de la partition /boot.

  Un endroit recommandé pour le répertoire persistant est situé dans la partition
  /data (niveau supérieur) ou sur une clé USB.  de la clé USB
  doit être de type VFAT ou activer le fichier système pour OPT\_HD avec les
  autorisations en écriture et lecture.}

  \config{IP\_CONNTRACK\_MAX}{IP\_CONNTRACK\_MAX}{IPCONNTRACKMAX}

  Configuration par défaut~: \var{IP\_CONNTRACK\_MAX=''}

  Avec cette variable, vous pouvez régler manuellement la quantité
  maximum de connexions simultanées. Normalement une valeur rationnelle
  est trouvée automatiquement par rapport à la mémoire vive installée.
  Le tableau \ref{tab:connectiontracking} représente la configuration par défaut.

    \begin{table}[ht!]
        \centering
        \caption{Réglage Automatique du nombre de connexions maximum}\marklabel{tab:connectiontracking}{}
        \begin{tabular}{p{6cm}p{6cm}}
          Mémoire RAM Mio   &    Connexions simultanées \\\hline
          16                      &    1024 \\
          24                      &    1280 \\
          32                      &    2048 \\
          64                      &    4096 \\
          128                     &    8192 \\
        \end{tabular}
    \end{table}

  Si vous utilisez sur le routeur des programmes de partage de fichiers en 
  arrière ou si le routeur a peu de RAM. Le nombre maximal de connexions 
  simultanées sera rapidement atteint et les connexions supplémentaires ne
  pourront plus être développées.\\
  Cela se traduit par un message erreur qui s'affichera~:

\begin{example}
\begin{verbatim}
        ip_conntrack: table full, dropping packet
\end{verbatim}
\end{example}

   Autre message

\begin{example}
\begin{verbatim}
        ip_conntrack: Maximum limit of XXX entries exceeded
\end{verbatim}
\end{example}

    Maintenant au moyen de la variable \var{IP\_\-CONNTRACK\_\-MAX} vous pouvez
    régler précisément la valeur du nombre maximum de connexions simultanées.
    Cependant vous devez savoir. Pour chaque connexions simultanées cela coûte
    350 Octets de mémoire RAM en moins, qui ne seront plus utilisés pour autre
    chose. Si vous indiquez 10000, on perd à peu près 3,34 Mo de mémoire RAM
    pour l'utilisation (du kernel, de Ramdisks et des programmes).

    Avec 32 Mio de RAM, il ne devrait pas y avoir de problème, pour la table
    ip\_conntrack 2 ou 3 Mio seront réservés, voir le tableau. Avec 16 Mio de
    RAM c'est juste, mais avec 12 ou même 8 Mio on est sur d'avoir un message erreur.

    Le réglage en cours d'utilisation peuvent être affichées sur la console en tapant

\begin{example}
\begin{verbatim}
        cat /proc/sys/net/ipv4/ip_conntrack_max
\end{verbatim}
\end{example}

    et peut être modifié à la volée en tapant

\begin{example}
\begin{verbatim}
        echo "XXX" > /proc/sys/net/ipv4/ip_conntrack_max
\end{verbatim}
\end{example}

    "XXX" indique la quantité de connexions simultanées à entrer. Vous pouvez
    afficher sur la console, le nombre de connexion de la variable \var{IP\_CONNTRACK}
    en tapant

\begin{example}
\begin{verbatim}
        cat /proc/net/ip_conntrack
\end{verbatim}
\end{example}

    Pour voir les détails

\begin{example}
\begin{verbatim}
        cat /proc/net/ip_conntrack | grep -c use
\end{verbatim}
\end{example}

  \config{LOCALE}{LOCALE}{LOCALE}

  Configuration par défaut~: \var{LOCALE}='de'

  Certains composants sont devenus entre-temps multi langues. Par exemple,
  le menu de l'interface Web. Avec cette variable, vous pouvez choisir votre
  langue préférée. Différents composants ont leur propre paramètre de base,
  avec ce réglage le paramètre sera tronqué, si la langue indiquée n'est pas
  (encore) disponible pour ces composants, l'anglais sera la langue par défaut.

  Si la variable est sur \var{KEYBOARD\_LOCALE}='auto' on utilise le clavier
  commun à la langue qui est indiquée dans la variable \var{LOCALE}.

  Les réglages suivants sont possibles~: de, en, es, fr, hu, nl.

\end{description}
