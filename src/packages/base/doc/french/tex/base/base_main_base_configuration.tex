% Synchronized to r49626

\chapter{Configuration de la base}

A partir de la version 2.0 la distribution fli4l est devenue modulaire, elle est
partagée en plusieurs paquetages, ils peuvent être téléchargés séparément.
Le paquetage \texttt{fli4l-\version.tar.gz} contient uniquement le logiciel de base
pour le routeur Ethernet. On téléchargera ensuite les paquetages dont on a besoin
pour une connexion DSL, ISDN ils seront extraits dans le répertoire \texttt{fli4l-\version/} (!)
Vous avez le choix du Kernel (ou noyau) pour le système d'exploitation fli4l, les Kernels
ont été sous-traitées dans des paquetages différents. Vous avez besoin au minimum
du paquetage de base et d'un Kernel pour l'installation. Dans le tableau \ref{tab:zusatzpakete}
vous trouverez un aperçu des paquetages supplémentaires.

\begin{table}[ht!]
 \caption{Aperçu des paquetages supplémentaires}\marklabel{tab:zusatzpakete}{}
  \begin{center}
    \begin{tabular}{ll}
      \textbf{Archive à télécharger}  &    \textbf{Paquetage} \\
      \hline
      \texttt{fli4l-\version}         &    Base, nécessaire~!\\
      \verb*zkernel_4_19z             &    Kernel Linux, nécessaire~!\\
      \texttt{fli4l-\version-doc}     &    Documentation complète \\
      \verb*zadvanced_networkingz     &    Configuration pour réseau étendue\\
      \verb*zcertz                    &    Gestion des certificats\\
      \verb*zchronyz                  &    Serveur/Client de temps\\
      \verb*zdhcp_clientz             &    Divers clients DHCP\\
      \verb*zdns_dhcpz                &    Serveur DNS et serveur DHCP\\
      \verb*zdslz                     &    Routeur DSL (PPPoE, PPTP)\\
      \verb*zdyndnsz                  &    Supporte le service DYNDNS\\
      \verb*zeasycronz                &    Service de planification\\
      \verb*zhdz                      &    Installation sur disque dur\\
      \verb*zhwsuppz                  &    Supporte du matériel spécifique\\
      \verb*zhttpdz                   &    Mini serveur Web pour le statut - information\\
      \verb*zimonc_windowsz           &    Imonc pour Windows\\
      \verb*zimonc_unixz              &    Imonc pour GTK-Unix\\
      \verb*zipv6z                    &    Internet Protocole Version 6\\
      \verb*zisdnz                    &    Routeur ISDN\\
      \verb*zopenvpnz                 &    Supporte le VPN\\
      \verb*zpcmciaz                  &    Supporte les cartes PCMCIA\\
      \verb*zpppz                     &    Liaison PPP sur interface série\\
      \verb*zproxyz                   &    Serveur proxy\\
      \verb*zqosz                     &    Quality of Service (ou service de qualité)\\
      \verb*zsshdz                    &    Serveur SSH\\
      \verb*ztoolsz                   &    Divers outils et programmes pour Linux\\
      \verb*zumtsz                    &    Connexion UMTS via Internet\\
      \verb*zusbz                     &    Supporte les interfaces USB\\
      \verb*zwlanz                    &    Supporte les cartes WLAN
    \end{tabular}
  \end{center}
 \end{table}

Les fichiers utilisés pour configurer le routeur fli4l se trouvent dans le répertoire
\texttt{config/} et seront décrits dans les pages suivantes de la documentation.

Ces fichiers peuvent être modifiés avec un \emph{simple} éditeur de texte ou
avec un éditeur spécialement adapté pour fli4l. Vous trouverez cette éditeur et
d'autres logiciels sous Windows pour vous aidé à configurer fli4l à cette adresse

\par

\altlink{http://www.fli4l.de/fr/telechargement/paquetages-annexes/addons/}.

Si des adaptations/extensions sont nécessaires pour des réglages spécifiques,
autres que ceux décrits ci-dessus, vous aurez besoin d'installer un système
linux afin d'éditer le rootf. Dans ce cas vous devriez lire le fichier README
dans le répertoire \verb+src/README+.

\newpage

