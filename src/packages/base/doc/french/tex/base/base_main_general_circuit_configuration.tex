% Do not remove the next line
% Synchronized to r29829

  \section{Configuration du circuit général}

  \begin{description}
    \config{IP\_DYN\_ADDR}{IP\_DYN\_ADDR}{IPDYNADDR}

    {Si une connexion avec une IP dynamique est utilisée, vous
     devez placée la variable \var{IP\_\-DYN\_\-ADDR} sur 'yes', ou
     sur 'no' si statique. La plupart des fournisseurs d'accés utilisent
     une IP dynamique.

      Configuration par défaut~: \var{IP\_\-DYN\_\-ADDR}='yes'}

    \config{DIALMODE}{DIALMODE}{DIALMODE}

    {Par défaut fli4l utilise 'auto' pour le mode de numérotation,
     c-à-d. une connexion sera établie automatiquement dès qu'un ordinateur
     du réseau local essayera d'accéder à une adresse IP extérieure par ex.
     Internet. Il est également possible de spécifier le modes de connexion
     'manual' ou 'off'. Dans ce cas, la connexion peut uniquement être déclenchée
     en utilisant le client imonc.

      Configuration par défaut~: \var{DIALMODE}='auto'}

  \end{description}
