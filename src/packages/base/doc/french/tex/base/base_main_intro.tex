% Synchronized to r34514
\chapter{Documentation du paquetage base}

\section{Introduction}

fli4l est un routeur basé sur Linux, il est capable de traiter les connexions
ISDN (en france RNIS), DSL, UMTS et Ethernet, avec une petite configuration
matériel~: une clé USB pour booter, un processeur Intel Pentium MMX, 64 Mio de RAM,
(au moins) une carte réseau Ethernet, cela est tout à fait suffisant pour créer
un routeur. Les médias nécessaires pour le Boot peuvent être créés sous Linux
ou sous MS~Windows. Vous n'avez pas besoin de connaissances spécifiques sur Linux,
mais cela est utile. Cependant, vous devez posséder quelques connaissances sur
les réseaux TCP/IP, DNS et sur le routage. Pour développer vos propres extensions
qui seront ajoutées à la configuration de base, vous aurez besoin d'un système
Linux ainsi que des compétences Linux.

fli4l prend en charge différents médias de boot, parmi eux, les clés USB,
les disques durs, les CDs et en particulier le boot par le réseau. Une clé USB
est l'idéal à bien des égards~: aujourd'hui, presque tous les PC peut démarrer
à partir de celle-ci, elle est relativement pas cher, elle a une grand capacitée
et l'installation de fli4l est relativement facile sous MS~Windows et Linux. En
outre, elle est ouverte en écriture et peut contenir des données de configuration
non volatiles (par exemple, les baux du serveur DHCP) contrairement à un CD.

\begin{itemize}
\item Caractéristiques générales

\begin{itemize}
\item Création du média de Boot sous \jump{sec:bootmedien_linux}{Linux},
      \jump{sec:bootmedien_linux}{Mac OS~X} et
      \jump{sec:bootmedien_windows}{MS~Windows}
\item Configuration des fichiers via ASCII/UTF-8
\item Supporte l'IP Masquerading et le Port Forwarding
\item Least-Cost-Routing (LCR) (ou Routage à Moindre-Coût)~: pour choisir
      automatiquement le fournisseur d'accès Internet, selon l'heure d'utilisation
\item Affiche/Calcul/Enregistre les temps de connexion et les coûts
\item Client imonc pour MS~Windows/Linux converse avec imond et telmond
\item Télécharge les mis à jour des fichiers de configurations via le client
      imonc sous MS~Windows ou via le SCP sous Linux
\item Les médias de Boot utilisent le système de fichier VFAT pour le stockage
      durable les données
\item Filtrage de paquets~: les accès aux ports externes bloqués sont enregistrés
\item Affectation uniforme des interfaces WAN et des soi-disant Circuits
\item Utilsation possible des Circuits ISDN et DSL/UMTS en parallèle
\end{itemize}

\item Fonctionnalité basic du routeur

\begin{itemize}
\item Kernels Linux 3.18 ou 3.19
\item Filtrage de paquets et IP Masquerading
\item Serveur DNS local afin de réduire le nombre de requêtes DNS sur les serveurs
      DNS externes
\item Accessibilité à distance du serveur du démon imond pour surveiller et
      contrôler le Least Cost Routing (ou Moindre-Coût-Routage)
\item Accessibilité à distance du serveur du démon telmond pour les détails des
      appels téléphoniques entrants
\end{itemize}

\item Supporte l'Ethernet

\begin{itemize}
\item Pilote de carte réseau~: actuellement supporte plus de 140 types de cartes
\end{itemize}

\item Supporte la DSL

\begin{itemize}
\item Le pilote Roaring Penguin PPPoE, supporte la connexion à la demande (peut
      être désactivé)
\item PPTP pour les fournisseurs DSL en Autriche et aux Pays-Bas
\end{itemize}

\item Supporte l'ISDN

\begin{itemize}
\item Supporte aux moins 60 types d'adaptateurs
\item Multiples possibilités de connexions ISDN~: entrant/sortant/rappel,
      "roh"/point-to-point (ppp)
\item Regroupement de canaux~: adaptation automatique de la bande passante ou
      activation manuelle du deuxième canal en utilisant le logiciel client
      sous MS~Windows/Linux
\end{itemize}

\item Paquetages optionnels

\begin{itemize}
\item Serveur DNS
\item Serveur DHCP
\item Serveur SSH
\item Affichage online/offline par simple LED
\item Console série
\item Serveur Web minimaliste pour la surveillance des connexions RNIS et DSL
      ainsi que pour la reconfiguration et/ou la mise à jour du routeur
\item Droit d'accès pour configurer certains réseaux extérieur
\item Possibilité d'installer des carte PCMCIA (appelé de nos jours carte PC)
\item Enregistrement des messages du système
\item Configuration des cartes ISAPnP en l'utilisant l'outil isapnp
\item Outils supplémentaires pour le débogage (ou correction d'erreurs)
\item Configuration de l'interface série
\item Système de sauvetage avec l'administration à distance via le réseau ISDN
\item Logiciel pour afficher des informations de configurable sur un écran LCD,
      par exemple les taux de transmissions, la charge du CPU, etc.
\item Serveur/Routeur PPP par l'interface série
\item Modem ISDN par l'interface série
\item Serveur d'impression
\item Synchronisation de l'heure avec les serveurs de temps externe
\item Exécution des commandes définies par l'utilisateur, pour les appels
      téléphoniques entrants (par ex. pour composer un numéro sur Internet)
\item Supporte l'IP Aliasing (plusieurs adresses IP par interface réseau)
\item Supporte le VPN
\item Supporte l'IPv6
\item Supporte le WLAN~: fli4l peut être à la fois point d'accès et client
\item Outil RRD pour la surveillance du routeur fli4l
\item Et beaucoup plus \ldots
\end{itemize}

\item Matériels requis

\begin{itemize}
\item Un processeur Intel Pentium avec le support MMX
\item 64 Mio de RAM, mieux 128 Mio
\item Une carte réseau Ethernet
\item ISDN~: un adaptateur supportant l'ISDN
\item Une clé USB, un disque dur ATA ou d'une carte CF (qui sera accessible de
      la même manière qu'un disque dur ATA), il est également possible de booter
      à partir d'un CD
\end{itemize}

\item Logiciels requis

Sous Linux, les programmes suivants sont demandés~:

\begin{itemize}
\item GCC et GNU make
\item syslinux
\item mtools (mcopy)
\end{itemize}

Sous MS~Windows, aucun outil supplémentaire n'est demandé, fli4l apporte tout
le nécessaire.
\end{itemize}

Vous avez en plus, le client imonc qui commande et affiche l'état du routeur
fli4l. Ce programme est disponible pour Windows (windows/imonc.exe) et pour
Linux (unix/gtk-imonc).

Et maintenant \ldots \bigskip

Amusez-vous bien avec fli4l~!\bigskip

Frank Meyer et l'équipe fli4l

\email{team@fli4l.de}
