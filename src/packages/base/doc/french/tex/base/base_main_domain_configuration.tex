% Do not remove the next line
% Synchronized to r40999

\marklabel{sec:domainkonfiguration}{\section{Configuration du domaine}}

  Dans un LAN les ordinateurs Windows ont des caractéristiques désagréables~:
  si vous avez besoin d'utiliser un serveur de nom (ou DNS), vous devez
  configurer vos PC Windows pour ce service. Le problème c'est que les PCs
  Windows questionnent à intervalle régulier le serveur~-- même si personne
  n'utilise l'ordinateur~!
  Si vous configurez un serveur-DNS sur Internet pour votre PC Windows, cela
  pourrait revenir très cher \ldots

  Le truc est le suivant~: s'il n'y a pas de serveur DNS disponible sur
  le LAN (ou réseau local), on peut utiliser le routeur fli4l comme
  serveur DNS.

  DNSMASQ est utilisé en tant que serveurs DNS.

  Avant que nous commençions la configuration du DNS, vous devez d'abord
  réfléchir au nom de domaine et aux noms des PCs avant de les écrire dans
  votre réseau local. Le nom de domaine que vous emploierez ne sera pas
  visible sur Internet. Ainsi vous êtes libre pour employer presque
  n'importe quel nom de domaine.

  Vous devez donner un nom à chaque ordinateur de Windows dans votre réseau.
  En outre le routeur-fli4l doit avoir connaissance de ces noms.

  \begin{description}
    \config{DOMAIN\_NAME}{DOMAIN\_NAME}{DOMAINNAME}

    Configuration par défaut~: \var{DOMAIN\_NAME='lan.fli4l'}

      {Dans la version fli4l le nom de domaine "lan.fli4l" est paramètrè par 
      défaut. Vous êtes libre d'écrire votre propre nom de domaine. Vous devez
      éviter d'utiliser un nom qui pourrait exister sur Internet.
      Si vous employez un nom de domaine existant (par ex. france3.fr), vous ne
      pourrez pas accéder à ce domaine.}

    \config{DNS\_FORWARDERS}{DNS\_FORWARDERS}{DNSFORWARDERS}

    Configuration par défaut~: \var{DNS\_FORWARDERS=''}

      {On indique dans cette variable l'adresse IP du serveur DNS de votre
      Fournisseur Accès Internet (ou FAI), si fli4l est utilisé comme routeur
      pour Internet. Le routeur fli4l expédie à cette adresse toutes les
      requêtes DNS auxquelles il ne peut pas répondre.

      Vous pouvez entrer plusieurs adresses IP pour les serveurs DNS, vous devez
      séparer toutes les adresses IP par un espace (ou un blanc).

      Si plusieurs serveurs DNS sont configurés, les requêtes DNS seront utilisés
	  dans l'ordre de configuration des serveurs, ainsi le deuxième serveur spécifié
	  sera utilisé seulement si le premier n'a pas répondu à la requête DNS, etc.

      Il est également possible d'ajouter en option un numéro de port à
      l'adresse IP, pour cela, il faut séparer l'adresse IP et le port par deux
      points. Toutefois, il est nécessère d'activer la variable
      \jump{OPTDNS}{\var{OPT\_DNS='yes'}} dans le (paquetage
      \jump{sec:dnsdhcp}{dns\_dhcp}), cependant, cette variable
      ne doit jamais être substituée à la variable \var{*\_USEPEERDNS}.

      Attention~:
        \begin{itemize}
        \item \jump{PPPOEUSEPEERDNS}{\var{PPPOE\_USEPEERDNS}},
        \item \jump{ISDNCIRCxUSEPEERDNS}{\var{ISDN\_CIRC\_x\_USEPEERDNS}}
        ou
        \item \jump{DHCPCLIENTxUSEPEERDNS}{\var{DHCPCLIENT\_x\_USEPEERDNS}}
        \end{itemize}

        L'une de ces variables doit être paramétrées sur (='yes'), cela est
        nécessaire pour que le serveur DNS externe soit enregistré, sinon après
        le démarrage du routeur aucune résolution de nom ne sera possible.
        Le serveur DNS externe ne fonctionnera pas.

      Exception~: si vous configurez le routeur fli4l dans un réseau local 
      \emph{sans} connexion Internet ou dans un réseau avec un serveur DNS
      supplémentaire (réseau d'entreprise). Dans ce cas vous devez paramètrer
      l'adresse IP 127.0.0.1 pour empêcher le forwarding (empêcher les connexions
      extérieure).}

    \config{HOSTNAME\_IP}{HOSTNAME\_IP}{HOSTNAMEIP} (optionnelle)

    {Avec cette variable optionnelle vous pouvez définir, le réseau 'IP\_NET\_x'
    qui sera rattaché au \var{HOSTNAME} (ou nom d'Hôte).}

    \config{HOSTNAME\_ALIAS\_N}{HOSTNAME\_ALIAS\_N}{HOSTNAMEALIASN} (optionnelle)

    {Dans cette variable vous indiquez, le nombre d'alias (ou de surnom)
    supplémentaire pour le routeur.}

    \config{HOSTNAME\_ALIAS\_x}{HOSTNAME\_ALIAS\_x}{HOSTNAMEALIASx} (optionnelle)

    {Dans cette variable vous indiquez, l'alias pour le routeur.}

  \end{description}
