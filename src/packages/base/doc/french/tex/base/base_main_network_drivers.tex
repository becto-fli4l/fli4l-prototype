% Synchronized to r49626

\section{Pilotes des cartes réseaux Ethernet}

\begin{description}
    \config{NET\_DRV\_N}{NET\_DRV\_N}{NETDRVN}

    Configuration par défaut~: \var{NET\_DRV\_N='1'}

    {Indiquer ici le nombre de pilote de cartes réseau.

    Si le routeur est utilisé pour l'ISDN (ou numéris), il y a habituellement
    une seule carte réseau, la valeur par défaut est donc '1'.

    Avec l'utilisation d'un modem DSL, on installe souvent deux cartes réseau.

    Il faut distingue deux cas~:
    \begin{enumerate}
    \item Les deux cartes réseaux sont du même type (identique). On doit
      indiquer un seul pilote pour charger les deux cartes donc
      \var{NET\_DRV\_N}='1'.
    \item Les deux cartes réseaux sont de type différent, vous indiquez
      '2' et spécifier un pilote pour chaque carte.
    \end{enumerate}
    }

  \config{NET\_DRV\_x}{NET\_DRV\_x}{NETDRVx}

    Configuration par défaut~: \var{NET\_DRV\_1='ne2k-pci'}

    {On indique ici le pilote pour la ou les cartes réseaux. Dans la variable
    \var{NET\_DRV\_1} le pilote par défaut est NE2000 = carte réseau compatible
	elle sera chargée à l'installation, vous pouvez modifier le pilote selon votre
	configuration. L'ensemble des cartes réseaux sont indiquées dans les tableaux suivant
	\ref{tabkartentreiber} et \ref{tabwlankartentreiber}.

    Au sujet de la carte 3COM EtherLinkIII (3c509) vous avez un outil sous DOS, 3c509cfg.exe
    pour modifier les paramètre de la carte (téléchargeable ici
    \altlink{ftp://ftp.ihg.uni-duisburg.de/Hardware/3com/3C5x9n/3C5X9CFG.EXE})

    Vous pouvez éventuellement configurer l'IRQ et le port I/O pour les
    connecteurs (BNC/TP).}

    \config{NET\_DRV\_x\_OPTION}{NET\_DRV\_x\_OPTION}{NETDRVxOPTION}

    Configuration par défaut~: \var{NET\_DRV\_x\_OPTION=''}

    {En général la variable peut rester vide.

    Les pilotes de certaines cartes ISA ont besoin d'informations supplémentaires
    pour que le système trouve la carte, par exemple, l'adresse I/O. C'est
    le cas de la carte compatible NE2000 ISA et de EtherExpress16.
    Par exemple~:

\begin{example}
\begin{verbatim}
        NET_DRV_x_OPTION='io=0x340'
\end{verbatim}
\end{example}

    Indiquer (la valeur numérique correspondante).

    Si aucun paramètre est nécessaire, la variable peut rester vide.

    Si plusieurs paramètres sont nécessaires, ceux-ci sont à séparer par un espace
    (ou un blanc), par exemple~:
\begin{example}
\begin{verbatim}
        NET_DRV_x_OPTION='irq=9 io=0x340'
\end{verbatim}
\end{example}

    Si deux cartes réseaux identiques sont utilisées, par exemple avec la
    NE2000-ISA, les valeurs des adresses I/O des cartes seront donc différentes
    et doivent être séparées par une virgule.

\begin{example}
\begin{verbatim}
        NET_DRV_x_OPTION='io=0x240,0x300'
\end{verbatim}
\end{example}

    Les deux valeurs I/O doivent être séparées par une virgule sans espace~!

    Cela ne fonctionne pas avec tous les pilotes de carte réseau. Sur quelques
    une vous devez doubler le chargement du pilote, donc \var{NET\_\-DRV\_\-N}='2'.
    Dans ce cas, vous devez attribuer l'option "-o" avec un nom différent,
    par exemple

    \begin{example}
    \begin{verbatim}
          NET_DRV_N='2'
          NET_DRV_1='3c503'
          NET_DRV_1_OPTION='-o 3c503-0 io=0x280'
          NET_DRV_2='3c503'
          NET_DRV_2_OPTION='-o 3c503-1 io=0x300'
    \end{verbatim}
    \end{example}

    Notre conseil~: essayez la première méthode, puis essayez la seconde méthode
    avec l'option "-o".

    Quelques exemples pour la configuration des cartes réseaux~:

    \begin{itemize}
    \item 1 x NE2000 ISA
\begin{example}
\begin{verbatim}
          NET_DRV_1='ne'
          NET_DRV_1_OPTION='io=0x340'
\end{verbatim}
\end{example}

    \item 1 x 3COM EtherLinkIII (3c509)
\begin{example}
\begin{verbatim}
          NET_DRV_1='3c509'
          NET_DRV_1_OPTION=''
\end{verbatim}
\end{example}
      Voir aussi les faq sur les cartes (en Allemand)~:

    \begin{raggedright}
      \altlink{http://extern.fli4l.de/fli4l_faqengine/faq.php?display=faq&faqnr=132&catnr=7&prog=1}\\
      \altlink{http://extern.fli4l.de/fli4l_faqengine/faq.php?display=faq&faqnr=133&catnr=7&prog=1}\\
      \altlink{http://extern.fli4l.de/fli4l_faqengine/faq.php?display=faq&faqnr=135&catnr=7&prog=1}\par
    \end{raggedright}

    \item 2 x NE2000 ISA
\begin{example}
\begin{verbatim}
          NET_DRV_1='ne'
          NET_DRV_1_OPTION='io=0x320,0x340'
\end{verbatim}
\end{example}

      Les valeurs IRQ doivent être placées ici~:

\begin{example}
\begin{verbatim}
          NET_DRV_1_OPTION='io=0x320,0x340 irq=3,5'
\end{verbatim}
\end{example}

      Vous devriez d'abord essayer de booter sans indiquer des interruptions.
      Si le pilote réseau n'est pas identifié, alors ajouter les interruptions.

    \item 2 x NE2000 PCI
\begin{example}
\begin{verbatim}
          NET_DRV_1='ne2k-pci'
          NET_DRV_1_OPTION=''
\end{verbatim}
\end{example}
    \item  1 x NE2000 ISA, 1 x NE2000 PCI
\begin{example}
\begin{verbatim}
          NET_DRV_1='ne'
          NET_DRV_1_OPTION='io=0x340'
          NET_DRV_2='ne2k-pci'
          NET_DRV_2_OPTION=''
\end{verbatim}
\end{example}
    \item 1 x SMC WD8013, 1 x NE2000 ISA
\begin{example}
\begin{verbatim}
          NET_DRV_1='wd'
          NET_DRV_1_OPTION='io=0x270'
          NET_DRV_2='ne2k'
          NET_DRV_2_OPTION='io=0x240'
\end{verbatim}
\end{example}
    \end{itemize}

    Vous pouvez voir la liste de tous les pilotes qui peuvent être installés dans
    la documentation du paquetage kernel.

    \emph{Si vous avez besoin d'un périphérique factice, vous pouvez indiquer
    'dummy' dans la variable \var{NET\_DRV\_x} et\\
    dans la variable \jump{IPNETxDEV}{\var{IP\_NET\_x\_DEV}}='dummy$<$Numéro$>$'
    pour le nom du périphérique.}

    }
\end{description}
