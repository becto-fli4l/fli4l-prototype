% Do not remove the next line
% Synchronized to r34917

\marklabel{chap:bootmedien}{
  \chapter{Création une archive fli4l/Média de Boot}
  }

  Lorsque tous les fichiers de configuration seront paramétrés,
  l'archive fli4l/Média de Boot peut être construite, on peut
  soit utiliser une carte Compact Flash pour booter ou créer une image ISO,
  soit uniquement faire une mise à jour des fichiers.


\marklabel{sec:bootmedien_linux}{
  \section{Création de l'archive fli4l/Média de Boot sous Linux, dérivé Unix
  et Mac OS X}
  }

  La construction se fait à l'aide du Scripts (\texttt{.sh}) qui se trouve
  dans la racine du répertoire de fli4l.

  \begin{description}
    \item \texttt{mkfli4l.sh}
  \end{description}

  Build-Script (ou script de construction) reconnaît indépendamment
  les différentes \jump{BOOTTYPE}{variantes de Boot}.

  La simple commande sous Linux est~:
  \begin{verbatim}
    sh mkfli4l.sh
  \end{verbatim}

  Les trois mécanismes suivant gèrent le démarrage de Build-Scripts~:
  \begin{itemize}
    \item La configuration de la variable \var{BOOT\_TYPE} dans le
          fichier \texttt{$<$config$>$/base.txt}
    \item La configuration du fichier \texttt{$<$config$>$/mkfli4l.txt}
    \item Les paramètres du Build-Scripts
  \end{itemize}

  On décide au moyen de la variable \jump{BOOTTYPE}{\var{BOOT\_TYPE}},
  le type de support de construction (Build-Scripts) pour fli4l~:
  \begin{itemize}
    \item Démarrer fli4l avec un CD-ROM par une image ISO
    \item Faire une mise à jour des fichiers, pour une nouvelle
      version fli4l
    \item Créer les fichiers fli4l et faire une mise à jour à distance via SCP
    \item etc.
  \end{itemize}

  Vous trouverez la description des variables dans le fichier de
  configuration \texttt{$<$config$>$/mkfli4l.txt} et dans le chapitre
  \jump{sec:mkfli4lconf}{Paramètres mkfli4l.txt}.


  \subsection{Lignes de commandes optionnelle}

  Les mécanismes de contrôle sont à ajouter aux paramètres d'option
  lorsque vous appelez le script de compilation par ligne de commande.
  Les options de contrôle sont semblables à ceux du fichier de commande \texttt{mkfli4l.txt}.
  Les spécifications des paramètres d'options remplacent les valeurs du fichier
  de contrôle. Pour des raisons de confort on a différencié, les paramètres
  optionnels et les variables du fichier de construction. les paramètre existe
  sous une forme courte et longue~:

  \begin{verbatim}
Utiliser : mkfli4l.sh [options] [config-dir]

-c, --clean             cleanup the build-directory
-b, --build <dir>       sets build-directory to <dir> for the fli4l-files
-v, --verbose           verbose - some debug-output
    --filesonly         creates only fli4l-files - does not create a boot-media
    --no-squeeze        don't compress shell scripts
-h, --help              display this usage

config-dir              sets other config-directory - default is "config"

--hdinstallpath <dir>   install a pre-install environment directly to
                    usb/compact flash device mounted or mountable to
                    directory <dir> in order to start the real installation
                    process directly from that device
                    device either has to be mounted and to be writable
                    for the user or it has to be mountable by the user
                    Do not use this for regular updates!

*** Remote-Update options
--remoteupdate          remote-update via scp, implies "--filesonly"
--remoteremount         make /boot writable before copying files and
                        read only afterwards
--remoteuser <name>     user name for remote-update - default is "fli4l"
--remotehost <host>     hostname or IP of remote machine - default
                        is HOSTNAME set in [config-dir]/base.txt
--remotepath <path>     pathname on remote maschine - default is "/boot"
--remoteport <portnr>   portnumber of the sshd on remote maschine

*** Netboot options
--tftpbootpath <path>   pathname to tftpboot directory
--tftpbootimage <name>  name of the generated bootimage file
--pxesubdir <path>      subdirectory for pxe files relative to tftpbootpath

*** Developer options
-u, --update-ver    set version to <fli4l_version>-rev<svn revision>
-v, --verbose       verbose - some debug-output
-k, --kernel-pkg    create a package containing all available kernel
                    modules and terminate afterwards.
                    set COMPLETE_KERNEL='yes' in config-directory/_kernel.txt
                    and run mkfli4l.sh again without -k to finish
    --filesonly     create only fli4l-files - do not create a boot-media
    --no-squeeze    don't compress shell scripts
    --rebuild       rebuild mkfli4l and related tools; needs make, gcc
  \end{verbatim}

  Avec l'option \verb+--hdinstallpath <dir>+ il est possible de faire une
  pré-installation sur une carte compact-flash en utilisant un lecteur de carte
  USB ou sur une clé USB, les supports doivent être formater en (FAT16/FAT32).
  Cette fonction est surtout utilisée \emph{à vos propres risques} pour la
  création de carte compact-flash ou de clé USB. Les fichiers nécessaires pour
  fli4l seront copiés sur la partition spécifiée. Le script ci-dessous appelle
  le répertoire fli4l.

  \begin{verbatim}
     sh mkfli4l.sh --hdinstallpath <dir>
  \end{verbatim}
  \vspace{-2ex}

  Les fichiers fli4l seront copiés sur la carte CF ou sur la clé USB.

  Pour effectuer les prochaines étapes, les conditions suivantes doivent être remplies~:

   \begin{itemize}
        \item \verb+chmod 777 /dev/brain+
        \item Droits-super-utilisateur
        \item Installer \verb+syslinux+
        \item Installer \verb+fdisk+
   \end{itemize}

  Ensuite le script contrôle, si le support de données est un lecteur USB et si
  la première partition est une partition FAT. Puis le Bootloader et les fichiers
  nécessaires sont copiés sur le volume spécifié. A la fin du script, vous
  recevrez un message indiquant le succès ou l'échec de l'installation.

  Après la construction, vous devez exécuter.

 \begin{verbatim}
 syslinux --mbr /dev/brain

    # make partition bootable using fdisk
    #     p - print partitions
    #     a - toggle bootable flag, specify number of fli4l partition
    #         usually '1'
    #     w - write changes and quit
    fdisk /dev/brain

    # install boot loader
    syslinux -i /dev/brain
 \end{verbatim}
 \vspace{-2ex}

  Pour finir, la carte CF ou la clé USB sera amorçable. Ne pas oublier de
  démonter le périphérique (avec \texttt{umount}).

  \bigskip

  Avec les derniers paramètres d'optionnel, On peut créer un répertoire
  de configuration alternatif. Le répertoire de configuration normal s'appelle
  \texttt{config} et se trouve directement à la racine du répertoire de fli4l.
  Dans ce répertoire, sont enregistrés tous les fichiers de configuration des
  paquetages fli4l. Si on veut gérer plus d'une configuration, on peut créer un
  répertoire supplémentaire, par exemple \texttt{hd.conf}, ici une copie des
  fichiers de configuration est faite et si vous voulez vous pouvez modifies
  ces fichiers selon vos besoins. Quelque exemples~:
  \begin{verbatim}
     sh mkfli4l.sh --filesonly hd.conf
     sh mkfli4l.sh --no-squeeze config.test
  \end{verbatim}


\marklabel{sec:bootmedien_windows}{
  \section{Création d'une archive fli4l/Média de Boot sous Windows}
  }

  Le programme utilisé est 'AutoIt3' voir le site
  (\altlink{http://www.autoitscript.com/site/autoit/}). il permet une construction
  'graphique' de fli4l et aussi des dialogues dans lesquels les variables
  sont décrites dans ce paragraphe, voici la commande.

  \begin{description}
    \item \texttt{mkfli4l.bat}
  \end{description}

  Build-Script reconnaît indépendamment les différentes \jump{BOOTTYPE}{variantes de Boot}.

  Le démarrage de 'mkfli4l.bat' peut s'opérer directement dans
  l'Explorer de Windows, sans utiliser aucun paramètre optionnel.

  Les différents mécanismes gérent la construction du programme Build~:
  \begin{itemize}
    \item Configuration de la variable \var{BOOT\_TYPE} dans
      le fichier \texttt{$<$config$>$/base.txt}
    \item Configuration du fichier \texttt{$<$config$>$/mkfli4l.txt}
    \item Les Paramètres du Programme Build
    \item Le Réglage interactif avec le GUI
  \end{itemize}

  On décide au moyen de la variable \jump{BOOTTYPE}{\var{BOOT\_TYPE}},
  le type de média de construction (Build-Scripts) pour fli4l~:
  \begin{itemize}
    \item Démarrer fli4l avec un CD-ROM par une image ISO
    \item Faire une mise à jour des fichiers, les copier sur le média
    \item Faire une mise à jour des fichiers, les envoyer sur le routeur via SCP
    \item Pré-installer un Disque Dur ou un CF (Compact Flash) en
      utilisent un lecteur de carte
    \item etc.
  \end{itemize}

  Vous trouverez la description des variables dans le fichier de
  configuration \texttt{$<$config$>$/mkfli4l.txt} dans ce chapitre
  \jump{sec:mkfli4lconf}{Paramètres mkfli4l.txt}.

  \subsection{Ligne de commande en option}

  On a la possibilité de rajouter des paramètres optionnels dans le
  fichier de commande \texttt{mkfli4l.txt}, qui appel programme de construction
  (Build-Programms). Ces paramètres on les mêmes orientations que le programme
  'graphique'. Pour des raisons de confort on a différencié, les paramètres
  optionnels et les variables du fichier de construction. Les paramètres existent
  sous une forme courte et une forme longue les voici~:

  \begin{verbatim}
  Utilisation : mkfli4l.bat [options] [config-dir]

-c, --clean             cleanup the build-directory
-b, --build <dir>       sets build-directory to <dir> for the fli4l-files
-v, --verbose           verbose - some debug-output
    --filesonly         creates only fli4l-files - does not create a disk
    --no-squeeze        don't compress shell scripts
-h, --help              display this usage

config-dir              sets other config-directory - default is "config"

*** Remote-Update options
--remoteupdate          remote-update via scp, implies "--filesonly"
--remoteuser <name>     user name for remote-update - default is "fli4l"
--remotehost <host>     hostname or IP of remote machine - default
                        is HOSTNAME set in [config-dir]/base.txt
--remotepath <path>     pathname on remote maschine - default is "/boot"
--remoteport <portnr>   portnumber of the sshd on remote maschine

*** GUI-Options
--nogui                 disable the config-GUI
--lang                  change language
                        [deutsch|english|espanol|french|magyar|nederlands]

  \end{verbatim}

  Avec les derniers paramètres optionnel, Vous pouvez créer un répertoire
  de configuration alternatif. Le répertoire de configuration normal
  s'appelle \texttt{config} il se trouve directement à la racine du répertoire
  fli4l. Dans ce répertoire, sont enregistrés tous les fichiers de
  configuration des paquetages fli4l. Si on veut gérer plusieurs configurations,
  on peut créer un répertoire supplémentaire, par exemple \texttt{hd.conf},
  on copie dans celui-ci les fichiers de configuration du répertoire \texttt{config},
  vous poulez ensuite modifies ces fichiers selon vos besoins.
  Ici quelque exemples pour démarrer le Build~:
  \begin{verbatim}
     mkfli4l.bat hd.conf
     mkfli4l.bat -v
     mkfli4l.bat --no-gui config.hd
  \end{verbatim}

  \subsection{Boîte de dialogue~- Définition du répertoire de configuration}

  Il y a dans la fenêtre principale de la boîte de dialogue, une liste de
  paramètres de configurations pour différents réglages, on peut ouvrir la
  fenêtre de son choix pour paramètrer le programme.\\

  Attention dans 'Config-Dir' on peut modifier le répertoire des
  fichiers de constructions dans \jump{sec:mkfli4lconf}{Paramètres 'mkfli4l.txt'}
  qui est stocké sur votre disque.\\

  Si mkfli4l.bat ne trouve pas le fichier 'base.txt' dans le
  répertoire fli4l-x.y.z$\backslash$ une fenêtre s'ouvre
  immédiatement pour rechercher le fichier de configuration.
  Cela permet d'administrer facilement une liste de plusieurs
  configurations pour fli4l.\\

  Exemple~:

\begin{example}
\begin{verbatim}
          fli4l-x.y.z\config
          fli4l-x.y.z\config.fd
          fli4l-x.y.z\config.cd
          fli4l-x.y.z\config.hd
          fli4l-x.y.z\config.hd-construction
\end{verbatim}
\end{example}

  \subsection{Boîte de dialogue~-- Paramètres généraux}
  \begin{figure}[h!]
  \centering
  \includegraphics[width=\columnwidth]{win_build_build}
  \caption{Paramètre}
  \label{fig:win_build_build}
  \end{figure}

  On défini dans cette fenêtre, la sauvegarde des paramètres et la création du média~:
  \begin{itemize}
    \item Build-Dir~-- Répertoire pour l'archive/l'image CD/...
    \item \var{BOOT\_TYPE}~-- Régle l'affichage/utilisé \var{BOOT\_TYPE}~-- il ne peut pas être modifié ici
    \item Verbose~--- Affiche les informations pendant la construction du programme fli4l
    \item Filesonly~--- Sauvegarde uniquement les fichiers, pas de création d'image
    \item Remoteupdate~--- Active la mise à jour par SCP
  \end{itemize}

  Avec le bouton \textbf{les paramètres du programme fli4l-build peuvent être sauvegardés à tout moment},
       les paramètres seront enregistrés dans le fichier mkfli4l.txt, ils
       peuvent être modifié manuellement en ouvrant ce fichier.


  \subsection{Boîte de dialogue~-- Paramètres pour la mise à jour à distance}
  \begin{figure}[h!]
  \centering
  \includegraphics[width=\columnwidth]{win_build_remoteupdate}
  \caption{Paramètre pour la mise à jour}
  \label{fig:win_build_remoteupdate}
  \end{figure}

  On défini dans cette fenêtre, les réglages pour l'installation d'une mise à jour~:
  \begin{itemize}
    \item Adresse IP ou Nom d'Hôte
    \item Nom d'utilisateur sur l'hôte distant
    \item Remote-path (Par défaut: /boot)
    \item Remote-port (Port par défaut: 22)
    \item Utiliser SSH-Keyfile (Format ppk de Putty)
  \end{itemize}

  \subsection{Boîte de dialogue~-- Paramètres pour une pré-installation du HD}
  \begin{figure}[h!]
  \centering
  \includegraphics[width=\columnwidth]{win_build_hd_install}
  \caption{Paramètre pour pré-installation du DD}
  \label{fig:win_build_hd_install}
  \end{figure}

   On défini dans cette fenêtre, les paramètres pour la pré-installation
   d'un disque dur, une Carte CompactFlash, une clef USB formaté et partitionné.

   Options possibles~:
  \begin{itemize}
     \item Active la pré-installation du Disque Dur
     \item Lettre du lecteur ou de la Carte-CF
  \end{itemize}

  Information pour partitionner et formater CF (Compact Flash)~: Pour
  cette installation utiliser le TYPE A, de plus (nous avons besoin
  du paquetage HD), une partition FAT primaire doit être active et
  formatée sur la CF. Si l'on veut utiliser une partition bootable
  il faut installer une partition Linux supplémentaire formatée avec
  le système ext3, on aura besoin du fichier \texttt{hd.cfg} sur la partition
  FAT (pour cela il faut absolument installer et configurer le paquetage HD).


\marklabel{sec:mkfli4lconf}{
  \section{Paramètre pour le fichier mkfli4l.txt}}

  Il existe depuis la Version fli4l 2.1.9, le fichier de
  configuration \texttt{$<$config$>$/mkfli4l.txt}. Toutes les commandes
  du programme 'graphique' fli4l-Build sont enregistrées dans
  le fichier mkfli4l.txt. Le fichier est construit comme tous
  les fichiers fli4l. Toutes les variables de configuration sont
  optionnelles, mais il ne faut pas, modifier les variables spécifiques.

  \begin{description}

  \config {BUILDDIR}{BUILDDIR}{BUILDDIR}

  Valeur par défaut~: 'build'

  On indique ici le nom du répertoire pour enregistrer les fichiers de
  construction pour le boot de fli4l. Si la variable n'est pas définie, mkfli4l
  sous Windows utilisera par défaut le sous-répertoire \texttt{build} de la racine
  du répertoire fli4l~:
  \begin{verbatim}
     Chemin/fli4l-x.y.z/build
  \end{verbatim}
  \vspace{-2ex}
  En lancant mkfli4l le programme enregistre des fichiers de
  construction produits dans le répertoire \texttt{$<$config$>$/build}.

  Vous devez utiliser les conventions des systèmes d'exploitation de Windows ou
  *Unix pour paramétrer le chemin d'accès \var{BUILDDIR}. Si vous avez paramétré
  un chemin relatif, ce chemin sera converti par le processus de construction
  de Windows ou *Unix.

  \config {VERBOSE}{VERBOSE}{VERBOSE}

  Valeur par défaut~: \var{VERBOSE='no'}

  Valeurs possibles sont \var{'yes'} ou \var{'no'}. Affiche les
  \emph{les Informations} du processus Build (ou processus de construction).

  \config {FILESONLY}{FILESONLY}{FILESONLY}

  Valeur par défaut~: \var{FILESONLY='no'}

  Valeurs possibles \var{'yes'} ou \var{'no'}. Vous permet de créer un Boot
  média, peut être désactivé de sorte à créer uniquement les fichiers d'archives.

  \config {REMOTEUPDATE}{REMOTEUPDATE}{REMOTEUPDATE}

  Valeur par défaut~: \var{REMOTEUPDATE='no'}

  Valeurs possibles \var{'yes'} ou \var{'no'}. Si on veut transmettre
  automatiquement des fichiers de boot sur le Routeur au moyen de SCP.
  Cela suppose que le paquetage \jump{OPTSSHD}{SSHD} est installé et
  en plus la variable \texttt{scp} soit activée dans se paquetage.

  \config {REMOTEHOSTNAME}{REMOTEHOSTNAME}{REMOTEHOSTNAME}

  Valeur par défaut~: \var{REMOTEHOSTNAME=''}

  On indique ici le nom d'hôte du destinataire pour le transfert des
  données avec SCP. Si vous n'avez indiqué aucun nom, le nom de la
  variable \jump{HOSTNAME}{\var{HOSTNAME}} est utilisée pour le
  transfert des données.

  \config {REMOTEUSERNAME}{REMOTEUSERNAME}{REMOTEUSERNAME}

  Valeur par défaut~: \var{REMOTEUSERNAME='fli4l'}

  Nom d'utilisateur pour la transmission des données SCP.

  \config {REMOTEPATHNAME}{REMOTEPATHNAME}{REMOTEPATHNAME}

  Valeur par défaut~: \var{REMOTEPATHNAME='/boot'}

  Chemin d'accès du destinataire pour la transmission des données SCP.

  \config {REMOTEPORT}{REMOTEPORT}{REMOTEPORT}

  Valeur par défaut~: \var{REMOTEPORT='22'}

  Port du destinataire pour la transmission des données SCP.

  \config {SSHKEYFILE}{SSHKEYFILE}{SSHKEYFILE}

  Valeur par défaut~: \var{SSHKEYFILE=''}

  Ici on peut indiquer le fichier de clef-SSH pour la mise à jour
  avec SCP. Un mot de passe peut aussi être demandé pour la mise à jour.

  \config {REMOTEREMOUNT}{REMOTEREMOUNT}{REMOTEREMOUNT}

  Valeur par défaut~: \var{REMOTEREMOUNT='no'}

  Les valeurs possibles sont \var{'yes'} ou \var{'no'}. Si vous indiquez
  \var{'yes'}, vous remontez le boot device "/boot" en lecture/écriture
  si le boot est en lecture seule, c'est pour monter et rendre possible
  la mise à jour distante.

  \config {TFTPBOOTPATH}{TFTPBOOTPATH}{TFTPBOOTPATH}

  Le chemin d'accès pour installer l'image de boot par le réseau.

  \config {TFTPBOOTIMAGE}{TFTPBOOTIMAGE}{TFTPBOOTIMAGE}

  Nom de l'image de boot sur le réseau.

  \config {PXESUBDIR}{PXESUBDIR}{PXESUBDIR}

  Sous-répertoire pour les fichiers PXE qui est en rapport avec TFTPBOOTPATH.

  \config {SQUEEZE\_SCRIPTS}{SQUEEZE\_SCRIPTS}{SQUEEZESCRIPTS}

  Active ou désactive Squeeze (compression des scripts). Par ex. un
  Script qui contient en plus des lignes de commentaires, ces lignes
  serons supprimées à la compressé par Squeeze. Normalement on devrais
  toujours indiquer \var{'yes'} dans cette variable.

  \config {MKFLI4L\_DEBUG\_OPTION}{MKFLI4L\_DEBUG\_OPTION}{MKFLI4LDEBUGOPTION}

   Options supplémentaires de débugage, peut être transmis au \jump{mkfli4l}{Programme-mkfli4l}.

  \end{description}

  \chapter{Réglage des PCs dans le LAN}

  Réglage des ordinateurs dans le LAN (ou réseau local)~:

  \begin{enumerate}
  \item Adresse IP (voir \smalljump{sec:pc-lan-ip}{Adresse IP})
  \item Nom de l'ordinateur et Nom de Domaine
    (voir \smalljump{sec:pc-lan-name}{Nom de l'ordinateur et de Domaine})
  \item Gateway-Standard (Passerelle Standard) (voir \smalljump{sec:pc-lan-gateway}{Gateway})
  \item Adresse IP et serveur-DNS (voir \smalljump{sec:pc-lan-dns}{Serveur-DNS})
  \end{enumerate}


  \marklabel{sec:pc-lan-ip}{\section{Adresse IP}}
  Les adresses IP du réseau local doivent se trouver dans le même
  réseau que l'adresse IP du routeur fli4l (de l'interface Ethernet),
  par ex. 192.168.6.2 pour l'ordinateur local dans le cas ou le routeur
  aurait l'adresse IP 192.168.6.1. Les adresses IP doivent être uniques
  dans le réseau, changer uniquement le dernier chiffre de l'adresse IP
  est un bon moyen pour ne pas se tromper. Vous devez vous assurez que
  l'adresses IP indiqué ici est la même adresse IP que vous avez config
  pour cet ordinateur dans le fichier config/base.txt.

  \marklabel{sec:pc-lan-name}{\section{Nom de l'ordinateur et de domaine}}
  Le nom de l'ordinateur est par ex. "mon-pc", et le nom de Domaine "lan.fli4l".

  \wichtig{Le domaine qui est réglé dans le PC doit être identique
   au domaine choisi dans l'ordinateur fli4l, si on veut utiliser le
   routeur fli4l comme serveur DNS, il peut y avoir d'énormes problèmes
   dans le réseau si les domaines sont différents.}

  La raison~: les ordinateurs Windows cherchent régulièrement les
  ordinateurs avec le même nom de groupe de travail WORKGROUP.mon-domain.fli4l.
  Si fli4l ne répond pas à la requête du domaine (ici~: mon-domain.fli4l),
  alors fli4l essayera de chercher le domaine en se connectant sur Internet \ldots

  Le domaine doit être enregistré dans les réglages TCP/IP de l'ordinateur.

  \subsection{Windows 2000}

  Pour Windows 2000 se trouve sous~:

  \noindent Démarrer \pfeil\\
  \hspace*{2ex}Paramètre \pfeil\\
  \hspace*{4ex}Panneau de configuration \pfeil\\
  \hspace*{6ex}Connexion réseau \pfeil\\
  \hspace*{8ex}Connexion au réseau local \pfeil\\
  \hspace*{10ex}Bouton droit propriétés  \pfeil\\
  \hspace*{12ex}Protocole Internet (TCP/IP) \pfeil\\
  \hspace*{14ex}Sélectionner \pfeil\\
  \hspace*{16ex}Avancé \ldots \pfeil\\
  \hspace*{18ex}DNS \pfeil\\
  \hspace*{20ex}Suffix DNS pour cette connexion \pfeil\\

  Entrer "lan.fli4l" (ou indiquer votre domaine) (sans les "" !)
  \pfeil et appuyez sur OK.

\subsection{NT 4.0}

  Démarrer \pfeil\\
  \hspace*{2ex}Paramètre \pfeil\\
  \hspace*{4ex}Panneau de configuration  \pfeil\\
  \hspace*{6ex}Réseau \pfeil\\
  \hspace*{8ex}Protocole \pfeil\\
  \hspace*{10ex}TCP/IP \pfeil\\
  \hspace*{12ex}Propriétés \pfeil\\
  \hspace*{14ex}DNS \pfeil\\
  \hspace{16ex}\begin{itemize}
  \item Nom d'hôte entrer (le Nom de l'ordinateur)
  \item Domaine entrer (le même Nom que dans le fichier config/base.txt)
  \item Ajouter adresse IP le même réseau que le routeur fli4l
  \item Ajouter suffix DNS (Domaine le même que la ligne 2)
  \end{itemize}

\subsection{Windows 95/98}

  Démarrer \pfeil\\
  \hspace*{2ex}Paramètre \pfeil\\
  \hspace*{4ex}Panneau de configuration \pfeil\\
  \hspace*{6ex}Réseau \pfeil\\
  \hspace*{8ex}Configuration \pfeil\\
  \hspace*{10ex}TCP/IP (sélectionner la carte réseau qui va au routeur) \pfeil\\
  \hspace*{12ex}propriétés \pfeil\\
  \hspace*{14ex}Configuration DNS~:

  Cliquer activé DNS, dans le champ "Domaine"~: entrer "lan.fli4l"
  (ou indiquer votre domaine) (sans les "" !).

\subsection{Windows XP}

  Pour Windows XP se trouvent sous~:

  \noindent Démarrer \pfeil\\
  \hspace*{2ex}Paramètre \pfeil\\
  \hspace*{4ex}Panneau de configuration \pfeil\\
  \hspace*{6ex}Connexions réseau \pfeil\\
  \hspace*{8ex}Connexion au réseau local \pfeil\\
  \hspace*{10ex}Propriétés \pfeil\\
  \hspace*{12ex}Protocole Internet (TCP/IP) \pfeil\\
  \hspace*{14ex}Propriétés \pfeil\\
  \hspace*{16ex}Avancé\ldots \pfeil\\
  \hspace*{18ex}DNS \pfeil\\
  \hspace*{20ex}Suffixe DNS pour cette connexion \pfeil\\

  Indiquez "lan.fli4l" (ou indiquer votre domaine) (sans les "" !)
  \pfeil Cliquez sur OK.

  \subsection{Windows 7}

  Pour Windows 7 se trouvent sous~:

  \noindent Bouton Windows (ex. Démarrer) \pfeil\\
  \hspace*{2ex}Contrôle \pfeil\\
  \hspace*{4ex}Panneau de configuration \pfeil\\
  \hspace*{6ex}Centre Réseau et partage \pfeil\\
  \hspace*{8ex}Connexion au réseau local \pfeil\\
  \hspace*{10ex}Propriétés \pfeil\\
  \hspace*{12ex}Protocole Internet version 4 (TCP/IPv4) \pfeil\\
  \hspace*{14ex}Propriétés \pfeil\\
  \hspace*{16ex}Avancé\ldots \pfeil\\
  \hspace*{18ex}DNS \pfeil\\
  \hspace*{20ex}Suffixe DNS pour cette connexion \pfeil\\

  Indiquez "lan.fli4l" (ou indiquer votre domaine) (sans les "" !)
  \pfeil Cliquez sur OK.

\subsection{Windows 8}

  Pour Windows 8 se trouvent sous~:

  \noindent Appuyez simultanément sur la touche Windows et X \pfeil\\
  \hspace*{2ex}Contrôle \pfeil\\
  \hspace*{4ex}Connexions réseau \pfeil\\
  \hspace*{6ex}Sélectionnez votre réseau (Ehternet ou WLAN) \pfeil\\
  \hspace*{8ex}Clique droit \pfeil\\
  \hspace*{10ex}Propriétés \pfeil\\
  \hspace*{12ex}Protocole Internet version 4 (TCP/IPv4) \pfeil\\
  \hspace*{14ex}Propriétés \pfeil\\
  \hspace*{16ex}Avancé\ldots \pfeil\\
  \hspace*{18ex}DNS \pfeil\\
  \hspace*{20ex}Suffixe DNS pour cette connexion \pfeil\\

  Indiquez "lan.fli4l" (ou indiquer votre domaine) (sans les "" !)
  \pfeil Cliquez sur OK.

  \marklabel{sec:pc-lan-gateway}{\section{Gateway (ou Passerelle)}}

  Il est absolument nécessaire d'indiquer une adresse IP dans le paramètre
  passerelle par défaut de votre PC, car s'il n'y a pas d'adresse IP d'indiquée,
  rien ne fonctionnera. Ainsi vous devrez indiquer l'adresse IP du routeur
  fli4l - (Interface Ethernet) par exemple 192.168.6.4, selon l'adresse
  IP qui est configurée dans le fichier config/base.txt du routeur fli4l.

  Il est incorrect de configuter le routeur fli4l comme un proxy dans Windows
  ou dans de votre navigateur~-- sauf si vous définissez un proxy sur votre
  routeur fli4l. Normalement fli4l a pas de proxy, s'il vous plaît ne spécifiez
  \emph{pas} fli4l comme un proxy~!

\marklabel{sec:pc-lan-dns}{\section{Serveur DNS}}

  Pour l'adresse IP du serveur DNS, vous ne devez pas indiquer
  d'adresse IP de votre fournisseur d'accés Internet mais l'adresse IP
  du routeur (interface Ethernet), car le routeur peut répondre aux
  requêtes DNS et faire suivre ceux-ci par Internet si nécessaire.

  Quand fli4l est utilisé comme serveur DNS, beaucoup de requêtes
  DNS son envoyées par les PCs client Windows, c'est le routeur fli4l qui
  leur répond directement, elles ne sont pas expédiées sur Internet.

\marklabel{sec:pc-lan-misc}{\section{Divers points }}

  Les points 1 et 4 n'ont pas besoin d'être enregistrés avec un
  serveur DHCP puisque le routeur fli4l communique les données
  nécessaires automatiquement.

  \textbf{Dans Options Internet~:} et dans la fenètre connexion vous ne devez
  "sélectionner aucun lien". Dans Paramètre réseau local (LAN)~: ne RIEN
  indiquer (sauf si vous utilisez le paquetage \var{OPT\_\-P}roxy).
  Par défaut les deux paramètres n'ont pas besoin d'être modifié pour
  une utilisation normale.


  \marklabel{IMONDSCHNITTSTELLE}{
    \chapter{Interface client/serveur imon}
  }

  \marklabel{sec:imond}{
    \section{Server imon avec imond}}

  Imond est un programme serveur qui répond à certaines enquêtes sur
  la gestion du réseau et accepte aussi des commandes qui peuvent contrôler
  le routeur sur le réseau local.

  Imond contrôle également les Moindres-Coûts-Routages. Il utilise
  le fichier de configuration /etc/imond.conf qui est produit
  automatiquement au moment du boot, à partir de la variable \var{ISDN\_\-CIRC\_\-x\_\-XXX}
  du fichier config/isdn.txt, le fichier est généré par un script shell.

  imond est un démon qui fonctionne en permanence en tache de fond,
  il écoute le port 5000 TCP/IP sur le périphérique /dev/isdninfo.

  Voici toutes les commandes qui peuvent être envoyées par le port 5000 TCP/IP~:
  \begin{table}
    \textbf{Commandes Admin}

    \vspace{1ex}
    \begin{tabular}{lp{9cm}}

      addlink ci-index              & Ajouter un canal au circuit (Channel-Bundling) \\
      adjust-time seconds           & Incrémente la date sur le routeur en secondes \\
      delete filename pw            & Supprime le fichier sur le routeur \\
      hup-timeout \#ci-index [value]& Affiche ou compose le HUP-Timeout pour
                                      des circuits RNIS (ou numéris) \\
      removelink ci-index           & Enlever le canal supplémentaire \\
      reset-telmond-log-file        & Supprime le fichier journal de telmond \\
      reset-imond-log-file          & Supprime le fichier journal de imond \\
      receive filename \#octets pw  & Transfére d'un fichier au routeur.
                                      Imond donne l'ordre avec ACK (0x06). Après,
                                      le fichier est transféré par blocs de 1024
                                      Octets qui sont également confirmé avec ACK.
                                      En conclusion, imond répond OK. \\
      send filename pw              & Si le mot de passe est correct et que le
                                      fichier existe, imond répond OK avec un \#octet.
                                      Puis, imond transfère le fichier par blocs de
                                      1024 octets, chaque fois confirmés avec ACK
                                      (0x06). A la fin, imond répond OK. \\
      support pw                    & Montre le statut/configuration du routeur \\
      sync                          & Synchronise le Cache des lecteurs montés \\
    \end{tabular}
  \end{table}


  \begin{table}
    \textbf{Commandes Admin et Utilisateur}

    \vspace{1ex}
    \begin{tabular}{lp{9cm}}

      dial                      &    Choix du FAI (Defaut-Route-Circuit) \\
      dialmode [auto|manual|off]&    Réglage des actions dans Dialmode \\
      disable                   &    Raccroche et place dialmode sur "off" \\
      enable                    &    Mets dialmode sur "auto" \\
      halt                      &    Descend proprement le Routeur \\
      hangup [\#channel-id]     &    Raccroche \\
      poweroff                  &    Descent le routeur et mise hors tension \\
      reboot                    &    Reboot le routeur fli4l! \\
      route [ci-index]          &    Met le routeur par Defaut sur un Circuit X (0=automatique) \\
    \end{tabular}
  \end{table}


  \begin{table}
    \textbf{Commandes Utilisateur}

    \vspace{1ex}
    \begin{tabular}{lp{9cm}}
      channels                  & Nombre de Canaux ISDN disponibles \\
      charge \#channel-id       & Edite les frais de connexion pour un Canal en ligne \\
      chargetime \#channel-id   & Temps et frais de connexion pour un canal en ligne \\
      circuit [ci-index]        & Edite le numéro du Circuit \\
      circuits                  & Edite le nombre de Defaut-Route-Circuits \\
      cpu                       & Donne la charge du CPU en pourcentage \\
      date                      & Edite la date et heure \\
      device ci-index           & Circuits du périphérique utilisé \\
      driverid \#channel-id     & Edite Driver-ID pour le Canal X \\
      help                      & Edite l'aide \\
      inout \#channel-id        & Edite la direction (entrante/sortante) \\
      imond-log-file            & Edite le fichier du Protocole imond \\
      ip \#channel-id           & Edite l'adresse IP \\
      is-allowed command        & Edite si la commande est valide\newline
                                  commandes possibles~:
                                  dial|dialmode|route|reboot
                                  |imond-log|telmond-log|mgetty-log \\
      is-enabled                & Edite si dialmode est sur off (0) ou auto (1) \\
      links ci-index            & Edite le nombre de canaux 0, 1 ou 2, 0 utilisé,
                                  ou alors~: Aucun Channel-Bundling possible \\
      log-dir imond|telmond|mgetty& Donne la direction des fichiers Log \\
      mgetty-log-file           & Edite le protocole du fichier mgetty \\
      online-time \#channel-id  & Edite le temps en ligne, et de connexion en hh:mm:ss \\
      pass [password]           & Vérifie, le mot de passe qui a été saisit par\newline
                                  1 Mot de passe Utilisateur est fixé\newline
                                  2 Mot de passe Admin est fixé\newline
                                  4 imond se trouve dans le mode Admin \\
      phone \#channel-id        & Edite le numéro de Tél et le nom du "correspondant" \\
      pppoe                     & Donne le numéro du périphérique pppoe (0 ou 1) \\
      quantity \#channel-id     & Donne l'ensemble des transmissions (en octet) \\
      quit                      & Coupe la connexion avec imond \\
      rate \#channel-id         & Edite les connexions (entrant/sortant en Octet/sec) \\
      status \#channel-id       & Edite le statut pour le Canal X \\
      telmond-log-file          & Edite le protocole telmond \\
      time \#channel-id         & Edite le temps total en ligne, au Format hh:mm:ss \\
      timetable [ci-index]      & Edite la time-table LC-Routing \\
      uptime                    & Edite le temps d'utilisation du Routeur en secondes \\
      usage \#channel-id        & Edite les réponses des connexions: Fax, Répondeur, Net, Modem, Raw \\
      version                   & Edite la version du protocole et la version du Programme \\
    \end{tabular}
  \end{table}


  Le port 5000 TCP/IP est accessible uniquement depuis un réseau LAN
  masqué. Avec la configuration standard du firewall l'accés est bloqué
  de l'extérieur.

  Imond supporte deux modes d'administrations, le Mode Utilisateur et
  le Mode Admin. On peut installer un Mot de Passe pour ces deux modes
  au moyen des variables \var{IMOND\_\-PASS} et \var{IMOND\_ADMIN\_\-PASS}.
  Si le Mot de Passe n'est pas transmis au serveur imond le client imonc
  a accés uniquement à deux commandes "pass" et "quit" toutes les autres
  commandes sont rejetées et une erreur s'affiche.

  Si plus tard, vous voulez limiter l'accés au serveur imond à un seul PC,
  la configuration du Firewall doit être modifier.

  Les commandes

\begin{example}
\begin{verbatim}
         enable/disable/dialmode   dial/hangup   route   reboot/halt
\end{verbatim}
\end{example}

  peuvent être activées ou désactivées dans la variable
  \var{IMOND\_\-XXX} voir (le chapitre "configuration").

  Avec un ordinateur Unix/Linux (ou un ordinateur Windows par la
  fenètre DOS) vous pouvez facilement entrer les commandes aprés
  la connexion telnet.

  Connexion telnet~:

\begin{example}
\begin{verbatim}
        telnet fli4l 5000        \# ou le Nom correspondant au routeur fli4l
\end{verbatim}
\end{example}

  Vous pouvez directement entrer les commandes mentionnées ci-dessus.

  Par exemple la commande "help" active l'aide sur l'écran ou "quit"
  démonte (ou arrête) le serveur imond.

\marklabel{sec:leastcostrouting}{
  \subsection{Mode de fonctionnement du Moindre-Coût-Routage}
  }

  imond construit une Time-Table (ou Plage Horaire) à partir du fichier
  de configuration /etc/imond.conf (qui est créé au boot avec la variable
  de configuration \var{ISDN\_\-CIRC\_\-x\_\-TIMES}. Ce "calendrier" est
  composé d'une semaine par intervalle d'une heure, une semaine = 168
  heures = 168 octets. La table se compose de circuits, dans lesquel sont
  définis des Défaut-Routes (ou connexion par défaut au FAI).

  Avec la commandement "timetable" on peut voir la table imond.
  Exemple de configuration~:

Supposons que nous définissions 3 circuits de connexions pour chaque FAI c'est à dire~:

\begin{example}
\begin{verbatim}
        CIRCUIT_1_NAME='Addcom'
        CIRCUIT_2_NAME='AOL'
        CIRCUIT_3_NAME='Firma'
\end{verbatim}
\end{example}

  Les deux premiers circuits sont réglés avec Défaut-Route c.à d.
  que l'itinétaire par défaut est écrit dans la variable
  ISDN\_CIRC\_x\_ROUTE avec la valeur '0.0.0.0/0'.

  Les variables \var{ISDN\_\-CIRC\_\-x\_\-TIMES} se présentent de la manière suivante~:

\begin{example}
\begin{verbatim}
        ISDN_CIRC_1_TIMES='Mo-Fr:09-18:0.0388:N Mo-Fr:18-09:0.0248:Y
                      Sa-Su:00-24:0.0248:Y'

        ISDN_CIRC_2_TIMES='Mo-Fr:09-18:0.019:Y Mo-Fr:18-09:0.049:N
                      Sa-Su:09-18:0.019:N Sa-Su:18-09:0.049:N'

        ISDN_CIRC_3_TIMES='Mo-Fr:09-18:0.08:N Mo-Fr:18-09:0.03:N
                      Sa-Su:00-24:0.03:N'
\end{verbatim}
\end{example}

  Puis le fichier /etc/imond.conf est créé de cette façon~:

\begin{example}
\begin{verbatim}
        #day  hour  device  defroute  phone        name        charge  ch-int
        Mo-Fr 09-18 ippp0   no        010280192306 Addcom      0.0388   60
        Mo-Fr 18-09 ippp0   yes       010280192306 Addcom      0.0248   60
        Sa-Su 00-24 ippp0   yes       010280192306 Addcom      0.0248   60
        Mo-Fr 09-18 ippp1   yes       019160       AOL         0.019   180
        Mo-Fr 18-09 ippp1   no        019160       AOL         0.049   180
        Sa-Su 09-18 ippp1   no        019160       AOL         0.019   180
        Sa-Su 18-09 ippp1   no        019160       AOL         0.049   180
        Mo-Fr 09-18 isdn2   no        0221xxxxxxx  Firma       0.08     90
        Mo-Fr 18-09 isdn2   no        0221xxxxxxx  Firma       0.03     90
        Sa-Su 00-24 isdn2   no        0221xxxxxxx  Firma       0.03     90
\end{verbatim}
\end{example}

  imond produit alors Time-Table (ou Plage Horaire) dans la mémoire.
  voici la table des données sorties avec la commande "timetable"~:

\begin{example}
\begin{verbatim}
         0  1  2  3  4  5  6  7  8  9 10 11 12 13 14 15 16 17 18 19 20 21 22 23
     --------------------------------------------------------------------------
     Su  3  3  3  3  3  3  3  3  3  3  3  3  3  3  3  3  3  3  3  3  3  3  3  3
     Mo  2  2  2  2  2  2  2  2  2  4  4  4  4  4  4  4  4  4  2  2  2  2  2  2
     Tu  2  2  2  2  2  2  2  2  2  4  4  4  4  4  4  4  4  4  2  2  2  2  2  2
     We  2  2  2  2  2  2  2  2  2  4  4  4  4  4  4  4  4  4  2  2  2  2  2  2
     Th  2  2  2  2  2  2  2  2  2  4  4  4  4  4  4  4  4  4  2  2  2  2  2  2
     Fr  2  2  2  2  2  2  2  2  2  4  4  4  4  4  4  4  4  4  2  2  2  2  2  2
     Sa  3  3  3  3  3  3  3  3  3  3  3  3  3  3  3  3  3  3  3  3  3  3  3  3

     No.  Name                   DefRoute  Device  Ch/Min   ChInt
      1   Addcom                   no      ippp0   0.0388     60
      2   Addcom                   yes     ippp0   0.0248     60
      3   Addcom                   yes     ippp0   0.0248     60
      4   AOL                      yes     ippp1   0.0190    180
      5   AOL                      no      ippp1   0.0490    180
      6   AOL                      no      ippp1   0.0190    180
      7   AOL                      no      ippp1   0.0490    180
      8   Firma                    no      isdn2   0.0800     90
      9   Firma                    no      isdn2   0.0300     90
     10   Firma                    no      isdn2   0.0300     90
\end{verbatim}
\end{example}

  Pour le circuit 1 (Addcom) il y a trois éléments définis (1-3),
  pour le circuit 2 il y a quatre éléments (4-7), et pour le circuit 3
  il y a trois éléments (8-10).

  Les index des circuits activés sont inscris toutes les heures dans
  la Time-Table respectivement. Ici les index (2-4) apparaissent,
  car les autres ne passent pas par LC-Défaut-Route.

  Si vous avez des zéros dans Time-Table, c'est qu'il manque des données
  dans la variable \var{ISDN\_\-CIRC\_\-X\_\-TIMES}. Si vous avez des
  zéro sur certaine plage horaire, cela veux dire qu'il n'y aura pas de
  Défaut-Route et aucun accés Internet possible sur ces plages horaires!

  Au démarrage du programme, imond vérifie le jour de la semaine et
  l'heure, puis les index dans la Time-Table et enfin régle les Défauts-Routes
  correspondants. Le Défaut-Route (ou connexion par Défaut au FAI) est alors
  activé par rapport à l'indexation.

  Lors d'un changement de statut, par exemple sur un canal, une connexion
  ou un racrochement de la ligne, si la commande mais plus d'une minute,
  le processus de démarrage est réactualisé, vérification de l'horaire et
  du jour, consultation de la table, selection du Circuit-Défaut-route.

  Si par exemple le lundi à 18:00 la connexion change, Défaut-Route est
  supprimé, les connexions existantes sont arrêtées (désolé\ldots),
  ensuite imond controle dans la Time-Table si un nouveau Circuit-Défaut-route
  existe, si oui imond mettra environ 60 secondes pour se reconnecter.
  Donc la connexion se fera au plus tard à 18:00:59.

  Il n'y aura aucun changement pour les circuits qui n'utilisent pas
  un Defaut-Route. Le contenu \var{ISDN\_\-CIRC\_\-x\_\-TIMES} sera
  uniquement employé pour le calcul des frais téléphoniques. Ceci peut
  être pertinent, si vous arrêtez temporairement le client imonc et que
  vous choisissiez manuellement un Circuit-Défaut-route.

  Vous pouvez également regarder dans l'indexation de Time-Table (exemple
  précédent de 1 à 10) les circuits non activés "Non-LC-Default-Route-Circuits".

  Commande pour vérifier un index dans le Time-Table~:

\begin{example}
\begin{verbatim}
                    timetable "index"
\end{verbatim}
\end{example}

  Exemple~:

\begin{example}
\begin{verbatim}
                    telnet fli4l 5000
                    timetable 5
                    quit
\end{verbatim}
\end{example}

  La sortie des données apparaîtront comme ceci~:

\begin{example}
\begin{verbatim}
         0  1  2  3  4  5  6  7  8  9 10 11 12 13 14 15 16 17 18 19 20 21 22 23
     --------------------------------------------------------------------------
     Su  0  0  0  0  0  0  0  0  0  0  0  0  0  0  0  0  0  0  0  0  0  0  0  0
     Mo  5  5  5  5  5  5  5  5  5  0  0  0  0  0  0  0  0  0  5  5  5  5  5  5
     Tu  5  5  5  5  5  5  5  5  5  0  0  0  0  0  0  0  0  0  5  5  5  5  5  5
     We  5  5  5  5  5  5  5  5  5  0  0  0  0  0  0  0  0  0  5  5  5  5  5  5
     Th  5  5  5  5  5  5  5  5  5  0  0  0  0  0  0  0  0  0  5  5  5  5  5  5
     Fr  5  5  5  5  5  5  5  5  5  0  0  0  0  0  0  0  0  0  5  5  5  5  5  5
     Sa  0  0  0  0  0  0  0  0  0  0  0  0  0  0  0  0  0  0  0  0  0  0  0  0

     No.  Name                   DefRoute  Device  Ch/Min   ChInt
      5   AOL                      no      ippp1   0.0490    180
\end{verbatim}
\end{example}

  Tout est clair jusque là~?

  Avec la commande "Route" d'imond vous pouvez commuter "Marche/Arrêt"
  de LC-Routing, et vous pouvez indiquer l'index du Circuit-Défaut-Route
  (1\ldots N), il se connectera sur le circuit. Si l'index est 0, le
  LC-Routing est activé et le circuit sera choisi automatiquement.

  \subsection{Calcul des frais on-line (en ligne)}

  Le mode de calcul des frais de connexions fonctionnera correctement
  uniquement si l'unité téléphonique est constante tout au long
  de la semaine, elle doit est inscrit dans la variable
  \var{ISDN\_\-CIRC\_\-x\_\-CHARGEINT}) en seconde. Normalement c'est la
  régle pour les fournisseurs d'accés Internets. Toutefois, si vous choisissez
  Telekom (je ne parle pas de T-Online~!) par exemple, pour un réseau
  d'entreprise, qui sera considéré comme des conversations téléphoniques normales.
  et changement passe de 90 secondes à 4 minutes aprés 18:00 (Stand Juni
  00). Par conséquent, la définition

   Mais si vous utilisez votre
  société téléphonique par exemple pour un accés Internet avec TéléKom
  (Allemagne) l'unité Tél change (information juin 2000).\\
  En France l'unité Tél est toujours constante 60 secondes, on n'a
  pas ce problème, c'est juste le tarif qui change en heure creuse
  0,018 euro et en heure pleine 0,033 euro (8:00 à 19:00 heure pleine).

\begin{example}
\begin{verbatim}
        ISDN_CIRC_3_CHARGEINT='90'
        ISDN_CIRC_3_TIMES='Mo-Fr:09-18:0.08:N Mo-Fr:18-09:0.03:N Sa-Su:00-24:0.03:N'
\end{verbatim}
\end{example}

  est en fait pas tout à fait exact. Le tarif le soir est de 3 cents la minute
  (donc 12 cents les 4 minutes de télécommunication), mais la mesure est fausse.
  C'est pour cette raison qu'il se produit des différences d'affichage par
  rapport au prix réel.

  Il est possible que ce problème soit peut être corrigé plus tard.
  En attendant on peut définir dans la variable \var{ISDN\_\-CIRC\_\-x\_\-CHARGEINT})
  2 Circuits~: un pour la journée avec \var{ISDN\_\-CIRC\_\-1\_\-CHARGEINT}='90'
  et l'autre pour la soirée \var{ISDN\_\-CIRC\_\-2\_\-CHARGEINT}='240'
  naturellement vous devez configurer \var{ISDN\_\-CIRC\_\-x\_\-TIMES},
  avec cette configuration vous utiliser le Circuit 1 pendant la journée
  et le Circuit 2 en soirée.

  Comme nous l'avons dit plus haut~: l'utilisation des connexions avec
  un fournisseur d'accés Internet, ne pose pas de problème parce que
  l'unité Tél est toujours constante et le coût par minute ne change pas
  (il a encore quelque chose? je ne fais pas confiance à T-* pour tout :-).


  % Do not remove the next line
% Synchronized to r30003

  \marklabel{sec:winimonc}{
    \section{Client Windows imonc.exe}}

  \subsection{Introduction}

  Le démon Imond sur le routeur fli4l gère deux modes d'utilisations différents~:
  le mode Administrateur (Admin) et le mode Utilisateur. Dans le mode Admin
  toutes les commandes sont activées automatiquement. Dans le mode Utilisateur
  vous devez activer les variables \jump{IMONDENABLE}{\var{IMOND\_ENABLE}},
  \jump{IMONDDIAL}{\var{IMOND\_DIAL}}, \jump{IMONDROUTE}{\var{IMOND\_ROUTE}} et
  \jump{IMONDREBOOT}{\var{IMOND\_REBOOT}}, dans le fichier /config/base.txt pour avoir
  les commandes. Si les variables sont sur `no' les commandes ne seront pas activées,
  même les commandes Exit et mode Admin ne seront pas activées dans le client imonc.
  Le choix de l'utilisation entre le mode Utilisateur et le mode Admin se fait
  par l'intermédiaire d'un Mot de Passe qui sera transféré au routeur. Vous pouvez
  activer le mode Admin ou Utilisateur, en cliquant sur l'icone située dans
  la barre de taches et entrer le Mot de Passe n'oubliez pas de redémarrer
  le client imonc.

  Lorsque imonc a démarré, une icône supplémentaire apparait dans la barre de taches,
  il indique le statut des canaux de la connexion Internet pour le (numéris).

  Les couleurs de l'icône ignifient~:

  \begin{description}
    \item[Rouge]~: offline (déconnecté)
    \item[Jaune]~: en cours de connexion 
    \item[Vert clair]~: online (en ligne il y a du trafic sur le canal)
    \item[Vert foncé]~: online (en ligne il n'y a pas de trafic sur le canal)
  \end{description}

  \noindent Suivant le Windows que vous utilisez le comportement d'imonc diverge,
  il peut être réduit à une icone dans la barre des taches près de l'heure. pour
  ouvrir la fenêtre il suffit de faire un double clic avec le bouton gauche de
  la souris sur l'icone. Pour ouvrir le menu contextuel vous utilisez le bouton
  droit, delà vous pouvez choisir directement les commandes imonc.

  Un (grand nombre de paramètres) peuvent être adaptés selon vos propres besoins,
  ils seront enregistrés et sauvegardés dans la base de registre de Windows à cet
  endroit HKCU{\textbackslash}Software{\textbackslash}fli4l.

  Il y a toujours quelques erreurs dans la documentation d'imonc et du routeur
  fli4l, malgré des relectures. Si vous rencontrez des problèmes, allez dans la
  page "A propos" cliquer sur le bouton systeminfo puis sur le bouton support info,
  ensuite le mot de passe du routeur vous sera demandé (pas le mot de passe d'imond~!).
  Imonc produira un fichier fli4lsup.txt, qui inclura toutes les informations
  importantes sur le routeur fli4l et sur imonc. Ce fichier peut être ajouté
  dans le Newsgroup pour demander de l'aide. Cela maximisera les chances
  d'avoir de l'aide plus rapidement.

  Vous pouvez trouve des détails concernant le développement du client imonc pour
  Windows sur le site \altlink{http://www.imonc.de/}, vous trouverez des informations
  sur les nouveaux dispositifs les futures versions d'imonc, les résolutions de bug
  et aussi la dernière version à télécharger (si elle n'est pas déjà inclue dans
  la distribution fli4l).

  \subsection{Paramètre de démarrage}

  Le client imonc à besoin du Nom ou de Adresse IP du routeur fli4l pour pouvoir
  établir une connexion avec celui-ci "l'ordinateur fli4l". Si l'ordinateur du client
  imonc est enregistré correctement dans le DNS, il devrait fonctionner sans problème.
  Voici les paramètres que l'on peut transmettre~: 

  \begin{itemize}
    \item /Server:IP ou Nom d'Hôte du routeur (Forme abrégée~: /S:IP ou Nom d'Hôte)
    \item /Password:Mot de Passe (Forme abrégée: /P:Mot de Passe)
    \item /log Active le protocole de communication entre imonc et imond, lorsque
      cette option est activée un fichier imonc.log est créé. Ce fichier enregistre
      toutes les communications, il peut être très volumineux. C'est pour cette
      raison que l'on active ce paramètre uniquement si il y a des problèmes de
      configurations.
    \item /iport:N$^\circ$ port Par défaut imond écoute sur le Port~: 5000
    \item /tport:N$^\circ$ port Par défaut telmond écoute sur le Port~: 5001
    \item /rc:"Commande" Les commandes écrites ici sont transmis au routeur sans
      aucun contrôle supplémentaire. Si plusieurs commandes sont exportées
      simultanément elles doivent être séparées par un point virgule. Pour être
      sur du fonctionnement de imonc vous devez retaper le Mot de Passe
      (si configuré?) car il n'y aura aucune redemande de Mot de Passe.
      les commandes possibles sont documentées dans le Chapitre 8.1. La commande
      dialtimesync n'ai plus utilisée elle est remplacée par \flqq{}dial; timesync\frqq{},
      qui force le routeur à synchroniser l'heure avec le client.
    \item /d:"Répertoire-fli4l" Cette option permet d'écrire le répertoire du
      dossier fli4l avec des paramètres de démarrage, c'est intéressant pour ceux
      qui utilisent plusieurs versions de fli4l.
    \item /wait Si le Nom d'Hôte ne peut pas être résolu, imonc se bloque,
      il faut redémarrer imonc par un double clic sur l'icône de celui-ci.
    \item /nostartcheck Cela coupe le contrôle d'imonc, s'il est en fonction.
      c'est uniquement nécessaire si vous avez plusieurs routeurs fli4l différents
      à surveiller dans votre Réseau. Si des fonctions supplémentaires étaient
      connectées comme syslog ou e-mail ils resteront désactivées.
  \end{itemize}

  Utilisation (enregistrement de lien)~:

\begin{example}
\begin{verbatim}
X:\...imonc.exe [/Server:Nom d'Hôte] [/Password:Mot de passe] [/iport:Numéro port]
            [/log] [/tport:Numéro port] [/rc:"Commande"]
\end{verbatim}
\end{example}

  Exemple d'enregistrement avec une adresse-IP~:

\begin{example}
\begin{verbatim}
        C:\wintools\imonc /Server:192.168.6.4
\end{verbatim}
\end{example}

  Ou avec le nom et le Mot de Passe~:

\begin{example}
\begin{verbatim}
        C:\wintools\imonc /S:fli4l /P:secret
\end{verbatim}
\end{example}

  Ou avec le nom, le Mot de Passe et une commande au routeur~:

\begin{example}
\begin{verbatim}
        C:\wintools\imonc /S:fli4l /P:secret /rc:"dialmode manual"
\end{verbatim}
\end{example}

  \subsection{Concernant l'aperçu de imonc}

  Imonc client Windows interroge imond pour avoir les informations sur les
  connexions Internet existantes, il les affichent dans un tableau. Sur cette
  page il y a aussi le statut général du routeur, l'heure, la date, le bouton
  synchronisation, etc. Voici les descriptions de ces fenêtres~:

  \begin{tabular}{lp{9cm}}
    Statut             &Calling/Online/Offline (appel/en ligne/raccrocher)\\
    Nom                &Le numéro de Tél ou le Nom du circuit\\
    Direction          &On voie si c'est une connexion entrante ou sortante\\
    IP                 &Adresse IP qui à été assignée\\
    I/Octets           &Octets Entrants\\
    O/Octets           &Octets Sortants\\
    T/enligne          &Temps en ligne\\
    T/Total            &Temps total en ligne\\
    Prix/Unit          &Prix de l'unité par connexion\\
    Prix               &Prix total de la connexion\\
  \end{tabular}

  \medskip

  Les données seront actualisées toutes les deux secondes. (Maintenant) cette
  intervalle peut être changé. Dans le menu on est en mesure de voir le
  canal sur lequel le routeur est en ligne en temps réel. Copiez l'Adresse IP
  réelle dans le presse-papier et installez le canal indiqué explicitement.
  Ceci peut être intéressant s'il y a plusieurs connexions différents par ex.
  une pour naviguer sur Internet et l'autre connectée à votre entreprise,
  de cette façon vous pouvez débrancher l'une ou l'autre connexion.

  En plus si vous avez activé telmond sur votre routeur fli4l, imonc sera en
  mesure d'afficher les informations sur les appels téléphoniques entrants
  (le nom et le numéro de Tél du correspondant). Le dernier appel téléphonique
  reçu sera vu au-dessus des boutons de commande. Un protocole des appels
  téléphoniques entrants peut être vu en utilisant les pages d'appels.

  Les six boutons mentionnés ci-dessous vous permettront de choisir les commandes
  suivantes~:

  \begin{tabular}{clp{9cm}}
    Bouton & Description      & Fonction\\
    1& Connecter/Raccrocher   & Connecter ou raccrocher la ligne\\
    2& Ajout Canal/Supp Canal & Ajoute ou supprime un canal, cette caractéristique
                                n'est disponible que dans le Mode Admin\\
    3& Redémarrer             & Redémarre fli4l!\\
    4& Éteindre               & Arrête fli4l proprement et met le routeur hors tension\\
    5& Arrêter                & Arrête fli4l proprement, pour éteindre le routeur 
                                en toute sécurité\\
    6& Sortir                 & Sort du programme client imonc\\
  \end{tabular}

  \medskip

  \noindent Les cinq premières commandes en mode Utilisateur peuvent être activées
  ou désactivées dans le fichier de configuration /config/base.txt pour le
  routeur fli4l. En mode administrateur toutes les commandes sont toujours
  activées.
  Le choix de la commande Dialmode modifie le comportement du routeur~:

  \begin{tabular}{lp{9cm}}
    Auto    & Le routeur établira automatiquement une connexion
              Internet s'il y a une demande dans réseau local.\\
    Manuel  & L'utilisateur doit établir la connexion manuellement.\\
    Couper  & Il n'y a aucune connexion possible, ni manuellement ni
              automatiquement. La selection du bouton de "connexion" est
              désactivée.\\
  \end{tabular}

  \noindent La volonté de fli4l par défaut c'est d'établir automatiquement une
  connexion Internet sur une demande de requéte Internet par n'importe quel
  Hôte du réseau local. En principe on ne doit jamais modifier la commande pour
  se connecter \ldots

  Il y a également la possibilité de changer manuellement le Circuit-Défaut-Route,
  c.à d. commuter marche/arrêt ou automatique, c'est pourquoi la liste de sélection
  de "Default route" (choix du FAI) est prévu dans la version de Windows d'imonc.
  En outre, on peut maintenant configurer directement dans imonc l'heure de déconnexion.
  Utiliser le Bouton "config" en dessous de Défaut-Route ici la configuration de
  tous les circuits pour le routeur sont indiqués. La valeur de la variable
  Hup-timeout peut être éditée directement dans le fichier isdn.txt du Circuit ISDN
  (ne fonctionne pas pour le moment avec la DSL).

  Un aperçu de LCR-Routing se trouve sur la page Admin/Plage Horaire. Là,
  vous pouvez voir, le Circuit qui sera démarré automatiquement.

  \subsection{Paramètres de configuration}

  On peut accéder à la configuration par le bouton "config" dans la barre d'état.
  La fenêtre qui s'ouvre est divisée en deux, dans le tableau de gauche vous
  avez les répertoires et sous-répertoires, dans celui de droite la configuration
  de imonc. Voici les répertoires en détail~:

  \begin{itemize}
  \item Répertoire général~:
    \begin{itemize}
    \item Synchroniser tous les~: on ajuste ici le nombre de rafraîchissement
      en seconde de la page d'accueil.
    \item Synchroniser au démarrage~: synchronise l'heure et la date du routeur
      avec le client au démarrage. on peut activer cette fonction manuellement
      avec le bouton "Synchroniser" sur la page d'accueil. 
    \item Réduire au démarrage~: au démarrage le programme sera réduit en icône.
      Vous verrez seulement l'icône à côté de l'heure.
    \item Lancer imonc au démarrage Windows~: ici le client imonc démarre
      automatiquement aprés le démarrage de Windows. on peut entrer dans la
      fenêtre "paramètre" des commandes supplémentaires.
    \item Voir l'actualité de fli4l.de~: ici on peut recevoir les (News) du site
      fli4l.de chargées automatiquement par imonc, les titres sont alors montrés
      dans la fenêtre "Nouvelles" et qui pourront être lus.
    \item Appel du fichier log~: on indique ici le nom du fichier pour
      enregistrer la liste des appels locaux.
    \item Attendre la réponse du routeur~: temps d'attente d'une réponse du
      routeur en seconde, avant que la connexion soit perdue.
    \item Langue~: on choisit ici la langue pour imonc.
    \item Confirmer les commandes du routeur~: si la case est cochée, toutes
      les commandes envoyées au routeur demande une confirmation,
      ex. redémarrage, déconnexion etc \ldots
    \item Arrêt même avec trafic~: si aucune réponse n'aboutit, la connexion
      s'arrête même si il y a toujours du trafic sur cette connexion.
    \item Reconnexion automatique au routeur~: une reconnexion du routeur est
      faite automatiquement, si une coupure de la connexion a eu lieu,
      (p. ex. un redémarrage du routeur).
    \item Reduire la fenêtre système~: si activée, en cliquant sur le bouton
      "Sortir" imonc se réduit en icône vers la barre des taches à côté de
      l'heure, au lieu de s'arrêter.
    \end{itemize}

  \item Sous-répertoire proxy~: ici on enregistre le proxy pour les demandes http.
    Celui-ci est utilisé à présent pour l'actualisation des fenêtres, time-table
    et news.
    \begin{itemize}
    \item Active le proxy pour le protocole http~: ici on active Proxy
          \begin{itemize}
            \item Adresse~: ici l'adresse du serveur proxy
            \item Port~: ici le numéro de port du serveur proxy (defaut: 8080)
          \end{itemize}
    \end{itemize}

  \item Sous-répertoire icône~: ici on peut personnaliser les couleurs des icônes.
    Dans l'avenir on pourra choisir les couleurs de fond de l'icône pour dialmode
    (mode de connexion) qui sera placé dans la barre de tache.

  \item Répertoire d'appel, le réglage de la position de la fenêtre avis d'appel
    sur l'écran, sera stockée et sauvegardée dans la base de Registre. Vous pouvez
    déplacer la fenêtre à l'endroit de votre choix. Après ce réglage, la fenêtre
    apparaitra exactement cet endroit à chaque fois.
    \begin{itemize}
      \item Mise à jour~: on peut choisir ici, comment imonc reçois les
        informations des nouveaux appels tél, Il y a trois possibilités différentes.
        La premier consiste à interroger régulièrement de service telmond sur le
        routeur. Une autre possibilité consiste à interroger les annonces de
        Syslog, cette variante est la préférer~-- On doit Naturellement activer
        Syslog dans le client imonc. Imonc doit être connecté à une direction approprié,
        la troisième possibilité proposé est d'utiliser le paquetage Capi2Text pour
        la signalisation d'appel.
      \item Effacer premier zéro~: parfois devant le numéro de téléphonique est
        placé un zéro supplémentaire. Celui-ci peut être supprimé avec cette option.
      \item Indicatif régional~: la présélection personnelle du numéro de tél
        peut être écrite ici. Quand un appel arrive avec la même présélection.
        La présélection ne sera pas visible.
      \item Annuaire~: ici on indique le fichier dans lequel l'annuaire
        téléphonique local sera sauvegardé pour les numéros de téléphones.
        Si le fichier n'existe pas, il est automatiquement installé.
      \item Fichier log~: on indique ici le nom du fichier, utile pour enregistrer
        la liste des appels sur l'ordinateur local. Ce paramètre est visible
        uniquement si la variable \var{TELMOND\_\-LOG} être sur `yes', c'est
        également valable pour la liste des appels réelle.
      \item Recherche externe~: un programme peut être indiqué dans cette fenêtre,
        que l'on appelle si un numéro de téléphone ne peut pas être résolu au moyen
        de l'annuaire téléphonique local. Des infos plus précises devraient être
        jointes aux programmes correspondants. Il y a jusqu'à présent un CD
        d'annuaire téléphonique de Marcel Wappler KlickTel ainsi qu'un lien vers
        une base de données.
    \end{itemize}

  \item Sous-Répertoire des appels tél~:
    ces options sont destinées, à détailler des instructions des appels téléphoniques
    et de les afficher, voir les illustrations ci-dessous.
    \begin{itemize}
      \item Notification d'appel actif~: détermine si des appels doivent être signalés.
      \item Indication des notifications d'appels~: lors d'un appels tél une fenêtre d'
        apparait, elle détaille les Infos suivantes~: l'appel MSN, le numéro de tél du
        correspondant et la date/heure de l'appel. Pour cela il est nécessaire que la
        variable \var{OPT\_\-TELMOND} soit placée sur `yes' dans le fichier
        config/isdn.txt
        \begin{itemize}
          \item Ne pas enregister les numéros non transmis~: les appels ne doivent pas
            être écrit dans la fenêtre d'appel, si aucun numéro de Tél n'a été transféré.
          \item Temps d'affichage~: cette indication influe sur la durée de fermeture de
            la fenêtre avis d'appel, la fenêtre doit rester ouverte un certain temps.
            Si on indique "0" la fenêtre ne se fermera pas automatiquement.
          \item Fontsize (ou police)~: ici on choisit la taille des caractères pour la
            fenêtre. Celle-ci affecte la taille de la fenêtre, puisque la taille de la
            fenêtre sera calculée par rapport à la taille du message.
          \item Couleur~: ici on choisit la couleur des textes dans la fenêtre d'appel.
            J'emploie le rouge pour l'identification des messages.
      \end{itemize}
    \end{itemize}

  \item Sous-répertoire annuaire~: la fenêtre contient l'annuaire téléphonique
    qui est utilisé pour la définition des numéros de téléphones des appels
    entrants et aussi si vous possédez un MSN. Cette fenêtre apparait même si la
    variable \var{TELMOND\_\-LOG} est sur `no' parce que cette fenêtre est utilisée
    aussi pour le dernier appel entrant vu dans la fenêtre principale. On peut choisir
    un fichier qui sera placé sur le routeur.

    Construction d'un appel entrant~:

\begin{example}
\begin{verbatim}
  # Format:
  # Telefonnummer=anzuzeigender Name[, Wavefilename]
  # 0241123456789=Testuser
  00=unbekannt
  508402=Fax
  0241606*=Elsa AG Aachen
\end{verbatim}
\end{example}

    Les trois premières lignes sont des commantaires. La quatrième
    ligne est créée si aucun numéro n'est transmis, "unbekannt" (ou inconnu)
    sera affiché. La cinquième ligne indique le numéro de tél "508402" et le
    Nom "Fax" , dans tous les cas le format sera toujours le même, Numéro de
    Tél=Nom. La sixième ligne détermine l'ensemble des numéros de Tél, pour
    toutes appel ex. 0241606 le Nom sera affiché. Souvenez-vous que le dernier
    numéro d'appel du correspondant est indiqué sur la première fenêtre
    principale. Optionnel, un fichier son peut être défini et sera joué lors
    d'un appel Tél.

    Dés la Version 1.5.2, il est possible d'installer un annuaire Téléphonique
    sur le routeur sous la forme d'un fichier il sera enregistré et synchronisé
    dans (/etc/phonebook). Si un même numéro de téléphone avec un Nom différent
    sont enregistrés dans l'annuaire Du routeur et dans l'annuaire de imonc,
    il sera demandé à l'utilisateur qu'elle est l'entrée valide. les appels ne
    sont pas juste recopiés mais sont enregistrés sur les deux annuaires. La
    synchronisation du fichier d'annuaire est faite dans la mémoire RAM, cela
    veut dire, lorsque l'on reboot (redémarre) le routeur, le fichier sera perdu.

  \item Répertoire son, les fichiers son qui seront installés ici seront joués,
    si l'événement indiqué se produit.
    \begin{itemize}
      \item Courriel~: le fichier son sera joué, si un nouveau courriel se trouve
        sur votre Serveur POP3. 
      \item Erreur courriel~: le fichier son sera joué, si une erreur se produit
        lors de la réception du Courriel.
      \item Connexion perdu~: le fichier son sera joué, si la connexion avec
        le routeur est perdue (ex. redémarrage du routeur). Si l'option
        "reconnexion automatique au routeur" n'est pas activée, un messagebox
        s'ouvrira pour demander une nouvelle connexion au routeur.
      \item Connexion~: le fichier son sera joué, si le routeur établit
        une connexion Internet.
      \item Déconnexion~: le fichier son sera joué, lorsque le routeur déactive
        la connexion Internet.
      \item Avis appel~: le fichier son sera joué, si l'annonce des appels est
        activée et si un nouvel appel est reçu.
      \item Annonce de fax~: le fichier son sera joué, après la réception de
        nouveaux fax.
    \end{itemize}

  \item Répertoire courriel
    \begin{itemize}
      \item Comptes~: cette fenêtre sert à configurer les comptes POP3.
      \item Activer le contrôle courriel~: si vous avez un compte courriel il
        recherchera automatiquement les nouveaux courriels.
        \begin{itemize}
          \item Vérifier x/Min~: cette option définit un intervalle temps entre
            chaque contrôle Courriel sur le compte. Attention~: en définissant un 
            intervalle trop court, le routeur peut rester constamment en ligne!
            Ceci se produit lorsque l'intervalle est plus court que "Delai attente"
            du circuit utilisé.
          \item Temps d'attente x/Sec~: temps d'attente d'une réponse du Serveur POP3
            avant l'arrêt de celui-ci, si la valeur est à "0" cela signifie qu'aucun
            TimeOut (temps d'attente) n'est installé.
          \item routeur déconnecté~: cette option permet au routeur de se connecter
            automatiquement pour rechercher les nouveaux courriels sur le Serveur POP3.
            Aprés le téléchargement des courriels le routeur se déconnecte. pour pouvoir
            utiliser ce dispositif on doit mettre Dialmode sur 'auto'. Attention~:
            cela occasionne des frais supplémentaires de connexion si aucun tarif
            unitaire est utilisé~!
          \item Circuit à utiliser~: cette option définit le circuit qui sera utilisé
            pour la connexion aux courriels.
          \item Rester en ligne aprés contrôle~: la déconnexion doit être faire
            manuellement ou l'arrêt doit être réalisé automatiquement par l'option
            Délai attente.
          \item Charger en-têtes des courriels~: télécharger les en-têtes des courriels ou
            uniquement le nombre de courriel disponible? Cette option doit être activée
            pour supprimer les courriels directement sur le serveur POP3.
         \item M'avertir de nouveaux courriels~: faut-il un message sonore et une icône
            dans la barre de tâche pour m'annoncer de nouveaux courriels.
         \item Exécuter le programme de messagerie~: démarrer automatiquement le
            programme de messagerie pour lire les nouveaux courriels disponibles.
         \item Programme~: indiquer ici le programme de messagerie.
         \item Paramètre~: entrer les paramètres additionnels qui seront transférés
            au démarrage du programme de messagerie. Si Outlook est utilisé comme
            programme courriel (pas Outlook Express!) vous pouvez entrer comme paramètre
            "/recycle" empêche de lancer Outlook dans une nouvelle fenêtre s'il est
            déjà ouvert.
      \end{itemize}
    \end{itemize}

  \item Repertoire Admin
    \begin{itemize}
      \item Mot de passe Root~: ici on entre le mot de passe du routeur qui est
        dans le fichier (/config/base.txt dans la variable \verb+PASSWORD+) pour pouvoir
        par exemple configurer Portforwarding sur votre ordinateur et l'envoyer
        sur le routeur.
      \item Voir les fichiers sur le routeur~: tous les fichiers log (ou journal)
        qui se trouvent sur le routeur sont à indiquer ici, ils peuvent être lus,
        avec un simple clic de la souris dans la page Admin/fichier, ainsi on peut
        afficher les fichiers log du routeur directement dans imonc.
      \item Fichier de configuration~: ici on peut choisir, si tous les fichiers
        seront ouverts avec le programme éditeur de texte ou uniquement les fichiers
        *.txt pour étudier et travailler dessus. On peut égalment ouvrir un ensemble
        de fichiers.
      \item DynEisfairLog~: si vous avez créé un compte sur DynEisfair, vous pouvez
        enregistrer ici les données d'accés et de voir avec le fichier Log les mises
        à jours des fonctions sur la page Admin/DynEisfairLog.
    \end{itemize}

  \item Répertoire démarrage auto, sert à configurer une liste de programmes qui
    sera lancée automatiquement. Celle-ci est exportée après une connexion réussie
    si l'option "Activer la liste des programmes" est cochée.
    \begin{itemize}
      \item Programme~: tous les programmes installés ici seront lancés
        automatiquement, si le routeur est connecté et que La Liste des
        programmes est cochée.
      \item Activer la liste des programmes~: la liste doit-elle être activé pour
        exécution des programmes aprés une connexion réussie~?
    \end{itemize}

  \item Répertoire trafic du réseau, est utilisé pour la configuration
    (personnalisée) de la fenêtre de Info trafic. Un utilisateur m'a averti qu'il
    y avait quelques erreurs sur la définition des données avec des versions
    anciennes de DirectX.
    \begin{itemize}
      \item Voir les informations sur le trafic~: voulez-vous afficher une utilisation
        graphique des canaux dans une fenêtre à part? Dans le menu contextuel vous
        pouvez choisir l'attribut StayOnTop, cette option provoque l'affichage de la
        fenêtre sur toutes les autres fenêtres. Cette option sera enregistrée dans
        la base de registre et sera en service aprés un redémarrage du programme.
      \item Voir les titres~: doit-on monter la barre de titre dans la fenêtre
        Traffic-Info? Cette fenêtre montrera les informations des circuits utilisés
        par le routeur.
        \begin{itemize}
          \item Voir l'utilisation CPU~: montrer l'utilisation du CPU dans la barre
            de titre?
          \item Voir le temps de communication~: le temps en ligne du canal doit-il
            aussi être indiqué dans la barre de titre~?
        \end{itemize}
      \item Fenêtre semi-transparente~: la fenêtre doit-elle être représentée en
        transparence? Cette fonction n'est disponible que sous
        Windows 2000 et Windows XP.
      \item Couleur~: les couleurs sont définies ici pour la fenêtre Traffic-Info.
        Maintenant le canal DSL et le premier canal ISND utiliseront les mêmes
        couleurs. 
      \item Limite~: entrer les valeurs maximales des taux de transmission xDSL~--
        pour T-Online~: Débit Montant (upload) 128 Ko/s et Débit Descendant
       (download) 1024 Ko/s.
    \end{itemize}

  \item Répertoire Syslog, est utilisé pour la configuration de l'affichage
    des messages Syslog.
    \begin{itemize}
      \item Activer le client Syslog~: montrez les messages Syslog dans imonc~?
        Cette option doit être arrêtée, si vous utilisez un autre client Syslog
        externe, par exemple le client Kiwi's Syslog.
      \item Indiquer les messages Syslog~: monter les messages Syslog avec un
        niveau de prioritaire~? Vous pouvez indiquer ici les niveaux prioritaires
        des messages Syslog, par defaut le message débug est coché, vous pouvez
        cocher le niveau selon vos besoins.
      \item Enregistrer les messages Syslog~: les messages lus doivent-ils être
        sauvegardés? Dans la fenêtre on peut choisir les messages que l'on veut
        sauvegarder. On peut insérer des caractères supplémentaires avec nom de
        fichier à sauvegarder, les voici~:
        \begin{description}
          \item[\%y]~-- On l'ajoute pour avoir l'année actuel
          \item[\%m]~-- On l'ajoute pour avoir le mois actuel
          \item[\%d]~-- On l'ajoute pour avoir le jour actuel
        \end{description}
      \item Voir le nom des ports~: doit on afficher la description du port au
        lieu du numéro de port~?
      \item Voir les messages pare-feu~: ici, on indique les messages du firewall
        (ou pare-feu), il seront aussi indiqués en mode utilisateur.
    \end{itemize}

  \item Répertoire fax, sert à configurer les fax (ou télécopie) dans imonc. Pour
    que ce dossier soit visible vous devez installer sur le routeur le paquetage
    mgetty et/ou faxrcv, (vous pouvez les trouver sur le site de fli4l).
    \begin{itemize}
      \item Fichier Log pour fax~: ici on peut enregister les fax reçus sous forme
        de fichier dans un dossier de l'ordinateur.
      \item Répertoire local des fax~: configurer le répertoire pour stocker les
        fax reçus, avant de les consulter.
      \item Actualisation~: il y deux possibilités, lorsque imonc reçoit un
        nouveau fax. Soit c'est imonc Syslog qui reçoit les fax (naturellement
        le client imonc-Syslog doit être activé), soit imonc regarde régulièrement
        le fichier log. La première variante est la meilleure. Si vous utilisé la
        deuxième variante, vous pouvez indiquer combien de fois la page d'aperçu de
        fax doit être actualisée. Il faut faire attention cette valeur n'est pas
        une indication en seconde, mais c'est une indication en multiple,
        en général c'est une intervalle d'actualisation.
    \end{itemize}

  \item Répertoire tableau, sert à ajuster les colonnes des (tableaux) dans imonc
    par rapport à vos besoins. D'une part, pour chaque tableau on peut régler les
    en-têtes les colonnes qui doivent être affichées, d'autre part pour chaque
    service de communication il y a un tableau différent, appel Tél, fax, on peut
    régler le moment ou les Infos doivent être affichées.
  \end{itemize}

  \subsection{Concernant les appels tél}

  L'annuaire Téléphonique sera uniquement vu, que si la variable \var{TELMOND\_\-LOG}
  est placée sur 'yes', sinon aucun journal d'appels ne sera conservée. Dans cet
  annuaire sera enregistré tous les appels téléphoniques qui seront entrées sur
  le routeur. Vous pouvez commuter entre les appels enregistrés sur le PC local
  et les appels enregistrés sur le routeur, vous pouvez effacer le fichier sur
  le routeur avec le bouton-Réinitialisé.

  Dans l'annuaire téléphonique, vous pouvez cliquer avec le bouton droit
  de la souris sur le Numéro de Tél pour attribuer un Nom au numéro, comme
  cela le Nom apparaitra à la place du Numéro de Tél.

  \subsection{Concernant les connexions}

  L'affichage des connexions internet par le routeur dans une page est utilisé
  de puis la Version 1.4, elle donnera une bonne vue d'ensemble du comportement
  du routeur connecté à Internet. Pour voir cette page la variable \var{IMOND\_\-LOG}
  doit être placée sur `yes' dans le fichier /config/base.txt.

  De la même façon que l'annuaire-Tél, vous pouvez commuter les connexions
  enregistrées localement et celles enregistrées sur le routeur. Vous pouvez
  aussi effacer le fichier des données sur le routeur en cliquant sur
  le bouton-rafraîchir.

  Affichage du tableau de connexions.

  \begin{itemize}
  \item Nom du FAI
  \item Date et heure de départ
  \item Date et heure de fin
  \item Temps en ligne
  \item Prix de l'unité
  \item Prix Total
  \item Réception du signal
  \item Émission du signal
  \end{itemize}

  \subsection{Concernant les FAX}

  Pour que soit affichée la page FAX il faut installer le paquetage
  \var{OPT\_\-MGETTY} par M. Michael Heimbach sur le routeur ou
  \var{OPT\_\-MGETTY} par M. Felix Eckhofer. Sur le site Internet de
  fli4l, à la page d'accueil vous avez un raccourci pour les
  paquetages-OPT. Dans cette fenêtre tous les FAX reçus seront enregistrés,
  le menu contextuel offre plusieurs options de configuration qui seront
  uniquement disponibles en mode Administrateur~:

  \begin{itemize}
  \item Concernant les Fax reçus, il faut correctement configurer le chemin
  d'accès pour fli4l dans répertoire Admin/Remoteupdate, pour que les FAX
  reçus sur le routeur soient enregistrés et compressés avec le programme
  gzip, qui se trouve dans le paquetage fli4l, le programme gzip.exe et le
  fichier win32gnu.dll peuvent aussi être copié dans le répertoire imonc.
  Si gzip.exe n'est pas trouvé dans l'un des deux emplacements, et si le
  routeur est connecté à Internet, il recherchera le programme sur internet
  (directement sur le site CGIs).
  \item Supprimer un FAX reçu. Cela signifie que le FAX sera supprimé sur
    votre PC local et sur le routeur (le fichier FAX réel, et aussi dans
    le fichier log).
  \item Supprimer tous les FAX présents sur routeur. Ici tous les FAX
    sur le routeur dans le fichier log seront effacés. Les FAX ne seront
    pas effacés du fichier log de votre PC local.
  \end{itemize}

  Comme dans la page des appels Tél, vous pouvez commuter entre les Fax
  enregister localement et les Fax enregistrés sur le routeur.

  \subsection{Concernant les courriels}

  Cette page apparait, si dans le répertoire Config courriel il y a au moins
  un compte \mbox{courriel} avec serveur POP3 qui a été configuré et activé.

  Description de la page \mbox{courriel}. Maintenant on a intégré dans cette section
  le contrôleur de \mbox{courriel}. Si l'option "le routeur n'est pas en ligne" dans
  config \mbox{courriel} n'est pas activée, le contrôleur de Mail vérifiera tous les
  comptes \mbox{courriel}, ensuite il utilisera l'intervalle Temps pour vérifier le
  Serveur (le routeur doit être connecté, il utilise le circuit
  présélectionné). Si le routeur n'est pas connecté, activer l'option "le
  routeur n'est pas en ligne" et indiquer le circuit à utiliser, il établira
  une connexion en utilisant le circuit choisi et téléchargera les \mbox{courriels} de
  tous les comptes \mbox{courriels} configurés ensuite il fermera la connexion. Pour
  utiliser cette option vous devrez placer Dialmode sur "auto".

  Si des \mbox{courriels} sont disponible sur le serveur POP3, le programme
  \mbox{courriel} client sera démarré automatiquement ou une icône apparaitra
  prés de l'heure dans la barre de tache, il indiquera le nombre de \mbox{courriel}
  sur le serveur. En double cliquant dessus l'ensemble du \mbox{courriel} client
  sera lancés. Si une erreur se produit sur un compte \mbox{courriel}, d'une part,
  une note sur l'erreur sera écrit dans le dossier Histoire du \mbox{courriel},
  d'autre part, l'icone du \mbox{courriel} affichera dans le coin supérieur droit
  une couleur rouge.

  Dans la fenêtre \mbox{courriel}, on peut effacer directement les \mbox{courriels} sur le
  Serveur sans les avoir préalablement téléchargés. Il faut avoir téléchargé
  les en-têtes des courriels, vous devez marquer les cellules à supprimer, puis
  en cliquant sur le bouton droit de la souris le menu contextuel s'ouvre,
  et cliquer sur Delete MailMessage.

  \subsection{Admin}

  Cette partie est uniquement disponible si imonc est démarré en mode Admin.

  Premier point, cette page offre une vue d'ensemble des circuits utilisés,
  ~--les fournisseurs d'accès Internet~-- qui ont été choisis automatiquement
  par le routeur (par l'intermédiaire du LC Routing). En double cliquant sur un
  fournisseur d'accès dans l'aperçu fournisseur d'accès vous obtiendrez
  l'affichage des définitions des plages horaires pour ce fournisseur qui à été
  défini dans /config/base.txt.

  Deuxième point, cette page donne l'occasion d'installer les mises à jour à
  distance sur le routeur. Vous pouvez choisir l'un ou les cinq programmes
  (Kernel, fichier système, fichier OPT, rc.cfg et syslinux.cfg) qui seront
  copiés sur le routeur. Pour pouvoir faire la mise à jour à distance, vous
  devrez indiquer le répertoire de fli4l dans imonc et les fichiers nécessaires
  à copier. En plus, vous devez écrire le sous-répertoire des fichiers de
  configuration (par défaut: /config/*.txt) pour devez construire tous les
  fichiers systèmes de fli4l. Il est conseillé de Rebooter (ou redémarrer) après
  avoir envoyé les fichiers système sur le routeur pour que les modifications
  soient prises en compte. Si un mot de passe est demandé par le routeur, il
  est inscrit dans la variable PASSWORD dans /config/base.txt.

  Troisième point, cette page traite des contraintes du Port Forwarding,
  un port est connecté exactement et uniquement à un ordinateur client.
  Maintenant il est possible d'éditer et de configurer Port-Forwarding du
  routeur. aprés les modifications des ports ils seront activées, la
  connexion doit être active. Puisque les fichiers sont enregistés dans la
  mémoire virtuelle (Ramdisk), tous les changements seront uniquement
  sauvegardés jusqu'au prochain redémarrage du routeur. Pour sauvegarder des
  changements de manière permanente vous devez changer des Port Forward dans
  le fichier /config/base.txt et installer le nouveau fichier-OPT sur le routeur.

  Quatrième point, dans la fenêtre Admin, puis~-- fichier~-- vous pouvez utilisée
  et voir la configuration des fichiers Log du routeur, en cliquant
  simplement sur la souris. La liste de choix peut être configurée dans le
  dossier config-Admin d'imonc "voir les fichiers sur le routeur". Ensuite,
  vous pouvez simplement choisir les fichiers qui sont indiqués dans le menu
  déroulant.

  Cinquième point, cette fenêtre montre DynEisfair log, elle apparaît
  uniquement si dans le répertoire de configuration Config-Admin les
  enregistrements les données pour un accès à un compte DynEisfair a
  été configuré (pour simuler une IP fixe, lorsque l'on a une IP dynamique).
  Si cela est fait, le fichier log des services sera indiqué dans cette fenêtre.

  Dernier point, fenêtre hôtes, tous les ordinateurs enregistrés dans le fichier
  /etc/hosts sont indiqués ici, à l'avenir on essaiera de configurer chacun des
  ordinateurs enregistrés pour pouvoir les "pinger" (ou interroger)
  individuellement, ainsi on pourra rapidement vérifier l'ordinateur qui est 
  connecté au réseau local.

  \subsection{Concernant les erreurs syslog et firewall}

  Les pages erreur, syslog et Firewall (pare-feu), s'affiche uniquement
  s'il y a des événements enregistrés dans ce fichier, en plus il faut
  être en mode Admin pour que les pages soit affichées.

  Toutes les erreurs spécifiques à imonc/imond seront enregistrées dans la 
  fenêtre erreur. Si vous avez des problèmes vous pouvez aller vérifier dans
  cette liste pour voir les causes des erreurs que vous avez rencontrées.

  Dans la fenêtre Syslog les messages de syslog seront affichés, excepté des
  messages du pare-feu. Ceux-ci sont affichés dans une page indépendante
  (voir ci-dessous). Pour que la page syslog  fonctionne vous devrez placer
  la variable \var{OPT\_\-SYSLOGD} sur "yes" dans le fichier de configuration
  /config/base.txt En plus dans la variable \var{SYSLOGD\_\-DEST} on doit
  placer l'adresse IP du client qui bien entendu utilise imonc (par exemple~:
  \var{SYSLOGD\_\-DEST}='@ 100.100.100.100~-- adresse IP de votre client!).
  Il n'y aura pas que les messages syslog qui seront affichés, mais aussi
  la date, l'heure, l'IP et le niveau de priorité.

  Des messages du Firewall (pare-feu) seront affichés dans une page
  indépendante. Pour que la page fonctionne, vous devez placer la variable
  \var{OPT\_\-KLOGD} sur 'yes' dans le fichier de configuration /config/base.txt.

  \subsection{Concernant les News}

  Cette page News (ou d'actualité), doit être activée dans le répertoire
  config-Imonc. Les News mentionnés sur la page accueil du site fli4l, seront
  visibles directement dans Imonc à la page accueil. On peut directement aller
  sur le site http://www.fli4l.de/german/news.xml avec le bouton-plus. Vous
  avez une fenêtre à côté des titres des News, qui indique les 10 derniers
  paquetages-OPT enregistrés sur le site
  http://www.fli4l.de/german/imonc\_opt\_show.php, en double cliquant sur le
  paquetage choisi, vous allez directement sur le site. En plus, il est
  indiqué dans la barre de statut en bas de Imonc, les titres des News.


  \marklabel{sec:imonc}{
    \section{Client imonc pour Unix/Linux}}

  Il y a deux versions pour le Linux~: une version de base (imonc)
  en texte uniquement et une version avec une interface graphique
  (ximonc). On peut trouver dans le répertoire /src les fichiers
  sources pour ximonc. La documentation pour le ximonc sera disponible
  dans la version 1.5 finale. Les utilisateurs expérimentés de Linux  ne
  devraient pas avoir de problème avec les fichiers sources.

  Nous nous limiterons ici à la version de base imonc en texte~: C'est
  un programme qui fonctionne uniquement par commande clavier. il n'a
  donc aucune interface graphique. Les fichiers sources peuvent être
  trouvés dans le répertoire unix.

  Installation~:

\begin{example}
\begin{verbatim}
        cd unix
        make install
\end{verbatim}
\end{example}

  Imonc est installé dans /usr/local/bin

  Démarrer le programme~:

\begin{example}
\begin{verbatim}
        imonc "hostname"
\end{verbatim}
\end{example}

  Le nom ou adresse IP du routeur fli4l doit être indiqué à la place
  de "hostname", par exemple.

\begin{example}
\begin{verbatim}
        imonc fli4l
\end{verbatim}
\end{example}

  imonc montre les information suivantes~:

  \begin{itemize}
  \item Data/Heure du routeur fli4l

  \item La connexion du FAI du moment

  \item Le Circuit par défaut (Default-Route-Circuits)

  \item Le canal ISDN (numéris)
    \begin{description}
    \item[Status]~:         Appel Tél en-ligne/déconnecté
    \item[Name]~:           Le numéro de Téléphone du Fournisseur d'accés
    \item[Time]~:           Temps de connexion
    \item[Charge-Time]~:    Connexion par unité de temps
    \item[Charge]~:         Prix de la connexion
    \end{description}
  \end{itemize}

  Les commandes sont~:

  \begin{tabular}{lll}
    N$^\circ$   &Commande              &Signification\\
    0   &quit                &Arrêt du programme\\
    1   &enable              &Activer\\
    2   &disable             &Déactiver\\
    3   &dial                &Composer le N$^\circ$\\
    4   &hangup              &Raccrocher\\
    5   &reboot              &Redémarrer\\
    6   &timetable           &Table de plage horaire \\
    7   &dflt route          &Nouveau Default-Route-Circuit \\
    8   &add channel         &Ajouter le deuxième canal\\
    9   &rem channel         &Supprimer le deuxième canal\\
  \end{tabular}

  \medskip

  \noindent Explication des commandes~:

  \begin{description}
  \item[0~-- quit] Quitter le serveur imond, le programme est arrêté.


  \item[1~-- enable] Tous les circuits seront placés en numérotation
    "auto". C'est l'état par défaut de fli4l après l'avoir initialisé.
    Cela signifie~: lorsqu'il y a une demande de connexion du réseau
    interne sur Internet, fli4l composera automatiquement le numéro
    de Tél du FAI.


  \item[2~-- disable] Tous les circuits du mode de composition seront
    placés sur "OFF". Après cette action fli4l est presque "mort"
    (ou en sommeil). fli4l sera réveillé au moyen de la commande "enable".


  \item[3~-- dial] Composer manuellement le numéro de Tél du FAI, cette
    fonction sert de test. Puisque cette commande est normalement sur
    automatique par l'intermédiaire du circuit par défaut. Elle est
    utilisée pour des essais, depuis que fli4l existe la connexion est
    habituellement automatique.


  \item[4~-- hangup] Raccrocher manuellement~: de cette façon, on peut
    devancer le raccrochement automatique de fli4l.


  \item[5~-- reboot] fli4l sera redémarré. Commande pas vraiment utile\ldots


  \item[6~-- timetable] Table des Horaires pour arrêter ou démarrer
    les circuits par défaut voir les détails page précédente.


  \item[7~-- default route circuit] Changer manuellement un circuit
    par défaut. Peut être logique par ex. pour arrêter un moment
    LC-Routing automatique de fli4l, parfois les fournisseurs ne
    permettent pas l'accèder à votre propre boîte mail, vous devez
    utiliser un autre fournisseur d'accés.


  \item[8~-- add channel] On l'utilise pour ajouter le deuxième canal
    ISDN (ou numéris en français). Vous devez placer la variable
    \var{ISDN\_\-CIRC\_\-x\_\-BUNDLING} sur `yes'.


  \item[9~-- remove channel] Coupe le deuxième canal ISDN. Voir
    également "add channel".

  \end{description}

  \noindent Avec les commandes imond, les mêmes remarques son valables
  qu'avec le client \verb+imonc.exe+ sous Windows.

  Remarque complémentaire~: Avec la version 1.4 de fli4l il est maintenant
  possible d'installer un client imonc "allégé" sur le Routeur de fli4l.
  Pour se faire il faut plaçer le paquetage option sur
  \smalljump{OPTIMONC}{\var{OPT\_\-IMONC}}='yes' dans le paquetage
  \smalljump{sec:tools}{\var{TOOLS}}.

  De cette façon, on peut maintenant configurer certains paramètres avec imonc
  par ex. pour faire du routage, etc. en utilisant la console fli4l. Attention~:
  Ce Mini-imonc fonctionne uniquement sur le routeur fli4l~! Sous Linux/Unix,
  il faut toujours utiliser le Client imonc/unix "son grand frère".
