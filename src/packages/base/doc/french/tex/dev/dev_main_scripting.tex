% Synchronized to r35952

\section{Conditions générales de création de script pour fli4l}

Nous n'avons \emph{pas} écrit d'introduction générale pour le Scripts-Shell,
mais vous pouvez lire cette introduction sur Internet, ici nous traitons que des
situations particulières pour fli4l. Des informations complémentaires sont disponibles
dans les différentes pages du manuel Unix-/Linux-. Les liens suivants peuvent servir
de point de dépard pour ce sujet~:

\begin{itemize}
\item Introduction au Scripts-Shell~:
  \begin{itemize}
  \item \altlink{http://cip.physik.uni-freiburg.de/main/howtos/sh.php}
  \end{itemize}
\item Pages d'aide en ligne~:
   \begin{itemize}
   \item \altlink{http://linux.die.net/}
   \item \altlink{http://heapsort.de/man2web}
   \item \altlink{http://man.he.net/}
   \item \altlink{http://www.linuxcommand.org/superman_pages.php}
   \end{itemize}
\end{itemize}


\subsection{Structure}

    Dans le monde Unix, il est essentiel de démarrer le script avec le nom de
    l'interpréteur de commande, à la première ligne vous devez indiquer~:

\begin{example}
\begin{verbatim}
      #!/bin/sh
\end{verbatim}
\end{example}

    Pour que l'on puisse identifier plus facilement le script, à savoir, à
    quoi il sert, qui l'a écrit, il faut maintenant que celui-ci soit suivi
    d'une en-tête à peu près comme ceci~:

\begin{example}
\begin{verbatim}
      #--------------------------------------------------------------------
      # /etc/rc.d/rc500.dummy - start my cool dummy server
      #
      # Creation:     19.07.2001  Toller Hecht <toller-hecht@example.net>
      # Last Update:  11.11.2001  Süße Maus <suesse-maus@example.net>
      #--------------------------------------------------------------------
\end{verbatim}
\end{example}

    Vous pouvez maintenant poursuivre le script ...


\marklabel{dev:sec:config-variables}{
\subsection{Gestion des variables de configuration}
}

\subsubsection{Généralités}

    La composition de la configuration de fli4l est dans le fichier
    \texttt{config/<PAQUETAGE>.txt}. 
    Cette documentation contient les \jump{subsec:dev:var-check}{Variables actives}
	et la création du support de boot avec le fichier \texttt{rc.cfg}. Lors du boot
	du routeur, ce fichier est lu avant tous les scripts-rc (les scripts sont dans
	\texttt{/etc/rc.d/}). Ce script peut accéder avec le \var{\$<nom de variable>} à
	toutes les variables de configuration du routeur.

    Avez-vous besoin des valeurs des variables de configuration, même après
	le boot~? Vous pouvez à partir du fichier \texttt{/etc/rc.cfg}, avec lequel
	vous avez écrit la configuration pour le support de boot. Par exemple, vous
	pouvez lire la valeur de la variable \texttt{OPT\_DNS}, avec un script, vous
	devez le faire de la manière suivante~:

\begin{example}
\begin{verbatim}
    eval $(grep "^OPT_DNS=" /etc/rc.cfg)
\end{verbatim}
\end{example}

    Cela fonctionne également avec plusieurs variables (c'est-à-dire en
	utilisant une seul fois le programme \texttt{grep})~:

\begin{example}
\begin{verbatim}
    eval $(grep "^\(HOSTNAME\|DOMAIN_NAME\|OPT_DNS\|DNS_LISTEN_N\)=" /etc/rc.cfg)
\end{verbatim}
\end{example}


\marklabel{dev:sec:persistent-data}{
\subsubsection{Stockage persistant des données}
}

Les paquetages on parfois besoins de stocker des données sur un support
persistant, ils pourront survivre au redémarrage du routeur. Il existe une
fonction \texttt{map2persistent}, elle peut être utilisée depuis un script
qui sera enregistré dans \texttt{/etc/rc.d/}. Ce script doit contenir
la variable avec le chemin et le sous-répertoire. L'idée est que la variable
est configurée avec un chemin réel~-- Alors, ce chemin sera utilisé, car l'utilisateur
l'a souhaité ou la variable sera configurée sur \og{}auto\fg{}~-- Alors, un
sous-répertoire correspondant sera créé en-dessous du répertoire sur un support
persistant selon le deuxième paramètre. La fonction retourne le résultat de la variable,
avec le nom qui à été indiqué dans le premier paramètre.

Un exemple permettra de clarifier cela. Soit la variable \var{VBOX\_SPOOLPATH}, qui
est paramétrée avec un chemin ou avec la valeur \og{}auto\fg{}. Pour l'activation

\begin{example}
\begin{verbatim}
    begin_script VBOX "Configuring vbox ..."
    [...]
    map2persistent VBOX_SPOOLPATH /spool
    [...]
    end_script
\end{verbatim}
\end{example}

Cela signifie que la variable \var{VBOX\_SPOOLPATH}, ne sera pas modifié
(si elle contient un chemin) ou le chemin sera remplacé par
\texttt{/var/lib/persistent/vbox/spool} (Si elle contient la valeur \og{}auto\fg{}).
La valeur se réfère \footnote{à l'aide d'un soi-disant \og{}lien\fg{} monté}
\texttt{/var/lib/persistent} est le répertoire pour enregistrer les données sur
un support de stockage non volatile, \var{<SCRIPT>} représente un minuscule script
d'exécution (ce nom est dérivé du premier argument
\jump{subsec:dev:bug-searching}{exécute \texttt{begin\_script}}). Si aucun support
convenable n'existe (cela peut être possible), le répertoire \texttt{/var/lib/persistent}
sera enregistré dans le disque RAM.

Il convient de noter, que le chemin utilisé par \texttt{map2persistent} n'est \emph{pas}
généré automatiquement~-- Cela doit être fait soi-même (peut-être avec la commande
\texttt{mkdir -p <chemin>}).

Dans le fichier \texttt{/var/run/persistent.conf} vous pouvez vérifier si le support de
stockage persistant pour les données est possible. Exemple~:

\begin{example}
\begin{verbatim}
    . /var/run/persistent.conf
    case $SAVETYPE in
    persistent)
        echo "Stockage persistant possible!"
        ;;
    transient)
        echo "Stockage persistant PAS possible!"
        ;;
    esac
\end{verbatim}
\end{example}


\marklabel{subsec:dev:bug-searching}{
\subsection{Recherche d'erreur}
}

    Au démarrage d'un script, il est souvent utile d’activer le mode débogage,
    pour déterminer si \og{}un ver est dans le script\fg{} et pour savoir s'il est inséré
    au début ou à la fin du texte~:

\begin{example}
\begin{verbatim}
      begin_script <OPT-Name> "start message"
      <script code>
      end_script
\end{verbatim}
\end{example}

En fonctionnement normal, un texte apparait au démarrage du script et à la fin
de se même texte le préfixe \og{}finished\fg{} sera spécifié.

Si vous voulez déboguer un script, vous devez faire deux choses~:

\begin{enumerate}
\item Il faut mettre \jump{DEBUGSTARTUP}{\var{DEBUG\_\-STARTUP}} sur \og{}yes\fg{}.
\item Vous devez activer le débogage de l'OPT choisi. On le fait en général
   par la variable suivante dans les fichiers de configurations~:\footnote{parfois,
   plusieurs scripts de démarrage sont utilisés pour chaque variable de débogage,
   ces variables ont des noms différents pour le débogage. Voici un rapide coup d'oeil
   sur ces scripts.}
\begin{example}
\begin{verbatim}
      <OPT-Name>_DO_DEBUG='yes'
\end{verbatim}
\end{example}

Maintenant vous pourrez voir sur la console, la représentation exact de l'exécution
du programme.
\end{enumerate}


\subsubsection{D'autres variables pour le débogage}

\begin{description}
  \config{DEBUG\_ENABLE\_CORE}{DEBUG\_ENABLE\_CORE}{DEBUGENABLECORE}

  Si cette variable est placée sur \og{}yes\fg{}, elle permet de créer un core-Dumps
  (ou image mémoire). Si un programme se bloque en raison d'une erreur, un
  fichier image enregistre l'état actuel du système, il pourra ensuite être
  utilisée pour une analyser du problème. L'image core-Dumps sera enregistrée
  dans le dossier \texttt{/var/log/dumps}.

  \config{DEBUG\_IP}{DEBUG\_IP}{DEBUGIP}

  Si cette variable est placée sur \og{}yes\fg{}, tous les appels du programme par
  le protocol \texttt{ip} seront enregistrés.

  \config{DEBUG\_IPUP}{DEBUG\_IPUP}{DEBUGIPUP}

  Si cette variable est placée sur \og{}yes\fg{}, lors de l'exécution des scripts
  \texttt{ip-up}/\texttt{ip-down} les instructions exécutées seront stockées dans
  le système de journalisation.

  \config{LOG\_BOOT\_SEQ}{LOG\_BOOT\_SEQ}{LOGBOOTSEQ}

  Si cette variable est placée sur \og{}yes\fg{}, elle enregistre dans \texttt{bootlogd}
  tous le processus de Boot visible sur la console. Cette variable est placée
  par défaut sur \og{}yes\fg{}.

  \config{DEBUG\_KEEP\_BOOTLOGD}{DEBUG\_KEEP\_BOOTLOGD}{DEBUGKEEPBOOTLOGD}

  Normalement \texttt{bootlogd} se termine à la fin du processus de boot. Si on active
  cette variable, cela permet l'enregistrement au-delà de l'arrêt du processus
  de boot visible sur la console.

  \config{DEBUG\_MDEV}{DEBUG\_MDEV}{DEBUGMDEV}

  Si on active cette variable cela génère le protocole Démons-\texttt{mdev} et produit
  un fichier sur tous les périphériques dans le dossier \texttt{/dev}
\end{description}


\subsection{Remarques}

\begin{itemize}
\item Il est \emph{toujours} préférable d'utiliser les accolades \og{}\{\ldots\}\fg{} à
   la place des parenthèses \og{}(\ldots)\fg{}. Il convient après l’ouverture de l’accolade
   de placer, un espace ou un saut de ligne avant la prochaine commande et de placer avant
   la fermeture de l’accolade un point-virgule ou un nouveau saut de ligne. Par exemple~:

\begin{example}
\begin{verbatim}
        { echo "cpu"; echo "quit"; } | ...
\end{verbatim}
\end{example}

        Équivaut à~:

\begin{example}
\begin{verbatim}
        {
                echo "cpu"
                echo "quit"
        } | ...
\end{verbatim}
\end{example}

      \item Un script peut être arrêté prématurément avec la commande \og{}exit\fg{}.
        Mais pour le Script de démarrage (\texttt{opt/etc/boot.d/...}, \texttt{opt/etc/rc.d/...}),
        le Script d’arrêt (\texttt{opt/etc/rc0.d/...}) et les Scripts \texttt{ip-up}/\texttt{ip-down}
		avec (opt/etc/ppp/*) cela est carrément mortelle, il faut dire aussi
        que ces Scripts ne seront plus exécuté. En cas de doute, ne toucher
        à rien.

      \item KISS - Keep it small and simple. Vous voulez utiliser Perl en tant
        que langage script~? Les possibilités d’écriture pour fli4l ne te
        suffisent pas~? Penser à votre installation~! Votre OPT en a vraiment
        besoin~? fli4l est toujours \og{}uniquement\fg{} un routeur, un routeur
		ne doit pas offrir de services, comme un serveur.

      \item Le message d'erreur~: \og{}not found\fg{} signifie le plus souvent
		que le script est encore au format-DOS. Autre problème~: si le script n'est
		pas exécuté.
        Dans les deux cas, vous devez vérifier le fichier \texttt{opt/<PACKAGE>.txt},
		pour voir si les options sont correctes (par rapport au \og{}mode\fg{}, \og{}gid\fg{},
		\og{}uid\fg{} et Flags). Si le script est produit seulement au démarrage du
		routeur, vous devez exécuter la commande \og{}chmod +x <nom du script>\fg{}.

      \item Pour les fichiers temporaires, vous devez utiliser le chemin \texttt{/tmp}.
        Mais il est essentiel de veiller à ne pas utiliser trop d’espace, parce que
		le dossier est dans un Ramdisk-RootFS~! Si vous avez besoin de plus d'espace,
		il faut créer un Ramdisk et le monter. L’ensemble des détails à ce sujet se
		trouvent dans le paragraphe "RAM-Disks" de la documentation Dev.

      \item Afin que les fichiers temporaires obtiennent un nom unique, vous devez
		ajouter un ID de processus actuelle, dans la variable du Shell le caractère
		\og{}\$\fg{} sera ajouté au nom du fichier. \texttt{/tmp/<OPT-Name>.\$\$} et le nom
		du fichier sera correct, mais \texttt{/tmp/<OPT-Name>} est plutôt moins, bien sûr
		\texttt{<OPT-Name>} ne doit pas laissé comme ceci, mais vous devez le modifier en fonction.

\end{itemize}

