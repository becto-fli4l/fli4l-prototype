% Synchronized to r29817

\section {CHRONY - Network Time Protocol Server/Client}

OPT\_CHRONY extends fli4l with the \jump{url:chronyntporg}{Network Time
Protocol} (NTP). Don't mix it up with the \emph{normal} Time Protocol,
the old OPT\_TIME provides. The protocols aren't compatible wich possibly raises the need 
of new client-side programs, which understand NTP. If you can't abandon the
simple time protocol, chrony can provide it as well.
OPT\_CHRONY works both as a server and client. Working as a client, OPT\_CHRONY
adjusts the time of the fli4l according to time referencies (time servers) in
the internet. The basic setting uses up to three time servers from 
\jump{url:chronypoolntporg}{pool.ntp.org}. However it's possible to use a different 
selection of time servers in the configuration file. Thus it's possible to use time servers
near you, if you choose de.pool.ntp.org if the router or your provider sits in Germany.
For more details, look at the website of \jump{url:chronypoolntporg}{pool.ntp.org}.\\

Working as a client, OPT\_CHRONY acts as a time reference for the local network (LAN). 
NTP uses port 123.

Chrony doesn't need a permanent connection to the internet. When the connection drops,
chrony get's a notice and ceases the adjustion with the internet time servers. Neither 
does chrony dial to raise the connection nor prevent the automatic hangup, if
\var{HUP\_TIMEOUT}, the duration where no data is routet to the internet, is reached.

In order to do time adjustments smoothly, the following should be in mind:
\begin{itemize}
  \item Chrony expects the BIOS-clock to be in UTC timezone. \\
        UTC = German time minus 1 (winter) or 2 (summer) hour(s)
  \item Since version 2.1.12 chrony sets the time during the first connection
        to the internet correct even if the time difference is huge (caused by an 
	empty mainboard battery for example). 
  \item If the BIOS isn't capable of handling years after 1999 (Year 2000 Bug) or
        if the BIOS clock is faulty, activate \jump{OPTY2K}{\var{OPT\_Y2K='yes'}}!
\end{itemize}

Only time servers in the internet which are reachable by the default route (\var{0.0.0.0/0})
can be used, because only the default route changes chrony into online mode.
As an ethernet router with no DSL or ISDN circuits configured, chrony acts permanently
in online mode.

\textbf{Disclaimer: }\emph{The author gives neither a guarantee of functionality nor is he liable 
for any damage or the loss of data when using OPT\_\-CHRONY.}


\marklabel{sec:konfigchrony}{
\subsection {Configuration of OPT\_CHRONY}
}

The configuration is made, as for all fli4l packages, by adjusting the file\\
\var{path/fli4l-\version/$<$config$>$/chrony.txt} to meet the own demands.
However almost all variables of OPT\_\-CHRONY are optional. Optional means, the
variables could, but need not be in the config file.
Thus the chrony config file is almost empty and all optional 
variables have a usefull setting by default.
To use different settings, the variables must be inserted into the config file 
by hand. Furthermore the description of every variable follows:


\begin{description}

\config {OPT\_CHRONY}{OPT\_CHRONY}{OPTCHRONY}

  Default: \var{OPT\_CHRONY='no'}

  The setting \var{'no'} deactivates OPT\_CHRONY completely. There will be no changes
  made on the fli4l boot medium or the archive \var{opt.img}.
  Further OPT\_CHRONY basically does not overwrite other parts of the fli4l installation
  with one exception. The paket filter file which is responsible for not counting the 
  traffic coming from the outside (to make sure fli4l will drop the line for sure, 
  when reaching the hangup time), will be exchanged. The new paket filter file ensures, 
  that chrony-traffic doesn't count for the hangup time.\\
  To activate OPT\_CHRONY, the variable \var{OPT\_CHRONY} has to be set to \var{'yes'}. 

  
\config {CHRONY\_TIMESERVICE}{CHRONY\_TIMESERVICE}{CHRONYTIMESERVICE}

  Default: \var{CHRONY\_TIMESERVICE='no'}

  With \var{CHRONY\_TIMESERVICE} an additional protocol to transfer time informations
  can be invoked. It's only neccesary when the local hosts can't deal with NTP. It is
  conform to RFC 868 and uses port 37. If possible, prefer NTP.

  Many thanks to Christoph Schulz, who provided the program \texttt{srv868}.


\config {CHRONY\_TIMESERVER\_N}{CHRONY\_TIMESERVER\_N}{CHRONYTIMESERVERN}

  Default: \var{CHRONY\_TIMESERVER\_N='3'}

  \var{CHRONY\_TIMESERVER\_N} sets the number of reference time servers chrony uses.
  Add the same number of \var{CHRONY\_TIMESERVER\_x} variables. The index \var{x} 
  must be increased up to the total number.\\
  The basic setting uses up to three time servers from
  \jump{url:chronypoolntporg}{pool.ntp.org}.
  

\config {CHRONY\_TIMESERVER\_x}{CHRONY\_TIMESERVER\_x}{CHRONYTIMESERVERx}

  Default: \var{CHRONY\_TIMESERVER\_x='pool.ntp.org'}
  
  With \var{CHRONY\_TIMESERVER\_x} an own list of internet time servers can be
  made. Either IP or DNS hostnames are possible.

\config {CHRONY\_LOG}{CHRONY\_LOG}{CHRONYLOG}

  Default: \var{CHRONY\_LOG='/var/log/chrony'}

  \var{CHRONY\_LOG} is the folder where chrony stores NTP measurements and
  information about applied time adjustments.

\end{description}
  
\marklabel{sec:chronysupport}{
\subsection{Support}
}
Support is only given in the \jump{url:chronyfli4lnews}{fli4l Newsgroups}. 
\subsection{Literature}

\marklabel{url:chronysource}{
 Homepage of chrony: \altlink{http://chrony.tuxfamily.org/}
 }

\marklabel{url:chronyntporg}{
 NTP: The Network Time Protocol: \altlink{http://www.ntp.org/}
 }

\marklabel{url:chronypoolntporg}{
 pool.ntp.org: public ntp time server for everyone: \altlink{http://www.pool.ntp.org/en/}
 }

\marklabel{url:rfc1305}{
 RFC 1305 - Network Time Protocol (Version 3) Specification, Implementation:\\
 \indent\altlink{http://www.faqs.org/rfcs/rfc1305.html}
 }

\marklabel{url:chronyfli4lnews}{
 fli4l Newsgroups and the rules: \altlink{http://www.fli4l.de/hilfe/newsgruppen/}
}
