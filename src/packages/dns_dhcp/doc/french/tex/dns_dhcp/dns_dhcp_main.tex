% Do not remove the next line
% Synchronized to r55089

\marklabel{sec:dnsdhcp}
{
\section{DNS\_DHCP - Serveur DNS et DHCP - Relay DHCP et serveur DNS esclave}
}

\subsection{Nom d'hôte}
\subsubsection{Hôte}
\begin{description}
    \config{OPT\_HOSTS}{OPT\_HOSTS}{OPTHOSTS}

    Avec la variable optionnelle \var{opt\_HOSTS}, la configuration des noms
	d'hôtes peut être désactivée~!

    \configlabel{HOST\_x\_NAME}{HOSTxNAME}
    \configlabel{HOST\_x\_IP4}{HOSTxIP4}
    \configlabel{HOST\_x\_IP6}{HOSTxIP6}
    \configlabel{HOST\_x\_DOMAIN}{HOSTxDOMAIN}
    \configlabel{HOST\_x\_ALIAS\_N}{HOSTxALIASN}
    \configlabel{HOST\_x\_ALIAS\_x}{HOSTxALIASx}
    \configlabel{HOST\_x\_MAC}{HOSTxMAC}
    \configlabel{HOST\_x\_MAC2}{HOSTxMAC2}
    \configlabel{HOST\_x\_DHCPTYP}{HOSTxDHCPTYP}
    \config{HOST\_N HOST\_x\_\{attribute\}}{HOST\_N}{HOSTN}

      {Tous les ordinateurs du réseau local doivent être enregistrés
      avec une adresse IP, un nom, un alias (ou pseudo) et éventuellement
      une adresse MAC pour la configuration du dhcpd. Pour ce faire,
      on indique d'abord le nombre d'ordinateur dans la variable \var{HOST\_\-N}.

     \textbf{Remarque~:} depuis la version 3.4.0, l'enregistrement des
     informations du routeur sont générées dans le fichier \var{$<$config$>$/base.txt}.
     Des alias supplémentaires peuvent être ajoutés dans celui-ci, voir la
     variable \jump{HOSTNAMEALIASN}{\var{HOSTNAME\_ALIAS\_N}}.

      Ensuite on définit les caractéristiques de chaque hôte dans ces variables.
      Certain paramètres sont obligatoires comme par ex. l'adresse IP, le nom et
      d'autres sont optionnels, c.-à-d. qu'ils ne sont pas obligatoires.

    \begin{description}
      \item[NAME]         -- Nom d'hôte par n-fois
      \item[IP4]          -- Adresse IP (ipv4) de l'hôte par n-fois
      \item[IP6]          -- Adresse IP (ipv6) de l'hôte par n-fois (optionnelle)
                             Si vous indiquez 'auto', l'adresse sera automatiquement
                             composée du préfixe IPv6 (avec le masque de
                             sous-réseau /64) et l'adresse MAC correspondant
                             à l'hôte si vous avez activé OPT\_IPV6. Pour que
                             cela fonctionne, vous devrez configurer
                             \var{HOST\_x\_MAC} (voir ci-dessous) et configurer
                             le paquetage \texttt{ipv6}.
      \item[DOMAIN]       -- Domaine DNS de l'hôte par n-fois (optionnelle)
      \item[ALIAS\_N]     -- Nombre d'alias (ou pseudo)
      \item[ALIAS\_m]     -- m-fois le nom d'alias de l'hôte par n-fois
      \item[MAC]          -- Adresse MAC de l'hôte par n-fois
      \item[MAC2]         -- Adresse MAC pour une autre interface de l'hôte par n-fois
      \item[DHCPTYP]      -- Attribution de l'adresse IP par DHCP en fonction
                            de l'adresse MAC ou du Nom (optionnelle)
    \end{description}

      Dans l'exemple du fichier dns\_dhcp.txt, 4 ordinateurs sont configurés~-
      Pour les PCs "client1", "client2", "client3" et "client4".

\begin{example}
\begin{verbatim}
         HOST_1_NAME='client1'                # 1st host: ip and name
         HOST_1_IP4='192.168.6.1'
\end{verbatim}
\end{example}

      Les noms d'alias doivent être compatibles avec le nom complet du domaine
      spécifié dans fli4l.

      L'adresse MAC est optionnelle, elle est pertinente que si fli4l utilise
      un serveur DHCP. Pour cela vous devez voir la description des options
      de la variable "\var{OPT\_\-DHCP}" indiqué plus bas dans ce document. Si
      vous n'utilisez pas de serveur DHCP, vous pouvez juste indiquer l'adresse IP,
      le nom de l'ordinateur et peut-être un alias. L'adresse MAC à un adressage
      de 48 bits et se compose de 6 hexadécimales séparées par deux points.

      Exemple~:

\begin{example}
\begin{verbatim}
        HOST_2_MAC='de:ad:af:fe:07:19'
\end{verbatim}
\end{example}

      \emph{Remarque~:} si vous ajoutez à fli4l le paquetage IPv6, il n'y a pas
      besoin d'indiquer les adresses IPv6, si l'adresse MAC est présente dans
      la configuration, le paquetage IPv6 créera automatiquement les adresses
      IPv6 (EUI-64 modifié) en utilisant l'adresse MAC. Bien sûr, vous n'êtes
      pas obligé d'utiliser l'adressage automatique, vous pouvez indiquer
      les adresses IPv6 manuellement sur l'hôte, si vous le souhaitez.
      }
\end{description}


\subsubsection{Extra hôte}

\begin{description}
  \configlabel{HOST\_EXTRA\_x\_NAME}{HOSTEXTRAxNAME}
  \configlabel{HOST\_EXTRA\_x\_IP4}{HOSTEXTRAxIP4}
  \configlabel{HOST\_EXTRA\_x\_IP6}{HOSTEXTRAxIP6}
  \config{HOST\_EXTRA\_N HOST\_EXTRA\_x\_NAME HOST\_EXTRA\_x\_IP4 HOST\_EXTRA\_x\_IP6}{HOST\_EXTRA\_N}{HOSTEXTRAN}

  {
      Avec ces variables, vous pouvez ajouter d'autres hôtes qui
      n'appartiennent pas au domaine local, pas exemple des hôtes qui
      se trouvent sur un autre domaine à travers une connexion VPN
  }
\end{description}


\subsection{Serveur DNS}

\begin{description}
    \config{OPT\_DNS}{OPT\_DNS}{OPTDNS}

    {Pour activer le serveur DNS vous devez paramétrer la variable \var{OPT\_DNS}
    sur 'yes'.

    Si aucun ordinateur Windows n'est utilisé dans le LAN (ou réseau local), ou
    si un serveur DNS est déjà disponible dans le LAN, vous pouvez mettre la
    variable \var{OPT\_DNS} sur 'no' et ignorer le reste de ce paragraphe.

    Dans le doute, toujours utiliser la (configuration par défaut)~: \var{OPT\_DNS}='yes'}
\end{description}


\subsubsection{Option générale pour serveur DNS}

\begin{description}
    \configlabel{DNS\_LISTEN\_x}{DNSLISTENx}
    \config{DNS\_LISTEN\_N DNS\_LISTEN\_x}{DNS\_LISTEN\_N}{DNSLISTENN}

      {Si vous avez choisi d'activer la variable \var{OPT\_DNS}='yes', vous
      pouvez indiquer dans la variable \var{DNS\_LISTEN\_N} le nombre de
      variable à configurer et indiquez dans la variable \var{DNS\_LISTENIP\_1}
      l'adresse IP locale sur laquelle, les requêtes DNS utiliserons
      le programme \verb+Dnsmasq+. Si vous paramétrez la variable \var{DNS\_LISTEN\_N}
      sur '0' toutes les requêtes DNS des adresses IP locaux utiliserons
      le programme \verb+Dnsmasq+.

      À ce stade, seul les adresses IP des interfaces existantes (ethernet,
      wlan ...) peuvent être utilisées, sinon vous aurez un message d'avertissement
      au démarrage du routeur. Il est désormais possible d'utiliser alternativement
      les alias, par exemple \verb+IP_NET_1_IPADDR+.

      Pour toutes les adresses indiquées, les règles ACCEPT de la chaîne INPUT
      seront créées pour le pare-feu si \var{PF\_INPUT\_ACCEPT\_DEF='yes'}
      et/ou \var{PF6\_INPUT\_ACCEPT\_DEF='yes'} sont activées. Si la variable \var{DNS\_LISTEN='0'}
	  est à zéro, les règles qui permettent l'accès au DNS seront également générées pour
	  \emph{toutes} les interfaces configurées.

      \wichtig{Si vous souhaitez que le serveur DNS écoute les interfaces configurées
	  dynamiquement à l'installation, par exemple, une interface réseau pour un tunnel
	  VPN. Vous ne devez pas configurer cette liste de variables, il faut les laisser vide.
	  Sinon le serveur DNS ne répondra pas aux requêtes DNS effectuées via le tunnel VPN.}

      En cas de doute, vous pouvez utiliser les paramètres par défaut.}

    \config{DNS\_BIND\_INTERFACES}{DNS\_BIND\_INTERFACES}{DNSBINDINTERFACES}{
      Si vous voulez que le serveur DNS écouter uniquement les adresses configurées via
	  la variable \var{DNS\_LISTEN\_x} \emph{et} si vous voulez un serveur DNS
	  \emph{supplémentaire} pour écouter les \emph{autres} adresses du réseau. Avec cette
	  option, vous demandez au serveur DNS d'écouter uniquement les adresses assignées.
	  Par défaut, le serveur DNS écoute \emph{toutes} les interfaces et envoie les requêtes
	  DNS qui arrivent par les adresses qu'ils ne sont pas configurées dans la liste de variables
	  \var{DNS\_LISTEN\_x}. Cette option a un avantage, c'est que le serveur DNS peut aussi traiter
	  les interfaces configurées dynamiquement et un inconvénient c'est qu'aucun serveur DNS
	  alternatif peut fonctionner simultanément en utilisant le port 53 du DNS standard. Si
	  vous voulez utiliser le serveur DNS et si vous exécutez un deuxième serveur DNS esclave
	  comme "yadifa" directement sur le routeur fli4l, vous devez savoir que le serveur
	  Dnsmasq ne sera pas exclusivement utilisé par fli4l. Vous devez sélectionner
	  le paramètre 'yes' et configurer les adresses IP dans la variable \var{DNS\_LISTEN} pour
	  pouvoir utiliser le serveur Dnsmasq.}

    \config{DNS\_VERBOSE}{DNS\_VERBOSE}{DNSVERBOSE}

      {Enregistrement des requêtes DNS~: 'yes' ou 'no'

      Si vous voulez avoir des détails sur les requêtes DNS, mettez la variable
      \var{DNS\_VERBOSE} sur 'yes'. Dans ce cas, les messages des requêtes DNS
      seront consignés sur le serveur de nom~- A savoir sur l'interface syslog.
      Si vous voulez rendre visible et lire ce fichier jounal, vous devez activer
      la variable \jump{OPTSYSLOGD}{\var{OPT\_SYSLOGD='yes'}}, voir ci-dessous.}

    \config{DNS\_MX\_SERVER}{DNS\_MX\_SERVER}{DNSMXSERVER}

      {Avec cette variable, vous pouvez enregistrer un nom d'hôte pour le MX
      (Mail-Exchanger) et pour définir dans \var{DOMAIN\_NAME} un domaine. Si
      un MTA (Mail"=Transport"=Agent, par exemple sendmail) est installé sur le
      serveur interne, une demande de DNS sera faite par Mail-Exchanger, l'objectif
      est d'envoyer un mail au domaine recherché. Le serveur DNS fournit ici
      l'hôte au MTA, pour l'envoi du mail au \var{DOMAIN\_NAME} du domaine compétent.

      \achtung{ Il ne s'agit pas de configuration automatique un clients de
                messagerie, comme par exemple Outlook~! Veuillez ne pas enregistrer
                votre adresse mail ici, après ne vous étonnez pas si Outlook ne
                fonctionne pas.}
      }

    \configlabel{DNS\_FORBIDDEN\_x}{DNSFORBIDDENx}
    \config{DNS\_FORBIDDEN\_N DNS\_FORBIDDEN\_x}{DNS\_FORBIDDEN\_N}{DNSFORBIDDENN}

    {Avec ces variables, vous pouvez paramétrer les domaines pour lesquels les
    requêtes DNS n'auront "pas accés" au serveur DNS, ils n'auront aucune réponse.

      Exemple:

\begin{example}
\begin{verbatim}
        DNS_FORBIDDEN_N='1'
        DNS_FORBIDDEN_1='foo.bar'
\end{verbatim}
\end{example}

      Dans cet exemple une demande d'accés au domaine www.foo.bar répondra
      par une erreur.

      On peut aussi interdire un Top-Level-Domains (ou domaines de premier niveau)
      les plus connus sont .com, .fr, .de~:

\begin{example}
\begin{verbatim}
        DNS_FORBIDDEN_1='de'
\end{verbatim}
\end{example}

      Dans cet exemple, la résolution de nom pour 'de' du Top-level-domain sur
      le Web sera supprimé pour tous les ordinateurs du réseau local.}

    \configlabel{DNS\_REDIRECT\_x}{DNSREDIRECTx}
    \configlabel{DNS\_REDIRECT\_x\_IP}{DNSREDIRECTxIP}
    \config{DNS\_REDIRECT\_N DNS\_REDIRECT\_x DNS\_REDIRECT\_x\_IP}{DNS\_REDIRECT\_N}{DNSREDIRECTN}

    {Avec ces variables, vous pouvez spécifier les domaines sur lesquels les
    requêtes DNS seront redirigées vers une autre adresse IP du serveur DNS.

      Exemple~:

\begin{example}
\begin{verbatim}
        DNS_REDIRECT_N='1'
        DNS_REDIRECT_1='yourdom.dyndns.org'
        DNS_REDIRECT_1_IP='192.168.6.200'
\end{verbatim}
\end{example}

      Dans cet exemple, une demande d'accés au domaine yourdom.dyndns.org sera
      dérivé vers l'adresse IP 192.168.6.200. Ainsi vous pouvez dériver les
      domaines externe de votre choix, sur une adresse IP local de votre choix.}

     \config{DNS\_BOGUS\_PRIV}{DNS\_BOGUS\_PRIV}{DNSBOGUSPRIV}

      {Si vous placez cette variable sur 'yes' vous ne transmettez pas les
      recherches inversées pour les adresses IP RFC1918 (classe d'adresse IP
      privée, non routables sur Internet) ces adresses ne seront pas transmis
      au serveur DNS, mais le Dnsmasq répondra.}

     \config{DNS\_FORWARD\_PRIV\_x}{DNS\_FORWARD\_PRIV\_x}{DNSFORWARDPRIVx}

     Si vous avez besoin de transmettre la résolution d'adresse de certain sous-réseau
	 privés, malgré la configuration de la variable \var{DNS\_BOGUS\_PRIV} pour le serveur DNS.
	 La transmission est nécessaire, par exemple si le routeur gère une connexion montante pour
	 certain sous-réseaux privés. Cette ensemble de variables peut être utilisés pour définir
	 les sous-réseaux privés, ainsi, la résolution d'adresse sera transmise.

     \config{DNS\_FILTERWIN2K}{DNS\_FILTERWIN2K}{DNSFILTERWIN2K}

     {Si vous placez cette variable sur 'yes' les requêtes DNS du type SOA, SRV
     et ANY seront bloquées. Les services qui utilisent ce type de requète ne
     fonctionneront plus sans une configuration supplémentaire.\hfil\break
     Par exemple~:
     \begin{itemize}
     \item XMPP (Jabber)
     \item SIP
     \item LDAP
     \item Kerberos
     \item Teamspeak3 (à partir de la version du client 3.0.8)
     \item Minecraft (à partir de toutes les versions 1.3.1)
     \item Recherche pour la gestion du contrôleur de domaine (Win2k)
     \end{itemize}
     Pour plus d'informations~:}
     \begin{itemize}
     \item Voir les explications des types de requète DNS à cette adresse~:\hfil\break
           \altlink{http://en.wikipedia.org/wiki/List_of_DNS_record_types}
     \item Voir le manuel du Dnsmasq à cette adresse~:\hfil\break
           \altlink{http://www.thekelleys.org.uk/dnsmasq/docs/dnsmasq-man.html}
     \item Voir les requêtes SRV dans le détail à cette adresse~:\hfil\break
           \altlink{http://en.wikipedia.org/wiki/SRV_resource_record} 
     \end{itemize}
     \achtung{Si vous avez indiqué le paramètre 'no' en plus des problèmes de
     transmission des requêtes DNS, cela peut aussi provoquer des connexions
     indésirables ou empêcher la fermeture d'une connexion déjà existante.
     Surtout si vous utilisez l'ISDN (ou Numéris) ou l'UMTS, des coûts de connexion
     supplémentaires peuvent survenir. Vous devez choisir ce qui est le plus
     important pour vous.}

     \config{DNS\_FORWARD\_LOCAL}{DNS\_FORWARD\_LOCAL}{DNSFORDWARDLOCAL}

      {Si vous placez cette variable sur 'yes' le routeur fli4l peut être configuré
      dans un domaine avec la variable DOMAIN\_NAME='example.local', et via la
      variable DNS\_ZONE\_DELEGATION\_x\_DOMAIN='example.local'} les requêtes
      seront résolues à partir d'un autre serveur de nom.

     \config{DNS\_LOCAL\_HOST\_CACHE\_TTL}{DNS\_LOCAL\_HOST\_CACHE\_TTL}{DNSLOCALHOSTCACHETTL}

      {Vous indiquez dans cette variable le TTL (Time To Live, en seconde) pour
      les noms enregistrés dans le fichier /etc/hosts et pour les adresses IP
      affectés par le DHCP. La valeur par défaut pour fli4l est de 60 secondes.
      La valeur par défaut du TTL pour les noms enregistrés dans le Dnsmasq locale
      doit être de 0, en fait la mise en cache des entrées DNS sera désactivée.
      L'idée, est que l'exécution des baux du DHCP, etc. pourront être transmis
      rapidement. Exemple d'un Proxy IMAP local, il peut demander plusieurs fois
      par seconde un nom enregistré dans le DNS, cela occasionne des lourdes charges
      sur le réseau. Le compromis est donc un TTL relativement court, avec
      60 secondes. Il peut même fonctionner sans ce court TTL de 60 secondes,
      avec une simple mise hors tension à tous moment de l'hôte, de sorte que le
      logiciel qui interroge ne traitera pas de toute façon la réponce de l'hote.}

     \config{DNS\_SUPPORT\_IPV6}{DNS\_SUPPORT\_IPV6}{DNSSUPPORTIPV6} (optionelle)

      {Si vous placez cette variable sur 'yes' vous activez le serveur DNS
      supportant l'adressage IPv6.}
\end{description}

\subsubsection{Configuration d'une zone DNS}

      Le Dnsmasq peut également gérer un domaine DNS de manière autonome,
	  c'est à dire, il a "autorité" pour ce domaine. Par conséquent, il faut
	  faire deux choses~: la première consiste à spécifier le nom du service
	  DNS externe (!) sur fli4l pour l' envoie des requètes, la deuxième est
	  savoir sur quelle interface réseau que tous cela se passera. La spécification
	  du référencement externe est nécessaire, car le domaine qui gère fli4l,
	  est toujours un sous-domaine d'un autre domaine. \footnote{Nous allons
	  supposer que personne n'utilise fli4l comme serveur DNS à la racine...}
	  La spécification de l'interface "externe" est important parce que Dnsmasq
	  se comporte différemment par rapport à une autre interface "interne"~:
	  le Dnsmasq ne répond jamais aux requêtes de extérieur, en dehors de sa
	  propre configuration de nom de domaine. En interne le Dnsmasq fonctionne
	  naturellement comme un relay DNS pour que la résolution de noms qui n'est
	  pas sur le routeur, fonctionne sur Internet.

	  D'autre part, vous devez configurer le réseau pour que la résolution de nom
	  soit accessible vers l'extérieur. La configuration du réseaux ne doit être
	  spécifiée avec une adresses IP publiques, parce que les adresses des hôtes
	  privées, ne peuvent pas atteindre une IP publique qui est externe.

	  Ci-dessous, un exemple de configuration est décrite. Cet exemple suppose que
	  le paquet IPv6 ainsi que le préfixe IPv6 est routé vers le réseau publique.
	  Cette adresse peut par exemple, être fournis par le fournisseur de tunnel 6in4
	  comme Hurricane Electric.

\begin{description}

\config{DNS\_AUTHORITATIVE}{DNS\_AUTHORITATIVE}{DNSAUTHORITATIVE}

      Si vous activez la variable \verb+DNS_AUTHORITATIVE='yes'+, vous activez
	  le module Dnsmasq qui fait autorité. Toutefois, cela ne suffit pas, car
	  vous devez fournir plus d'information (voir ci-dessous).

	  Paramètre par défaut~:\verb+DNS_AUTHORITATIVE='no'+

	  Exemple~: \verb+DNS_AUTHORITATIVE='yes'+

\config{DNS\_AUTHORITATIVE\_NS}{DNS\_AUTHORITATIVE\_NS}{DNSAUTHORITATIVENS}

	  Dans cette variable vous configurez le nom du DNS externe pour fli4l, vous
	  indiquez ici le Nom de Domaine du DNS. Cela peut être un nom de DNS qui appartient
	  à un service de DNS Dynamique.

	  Exemple~: \verb+DNS_AUTHORITATIVE_NS='fli4l.noip.me'+

\config{DNS\_AUTHORITATIVE\_LISTEN}{DNS\_AUTHORITATIVE\_LISTEN}{DNSAUTHORITATIVELISTEN}

	  Dans cette variable vous configurez l'adresse ou l'interface, sur la quelle les
	  demandes de DNS du Dnsmasq doit répondre par le domaine qui fait autorité.
	  Les noms symboliques comme \verb+IP_NET_2_IPADDR+, \verb+IP_NET_1_DEV+ ou
    \verb+{LAN}+ sont autorisés. Le Dnsmasq peut répondre seulement à \emph{une}
    adresse/interface qui fait autorité.

	  \wichtig{Il faut faire attention à se que l'adresse/interface ne dépend jamais 
	  du réseau local, autrement aucun nom ne sera résolu dans le LAN~!}

	  Exemple~: \verb+DNS_AUTHORITATIVE_LISTEN='IP_NET_2_IPADDR'+

\configlabel{DNS\_ZONE\_NETWORK\_x}{DNSZONENETWORKx}
\config{DNS\_ZONE\_NETWORK\_N DNS\_ZONE\_NETWORK\_x}{DNS\_ZONE\_NETWORK\_N}{DNSZONENETWORKN}

	  Dans cette variable vous configurez l'adresse réseau, pour que le Dnsmasq qui fait
	  autorité puisse résoudre les noms. Il fonctionne à la fois en recherche normal
	  (le nom de l'adresse IP) ainsi qu'en recherche inversée (l'adresse IP du nom).

	  Un exemple complet~:

\begin{example}
\begin{verbatim}
        DNS_AUTHORITATIVE='yes'
        DNS_AUTHORITATIVE_NS='fli4l.noip.me'
        DNS_AUTHORITATIVE_LISTEN='IP_NET_2_IPADDR' # Uplink dépend de eth1
        DNS_ZONE_NETWORK_N='1'
        DNS_ZONE_NETWORK_1='2001:db8:11:22::/64'   # IPv6-LAN local
\end{verbatim}
\end{example}

	  Il est supposé que "2001:db8:11::/48" est le réseau public et sera routé vers le préfix IPv6
	  dans fli4l, et que 22 sous-réseau dans le LAN ont été sélectionnés.

\end{description}

\subsubsection{Délégation de zone DNS}

\begin{description}
    \configlabel{DNS\_ZONE\_DELEGATION\_x}{DNSZONEDELEGATIONx}
    \configlabel{DNS\_ZONE\_DELEGATION\_x\_UPSTREAM\_SERVER\_x}{DNSZONEDELEGATIONUPSTREAMSERVERx}
    \configlabel{DNS\_ZONE\_DELEGATION\_x\_UPSTREAM\_SERVER\_x\_IP}{DNSZONEDELEGATIONUPSTREAMSERVERxIP}
    \configlabel{DNS\_ZONE\_DELEGATION\_x\_UPSTREAM\_SERVER\_x\_quERYSOURCEIP}{DNSZONEDELEGATIONUPSTREAMSERVERxQUERYSOURCEIP}
    \configlabel{DNS\_ZONE\_DELEGATION\_x\_DOMAIN}{DNSZONEDELEGATIONxDOMAIN}
    \configlabel{DNS\_ZONE\_DELEGATION\_x\_NETWORK}{DNSZONEDELEGATIONxNETWORK}
    \config{DNS\_ZONE\_DELEGATION\_N DNS\_ZONE\_DELEGATION\_x}{DNS\_ZONE\_DELEGATION\_N}{DNSZONEDELEGATIONN}

      {Il y a des situations particulières, où le référencement d'un ou plusieurs
      serveurs DNS est utiliser, par exemple lorsque l'on utilise fli4l en
      Intranet sans connexion Internet ou un mélange des deux (un Intranet avec
      son propre serveur DNS et en plus une connexion Internet).

      Si nous imaginons le scénario suivant~:

      \begin{itemize}
      \item Circuit 1~: Avec une connexion Internet
      \item Circuit 2~: Avec une connexion à un réseau d'entreprises 192.168.1.0
          nom de domaine (firma.de)
      \end{itemize}

      Nous allons configurer \var{ISDN\_\-CIRC\_\-1\_\-ROUTE} sur '0.0.0.0' et
      \var{ISDN\_\-CIRC\_\-2\_\-ROUTE} sur '192.168.1.0'. Pour accéder aux
      ordinateurs avec l'adresse IP 192.168.1.x fli4l utilisera le circuit 2,
      autrement le circuit 1 sera utilisé. Si le réseau d'entreprise n'est pas
      public, il est possible de mettre en service un serveur DNS interne dans
      le réseau. Supposons, que l'adresse de ce serveur DNS est 192.168.1.12 et
      le nom de domaine est "firma.de".

      Vous devez alors paramétrer les variables suivantes~:

\begin{example}
\begin{verbatim}
        DNS_ZONE_DELEGATION_N='1'
        DNS_ZONE_DELEGATION_1_UPSTREAM_SERVER_N='1'
        DNS_ZONE_DELEGATION_1_UPSTREAM_SERVER_1_IP='192.168.1.12'
        DNS_ZONE_DELEGATION_1_DOMAIN_N='1'
        DNS_ZONE_DELEGATION_1_DOMAIN_1='firma.de'
\end{verbatim}
\end{example}

      Aprés cette configuration, les requêtes DNS seront envoyées au domaine
      firma.de ils utiliseront le serveur DNS interne de l'entreprise. Tous les
      autres requêtes DNS iront comme d'usage vers un serveur DNS sur Internet.

      Autre cas~:

      \begin{itemize}
      \item Circuit 1~: Internet
      \item Circuit 2~: Réseau d'entreprise 192.168.1.0 *avec* une connexion Internet
      \end{itemize}

      Ici vous avez deux possibilités d'accéder à Internet. Si vous souhaitez
      séparer le travail et la vie privée, vous pouvez alors paramètrer~:

\begin{example}
\begin{verbatim}
        ISDN_CIRC_1_ROUTE='0.0.0.0'
        ISDN_CIRC_2_ROUTE='0.0.0.0'
\end{verbatim}
\end{example}

      Vous définissez donc les deux circuits avec une route par défaut et vous
      commutez alors les circuits en utilisant le client-imond~- en fonction
      de la demande. Dans ce cas vous devez paramétrer les variables
      \var{DNS\_ZONE\_DELEGATION\_\-N} et \var{DNS\_ZONE\_DELEGATION\_x\_DOMAIN\_x}
      comme décrit ci-dessous.}

      Si vous voulez utiliser la résolution de DNS inversé sur votre réseau,
      par ex. faire une recherche inversée pour certains serveurs de messageries,
      vous pouvez indiquer dans la variable optionnelle \var{DNS\_ZONE\_DELEGATION\_x\_NETWORK\_x},
      le ou les réseaux (mis en oeuvres), cela active la recherche inversée.
      Voici un exemple~:

\begin{example}
\begin{verbatim}
        DNS_ZONE_DELEGATION_N='2'
        DNS_ZONE_DELEGATION_1_UPSTREAM_SERVER_N='1'
        DNS_ZONE_DELEGATION_1_UPSTREAM_SERVER_1_IP='192.168.1.12'
        DNS_ZONE_DELEGATION_1_DOMAIN_N='1'
        DNS_ZONE_DELEGATION_1_DOMAIN_1='firma.de'
        DNS_ZONE_DELEGATION_1_NETWORK_N='1'
        DNS_ZONE_DELEGATION_1_NETWORK_1='192.168.1.0/24'
        DNS_ZONE_DELEGATION_2_UPSTREAM_SERVER_N='1'
        DNS_ZONE_DELEGATION_2_UPSTREAM_SERVER_1_IP='192.168.2.12'
        DNS_ZONE_DELEGATION_2_DOMAIN_N='1'
        DNS_ZONE_DELEGATION_2_DOMAIN_1='bspfirma.de'
        DNS_ZONE_DELEGATION_2_NETWORK_N='2'
        DNS_ZONE_DELEGATION_2_NETWORK_1='192.168.2.0/24'
        DNS_ZONE_DELEGATION_2_NETWORK_2='192.168.3.0/24'
\end{verbatim}
\end{example}

        Avec l'option de configuration
        \var{DNS\_ZONE\_DELEGATION\_x\_UPTREAM\_SERVER\_x\_QUERYSOURCEIP} vous
        pouvez définir l'adresse IP sortante qui interrogera le serveur DNS amont.
        C'est utile, par exemple quand vous allez sur le serveur amont via un VPN
        et si vous ne voulez pas que l'adresse locale du VPN fli4l apparaît comme
        l'adresse IP source dans le serveur amont. Autre cas d'application,
        l'adresse IP du serveur DNS amont ne sera pas routable (cela se produit
        éventuellement à travers une interface VPN). Dans autre cas, il est
        logique que le Dnsmasq utilise l'adresse IP sortante qui est paramétré
        sur le routeur fli4l et que l'adresse IP du serveur DNS amont soit défini
        pour être accessible.

\begin{example}
\begin{verbatim}
        DNS_ZONE_DELEGATION_N='1'
        DNS_ZONE_DELEGATION_1_UPSTREAM_SERVER_N='1'
        DNS_ZONE_DELEGATION_1_UPSTREAM_SERVER_1_IP='192.168.1.12'
        DNS_ZONE_DELEGATION_1_UPSTREAM_SERVER_1_QUERYSOURCEIP='192.168.0.254'
        DNS_ZONE_DELEGATION_1_DOMAIN_N='1'
        DNS_ZONE_DELEGATION_1_DOMAIN_1='firma.de'
        DNS_ZONE_DELEGATION_1_NETWORK_N='1'
        DNS_ZONE_DELEGATION_1_NETWORK_1='192.168.1.0/24'
\end{verbatim}
\end{example}

    \configlabel{DNS\_REBINDOK\_x\_DOMAIN}{DNSREBINDOKxDOMAIN}
    \config{DNS\_REBINDOK\_N DNS\_REBINDOK\_x\_DOMAIN}{DNS\_REBINDOK\_N}{DNSREBINDOKN}

      Habituellement le serveur de noms \emph{Dnsmasq} refuse de répondre à d'autre
      serveur de nom, s'il contient des adresses IP de réseau privé. Il empêche
      ainsi une certaine forme d'attaques réseau. Mais si vous avez un nom de
      domaine avec une adresse IP dans votre réseau et si un serveur de nom
      distinct responsable du réseau privé fournit les réponses exactes,
      il sera rejeté par le serveur \emph{Dnsmasq}. On peut faire une liste
      de ces domaines dans la variable \var{DNS\_REBINDOK\_x}, les réponses
      appropriées des demandes au sujet de ce domaine, seront ensuite acceptées.
      Un autre exemple d'un serveur de nom qui fournirait les réponses aux
      adresses IP privées, ces serveurs sont soi-disant des "serveurs Blacklist
      en temps réel". Voici un exemple basé sur ces serveurs~:

\begin{example}
\begin{verbatim}
        DNS_REBINDOK_N='8'
        DNS_REBINDOK_1_DOMAIN='rfc-ignorant.org'
        DNS_REBINDOK_2_DOMAIN='spamhaus.org'
        DNS_REBINDOK_3_DOMAIN='ix.dnsbl.manitu.net'
        DNS_REBINDOK_4_DOMAIN='multi.surbl.org'
        DNS_REBINDOK_5_DOMAIN='list.dnswl.org'
        DNS_REBINDOK_6_DOMAIN='bb.barracudacentral.org'
        DNS_REBINDOK_7_DOMAIN='dnsbl.sorbs.net'
        DNS_REBINDOK_8_DOMAIN='nospam.login-solutions.de'
\end{verbatim}
\end{example}
\end{description}


\subsection{Serveur DHCP}

\begin{description}
    \config{OPT\_DHCP}{OPT\_DHCP}{OPTDHCP}

    {Avec la variable \var{OPT\_DHCP}, vous pouvez activer un serveur DHCP.}

    \config{DHCP\_TYPE}{DHCP\_TYPE}{DHCPTYPE} (optionnelle)

    {Avec cette variable, vous déterminez si vous voulez utiliser la fonction
    DHCP interne avec Dnsmasq, ou si vous voulez recourir à la fonction ISC-DHCPD
    externe. Dans ce cas avec ISC-DHCPD le support DDNS (ou DNS dynamique)
    sera supprimé.}

    \config{DHCP\_VERBOSE}{DHCP\_VERBOSE}{DHCPVERBOSE}

    {Avec cette variable, vous activez les messages sur les transactions DHCP
    dans le log (ou fichier journal).}

    \config{DHCP\_LS\_TIME\_DYN}{DHCP\_LS\_TIME\_DYN}{DHCPLSTIMEDYN}

    {Avec cette variable, vous indiquez le Lease-Time (ou délai du bail) standard
    pour des adresses IP fournies dynamiquement.}

    \config{DHCP\_MAX\_LS\_TIME\_DYN}{DHCP\_MAX\_LS\_TIME\_DYN}{DHCPMAXLSTIMEDYN}

    {Avec cette variable, vous indiquez le Lease-Time (ou délai du bail) maximum
    pour des adresses IP fournies dynamiquement.}

    \config{DHCP\_LS\_TIME\_FIX}{DHCP\_LS\_TIME\_FIX}{DHCPLSTIMEFIX}

    {Avec cette variable, vous indiquez le Lease-Time standard pour des
    adresses IP assignées statiquement.}

    \config{DHCP\_MAX\_LS\_TIME\_FIX}{DHCP\_MAX\_LS\_TIME\_FIX}{DHCPMAXLSTIMEFIX}

    {Avec cette variable, vous indiquez le Lease-Time maximum pour des
    adresses IP assignées statiquement.}

    \config{DHCP\_LEASES\_DIR}{DHCP\_LEASES\_DIR}{DHCPLEASESDIR}

    {Avec cette variable, vous indiquez le répertoire pour le fichier du bail DHCP.
    Il est possible de spécifier un chemin absolu ou d'indiquer le paramètre \emph{auto}.
    Si vous avez défini \emph{auto} le fichier du bail sera stocké dans le sous-répertoire
    persistant du DHCP (voir la documentation de la Base)}

    \config{DHCP\_LEASES\_VOLATILE}{DHCP\_LEASES\_VOLATILE}{DHCPLEASESVOLATILE}

    {Le répertoire \emph{Leases} se trouve dans le disque RAM (car avec une
    installation par CD le routeur n'a pas autres supports), au boot le routeur
    enverra un message d'avertissement, à cause de l'absent du répertoire \emph{Leases}
    pour l'installation des fichiers. Cet avertissement sera annulé, si vous
    indiquez dans la variable \var{DHCP\_LEASES\_VOLATILE} la valeur \emph{yes}.}

    \config{DHCP\_DNS\_SERVERS}{DHCP\_DNS\_SERVERS}{DHCPDNSSERVERS}

    {Dans cette variable vous indiquez l'adresse du serveur DNS. \\
	Plusieurs serveurs DNS peuvent être entrés, vous devez les séparer par un espace.
	Cette variable est optionnelle. Si vous la laissez vide ou si vous l'oubliez,
	l'adresse IP du réseau correspondant sera utilisée lorsque le serveur DNS du routeur
	sera activé. En outre, il est possible de mettre la valeur 'none' dans cette variable.
	Il n'y aura aucun serveur DNS attribué par la suite. Ce paramètre peut être écrasé
	par la variable
	\smalljump{DHCPRANGExDNSSERVERS}{\var{DHCP\_RANGE\_x\_DNS\_SERVERS}}.}
    
    \config{DHCP\_WINS\_SERVERS}{DHCP\_WINS\_SERVERS}{DHCPWINSSERVERS}

    {Dans cette variable vous indiquez l'adresse du serveur WINS. \\
	Plusieurs serveurs WINS peuvent être entrés, vous devez les séparer par un espace.
	Cette variable est optionnelle. Si vous la laissez vide ou si vous l'oubliez, et
	si le serveur WINS est configuré et activé dans le paquetage SAMBA, les paramètres à
	partir de ce paquetage seront utilisés. En outre, il est possible de mettre la valeur
	'none' dans cette variable. Il n'y aura aucun serveur WINS attribué par la suite.
	Ce paramètre peut être écrasé par la variable
	\smalljump{DHCPRANGExWINSSERVERS}{\var{DHCP\_RANGE\_x\_WINS\_SERVERS}}.}

    \config{DHCP\_NTP\_SERVERS}{DHCP\_NTP\_SERVERS}{DHCPNTPSERVERS}

    {Dans cette variable vous indiquez l'adresse du serveur NTP. \\
	Plusieurs serveurs NTP peuvent être entrés, vous devez les séparer par un espace.
	Cette variable est optionnelle. Si vous la laissez vide ou si vous l'oubliez,
	l'adresse IP du réseau correspondant sera utilisée lorsqu'un paquet d'un serveur temps
	du routeur sera activé. En outre, il est possible de mettre la valeur 'none' dans cette
	variable. Il n'y aura aucun serveur NTP attribué par la suite. Ce paramètre peut être
	écrasé par la variable
	\smalljump{DHCPRANGExNTPSERVERS}{\var{DHCP\_RANGE\_x\_NTP\_SERVERS}}.}
    
    \config{DHCP\_OPTION\_WPAD}{DHCP\_OPTION\_WPAD}{DHCPOPTIONWPAD}
     
     {Avec cette variable, vous activez ou désactivez la transmission du DHCP OPTION 252
	 (Web Proxy Autodiscovery Protocol) cela permet au navigateur d'acquérir les paramètres
	 du proxy automatiquement.
	 (Voir \altlink{http://de.wikipedia.org/wiki/Web_Proxy_Autodiscovery_Protocol})
     }
     
     \config{DHCP\_OPTION\_WPAD\_URL}{DHCP\_OPTION\_WPAD\_URL}{DHCPOPTIONWPADURL}
     
     {Dans cette variable vous indiquez l'URL du fichier wpad.dat, si vous laissez le champ
	 vide la réponse envoyée au navigateur qui recherche le fichier sera nul, ainsi il
	 n'effectura aucune autre requête.
     }

\end{description}


\subsubsection{Plage DHCP locale}

\begin{description}
    \config{DHCP\_RANGE\_N}{DHCP\_RANGE\_N}{DHCPRANGEN}

    {Avec cette variable, vous indiquez le nombre de plage d'adresses IP.}

    \config{DHCP\_RANGE\_x\_NET}{DHCP\_RANGE\_x\_NET}{DHCPRANGExNET}

    {Avec cette variable, vous indiquez le réseau défini dans la variable
	\var{IP\_NET\_x}.}

    \config{DHCP\_RANGE\_x\_START}{DHCP\_RANGE\_x\_START}{DHCPRANGExSTART}

    {Avec cette variable, vous indiquez la première adresse IP.}

    \config{DHCP\_RANGE\_x\_END}{DHCP\_RANGE\_x\_END}{DHCPRANGExEND}

    {Avec cette variable, vous indiquez la dernière adresse IP. Les deux
	variables \var{DHCP\_RANGE\_x\_START} et \var{DHCP\_RANGE\_x\_END} peuvent être
    vide. Alors, aucune plage DHCP ne sera créée. Mais les hôtes avec l'attribution
    d'une adresse MAC recevront les valeurs des autres variables.}

    \config{DHCP\_RANGE\_x\_DNS\_DOMAIN}{DHCP\_RANGE\_x\_DNS\_DOMAIN}{DHCPRANGExDNSDOMAIN}

    {Avec cette variable, vous définissez un domaine DNS spécifique pour les hôtes
    de la plage DHCP. Cette variable est optionnelle. Si rien n'est enregistré,
    la variable sera simplement omise et le domaine DNS par défaut \var{DOMAIN\_NAME}
    sera utilisé.}

    \config{DHCP\_RANGE\_x\_DNS\_SERVERS}{DHCP\_RANGE\_x\_DNS\_SERVERS}{DHCPRANGExDNSSERVERS}

    {Dans cette variable vous indiquez l'adresse du serveur DNS. \\
	Plusieurs serveurs DNS peuvent être entrés, vous devez les séparer par un espace.
	Cette variable est optionnelle. Si vous la laissez vide ou si vous l'oubliez, la valeur
	de la variable \smalljump{DHCPDNSSERVERS}{\var{DHCP\_DNS\_SERVERS}} sera utilisée.
	En outre, il est possible de mettre la valeur 'none' dans cette variable. Il n'y
	aura aucun serveur DNS attribué par la suite.}

    \config{DHCP\_RANGE\_x\_WINS\_SERVERS}{DHCP\_RANGE\_x\_WINS\_SERVERS}{DHCPRANGExWINSSERVERS}

    {Dans cette variable vous indiquez l'adresse du serveur WINS. \\
	Plusieurs serveurs WINS peuvent être entrés, vous devez les séparer par un espace.
	Cette variable est optionnelle. Si vous la laissez vide ou si vous l'oubliez, la valeur
	de la variable \smalljump{DHCPWINSSERVERS}{\var{DHCP\_WINS\_SERVERS}} sera utilisée.
	En outre, il est possible de mettre la valeur 'none' dans cette variable. Il n'y
	aura aucun serveur WINS attribué par la suite.}

    \config{DHCP\_RANGE\_x\_NTP\_SERVERS}{DHCP\_RANGE\_x\_NTP\_SERVERS}{DHCPRANGExNTPSERVERS}

    {Dans cette variable vous indiquez l'adresse du serveur NTP. \\
	Plusieurs serveurs NTP peuvent être entrés, vous devez les séparer par un espace.
	Cette variable est optionnelle. Si vous la laissez vide ou si vous l'oubliez, la valeur
	de la variable \smalljump{DHCPNTPSERVERS}{\var{DHCP\_NTP\_SERVERS}} sera utilisée.
	En outre, il est possible de mettre la valeur 'none' dans cette variable. Il n'y
	aura aucun serveur NTP attribué par la suite.}

    \config{DHCP\_RANGE\_x\_GATEWAY}{DHCP\_RANGE\_x\_GATEWAY}{DHCPRANGExGATEWAY}

    {Avec cette variable, vous définissez la Gateway (ou passerelle) pour les hôtes
    de la plage DHCP. Cette variable est optionnelle. Si rien n'est enregistré,
    la variable sera simplement omise et l'adresse IP référencée dans la variable
    \var{DHCP\_RANGE\_x\_NET} sera utilisée. Il est également possible de placer
    'none' dans cette variable, alors aucune Gateway ne sera utilisé.}

    \config{DHCP\_RANGE\_x\_MTU}{DHCP\_RANGE\_x\_MTU}{DHCPRANGExMTU}

    {Avec cette variable, vous définissez le MTU du client pour la plage d'adresses.
	Cette variable est optionnelle.}

    \config{DHCP\_RANGE\_x\_OPTION\_WPAD}{DHCP\_RANGE\_x\_OPTION\_WPAD}{DHCPRANGExOPTIONWPAD}
     
     {Avec cette variable, vous activez ou désactivez la transmission du DHCP OPTION 252
	 (Web Proxy Autodiscovery Protocol) pour une plage DHCP, cela permet au navigateur
	 d'acquérir les paramètres du proxy automatiquement.
	 (Voir \altlink{http://de.wikipedia.org/wiki/Web_Proxy_Autodiscovery_Protocol})
	 cette variable est optionnelle.
     }
     
     \config{DHCP\_RANGE\_x\_OPTION\_WPAD\_URL}{DHCP\_RANGE\_x\_OPTION\_WPAD\_URL}{DHCPRANGExOPTIONWPADURL}
     
     {Dans cette variable vous indiquez l'URL du fichier wpad.dat, si vous laissez le champ
	 vide la réponse envoyée au navigateur qui recherche le fichier sera nul, ainsi il
	 n'effectura aucune autre requête. Cette variable est optionnelle.
     }

    \configlabel{DHCP\_RANGE\_x\_OPTION\_x}{DHCPRANGExOPTIONx}
    \config{DHCP\_RANGE\_x\_OPTION\_N}{DHCP\_RANGE\_x\_OPTION\_N}{DHCPRANGExOPTIONN}

    {Avec ces variables vous pouvez définir des options spécifiques pour ce
    domaine. Les options peuvent être trouvées dans le Manuel Dnsmasq
    (\altlink{http://thekelleys.org.uk/dnsmasq/docs/dnsmasq.conf.example}). Ces
    options n'ont pas été contrôlées, ils peuvent causer des erreurs ou des
    problèmes avec le serveur DNS/DHCP. Cette variable est optionnelle.}
\end{description}


\subsubsection{Extra plage DHCP}

\begin{description}
    \config{DHCP\_EXTRA\_RANGE\_N}{DHCP\_EXTRA\_RANGE\_N}{DHCPEXTRARANGEN}

    {Avec cette variable, vous indiquez le nombre de plage d'adresse IP pour un serveur
	DHCP qui n'est pas dans le réseau local. Pour cela vous devez installer un relais
	DHCP qui sera sur le réseau de la gateway (ou passerelle).}

    \config{DHCP\_EXTRA\_RANGE\_x\_START}{DHCP\_EXTRA\_RANGE\_x\_START}{DHCPEXTRARANGExSTART}

    {Avec cette variable, vous indiquez la première adresse IP.}

    \config{DHCP\_EXTRA\_RANGE\_x\_END}{DHCP\_EXTRA\_RANGE\_x\_END}{DHCPEXTRARANGExEND}

    {Avec cette variable, vous indiquez la dernière adresse IP.}

    \config{DHCP\_EXTRA\_RANGE\_x\_NETMASK}{DHCP\_EXTRA\_RANGE\_x\_NETMASK}{DHCPEXTRARANGExNETMASK}

    {Avec cette variable, vous indiquez le masque de sous réseau.}

    \config{DHCP\_EXTRA\_RANGE\_x\_DNS\_SERVERS}{DHCP\_EXTRA\_RANGE\_x\_DNS\_SERVERS}{DHCPEXTRARANGExDNSSERVERS}

    {Dans cette variable vous indiquez l'adresse du serveur DNS \\
    (Voir \smalljump{DHCPRANGExDNSSERVERS}{\var{DHCP\_RANGE\_x\_DNS\_SERVERS}}).}

    \config{DHCP\_EXTRA\_RANGE\_x\_WINS\_SERVERS}{DHCP\_EXTRA\_RANGE\_x\_WINS\_SERVERS}{DHCPEXTRARANGExWINSSERVERS}

    {Dans cette variable vous indiquez l'adresse du serveur WINS \\
    (Voir \smalljump{DHCPRANGExWINSSERVERS}{\var{DHCP\_RANGE\_x\_WINS\_SERVERS}}).}
    
    \config{DHCP\_EXTRA\_RANGE\_x\_NTP\_SERVERS}{DHCP\_EXTRA\_RANGE\_x\_NTP\_SERVERS}{DHCPEXTRARANGExNTPSERVERS}

    {Dans cette variable vous indiquez l'adresse du serveur NTP \\
    (Voir \smalljump{DHCPRANGExNTPSERVERS}{\var{DHCP\_RANGE\_x\_NTP\_SERVERS}}).}
 
    \config{DHCP\_EXTRA\_RANGE\_x\_GATEWAY}{DHCP\_EXTRA\_RANGE\_x\_GATEWAY}{DHCPEXTRARANGExGATEWAY}

    {Avec cette variable, vous indiquez l'adresse de la Gateway par défaut pour ce domaine.}

    \config{DHCP\_EXTRA\_RANGE\_x\_MTU}{DHCP\_EXTRA\_RANGE\_x\_MTU}{DHCPEXTRARANGExMTU}

    {Avec cette variable, vous indiquez le MTU du client pour la plage d'adresses.
	Cette variable est optionnelle.}

    \config{DHCP\_EXTRA\_RANGE\_x\_DEVICE}{DHCP\_EXTRA\_RANGE\_x\_DEVICE}{DHCPEXTRARANGExDEVICE}

    {Avec cette variable, vous indiquez l'interface réseau pour accéder à ce domaine.}
\end{description}


\subsubsection{Clients DHCP non autorisés}

\begin{description}
    \config{DHCP\_DENY\_MAC\_N}{DHCP\_DENY\_MAC\_N}{DHCPDENYMACN}

    {Avec cette variable, vous indiquez le nombre de variable, dont l'accès au serveur DHCP
	sera refusé par les adresses MAC des hôtes.}

    \config{DHCP\_DENY\_MAC\_x}{DHCP\_DENY\_MAC\_x}{DHCPDENYMACx}

    {Avec cette variable, vous indiquez les adresses MAC des hôtes, dont l'accès
     au serveur DHCP sera refusé.}
\end{description}


\subsubsection{Supporte le boot par le réseau}

  Dnsmasq supporte les clients, qui lancent le Bootp/PXE via le réseau pour
  booter (ou démarrer) fli4l. Les informations nécessaires sont fournies par le
  Dnsmasq pour configurer l'hôte sur le sous-réseau. Les variables nécessaires
  sont DHCP\_RANGE\_\%- et HOST\_\%- Ce paragraphe décrit l'installation et le
  fichier de boot avec (*\_PXE\_FILENAME), le serveur met à disposition les variables
  (*\_PXE\_SERVERNAME et *\_PXE\_SERVERIP), éventuellement (*\_PXE\_OPTIONS)
  si nécessaire pour les options. De plus, on peut activer un serveur TFTP
  interne, si bien que le boot sera complètement supporté par Dnsmasq.

\begin{description}
    \configlabel{HOST\_x\_PXE\_FILENAME}{HOSTxPXEFILENAME}
    \config{HOST\_x\_PXE\_FILENAME DHCP\_RANGE\_x\_PXE\_FILENAME}{DHCP\_RANGE\_x\_PXE\_FILENAME}{DHCPRANGExPXEFILENAME}

    Avec cette variable, vous indiquez l'image boot à lancer. Avec PXE vous
    indiquez ici le pxe-Bootloader à charger, par exemple pxegrub, pxelinux
    ou un autre Bootloader qui convient.

    \configlabel{HOST\_x\_PXE\_SERVERNAME}{HOSTxPXESERVERNAME}
    \configlabel{HOST\_x\_PXE\_SERVERIP}{HOSTxPXESERVERIP}
    \configlabel{DHCP\_RANGE\_x\_PXE\_SERVERNAME}{DHCPRANGExPXESERVERNAME}
    \config{HOST\_x\_PXE\_SERVERNAME HOST\_x\_PXE\_SERVERIP
    DHCP\_RANGE\_x\_PXE\_SERVERNAME
    DHCP\_RANGE\_x\_PXE\_SERVERIP}{DHCP\_RANGE\_x\_PXE\_SERVERIP}{DHCPRANGExPXESERVERIP}

    Avec ces variables, vous indiquez Le nom et l'adresse IP du serveur TFTP,
    ces variables doivent rester vides, si le routeur est utilisé en tant que
    serveur TFTP.

    \configlabel{HOST\_x\_PXE\_OPTIONS}{HOSTxPXEOPTIONS}
    \config{DHCP\_RANGE\_x\_PXE\_OPTIONS HOST\_x\_PXE\_OPTIONS}{DHCP\_RANGE\_x\_PXE\_OPTIONS}{DHCPRANGExPXEOPTIONS}

    Certains Bootloader ont besoin d'options pour booter. Il demande par exemple,
    avec pxegrub, l'option 150 avec le nom du fichier menu. Cette option peut
    être indiquer dans cette variable et sera reprise alors par le fichier config.
    Dans l'exemple pxegrub, on pourrait paramètrer comme ceci~: \\
    \begin{example}
      \begin{verbatim}
    HOST_x_PXE_OPTIONS='150,"(nd)/grub-menu.lst"'
      \end{verbatim}
    \end{example}

    S'il est nécessaire d'indiquer plusieurs options, ils seront simplement
    séparés par un espace.
\end{description}


\subsection {Relais DHCP}

    Le relais DHCP est utilisé, lorsqu'un autre serveur DHCP assume la
    gestion de la plage d'adresses IP et qui ne peut pas directement être
    atteint par les clients.

\begin{description}
\config{OPT\_DHCPRELAY}{OPT\_DHCPRELAY}{OPTDHCPRELAY}

    Vous devez paramétrer cette variable sur 'yes' pour que le routeur puisse
    faire fonctionner le relais DHCP. Il ne faut pas activer un serveur DHCP
	en même temps.

    Configuration par défaut~: \var{OPT\_\-DHCPRELAY}='no'

\config{DHCPRELAY\_SERVER}{DHCPRELAY\_SERVER}{DHCPRELAYSERVER}

    À ce stade, le serveur DHCP est correctement enregistré, pour que les
    demandes puissent passer.

\configlabel{DHCPRELAY\_IF\_N}{DHCPRELAYIFN}
\configlabel{DHCPRELAY\_IF\_x}{DHCPRELAYIFx}
\configvar{DHCPRELAY\_IF\_N DHCPRELAY\_IF\_x}

    Avec la variable \var{DHCPRELAY\_\-IF\_N}, on indique le nombre de cartes
    réseaux sur lesquelles le serveur Relay doit écouter. Dans la variable
    \var{DHCPRELAY\_IF\_x} on indique la carte réseau correspondantes.

    L'interface sur laquelle le serveur DHCP répond aux demandes, doit être
    mentionnée dans la liste. En outre, il faut s'assurer que les routes de
    l'ordinateur, sur lequel le serveur DHCP est installé fonctionne correctement.
    La réponse du serveur DHCP doit provenir via l'adresse IP de l'interface
    sur laquelle le client DHCP dépend. Prenons le scénario suivant~:

    \begin{itemize}
    \item Relais sur deux interfaces
    \item Interface client~: eth0, 192.168.6.1
    \item Interface serveur DHCP~:  eth1, 192.168.7.1
    \item Serveur DHCP~:  192.168.7.2
    \end{itemize}

    Il doit y avoir pour le serveur DHCP, une route qui accéde à l'objectif,
    ici il faut répondre à l'adresse 192.168.6.1, est-ce que le routeur sur
    lequel le relais fonctionne par défaut sur la passerelle peux accéder
    au serveur DHCP, si oui tout est ok. \\
    Si ce n'est pas le cas, nous allons avoir besoin d'une extra route
    supplémentaire. Si le serveur DHCP veut accéder au client par le
    routeur fli4l, vous devez enregistrer dans config/base.txt~:
    IP\_ROUTE\_x='192.168.6.0/24 192.168.7.1'

    Pendant le fonctionnement, il y a parfois des messages d'avertissements
    au sujet de certains paquets ignorés, ne tenez pas compte de ces avertissements,
    cela a aucune incidance sur le fonctionnement normal.

   Exemple~:

\begin{example}
\begin{verbatim}
        OPT_DHCPRELAY='yes'
        DHCPRELAY_SERVER='192.168.7.2'
        DHCPRELAY_IF_N='2'
        DHCPRELAY_IF_1='eth0'
        DHCPRELAY_IF_2='eth1'
\end{verbatim}
\end{example}
\end{description}

\marklabel{sec:dhcp }
{
\subsection {Client-DHCP}
}

Le client DHCP peut être utilisé pour recevoir une adresse IP sur une ou
plusieurs interface(s) du routeur~- cela émane le plus souvent du fournisseur
d'accès. Actuellement cette possibilité de liaison provient principalement de
la Suisse, des Pays-Bas et de la France, avec l'utilisation d'un modem câblé.
On a aussi parfois besoin de cette configuration, quand le routeur est intégré
derrière un autre routeur qui distribue les adresses IP par DHCP (par ex.
derrière une FritzBox).

Au démarrage du routeur, les interfaces spécifiées obtiennent une adresse IP.
Ensuite, cette interface est affectée et si besoin une route par défaut est
définie pour cette interface.

\begin{description}
\config{OPT\_DHCP\_CLIENT}{OPT\_DHCP\_CLIENT}{OPTDHCPCLIENT}

Il faut paramétrer 'yes' dans cette variable, si vous voulez utiliser
le client DHCP.

      Configuration par défaut~: \var{OPT\_\-DHCP\_CLIENT}='no'

\config{DHCP\_CLIENT\_TYPE}{DHCP\_CLIENT\_TYPE}{DHCPCLIENTTYPE}

Le paquetage client DHCP a actuellement deux protocoles différents le dhclient
et le dhcpcd. Vous pouvez choisir et indiquer ici le protocole que vous souhaitez.

      Configuration par défaut~: \var{DHCP\_CLIENT\_TYPE}='dhcpcd'

\config{DHCP\_CLIENT\_N}{DHCP\_CLIENT\_N}{DHCPCLIENTN}

Vous indiquez ici, le nombre d'interfaces à configurer.

\config{DHCP\_CLIENT\_x\_IF}{DHCP\_CLIENT\_x\_IF}{DHCPCLIENTxIF}

Vous indiquez ici, l'interface à configurer qui est référencée par \var{IP\_NET\_x\_DEV},
par ex. \var{DHCP\_CLIENT\_1\_IF}='\var{IP\_NET\_1\_DEV}'. Le client DHCP récupère
le périphérique associé à la variable correspondante. Celle-ci doit être
enregistrée dans le fichier base.txt, toujours dans le fichier base.txt à la
place de l'adresse IP et du masque de sous réseau du périphérique vous devez
paramétrer '\emph{dhcp}'

\config{DHCP\_CLIENT\_x\_ROUTE}{DHCP\_CLIENT\_x\_ROUTE}{DHCPCLIENTxROUTE}

Si vous voulez appliquée une route pour l'interface, vous pouvez l'indiquer ici.
La variable peut être paramétrée avec les valeurs suivantes~:
\begin{description}
\item[none] Aucune route n'est appliquée sur l'interface.
\item[default] Une route par défaut est appliquée sur l'interface.
\item[imond] Imond gère la route par défaut pour cette interface.
\end{description}

      Configuration par défaut~: \var{DHCP\_CLIENT\_x\_ROUTE}='default'

\config{DHCP\_CLIENT\_x\_USEPEERDNS}{DHCP\_CLIENT\_x\_USEPEERDNS}{DHCPCLIENTxUSEPEERDNS}

Si cette variable est paramétrée sur 'yes' et si la route par défaut est
configurée sur ce périphérique, alors les demandes DNS du routeur seront
transférées au serveur DNS du fournisseur d'accés Internet, pour que cela
fonctionne vous devez paramétrer les serveurs DNS dans le fichier base.txt.

      Configuration par défaut~: \var{DHCP\_CLIENT\_x\_\-USEPEERDNS}='no'

\config{DHCP\_CLIENT\_x\_HOSTNAME}{DHCP\_CLIENT\_x\_HOSTNAME}{DHCPCLIENTxHOSTNAME}

Certains fournisseurs d'accés Internet demandent un nom d'hôte pour la connexion
Internet. Ce nom doit être fournie par le FAI et doit être indiqué ici. Ce nom
ne doit pas être identique au nom d'hôte du routeur.

      Configuration par défaut~: \var{DHCP\_CLIENT\_x\_HOSTNAME}=''

\config{DHCP\_CLIENT\_x\_STARTDELAY}{DHCP\_CLIENT\_x\_STARTDELAY}{DHCPCLIENTxSTARTDELAY}

Cette variable est optionnelle, elle sert à retarder le départ du client DHCP.

Dans certaines installations (par exemple lorsque fli4l est configuré en tant
que client DHCP derrière un modem câblé, une FritzBox, ...) il est nécessaire
d'attendre que le serveur DHCP soit redémarré pour le paramétrage du client,
par exemple lors d'une coupure d'électricité.

      Configuration par défaut~: \var{DHCP\_CLIENT\_x\_STARTDELAY}='0'

\config{DHCP\_CLIENT\_x\_WAIT}{DHCP\_CLIENT\_x\_WAIT}{DHCPCLIENTxWAIT}

Normalement, le client DHCP démarre en arrière-plan. Cela signifie que le
processus de Boot n'est pas retardé par la création de l'adresse IPv4. si un
paquetage installé sur le routeur à besoin rapidement d'une adresse configurée,
il est nécessaire, que l'adresse IP soit créée avant que le processus de Boot
ne démarre, (par exemple pour l'OPT\_IGMP). Dans ce cas, vous pouvez activer
la variable \verb+DHCP_CLIENT_x_WAIT='yes'+, pour forcer la surveillance de
l'adresse IP.

      Configuration par défaut~: \var{DHCP\_CLIENT\_x\_WAIT}='no'

\config{DHCP\_CLIENT\_DEBUG}{DHCP\_CLIENT\_DEBUG}{DHCPCLIENTDEBUG}

Vous enregistrez avec cette variable des informations supplémentaires,
lorsqu'une adresse IP est attribuée sur le client DHCP.

      Configuration par défaut~: ou à délaisser \var{DHCP\_CLIENT\_DEBUG}='no'

\end{description}

\subsection {Serveur TFTP}

    Un serveur TFTP peut être utilisé dans fli4l, pour la transmission de fichiers.
    Cela peut servir, par exemple, à un client pour récupérer des fichiers sur son
    portable par Internet.

\begin{description}

    \config{OPT\_TFTP}{OPT\_TFTP}{OPTTFTP}

    Cette variable active le serveur TFTP interne du Dnsmasq.
    Le paramètre par défaut est 'no'.

    \config{TFTP\_PATH}{TFTP\_PATH}{TFTPPATH}

    Vous indiquez ici le répertoire du serveur TFTP dans lequel sera placé les
    fichiers, pour que les clients puissent les récupérer. Vous pouvez déposer
    les fichiers dans le chemin correspondant, à l'aide d'un programme adapté
    (par ex. scp).
\end{description}


\subsection {YADIFA - Serveur DNS esclave}

\begin{description}

    \config{OPT\_YADIFA}{OPT\_YADIFA}{OPYADIFA}

    Avec cette variable vous activez YADIFA, un serveur DNS esclave.
    Le paramètre par défaut est 'no'.

    \config{OPT\_YADIFA\_USE\_DNSMASQ\_ZONE\_DELEGATION}{OPT\_YADIFA\_USE\_DNSMASQ\_ZONE\_DELEGATION}{OPTYADIFAUSEDNSMASQZONEDELEGATION}

    Si cette variable est activé, yadifa produira un script de démarrage qui
    génèrera automatiquement les entrées de toutes les zones esclaves
    correspondant au délégation de zone pour Dnsmasq. Ainsi, les zones d'esclaves
    sont directement interrogés par le Dnsmasq, en principe il ne sera pas
    nécessaire de configurer plusieurs fois YADIFA\_LISTEN\_x. Les réponces des
    requêtes de Dnsmasq sont transmises à yadifa qui écoute uniquement
    sur le port localhost:35353.

    \config{YADIFA\_LISTEN\_N}{YADIFA\_LISTEN\_N}{YADIFALISTENN}

    Si vous avez activez \var{OPT\_YADIFA}='yes', avec l'aide de la variable
    \var{YADIFA\_LISTEN\_N} vous indiquez le nombre d'adresses, si vous indiquez
    \var{YADIFA\_LISTEN\_1} vous devez spécifier une adresse IP du réseau local,
    sur laquelle YADIFA devra accepter les requêtes DNS. Un numéro de port est
    facultatif, si vous indiquez 192.168.1.1:5353 le serveur DNS esclave YADIFA
    écoutera les requêtes DNS sur le port 5353. Assurez-vous que le Dnsmasq
    n'écoute pas sur toutes les interfaces (voir \var{DNS\_BIND\_INTERFACES}).
    A ce stade, seuls les adresses IP des interfaces existantes (Ethernet, Wlan, ...)
    peuvent être utilisé, sinon il y aura un message d'avertissement au démarrage
    du routeur. Il est également possible d'utiliser les alias, par ex. \verb+IP_NET_1_IPADDR+

    \config{YADIFA\_ALLOW\_QUERY\_N}{YADIFA\_ALLOW\_QUERY\_N}{YADIFAALLOWQUERYN}
    \config{YADIFA\_ALLOW\_QUERY\_x}{YADIFA\_ALLOW\_QUERY\_x}{YADIFAALLOWQUERYX}

    Dans ces variables vous indiquer le nombre et les adresses IP ou les
    réseaux pour que YADIFA puis avoir une autorisation d'accès. YADIFA utilise
    les informations du filtrage de paquets de fli4l qui doit être configuré en
    conséquence, il faut aussi paramétrer les fichiers de configuration de YADIFA.
    En ajoutant le préfixe '!' à l'adresse, l'accès à l'adresse IP ou au réseau
    sera rejeté par YADIFA.

    Le filtrage de paquets de fli4l est configuré pour YADIFA de sorte que tous
    les réseaux autosisés soient ajoutés dans chaque zone pour l'ensemble de la
    liste ipset (yadifa-allow-query). Une différenciation entre les zones pour
    le filtrage de paquets n'est pas possible. De plus toutes les adresses IP
    et de réseaux de la configuration globale dont l'accès est refusé seront
    ajoutées à cette liste. Il n'est donc pas possible d'étendre l'accès de
    chaque zone ultérieurement.

    \config{YADIFA\_SLAVE\_ZONE\_N}{YADIFA\_SLAVE\_ZONE\_N}{YADIFASLAVEZONEN}

    Dans cette variable vous indiquez le nombre de zone DNS pour YADIFA esclave.

    \config{YADIFA\_SLAVE\_ZONE\_x}{YADIFA\_SLAVE\_ZONE\_x}{YADIFASLAVEZONEx}

    Dans cette variable vous indiquez le nom de la zone DNS esclave.

    \config{OPT\_YADIFA\_SLAVE\_ZONE\_USE\_DNSMASQ\_ZONE\_DELEGATION}{OPT\_YADIFA\_SLAVE\_ZONE\_USE\_DNSMASQ\_ZONE\_DELEGATION}{OPTYADIFASLAVEZONEUSEDNSMASQZONEDELEGATION}

    Dans cette variable vous activez (='yes') ou déactivez (='no')
    la délégation de zone pour la zone esclave du Dnsmasq.

    \config{YADIFA\_SLAVE\_ZONE\_x\_MASTER}{YADIFA\_SLAVE\_ZONE\_x\_MASTER}{YADIFASLAVEZONExMASTER}

    Dans cette variable vous indiquez l'adresse IP avec le numéro de port
    facultatif du serveur DNS maître.

    \config{YADIFA\_SLAVE\_ZONE\_x\_ALLOW\_QUERY\_N}{YADIFA\_SLAVE\_ZONE\_x\_ALLOW\_QUERY\_N}{YADIFASLAVEZONExALLOWQUERYN}
    \config{YADIFA\_SLAVE\_ZONE\_x\_ALLOW\_QUERY\_x}{YADIFA\_SLAVE\_ZONE\_x\_ALLOW\_QUERY\_x}{YADIFASLAVEZONExALLOWQUERYx}

    Dans ces variables vous indiquez le nombre et les adresses IP ou
    les réseaux, pour que YADIFA puis avoir une autorisation d'accès. En outre
    l'accès peut être limité à des zones DNS spécifiques. YADIFA utilise ces
    informations pour créer un fichier de configuration YADIFA.

    En ajoutant le préfixe '!' à l'adresse, l'accès à l'adresse IP ou du réseau
    sera rejeté par YADIFA.

\end{description}
