% Last Update: $Id: lnk_main.tex 43058 2015-11-27 12:03:32Z cspiess $
\marklabel{sec:opt-lnk}
{
\section {OPT\_LNK - Erstellen von Verknüpfungen}
}

\subsection {Einleitung}

Mit diesen Paket ist es möglich, bestimmte Dateien/Ordner während des
Boot-Vorgangs zu verknüpfen\footnote{Eine Verknüpfung wird auch als \emph{Link}
bezeichnet, daher rührt auch der Name des Pakets.}. Dies ist besonders für
normalerweise flüchtige Konfigurations- und Log-Dateien wie das
imonc-Telefonbuch \texttt{/etc/phonebook} sinnvoll, um zu verhindern, dass
jedesmal alle Änderungen nach dem nächsten Neustart verloren gehen. Auch ganze
Verzeichnisse kann man ``umlegen'', etwa \texttt{/var/log}.

Dieses Paket ist eine Anpassung des OPT\_LNK-Pakets von Alexander Krause an
fli4l 3.x.

\subsection {Beispiele}

Um das \texttt{/var/log}-Verzeichnis auf Ihre Festplatte abzubilden, verändern
Sie die Variablen in \texttt{config/lnk.txt} wie folgt:

\begin{small}
\begin{example}
\begin{verbatim}
OPT_LNK='yes' # Bei 'yes' wird das Paket installiert, andernfalls nicht
LNK_N='1'                        # Anzahl der Links, hier '1'
LNK_1_OPT='-fs'                  # Link 1: Optionen
LNK_1_DST='/var/log'             # Link 1: Ziel
LNK_1_SRC='/data/var/log'        # Link 1: Quelle
\end{verbatim}
\end{example}
\end{small}

\wichtig{Das Ziel der Verknüpfung (hier: \texttt{/var/log}) muss angelegt werden
können; das bedeutet in diesem Beispiel, dass \texttt{/var} ein beschreibbares
Verzeichnis sein muss. Ist das nicht der Fall, kann die Verknüpfung folglich
nicht angelegt werden. Genauso muss die Quelle (hier: \texttt{/data/var/log})
natürlich zum Zeitpunkt des Verknüpfens vorhanden sein.
}

\achtung{Falls das Ziel (hier: \texttt{/var/log}) vor dem Verknüpfen bereits
existiert, wird es gelöscht! (Das ist in dem Beispiel unproblematisch, da
dieses Verzeichnis zum Zeitpunkt der Erstellung der Verknüpfung ohnehin
leer ist.)}

Um zusätzlich das \texttt{imonc}-Telefonbuch persistent zu machen, kopieren Sie
es zunächst in ein Verzeichnis auf Ihrer Festplatte (etwa nach
\texttt{/data/etc/phonebook}) und legen dann eine zusätzliche Verknüpfung an:

\begin{small}
\begin{example}
\begin{verbatim}
OPT_LNK='yes' # Bei 'yes' wird das Paket installiert, andernfalls nicht
LNK_N='2'                        # Anzahl der Links, hier '2'
LNK_1_OPT='-fs'                  # Link 1: Optionen
LNK_1_DST='/var/log'             # Link 1: Ziel
LNK_1_SRC='/data/var/log'        # Link 1: Quelle
LNK_2_OPT='-fs'                  # Link 2: Optionen
LNK_2_DST='/etc/phonebook'       # Link 2: Ziel
LNK_2_SRC='/data/etc/phonebook'  # Link 2: Quelle
\end{verbatim}
\end{example}
\end{small}

\subsection {Konfiguration}

\begin{description}
\config{OPT\_LNK}{OPT\_LNK}{OPTLNK}{
Steht dieser Wert auf ``yes'', wird das Paket aktiviert.
}
\config{LNK\_N}{LNK\_N}{LNKN}{
Diese Variable nimmt die Anzahl der zu erstellenden Verknüpfungen auf.

\achtung {In den folgenden Variablen ist \var{x} durch einen Index zu ersetzen!}
}
\config{LNK\_x\_OPT}{LNK\_x\_OPT}{LNKxOPT}{
Über diese Variable können Sie die Optionen verändern, die beim Verknüpfen von
\smalljump{LNKxSRC}{Quelle} und \smalljump{LNKxDST}{Ziel} an das Programm ``ln''
übergeben werden. Der Standard-Wert ``-fs'' ist in den allermeisten Fällen
korrekt und sollte nicht verändert werden.
}
\config{LNK\_x\_DST}{LNK\_x\_DST}{LNKxDST}{
In dieser Variable steht das Ziel der anzulegenden Verknüpfung. Über den hier
angegebenen Namen kann die \smalljump{LNKxSRC}{Quelle} nach dem Verknüpfen
angesprochen werden. In der Regel sollte das hier angegebene Ziel nicht
existieren.

\achtung {Achtung: Wenn das Ziel existieren sollte, wird es ohne Vorwarnung
gelöscht!}
}
\config{LNK\_x\_SRC}{LNK\_x\_SRC}{LNKxSRC}{
In dieser Variable steht die Quelle der anzulegenden Verknüpfung. Dabei kann
es sich sowohl um eine Datei als auch um ein Verzeichnis handeln.

\wichtig {Die hier angegebene Quelle muss zum Zeitpunkt des
Verknüpfens vorhanden sein!}
}
\end{description}
