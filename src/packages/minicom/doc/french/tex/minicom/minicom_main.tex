% Synchronized to r32886
\marklabel{sec:minicom}
{
\section {Minicom}
}

\subsection {Description}
  Ce package contient le programme minicom \\
  Minicom est un programme d'émulation de terminal. Il prend en charge les terminaux
  ANSI et VT.102. Pour accéder à la configuration par défaut vous devez utiliser le paramètre
  '-s', la configuration se fait par ligne de commande. Vous pouvez également accéder au menu
  de configuration lorque le programme est exécuté en utilisant le raccourci clavier 'CTRL+A'
  ensuite en appuyant sur la touche 'Z'.

\subsection{Configuration}

\begin{description}

\config{OPT\_MINICOM}{OPT\_MINICOM}{OPTMINICOM}{}
  Avec le paramètre \var{'no'} vous déactivez complètement l'OPT\_MINICOM.\\
  Pour activer \var{OPT\_MINICOM}, vous devez paramétrer la variable \var{OPT\_MINICOM}
  avec la valeur \var{'yes'}.

\end{description}
