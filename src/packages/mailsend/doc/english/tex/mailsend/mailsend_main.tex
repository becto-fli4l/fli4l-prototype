% Synchronized to r0

% ##TRANSLATE## : FFL-1570: begin new package

\section {MAILSEND - send eMail}

\subsection {Description}
    The package MAILSEND provides an API for mail transmission to other packages.
    
    Additionally the package EASYCRON is required (\var{OPT\_EASYCRON='yes'}).

\subsection {Configuration of the MAILSEND package}

    The configuration is made, as of all fli4l packages, by adjusting the file\\
    \var{path/fli4l-\version/\emph{\flq config\frq}/mailsend.txt} to meet your own demands.

\subsubsection {General}
\begin{description}

\config {OPT\_MAILSEND}{OPT\_MAILSEND}{OPTMAILSEND}

    The setting \var{'no'} deactivates OPT\_MAILSEND completely.
    There will be no changes made on the fli4l boot medium \var{rootfs.img} 
    or the archive \var{opt.img}. 
    OPT\_MAILSEND does not overwrite other parts of the fli4l installation.
    
    To activate OPT\_MAILSEND set the variable \var{OPT\_MAILSEND} to \var{'yes'}

\config {MAILSEND\_SPOOL\_DIR}{MAILSEND\_SPOOL\_DIR}{MAILSENDSPOOLDIR}
    
    This optional setting determines the spool directory.
    The spool directory should be placed on a persistent partition.
    
    As defaultvar{'/data/spool/mail'} is used.
     
\config {MAILSEND\_LOGFILE}{MAILSEND\_LOGFILE}{MAILSENDLOGFILE}

    This optional setting determines the log file.
    This variable has to be set to \var{'syslog'} if logging shall be driected 
    to the system log. 
    
    This variable has to be set to \var{'{}'} to disable logging (default value).

\config {MAILSEND\_SEND\_DELAYED}{MAILSEND\_SEND\_DELAYED}{MAILSENDSENDDELAYED}
    
    When this setting is set to \var{'yes'}, the mail will be delivered delayed 
    by a cron job or an ip-up event.
    When set to \var{'no'} (default value) the mail will be sent immediately.
    
\config {MAILSEND\_CHARSET}{MAILSEND\_CHARSET}{MAILSENDCHARSET}
    
    Thie optional setting defines the character set for string encoding.
    As default value \var{'ISO-8859-1'} is used, e.g. \var{'UTF-8'} can be set. 
                                
\config {MAILSEND\_N}{MAILSEND\_N}{MAILSENDN}
  
    Defines the number of mail accounts. 

\end{description}

\subsubsection {Mail accounts}
    Settings for mail accounts, these settings are made separately for each 
    mail account.
     
\begin {description}

\config {MAILSEND\_\emph{x}\_ACCOUNT}{MAILSEND\_\emph{x}\_ACCOUNT}{MAILSENDACCOUNT}
    
    Defines the logical name of the mail account.
    
    Through this name multiple mail accounts can be distingueshed. 
    This name has to be different for all accounts.
    
    When no account has the name \var{'default'}, the account with index \var{'1'} 
    can be addressed aditionaly with this name.
    
\config {MAILSEND\_\emph{x}\_AUTH}{MAILSEND\_\emph{x}\_AUTH}{MAILSENDAUTH}
    
    Determines the authentification method. Possible values are:
    
    \begin {itemize}
      \item [\var{'none'}] no authentification
      \item [\var{'pop'}] POP3 before SMTP
      \item [\var{'plain'}] unencrypted password
      \item [\var{'cram-md5'}] MD5 hash values
      \item [\var{'login'}] unencrypted password (similar \var{'plain'})
    \end {itemize}
    
\config {MAILSEND\_\emph{x}\_CRYPT}{MAILSEND\_\emph{x}\_CRYPT}{MAILSENDCRYPT}

    Determines the encryption method. Possible values are:
    
    \begin {itemize}
      \item [\var{'none'}] no encryption
      \item [\var{'TLS'}] SSL/TLS
      \item [\var{'STARTTLS'}] TLS with STARTTLS
    \end {itemize}

\config {MAILSEND\_\emph{x}\_FROM}{MAILSEND\_\emph{x}\_FROM}{MAILSENDFROM}

    This optional setting defines the sender address.
    
    When this stting is empty 
    \smalljump{MAILSENDSMTPUSERNAME}{\var{MAILSEND\_\emph{x}\_SMTP\_USERNAME}} is used.
    
    The sender address can be overritten with the \smalljump{sec:mailsendapi}{API} call. 
 
\end{description}

\subsubsection {SMTP server}
\begin {description}

\config {MAILSEND\_\emph{x}\_SMTP\_SERVER}{MAILSEND\_\emph{x}\_SMTP\_SERVER}{MAILSENDSMTPSERVER}
    
    Defines the SMTP server. Hostname or IP address can be entered her.
            
\config {MAILSEND\_\emph{x}\_SMTP\_PORT}{MAILSEND\_\emph{x}\_SMTP\_PORT}{MAILSENDSMTPPORT}
    
    With this optinal setting the SMTP port number can be defined,
    as default value the  SMTP standard port \var{'25'} will be used.
    
\config {MAILSEND\_\emph{x}\_SMTP\_USERNAME}{MAILSEND\_\emph{x}\_SMTP\_USERNAME}{MAILSENDSMTPUSERNAME}
    
    The user name for SMTP authentification.
                
\config {MAILSEND\_\emph{x}\_SMTP\_PASSWORD}{MAILSEND\_\emph{x}\_SMTP\_PASSWORD}{MAILSENDSMTPPASSWORD}
    
    The password for SMTP authentification.

\end {description}

\subsubsection {Optional: POP3 server}

    These settings are only required when 
    \smalljump{MAILSENDAUTH}{\var{MAILSEND\_\emph{x}\_AUTH}} is set to \var{'pop'}. 

\begin {description}
\config {MAILSEND\_\emph{x}\_POP3\_SERVER}{MAILSEND\_\emph{x}\_POP3\_SERVER}{MAILSENDPOP3SERVER}

    Defines the POP3 server. Hostname or IP address can be entered her.
            
\config {MAILSEND\_\emph{x}\_POP3\_PORT}{MAILSEND\_\emph{x}\_POP3\_PORT}{MAILSENDPOP3PORT}
    
    With this optinal setting the POP3 port number can be defined,
    as default value the  POP3 standard port \var{'110'} will be used.
    
\config {MAILSEND\_\emph{x}\_POP3\_USERNAME}{MAILSEND\_\emph{x}\_POP3\_USERNAME}{MAILSENDPOP3USERNAME}
    
    The user name for POP3 before SMTP authentification.
    
\config {MAILSEND\_\emph{x}\_POP3\_PASSWORD}{MAILSEND\_\emph{x}\_POP3\_PASSWORD}{MAILSENDPOP3PASSWORD}
    
    The password for POP3 before SMTP authentification.

\end {description}

\subsubsection {Enhanced features}

\paragraph {Server certificate}
Server certificate examination

\begin {description}
    
\config {MAILSEND\_\emph{x}\_TEST\_SCERT}{MAILSEND\_\emph{x}\_TEST\_SCERT}{MAILSENDTESTSCERT}

    When this setting is set to \var{'yes'}, the SMTP server certificate will be
    testet against the certificate in \smalljump{MAILSENDSCERT}{MAILSEND\_\emph{x}\_SCERT}.

\config {MAILSEND\_\emph{x}\_SCERT}{MAILSEND\_\emph{x}\_SCERT}{MAILSENDSCERT}
    
    File name including path of the X.509 server certificate in PEM format,
    e.g. \var{'/etc/mailsend/server-cert.pem'}
    
    As path \var{'/etc/mailsend/\emph{server}.pem'} is recommended.
    The certificate can be placed in the configuration directory at 
    \var{\emph{\flq config\frq}/etc/mailsend/}.

\end {description}

\paragraph {Client certifikate/key}
\begin {description}

\config {MAILSEND\_\emph{x}\_USE\_CCERT}{MAILSEND\_\emph{x}\_USE\_CCERT}{MAILSENDUSECCERT}

    When this setting is set to \var{'yes'}, the SMTP client will be authenticated
    with the certicicate in \smalljump{MAILSENDCCERT}{MAILSEND\_\emph{x}\_CCERT}
    and the private key in \smalljump{MAILSENDCKEY}{MAILSEND\_\emph{x}\_CKEY}. 
    
\config {MAILSEND\_\emph{x}\_CCERT}{MAILSEND\_\emph{x}\_CCERT}{MAILSENDCCERT}
    
    File name of the X.509 client certificate in PEM format, \\
    e.g. \var{'/etc/mailsend/client-cert.pem'}

\config {MAILSEND\_\emph{x}\_CKEY}{MAILSEND\_\emph{x}\_CKEY}{MAILSENDCKEY}
    
    File name of the X.509 client key in PEM format, \\
    e.g. \var{'/etc/mailsend/client-key.pem'}
    
\end {description}

\marklabel{sec:mailsendapi}{
\subsection {API}}
\subsubsection{\var{mailsend}}
    The mail body can be passed to the API \var {/usr/local/bin/mailsend} 
    either as file, as quoted parameter or by the standard input pipe
    (see \smalljump{sec:mailsendexample}{Examples}).

    The API can be called with following parameters.
    
\begin {itemize}
  \item [\var{-t},] \var{-{}-to} \var{"\emph{\flq address\frq}"}
    \\required: Defines one or more comma separated recipient addresses
    the mail will be sent to.
    \\ This option can be set multple times.

  \item [\var{-s},] \var{-{}-subject} \var{"\emph{\flq subject\frq}"}
    \\required: Determines the subject of the mail.
  
  \item [\var{-A},] \var{-{}-account} \var{"\emph{\flq account\frq}"}
    \\optional: Defines the \smalljump{MAILSENDACCOUNT}{account name}
    used to send the mail. When this parameter is missing the account
    with the \smalljump{MAILSENDACCOUNT}{name} \var{default} is used.
  
  \item [\var{-f},] \var{-{}-from} \var{"\emph{\flq address\frq}"}
    \\optional: Defines the sender address . 
    If missing the address in 
    \smalljump{MAILSENDFROM}{\var{MAILSEND\_\emph{x}\_FROM}} is used.
  
  \item [\var{-c},] \var{-{}-cc} \var{"\emph{\flq address\frq}"}
    \\optional: Defines one or more comma separated CC addresses
    the mail will be sent to.
    \\ This option can be set multple times.

  \item [\var{-b},] \var{-{}-bcc} \var{"\emph{\flq address\frq}"}
    \\optional: Defines one or more comma separated BCC addresses
    the mail will be sent to.
    \\ This option can be set multple times.

  \item [\var{-r},] \var{-{}-reply-to} \var{"\emph{\flq address\frq}"}
    \\optional: Defines one or more comma separated reply addresses,
    an answer to the mail shall be sent to.
    \\ This option can be set multple times.

  \item [\var{-B},] \var{-{}-body} \var{"\emph{\flq file\frq}"}
    \\optional: Load the mail body from a file  (absolute path).
    \\When missing the mail body has to be passed as paramtere or by pipe 
    (see \smalljump{sec:mailsendexample}{Examples}).

  \item [\var{-a},] \var{-{}-attach-to} \var{"\emph{\flq file\frq}"}
    \\optional: Adds a file as mail attachment (absolute path).
    \\ This option can be set multple times.
    \\ Also multiple semicolon (;) separated files are allowed.
  
  \item [\var{-C},] \var{-{}-charset} \var{"\emph{\flq charset\frq}"}
    \\optional: Defines the character set for string encoding.
    When missing the character set defined with  
    \smalljump{MAILSENDCHARSET}{\var{MAILSEND\_CHARSET}} will be used.
\end{itemize}

\subsubsection{\var{mailsend} mail file}
    Alternatively a \smalljump{rfc5322}{RFC5322} compatible
    mail file can be sent. All parameteres defined in this file will be used. 
    
    Other parameters (like e.g. \var{-{}-to}) must not be used!  

\begin {itemize}
  \item [\var{-m},] \var{-{}-mailfile} \var{"\emph{\flq file\frq}"}
    \\required: Name of the mail file (absolute path).
  
  \item [\var{-A},] \var{-{}-account} \var{"\emph{\flq account\frq}"}
    \\optional: Defines the \smalljump{MAILSENDACCOUNT}{account name}
    used to send the mail. When this parameter is missing the account
    with the \smalljump{MAILSENDACCOUNT}{name} \var{default} is used.
\end{itemize}

% ##TRANSLATE## : FFL-1570: end
